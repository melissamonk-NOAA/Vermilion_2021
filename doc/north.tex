\input{input_accessability.tex}
\documentclass[11pt,
  english,
]{article}
\usepackage{sa4ss}
\usepackage{amsmath,amssymb,array}
\usepackage{booktabs}

% From tagged-template.latex
\usepackage{lmodern}
\usepackage{ifxetex,ifluatex}
\ifnum 0\ifxetex 1\fi\ifluatex 1\fi=0 % if pdftex
  \usepackage[T1]{fontenc}
  \usepackage[utf8]{inputenc}
  \usepackage{textcomp} % provide euro and other symbols
\else % if luatex or xetex
  \usepackage{unicode-math}
  \defaultfontfeatures{Scale=MatchLowercase}
  \defaultfontfeatures[\rmfamily]{Ligatures=TeX,Scale=1}
\fi

% Use upquote if available, for straight quotes in verbatim environments
\IfFileExists{upquote.sty}{\usepackage{upquote}}{}
\IfFileExists{microtype.sty}{% use microtype if available
  \usepackage[]{microtype}
  \UseMicrotypeSet[protrusion]{basicmath} % disable protrusion for tt fonts
}{}
\makeatletter
\@ifundefined{KOMAClassName}{% if non-KOMA class
  \IfFileExists{parskip.sty}{%
    \usepackage{parskip}
  }{% else
    \setlength{\parindent}{0pt}
    \setlength{\parskip}{6pt plus 2pt minus 1pt}}
}{% if KOMA class
  \KOMAoptions{parskip=half}}
\makeatother
\usepackage{xcolor}
\IfFileExists{xurl.sty}{\usepackage{xurl}}{} % add URL line breaks if available
\hypersetup{
  pdftitle={DRAFT The status of Vermilion Rockfish (Sebastes miniatus) and Sunset Rockfish (Sebastes crocotulus) in U.S. waters off the coast of California north of Point Conception in 2021},
  pdflang={en},
  hidelinks,
  pdfcreator={LaTeX via pandoc}}
\urlstyle{same} % disable monospaced font for URLs
\usepackage{longtable}
% Correct order of tables after \paragraph or \subparagraph
\usepackage{etoolbox}
\makeatletter
\patchcmd\longtable{\par}{\if@noskipsec\mbox{}\fi\par}{}{}
\makeatother
% Allow footnotes in longtable head/foot
\IfFileExists{footnotehyper.sty}{\usepackage{footnotehyper}}{\usepackage{footnote}}
\makesavenoteenv{longtable}
\usepackage{graphicx}
\makeatletter
\def\maxwidth{\ifdim\Gin@nat@width>\linewidth\linewidth\else\Gin@nat@width\fi}
\def\maxheight{\ifdim\Gin@nat@height>\textheight\textheight\else\Gin@nat@height\fi}
\makeatother
% Scale images if necessary, so that they will not overflow the page
% margins by default, and it is still possible to overwrite the defaults
% using explicit options in \includegraphics[width, height, ...]{}
\setkeys{Gin}{width=\maxwidth,height=\maxheight,keepaspectratio}
% Set default figure placement to htbp
\makeatletter
\def\fps@figure{htbp}
\makeatother
\setlength{\emergencystretch}{3em} % prevent overfull lines
\providecommand{\tightlist}{%
  \setlength{\itemsep}{0pt}\setlength{\parskip}{0pt}}
\setcounter{secnumdepth}{5}
\usepackage{booktabs}
\usepackage{longtable}
\usepackage{array}
\usepackage{multirow}
\usepackage{wrapfig}
\usepackage{float}
\usepackage{colortbl}
\usepackage{pdflscape}
\usepackage{tabu}
\usepackage{threeparttable}
\usepackage[normalem]{ulem}
\usepackage{makecell}
\usepackage{placeins}
\usepackage{caption}
\ifxetex
  % Load polyglossia as late as possible: uses bidi with RTL langages (e.g. Hebrew, Arabic)
  \usepackage{polyglossia}
  \setmainlanguage[]{english}
\else
  \usepackage[shorthands=off,main=english]{babel}
\fi

%Define cslreferences environment, required by pandoc 2.8
%https://github.com/rstudio/rmarkdown/issues/1649
\newlength{\csllabelwidth}
\setlength{\csllabelwidth}{3em}
\newlength{\cslhangindent}
\setlength{\cslhangindent}{1.5em}
% for Pandoc 2.8 to 2.10.1
\newenvironment{cslreferences}%
  {}%
  {\par}
% For Pandoc 2.11+
\newenvironment{CSLReferences}[2] % #1 hanging-ident, #2 entry spacing
 {% don't indent paragraphs
  \setlength{\parindent}{0pt}
  % turn on hanging indent if param 1 is 1
  \ifodd #1 \everypar{\setlength{\hangindent}{\cslhangindent}}\ignorespaces\fi
  % set entry spacing
  \ifnum #2 > 0
  \setlength{\parskip}{#2\baselineskip}
  \fi
 }%
 {}
\usepackage{calc}  % for \widthof, \maxof in minipage
\newcommand{\CSLBlock}[1]{#1\hfill\break}
\newcommand{\CSLLeftMargin}[1]{\parbox[t]{\csllabelwidth}{#1}}
\newcommand{\CSLRightInline}[1]{\parbox[t]{\linewidth - \csllabelwidth}{#1}\break}
\newcommand{\CSLIndent}[1]{\hspace{\cslhangindent}#1}


\providecommand{\tightlist}{%
  \setlength{\itemsep}{0pt}\setlength{\parskip}{0pt}}

\usepackage{booktabs}
\usepackage{longtable}
\usepackage{array}
\usepackage{multirow}
\usepackage{wrapfig}
\usepackage{float}
\usepackage{colortbl}
\usepackage{pdflscape}
\usepackage{tabu}
\usepackage{threeparttable}
\usepackage[normalem]{ulem}
\usepackage{makecell}
\usepackage{placeins}
\usepackage{caption}
\date{}
\newcommand{\trTitle}{DRAFT The status of Vermilion Rockfish (\emph{Sebastes miniatus}) and Sunset Rockfish (\emph{Sebastes crocotulus}) in U.S. waters off the coast of California north of Point Conception in 2021}
\newcommand{\trYear}{2021}
\newcommand{\trMonth}{August}
\newcommand{\trAuthsLong}{truetruetruetruetrue}
\newcommand{\trAuthsBack}{Monk, M.H., E.J. Dick, J.C. Field, E.M. Saas, T.L. Rogers}
\newcommand{\trCitation}{
\begin{hangparas}{1em}{1}
\trAuthsBack{}. \trYear{}. \trTitle{}. \glsentrylong{pfmc}, Portland, Oregon. \pageref{LastPage}{}\,p.
\end{hangparas}}

\AtBeginDocument{\tagstructbegin{tag=Document}}
\AtEndDocument{\tagstructend}
\pretocmd{\maketitle}{\tagstructbegin{tag=H1}\tagmcbegin{tag=H1}}{}{}
\apptocmd{\maketitle}{\tagmcend\tagstructend}{}{}

\begin{document}

%%%%% Frontmatter %%%%%

% Footnote symbols in front matter
\renewcommand*{\thefootnote}{\fnsymbol{footnote}}

\small
\thispagestyle{empty}
\pagenumbering{roman}
\noindent
\begin{center}
\title{DRAFT The status of Vermilion Rockfish (\emph{Sebastes miniatus}) and Sunset Rockfish (\emph{Sebastes crocotulus}) in U.S. waters off the coast of California north of Point Conception in 2021}
% \textnormal{\MakeTextUppercase{\trTitle{}}}
\vspace{1.5cm}
{\Large\textbf\newline{DRAFT The status of Vermilion Rockfish (\emph{Sebastes miniatus}) and Sunset Rockfish (\emph{Sebastes crocotulus}) in U.S. waters off the coast of California north of Point Conception in 2021}}
\vfill
by\\
Melissa H. Monk\textsuperscript{1}\\
E. J. Dick\textsuperscript{1}\\
John C. Field\textsuperscript{1}\\
Emma M. Saas\textsuperscript{2}\\
Tanya L. Rogers\textsuperscript{1}\vfill
\textsuperscript{1}Southwest Fisheries Science Center, U.S. Department of Commerce, National Oceanic and Atmospheric Administration, National Marine Fisheries Service, 110 McAllister Way, Santa Cruz, California 95060\\
\textsuperscript{2}Fisheries Collaborative Program, Insitute of Marine Sciences, University of California, Santa Cruz, 110 McAllister Way, Santa Cruz, California 95060\vfill
\trMonth{} \trYear{}
\end{center}
\clearpage

% Fourth page: Colophon
\thispagestyle{empty}
\vspace*{\fill}
\begin{center}
\copyright{} \glsentrylong{pfmc}, \trYear{}\\
\end{center}
\par
\bigskip
\noindent
Correct citation for this publication:
\bigskip
\par
\trCitation{}
\clearpage

% Add TOC to pdf bookmarks (clickable pdf)
\pdfbookmark[1]{\contentsname}{toc}

% Table of contents page, lists of figures and tables
\tableofcontents\clearpage
\label{TRlastRoman}
\clearpage

% Table of contents
\newpage
\thispagestyle{empty} % to remove page number

% Settings for the main document
\pagenumbering{arabic}  % Regular page numbers
\pagestyle{plain}  % No page number on first page of main document, use 'empty'
\renewcommand*{\thefootnote}{\arabic{footnote}}  % Back to numeric footnotes
\setcounter{footnote}{0}  % And start at 1
\renewcommand{\headrulewidth}{0.5pt}
\renewcommand{\footrulewidth}{0.5pt}
%\pagestyle{fancy}\fancyhead[c]{Draft: Do not cite or circulate}

\newcommand{\lt}{\ensuremath <}
\newcommand{\gt}{\ensuremath >}

\newcommand\CapeM{$40^\circ 10^\prime N$}
\newcommand\PtC{$34^\circ 27^\prime N$}
\newcommand\CAOR{$42^\circ 00^\prime N$}

\newpage

\includegraphics{cover_photo.png} Two fish of the vermilion/sunset rockfish cryptic species pair. Confirmation of species can only be determined via genetic analysis and species identification of these two fish caught in the Santa Barbara channel at approximately 250 ft depth is unknown. Photo courtesy of Sabrina Beyer (UCSC/NOAA).

\pagebreak
\pagenumbering{roman}
\setcounter{page}{1}

\renewcommand{\thetable}{\roman{table}}
\renewcommand{\thefigure}{\roman{figure}}

\setlength\parskip{0.5em plus 0.1em minus 0.2em}

\tagstructbegin{tag=H1}\tagmcbegin{tag=H1}

\hypertarget{executive-summary}{%
\section*{Executive Summary}\label{executive-summary}}
\addcontentsline{toc}{section}{Executive Summary}

\leavevmode\tagmcend\tagstructend

\tagstructbegin{tag=H2}\tagmcbegin{tag=H2}

\hypertarget{stock}{%
\subsection*{Stock}\label{stock}}
\addcontentsline{toc}{subsection}{Stock}

\leavevmode\tagmcend\tagstructend

This assessment reports the combined status of the vermilion rockfish (\emph{Sebastes miniatus}) and sunset rockfish (\emph{Sebastes crocotulus}), referred to as ``vermilion rockfish'' throughout this document, in U.S. waters off the coast of California north of Point Conception ($34^\circ 27^\prime N$) using data through 2020. Genetic evidence suggests overlapping distributions for the two species, with the majority of the sunset rockfish population occupying waters south of Point Conception. Alternative spatial structures for the vermilion rockfish assessment should be considered if additional data on stock structure and the distribution of the two species become available.

\tagstructbegin{tag=H2}\tagmcbegin{tag=H2}

\hypertarget{catches}{%
\subsection*{Catches}\label{catches}}
\addcontentsline{toc}{subsection}{Catches}

\leavevmode\tagmcend\tagstructend

Over the past decade, vermilion rockfish in the assessed area off the coast of California have been primarily caught by the recreational fishery (Table \ref{tab:removalsES}). Annual total mortality of catch and discards of vermilion rockfish have ranged between 76-204 mt, with total mortality (catch + discards) in 2020 of 139 mt. Vermilion and sunset rockfish landings from all sectors have historically been recorded as ``vermilion rockfish'' and sampling programs in California currently do not differentiate between the two species.

Recreational removals in California prior to 2004 were only estimated at large spatial scales (north and south of Point Conception) following the design of the Marine Recreational Fisheries Statistics Survey (MRFSS). Recent sampling (2004 -- present) by the California Recreational Fisheries Survey (CRFS) produces estimates of vermilion rockfish landings and discard at a finer spatial resolution. Total removals north of Point Conception increased steadily following World War II, peaking in the late 1970s and 1980s with annual removals of 365 mt per year (Figure \ref{fig:catchES}). Recent years have seen a steady increase in landings, with recreational fleets accounting for the majority of total mortality.

\FloatBarrier

\begin{figure}
\centering
\includegraphics[width=1\textwidth,height=1\textheight]{C:/Stock_Assessments/VRML_Assessment_2021/Model_files/NCA/FINAL_base/Verm_NoCA_base_files_Day1_Request1_drop2020PR_weight_CCFRP_lengths_hessian - Copy/plots/catch2 landings stacked.png}
\caption{Catch histories by fleet used in the base model (Commercial hook-and-line = COM\_HKL, Commercial trawl = COM\_TWL, Commercial net = COM\_NET, Recreational party/charter retained = REC\_PC, Recreational private/rental retained = REC\_PR, Recreational party/charter dead discards = REC\_PC\_DIS, Recreational private/rental dead discards = REC\_PR\_DIS).\label{fig:catchES}}
\end{figure}

\begin{table}[H]

\caption{\label{tab:removalsES}Recent mortality (mt) by fleet and total landings summed across 
                all fleets in the model.}
\centering
\resizebox{\linewidth}{!}{
\begin{tabular}[t]{c>{\centering\arraybackslash}p{.6in}>{\centering\arraybackslash}p{.6in}>{\centering\arraybackslash}p{.6in}>{\centering\arraybackslash}p{.6in}>{\centering\arraybackslash}p{.6in}>{\centering\arraybackslash}p{.6in}>{\centering\arraybackslash}p{.6in}>{\centering\arraybackslash}p{.6in}}
\toprule
\multicolumn{1}{c}{\textbf{ }} & \multicolumn{3}{c}{\textbf{Commercial}} & \multicolumn{4}{c}{\textbf{Recreational}} & \multicolumn{1}{c}{\textbf{ }} \\
\multicolumn{4}{c}{ } & \multicolumn{2}{c}{Party/charter} & \multicolumn{2}{c}{Private/rental} & \multicolumn{1}{c}{ } \\
\cmidrule(l{3pt}r{3pt}){5-6} \cmidrule(l{3pt}r{3pt}){7-8}
Year & Hook-and-Line & Trawl & Net & Retained &  Dead Discards & Retained & Dead Discards & Total Mortality\\
\midrule
\cellcolor{gray!6}{2011} & \cellcolor{gray!6}{10.0} & \cellcolor{gray!6}{0.0} & \cellcolor{gray!6}{0} & \cellcolor{gray!6}{40.3} & \cellcolor{gray!6}{0.3} & \cellcolor{gray!6}{49.4} & \cellcolor{gray!6}{0.1} & \cellcolor{gray!6}{100.1}\\
2012 & 9.4 & 0.0 & 0 & 36.0 & 0.2 & 41.2 & 0.2 & 87.1\\
\cellcolor{gray!6}{2013} & \cellcolor{gray!6}{13.8} & \cellcolor{gray!6}{0.0} & \cellcolor{gray!6}{0} & \cellcolor{gray!6}{21.1} & \cellcolor{gray!6}{0.1} & \cellcolor{gray!6}{40.6} & \cellcolor{gray!6}{0.1} & \cellcolor{gray!6}{75.8}\\
2014 & 14.1 & 0.0 & 0 & 21.1 & 0.0 & 41.7 & 0.2 & 77.2\\
\cellcolor{gray!6}{2015} & \cellcolor{gray!6}{18.2} & \cellcolor{gray!6}{0.4} & \cellcolor{gray!6}{0} & \cellcolor{gray!6}{40.0} & \cellcolor{gray!6}{0.1} & \cellcolor{gray!6}{64.6} & \cellcolor{gray!6}{0.2} & \cellcolor{gray!6}{123.5}\\
2016 & 13.3 & 0.1 & 0 & 38.0 & 0.2 & 60.3 & 0.3 & 112.1\\
\cellcolor{gray!6}{2017} & \cellcolor{gray!6}{14.2} & \cellcolor{gray!6}{0.1} & \cellcolor{gray!6}{0} & \cellcolor{gray!6}{92.3} & \cellcolor{gray!6}{0.4} & \cellcolor{gray!6}{58.3} & \cellcolor{gray!6}{0.3} & \cellcolor{gray!6}{165.6}\\
2018 & 19.0 & 0.6 & 0 & 88.0 & 0.1 & 72.4 & 0.2 & 180.5\\
\cellcolor{gray!6}{2019} & \cellcolor{gray!6}{19.6} & \cellcolor{gray!6}{0.0} & \cellcolor{gray!6}{0} & \cellcolor{gray!6}{92.0} & \cellcolor{gray!6}{0.3} & \cellcolor{gray!6}{91.9} & \cellcolor{gray!6}{0.6} & \cellcolor{gray!6}{204.4}\\
2020 & 19.9 & 0.0 & 0 & 55.4 & 0.2 & 63.3 & 0.3 & 139.0\\
\bottomrule
\end{tabular}}
\end{table}

\FloatBarrier

\tagstructbegin{tag=H2}\tagmcbegin{tag=H2}

\hypertarget{data-and-assessment}{%
\subsection*{Data and Assessment}\label{data-and-assessment}}
\addcontentsline{toc}{subsection}{Data and Assessment}

\leavevmode\tagmcend\tagstructend

A full assessment was attempted in 2005, but not accepted for management and a data-moderate assessment in 2013 was not reviewed. As such, this is the first benchmark assessment for vermilion and sunset rockfish. The 2021 assessment uses Stock Synthesis 3 (version V3.30.17.0). The assessment is a two-sex model, with the population spanning from Point Conception ($34^\circ 27^\prime N$) to the California/Oregon border ($42^\circ 00^\prime N$). The assessment model operates on an annual time step covering the period 1875 to 2020 (not including forecast years) and assumes an unfished population prior to 1875. Population dynamics are modeled for ages 0 through 70, with age-70 being the accumulator age.

The model is conditioned on catch from two sectors (commercial and recreational) divided among seven fleets, and is informed by five abundance indices (one fishery-independent survey, two CPUE indices from shore-based recreational fishery sampling programs, and two CPUE indices from recreational onboard party/charter boat observer programs). The model is also fit to length composition data from fishery-independent and fishery-dependent sources, as well as age compositions conditioned on length. Discards for the commercial fleets are not included in the model. Commercial discards of vermilion rockfish are a small fraction of the total mortality and data on commercial discard length composition is limited. The recreational fishery is split into four fleets, one discard and one retained fish fleet each for the private/rental and the party/charter boat modes. The model also incorporates an updated length-weight relationship, length-based maturity schedule, and fecundity-at-length function.

The assessment estimates parameters for natural mortality of females and males, and sex-specific growth parameters. Year class strength is estimated as deviations from a Beverton-Holt stock-recruitment relationship beginning in 1970. Steepness of the Beverton-Holt stock-recruitment relationship is fixed at the mean of the prior ({\tagstructbegin{tag=Formula}\tagmcbegin{tag=Formula}\(h\)\leavevmode\tagmcend\tagstructend} = 0.72).

\FloatBarrier

\tagstructbegin{tag=H2}\tagmcbegin{tag=H2}

\hypertarget{stock-biomass}{%
\subsection*{Stock Biomass}\label{stock-biomass}}
\addcontentsline{toc}{subsection}{Stock Biomass}

\leavevmode\tagmcend\tagstructend

Spawning output of vermilion rockfish was estimated to be 489 million eggs in 2021 (95\% asymptotic interval: 263 - 716 million eggs) or 43\% (95\% asymptotic interval: 25\% - 61\%) of unfished spawning output (``depletion,'' Table \ref{tab:ssbES}). Depletion is a ratio of the estimated spawning output in a particular year relative to estimated unfished, equilibrium spawning output.

In northern California, spawning output declined rapidly in the 1970s and early 1980s, likely falling below the minimum stock size threshold for a number of years in the 1990s and early 2000s, followed by a steady recovery since the late 2000s (Figures \ref{fig:ssbES} and \ref{fig:deplES}). The point estimate for spawning output in 2021 is just above the management target (40\% of unfished spawning output).

\begin{figure}
\centering
\includegraphics[width=1\textwidth,height=1\textheight]{C:/Stock_Assessments/VRML_Assessment_2021/Model_files/NCA/FINAL_base/Verm_NoCA_base_files_Day1_Request1_drop2020PR_weight_CCFRP_lengths_hessian - Copy//plots/ts7_Spawning_output_with_95_asymptotic_intervals_intervals.png}
\caption{Estimated time series of spawning output (solid line with circles) with approximate 95\% asymptotic confidence intervals (dashed lines).\label{fig:ssbES}}
\end{figure}

\begin{figure}
\centering
\includegraphics[width=1\textwidth,height=1\textheight]{C:/Stock_Assessments/VRML_Assessment_2021/Model_files/NCA/FINAL_base/Verm_NoCA_base_files_Day1_Request1_drop2020PR_weight_CCFRP_lengths_hessian - Copy//plots/ts9_Relative_spawning_output_intervals.png}
\caption{Estimated time series of spawning output relative to unfished spawning output (solid line with circles) with approximate 95\% asymptotic confidence intervals (dashed lines).\label{fig:deplES}}
\end{figure}

\begin{table}[H]

\caption{\label{tab:ssbES}Estimated recent trend in spawning output and the fraction unfished and the approximate 95\% asymptotic confidence intervals.}
\centering
\begin{tabular}[t]{c>{\centering\arraybackslash}p{.6in}>{\centering\arraybackslash}p{.6in}>{\centering\arraybackslash}p{.6in}|>{\centering\arraybackslash}p{.6in}>{\centering\arraybackslash}p{.6in}>{\centering\arraybackslash}p{.6in}}
\toprule
\multicolumn{1}{c}{\textbf{ }} & \multicolumn{3}{c}{\textbf{Spawning Output}} & \multicolumn{3}{c}{\textbf{Fraction Unfished}} \\
\cmidrule(l{3pt}r{3pt}){2-4} \cmidrule(l{3pt}r{3pt}){5-7}
Year & Estimate & Lower Interval & Upper Interval & Estimate & Lower Interval & Upper Interval\\
\midrule
2011 & 431.973 & 244.002 & 619.944 & 0.377 & 0.227 & 0.527\\
2012 & 435.431 & 244.955 & 625.907 & 0.380 & 0.229 & 0.531\\
2013 & 442.395 & 249.226 & 635.564 & 0.386 & 0.234 & 0.539\\
2014 & 454.034 & 257.314 & 650.754 & 0.396 & 0.241 & 0.552\\
2015 & 469.146 & 267.897 & 670.395 & 0.410 & 0.251 & 0.568\\
2016 & 479.639 & 273.578 & 685.700 & 0.419 & 0.257 & 0.581\\
2017 & 490.602 & 279.902 & 701.302 & 0.428 & 0.263 & 0.594\\
2018 & 490.707 & 275.944 & 705.470 & 0.428 & 0.260 & 0.597\\
2019 & 487.751 & 269.376 & 706.126 & 0.426 & 0.254 & 0.598\\
2020 & 482.178 & 260.377 & 703.979 & 0.421 & 0.246 & 0.596\\
2021 & 489.439 & 263.228 & 715.650 & 0.427 & 0.249 & 0.606\\
\bottomrule
\end{tabular}
\end{table}

\FloatBarrier

\tagstructbegin{tag=H2}\tagmcbegin{tag=H2}

\hypertarget{recruitment}{%
\subsection*{Recruitment}\label{recruitment}}
\addcontentsline{toc}{subsection}{Recruitment}

\leavevmode\tagmcend\tagstructend

Recruitment deviations were estimated from 1970-2020 with a recent, strong recruitment in 2016 that has contributed to the recent increase in vermilion rockfish biomass in northern California (Table \ref{tab:recrES}; Figure \ref{fig:recruitsES}). The second highest estimated recruitment occurred in 1985 and is more certain than the estimated 2016 recruitment.

\begin{figure}
\centering
\includegraphics[width=1\textwidth,height=1\textheight]{C:/Stock_Assessments/VRML_Assessment_2021/Model_files/NCA/FINAL_base/Verm_NoCA_base_files_Day1_Request1_drop2020PR_weight_CCFRP_lengths_hessian - Copy//plots/ts11_Age-0_recruits_(1000s)_with_95_asymptotic_intervals.png}
\caption{Age-0 recruits (1,000s) with approximate 95\% asymptotic confidence intervals.\label{fig:recruitsES}}
\end{figure}

\begin{table}[H]

\caption{\label{tab:recrES}Estimated recent trend in recruitment and recruitment 
                deviations and the approximate 95\% asymptotic confidence intervals.}
\centering
\begin{tabular}[t]{r>{\raggedleft\arraybackslash}p{.6in}>{\raggedleft\arraybackslash}p{.6in}>{\raggedleft\arraybackslash}p{.6in}|>{\raggedleft\arraybackslash}p{.6in}>{\raggedleft\arraybackslash}p{.6in}>{\raggedleft\arraybackslash}p{.6in}}
\toprule
\multicolumn{1}{c}{\textbf{ }} & \multicolumn{3}{c}{\textbf{Recruitment}} & \multicolumn{3}{c}{\textbf{Recruitment Deviations}} \\
\cmidrule(l{3pt}r{3pt}){2-4} \cmidrule(l{3pt}r{3pt}){5-7}
Year & Estimate & Lower Interval & Upper Interval & Estimate & Lower Interval & Upper Interval\\
\midrule
2011 & 225 & 116 & 437 & -0.397 & -0.956 & 0.163\\
2012 & 408 & 224 & 741 & 0.196 & -0.279 & 0.672\\
2013 & 466 & 242 & 896 & 0.326 & -0.220 & 0.872\\
2014 & 476 & 239 & 946 & 0.341 & -0.248 & 0.930\\
2015 & 277 & 125 & 616 & -0.215 & -0.937 & 0.506\\
2016 & 1536 & 814 & 2901 & 1.472 & 0.963 & 1.980\\
2017 & 163 & 65 & 409 & -0.800 & -1.680 & 0.081\\
2018 & 387 & 147 & 1022 & 0.048 & -0.892 & 0.988\\
2019 & 373 & 138 & 1004 & 0.003 & -0.964 & 0.970\\
2020 & 374 & 138 & 1010 & 0.009 & -0.961 & 0.978\\
2021 & 372 & 140 & 991 & 0.000 & -0.980 & 0.980\\
\bottomrule
\end{tabular}
\end{table}

\FloatBarrier

\tagstructbegin{tag=H2}\tagmcbegin{tag=H2}

\hypertarget{exploitation-status}{%
\subsection*{Exploitation Status}\label{exploitation-status}}
\addcontentsline{toc}{subsection}{Exploitation Status}

\leavevmode\tagmcend\tagstructend

The annual (equilibrium) spawning potential ratio (SPR) for vermilion rockfish was above target from 2017-2019 (Table \ref{tab:exploitES}, Figure \ref{fig:1-sprES}). Prior to 2011, the fishing intensity exceeded the target for a number of years, regularly reaching levels 50\% above target in the 1980s and 1990s (Figure \ref{fig:1-sprES}). As with current estimates of spawning output, recent estimates of equilibrium SPR are highly uncertain, ranging from 68\% to 129\% of target in 2020 (Table \ref{tab:exploitES}). As a percentage of total biomass (ages 4+), California harvest rates peaked in the 1980s and 1990s, but have since declined to levels below 10\% for the past decade (Figure \ref{fig:FmortalityES}). Harvest rates in northern California were near target in 2020, but above target in the three previous years (Figure \ref{fig:phaseES}). However, the harvest rate in 2019 was above target, and may be more representative of future catches, all else equal, given reductions in fishing activity during the 2020 pandemic. The equilibrium yield curve is shifted left, as expected from the Beverton-Holt steepness parameter fixed at 0.72 (Figure \ref{fig:yield2ES}).

\begin{table}[H]

\caption{\label{tab:exploitES}Estimated recent trend in the relative fishing intensity
                ($\frac{1-SPR}{1-SPR_{50\%}}$, 
                where SPR is the spawning potential ratio) and the exploitation rate, 
                with approximate 95\% asymptotic confidence intervals.}
\centering
\begin{tabular}[t]{r>{\raggedleft\arraybackslash}p{.6in}>{\raggedleft\arraybackslash}p{.6in}>{\raggedleft\arraybackslash}p{.6in}|>{\raggedleft\arraybackslash}p{.6in}>{\raggedleft\arraybackslash}p{.6in}>{\raggedleft\arraybackslash}p{.6in}}
\toprule
\multicolumn{1}{c}{\textbf{ }} & \multicolumn{3}{c}{\textbf{Relative Fishing Intensity}} & \multicolumn{3}{c}{\textbf{Exploitation Rate}} \\
\cmidrule(l{3pt}r{3pt}){2-4} \cmidrule(l{3pt}r{3pt}){5-7}
Year & Estimate & Lower Interval & Upper Interval & Estimate & Lower Interval & Upper Interval\\
\midrule
2011 & 0.939 & 0.653 & 1.224 & 0.061 & 0.037 & 0.085\\
2012 & 0.826 & 0.558 & 1.094 & 0.051 & 0.031 & 0.071\\
2013 & 0.715 & 0.469 & 0.961 & 0.041 & 0.025 & 0.056\\
2014 & 0.701 & 0.461 & 0.941 & 0.040 & 0.024 & 0.055\\
2015 & 0.966 & 0.684 & 1.249 & 0.062 & 0.038 & 0.087\\
2016 & 0.905 & 0.629 & 1.181 & 0.058 & 0.035 & 0.080\\
2017 & 1.108 & 0.808 & 1.408 & 0.077 & 0.045 & 0.108\\
2018 & 1.164 & 0.861 & 1.467 & 0.081 & 0.047 & 0.115\\
2019 & 1.248 & 0.943 & 1.554 & 0.094 & 0.054 & 0.133\\
2020 & 0.990 & 0.684 & 1.296 & 0.061 & 0.035 & 0.088\\
\bottomrule
\end{tabular}
\end{table}

\begin{figure}
\centering
\includegraphics[width=1\textwidth,height=1\textheight]{C:/Stock_Assessments/VRML_Assessment_2021/Model_files/NCA/FINAL_base/Verm_NoCA_base_files_Day1_Request1_drop2020PR_weight_CCFRP_lengths_hessian - Copy//plots/SPR3_ratiointerval.png}
\caption{Timeseries of relative fishing intensity ({\tagstructbegin{tag=Formula}\tagmcbegin{tag=Formula}\(\frac{1-SPR}{1-SPR_{50\%}}\)\leavevmode\tagmcend\tagstructend} where SPR is the spawning potential ratio) with approximate 95\% asymptotic confidence intervals (dashed lines).\label{fig:1-sprES}}
\end{figure}

\begin{figure}
\centering
\includegraphics[width=1\textwidth,height=1\textheight]{C:/Stock_Assessments/VRML_Assessment_2021/Model_files/NCA/FINAL_base/Verm_NoCA_base_files_Day1_Request1_drop2020PR_weight_CCFRP_lengths_hessian - Copy//plots/ts_summaryF.png}
\caption{Time-series of estimated summary harvest rate (total catch divided by age-4 and older biomass) for the base case model with approximate 95\% asymptotic confidence intervals (vertical lines).\label{fig:FmortalityES}}
\end{figure}

\begin{figure}
\centering
\includegraphics[width=1\textwidth,height=1\textheight]{C:/Stock_Assessments/VRML_Assessment_2021/Model_files/NCA/FINAL_base/Verm_NoCA_base_files_Day1_Request1_drop2020PR_weight_CCFRP_lengths_hessian - Copy//plots/SPR4_phase.png}
\caption{Phase plot of the relative biomass (also referred to as fraction unfished) versus the SPR ratio where each point represents the biomass ratio at the start of the year and the relative fishing intensity in that same year. Lines through the final point (representing 2020) show the 95\% intervals based on the asymptotic uncertainty for each dimension. The shaded ellipse is a 95\% region which accounts for the estimated correlations between the biomass ratio and SPR ratio. Fishing intensity in 2020 was reduced to due the pandemic.\label{fig:phaseES}}
\end{figure}

\begin{figure}
\centering
\includegraphics[width=1\textwidth,height=1\textheight]{C:/Stock_Assessments/VRML_Assessment_2021/Model_files/NCA/FINAL_base/Verm_NoCA_base_files_Day1_Request1_drop2020PR_weight_CCFRP_lengths_hessian - Copy//plots/yield2_yield_curve_with_refpoints.png}
\caption{Equilibrium yield curve for the base case model with management quantities. Values are based on the 2020 fishery selectivities.\label{fig:yield2ES}}
\end{figure}

\FloatBarrier

\tagstructbegin{tag=H2}\tagmcbegin{tag=H2}

\hypertarget{ecosystem-considerations}{%
\subsection*{Ecosystem Considerations}\label{ecosystem-considerations}}
\addcontentsline{toc}{subsection}{Ecosystem Considerations}

\leavevmode\tagmcend\tagstructend

In this assessment, ecosystem considerations were not explicitly included in analyses. This is primarily due to a lack of relevant data that could contribute ecosystem-related quantitative information for the assessment.

Vermilion/sunset rockfish are described as feeding on a wide range of both pelagic and benthic prey items, including forage fish species such as anchovies and mesopelagic fishes, squid, krill and octopus, as well as sporadically abundant pelagic organisms such as pyrosomes, salps and pelagic red crabs.

As with most other rockfish and groundfish in the California Current, recruitment, or cohort (year-class) strength appears to be highly variable for the vermilion/sunset rockfish complex, with only a modest apparent relationship to estimated levels of spawning output. Oceanographic and ecosystem factors are widely recognized to be key drivers of recruitment variability for most species of groundfish, as well as most elements of California Current food webs. With additional research, it may be feasible to incorporate ecosystem factors using results of pre-recruit surveys for co-occurring species or results from more data-rich groundfish assessments Such approaches would require more development and evaluation. Consequently, environmental factors are not explicitly considered in this assessment.

\FloatBarrier

\tagstructbegin{tag=H2}\tagmcbegin{tag=H2}

\hypertarget{reference-points}{%
\subsection*{Reference Points}\label{reference-points}}
\addcontentsline{toc}{subsection}{Reference Points}

\leavevmode\tagmcend\tagstructend

Reference point and management quantities for the vermilion rockfish base case model can be found in Table \ref{tab:referenceES}. In 2021, spawning output relative to unfished spawning output (``depletion'') is estimated at 43\% (95\% asymptotic interval: 25\% - 61\%). This stock assessment estimates that vermilion rockfish in the north is above the biomass target ({\tagstructbegin{tag=Formula}\tagmcbegin{tag=Formula}\(SB_{40\%}\)\leavevmode\tagmcend\tagstructend}), and well above the minimum stock size threshold ({\tagstructbegin{tag=Formula}\tagmcbegin{tag=Formula}\(SB_{25\%}\)\leavevmode\tagmcend\tagstructend}). Unfished age four-plus biomass is estimated to be 6342 mt in the base case model (95\% asymptotic interval: 5667 - 7017 mt). The target spawning output ({\tagstructbegin{tag=Formula}\tagmcbegin{tag=Formula}\(SB_{40\%}\)\leavevmode\tagmcend\tagstructend}) is 458 million eggs (95\% asymptotic interval: 366 - 550 million eggs). Equilibrium yield at the proxy {\tagstructbegin{tag=Formula}\tagmcbegin{tag=Formula}\(F_{MSY}\)\leavevmode\tagmcend\tagstructend} harvest rate corresponding to {\tagstructbegin{tag=Formula}\tagmcbegin{tag=Formula}\(SPR_{50\%}\)\leavevmode\tagmcend\tagstructend} is 139 mt (95\% asymptotic interval: 118 - 160 mt, Table \ref{tab:referenceES} and Figure \ref{fig:yield2ES}).

\begin{table}[H]

\caption{\label{tab:referenceES}Summary of reference points and management quantities including estimates of the approximate 95\% asymtotic confidence intervals.}
\centering
\resizebox{\linewidth}{!}{
\begin{tabular}[t]{lrrr}
\toprule
 & Estimate & Lower Interval & Upper Interval\\
\midrule
Unfished Spawning Output & 1145.180 & 914.835 & 1375.525\\
Unfished Age 4+ Biomass (mt) & 6341.790 & 5666.596 & 7016.984\\
Unfished Recruitment ($R_0$) & 420.186 & 299.040 & 541.332\\
Spawning Output (2021) & 489.439 & 263.228 & 715.650\\
Fraction Unfished (2021) & 0.427 & 0.249 & 0.606\\
\addlinespace[0.3em]
\multicolumn{4}{l}{\textbf{Reference Points Based on $SB_{40\%}$}}\\
\hspace{1em}Proxy Spawning Output $SB_{40\%}$ & 458.073 & 365.935 & 550.211\\
\hspace{1em}SPR Resulting in $SB_{40\%}$ & 0.458 & 0.458 & 0.458\\
\hspace{1em}Exploitation Rate Resulting in $SB_{40\%}$ & 0.071 & 0.060 & 0.083\\
\hspace{1em}Yield with SPR Based On $SB_{40\%}$ (mt) & 145.614 & 123.238 & 167.990\\
\addlinespace[0.3em]
\multicolumn{4}{l}{\textbf{Reference Points Based on SPR Proxy for MSY}}\\
\hspace{1em}Proxy Spawning Output ($SPR_{50\%}$) & 510.928 & 408.159 & 613.697\\
\hspace{1em}$SPR_{50\%}$ & 0.500 &  & \\
\hspace{1em}Exploitation Rate Corresponding to $SPR_{50\%}$ & 0.062 & 0.052 & 0.073\\
\hspace{1em}Yield with $SPR_{50\%}$ at $SB_{SPR}$ (mt) & 138.992 & 117.750 & 160.234\\
\addlinespace[0.3em]
\multicolumn{4}{l}{\textbf{Reference Points Based on Estimated MSY Values}}\\
\hspace{1em}Spawning Output at MSY ($SB_{MSY}$) & 308.931 & 249.480 & 368.382\\
\hspace{1em}$SPR_{MSY}$ & 0.341 & 0.332 & 0.349\\
\hspace{1em}Exploitation Rate Corresponding to $SPR_{MSY}$ & 0.104 & 0.087 & 0.121\\
\hspace{1em}MSY (mt) & 155.029 & 130.706 & 179.352\\
\bottomrule
\end{tabular}}
\end{table}

\FloatBarrier

\tagstructbegin{tag=H2}\tagmcbegin{tag=H2}

\hypertarget{management-performance}{%
\subsection*{Management Performance}\label{management-performance}}
\addcontentsline{toc}{subsection}{Management Performance}

\leavevmode\tagmcend\tagstructend

Vermilion rockfish have been managed as part of the minor shelf rockfish complex in the Pacific Coast Groundfish Fishery Management Plan. North of $40^\circ 10^\prime N$, total mortality of the minor shelf rockfish complex has exceeded the OFL since 2011. South of $40^\circ 10^\prime N$, total mortality of the minor shelf rockfish complex has exceeded the OFL since 2015, and exceeded the ABC in most years since 2011. Total mortality estimates from the NWFSC are not yet available for 2020. A summary of these values as well as other base case summary results can be found in Tables \ref{tab:managementES} and \ref{tab:summaryES}.

Results from post-STAR base models in all areas (southern California, northern California, Oregon, and Washington) are presented in Table \ref{tab:CombinedRefPtsES}. The fraction of the northern California model allocated to the northern management area (north of $40^\circ 10^\prime N$) is based on an Appendix in northern California assessment.

\begin{table}[H]

\caption{\label{tab:managementES}Annual estimates of total mortality, overfishing limit (OFL), acceptable biological catch (ABC), annual catch limit (ACL) for vermilion rockfish in the minor shelf rockfish complex as reported in the GEMM report (NWFSC).}
\centering
\resizebox{\linewidth}{!}{
\begin{tabular}[t]{lrrrrrrrrrrrr}
\toprule
 & 2011 & 2012 & 2013 & 2014 & 2015 & 2016 & 2017 & 2018 & 2019 & 2020 & 2021 & 2022\\
\midrule
\addlinespace[0.3em]
\multicolumn{13}{l}{\textbf{North of 40°10' N}}\\
\hspace{1em}OFL & 11.127 & 11.127 & 9.717 & 9.717 & 9.717 & 9.717 & 9.720 & 9.720 & 9.720 & 9.720 & 9.700 & 9.700\\
\hspace{1em}ABC & 5.564 & 5.564 & 8.104 & 8.104 & 8.104 & 8.104 & 8.104 & 8.104 & 8.104 & 8.104 & 7.547 & 7.547\\
\hspace{1em}Total landings & 15.249 & 18.695 & 14.149 & 10.504 & 13.472 & 12.104 & 20.602 & 22.949 & 25.696 &  &  & \\
\hspace{1em}CA rec. landings & 4.209 & 4.867 & 2.657 & 2.950 & 5.018 & 4.549 & 6.490 & 7.631 & 7.884 &  &  & \\
\hspace{1em}OR rec. landings & 6.102 & 9.150 & 6.305 & 3.949 & 4.653 & 3.689 & 8.798 & 9.199 & 9.252 &  &  & \\
\hspace{1em}WA rec. landings & 1.001 & 0.911 & 1.279 & 0.960 & 1.141 & 0.997 & 0.731 & 1.151 & 2.497 &  &  & \\
\hspace{1em}Commercial landings & 3.935 & 3.767 & 3.906 & 2.644 & 2.661 & 2.799 & 4.557 & 4.966 & 6.063 &  &  & \\
\hspace{1em}Research & 0.002 &  & 0.002 & 0.002 &  & 0.069 & 0.026 & 0.002 &  &  &  & \\
\midrule
\addlinespace[0.3em]
\multicolumn{13}{l}{\textbf{South of 40°10' N}}\\
\hspace{1em}OFL & 308.359 & 308.359 & 269.276 & 269.276 & 269.276 & 269.276 & 269.280 & 269.280 & 269.280 & 269.280 & 269.280 & 269.280\\
\hspace{1em}ABC & 154.179 & 154.179 & 224.576 & 224.576 & 224.576 & 224.576 & 224.580 & 224.580 & 224.580 & 224.580 & 209.515 & 209.515\\
\hspace{1em}Total landings & 210.310 & 235.216 & 237.074 & 197.043 & 334.984 & 292.375 & 341.207 & 344.454 & 484.967 &  &  & \\
\hspace{1em}CA rec. landings & 191.437 & 216.480 & 208.198 & 167.572 & 291.779 & 260.162 & 287.493 & 278.158 & 413.946 &  &  & \\
\hspace{1em}Commercial landings & 16.928 & 16.642 & 26.601 & 26.607 & 39.669 & 29.148 & 48.195 & 59.644 & 67.189 &  &  & \\
\hspace{1em}Research & 1.944 & 2.094 & 2.275 & 2.863 & 3.536 & 3.065 & 5.519 & 6.652 & 3.832 &  &  & \\
\bottomrule
\end{tabular}}
\end{table}

\begin{table}[H]

\caption{\label{tab:summaryES}Summary of recent estimates and managment quantities for vermilion rockfish in the assessed area.}
\centering
\resizebox{\linewidth}{!}{
\begin{tabular}[t]{lrrrrrrrrrrr}
\toprule
Quantity & 2011 & 2012 & 2013 & 2014 & 2015 & 2016 & 2017 & 2018 & 2019 & 2020 & 2021\\
\midrule
Total catch (mt) & 100.124 & 87.083 & 75.837 & 77.220 & 123.528 & 112.080 & 165.560 & 180.495 & 204.445 & 139.006 & \\
$(1-SPR)/(1-SPR_{50\%})$ & 0.939 & 0.826 & 0.715 & 0.701 & 0.966 & 0.905 & 1.108 & 1.164 & 1.248 & 0.990 & \\
Annual F & 0.061 & 0.051 & 0.041 & 0.040 & 0.062 & 0.058 & 0.077 & 0.081 & 0.094 & 0.061 & \\
Age 4+ Biomass (mt) & 2741.110 & 2813.220 & 2961.290 & 3037.340 & 3087.710 & 3118.040 & 3173.250 & 3184.580 & 3135.420 & 3393.480 & 6335.880\\
\addlinespace[0.3em]
\multicolumn{12}{l}{\textbf{Spawning Output ($10^6$)}}\\
\hspace{1em}Estimate & 431.973 & 435.431 & 442.395 & 454.034 & 469.146 & 479.639 & 490.602 & 490.707 & 487.751 & 482.178 & 489.439\\
\hspace{1em}Lower Interval & 244.002 & 244.955 & 249.226 & 257.314 & 267.897 & 273.578 & 279.902 & 275.944 & 269.376 & 260.377 & 263.228\\
\hspace{1em}Upper Interval & 619.944 & 625.907 & 635.564 & 650.754 & 670.395 & 685.700 & 701.302 & 705.470 & 706.126 & 703.979 & 715.650\\
\addlinespace[0.3em]
\multicolumn{12}{l}{\textbf{Recruits (1,000s)}}\\
\hspace{1em}Estimate & 224.973 & 407.824 & 465.847 & 475.537 & 277.184 & 1536.160 & 162.592 & 387.483 & 372.609 & 373.837 & 371.777\\
\hspace{1em}Lower Interval & 115.906 & 224.497 & 242.276 & 238.986 & 124.805 & 813.510 & 64.605 & 146.879 & 138.265 & 138.332 & 139.533\\
\hspace{1em}Upper Interval & 436.670 & 740.858 & 895.729 & 946.231 & 615.609 & 2900.748 & 409.194 & 1022.226 & 1004.144 & 1010.280 & 990.579\\
\addlinespace[0.3em]
\multicolumn{12}{l}{\textbf{Fraction Unfished}}\\
\hspace{1em}Estimate & 0.377 & 0.380 & 0.386 & 0.396 & 0.410 & 0.419 & 0.428 & 0.428 & 0.426 & 0.421 & 0.427\\
\hspace{1em}Lower Interval & 0.227 & 0.229 & 0.234 & 0.241 & 0.251 & 0.257 & 0.263 & 0.260 & 0.254 & 0.246 & 0.249\\
\hspace{1em}Upper Interval & 0.527 & 0.531 & 0.539 & 0.552 & 0.568 & 0.581 & 0.594 & 0.597 & 0.598 & 0.596 & 0.606\\
\bottomrule
\end{tabular}}
\end{table}

\begin{table}[H]

\caption{\label{tab:CombinedRefPtsES}Combined reference points for the four stock 
                assessments conducted for vermilion and sunset rockfish in 2021. The fraction of the northern California stock that is estimated to be north of $40^\circ 10^\prime N${} is 4.44\% (see the appendix in the northern CA model for more details). The projected OFLs (2023-2032) assume full attainment of GMT-projected catches for 2021-22, and catches based on the PFMC harvest control rule given $p\ast$ =  0.45 and $\sigma$ = 1.}
\centering
\resizebox{\linewidth}{!}{
\begin{tabular}[t]{>{\centering\arraybackslash}p{2.5in}|>{\centering\arraybackslash}p{.5in}>{\centering\arraybackslash}p{.5in}>{\centering\arraybackslash}p{.7in}>{\centering\arraybackslash}p{.7in}|>{\centering\arraybackslash}p{.7in}>{\centering\arraybackslash}p{.5in}>{\centering\arraybackslash}p{.5in}>{\centering\arraybackslash}p{.5in}}
\toprule
Description & CA South model & CA North model & $34^\circ 27^\prime N${} to $40^\circ 10^\prime N$& South of $40^\circ 10^\prime N$& $40^\circ 10^\prime N${} to CA/OR border & OR model & WA model & North of $40^\circ 10^\prime N$\\
\midrule
Unfished spawning output ($10^6$ eggs) & 977.83 & 1145.18 & 1094.79 & 2072.63 & 50.39 & 29.20 & 2.80 & 82.39\\
Total Biomass (mt) & 6263.31 & 6457.95 & 6173.80 & 12437.11 & 284.15 & 439.41 & 36.65 & 760.21\\
Unfished  Recruitment (1000s of fish) & 809.34 & 420.19 & 401.70 & 1211.04 & 18.49 & 16.30 & 2.50 & 37.29\\
Spawning Output (2021, $10^6$ eggs) & 471.18 & 489.44 & 467.90 & 939.08 & 21.54 & 21.40 & 1.50 & 44.44\\
Fraction Unfished (2021) & 0.48 & 0.43 &  &  &  & 0.73 & 0.56 & \\
\midrule
\addlinespace[0.3em]
\multicolumn{9}{l}{\textbf{Reference Points Based on $SPR_{50\%}$}}\\
\hspace{1em}Proxy Spawning Output ($10^6$ eggs) & 439.02 & 510.93 & 488.45 & 927.47 & 22.48 & 13.00 & 1.20 & 36.68\\
\hspace{1em}Proxy MSY, mt & 148.28 & 138.99 & 132.88 & 281.16 & 6.12 & 7.90 & 0.80 & 14.82\\
\midrule
GMT Projected Catch, 2021 (mt) & 210.30 & 226.77 & 216.79 & 427.09 & 9.98 & 12.96 & 2.69 & 25.63\\
GMT Projected Catch, 2022 (mt) & 210.30 & 226.77 & 216.79 & 427.09 & 9.98 & 12.96 & 3.26 & 26.20\\
OFL 2023 (mt) & 159.36 & 154.23 & 147.44 & 306.80 & 6.79 & 13.48 & 0.71 & 20.97\\
OFL 2024 (mt) & 158.81 & 157.80 & 150.86 & 309.67 & 6.94 & 13.38 & 0.71 & 21.03\\
OFL 2025 (mt) & 158.80 & 159.49 & 152.47 & 311.27 & 7.02 & 13.16 & 0.71 & 20.89\\
OFL 2026 (mt) & 159.01 & 159.86 & 152.82 & 311.83 & 7.03 & 12.89 & 0.72 & 20.64\\
OFL 2027 (mt) & 159.28 & 159.44 & 152.43 & 311.71 & 7.02 & 12.60 & 0.73 & 20.34\\
OFL 2028 (mt) & 159.58 & 158.66 & 151.68 & 311.26 & 6.98 & 12.31 & 0.74 & 20.03\\
OFL 2029 (mt) & 159.90 & 157.79 & 150.85 & 310.75 & 6.94 & 12.03 & 0.75 & 19.72\\
OFL 2030 (mt) & 160.25 & 156.98 & 150.08 & 310.33 & 6.91 & 11.76 & 0.76 & 19.42\\
OFL 2031 (mt) & 160.64 & 156.33 & 149.45 & 310.09 & 6.88 & 11.51 & 0.77 & 19.15\\
OFL 2032 (mt) & 161.06 & 155.87 & 149.01 & 310.07 & 6.86 & 11.27 & 0.78 & 18.91\\
\bottomrule
\end{tabular}}
\end{table}

\FloatBarrier

\tagstructbegin{tag=H2}\tagmcbegin{tag=H2}

\hypertarget{unresolved-problems-and-major-uncertainties}{%
\subsection*{Unresolved Problems and Major Uncertainties}\label{unresolved-problems-and-major-uncertainties}}
\addcontentsline{toc}{subsection}{Unresolved Problems and Major Uncertainties}

\leavevmode\tagmcend\tagstructend

The stratification of assessment areas was based on consideration of population structure identified in genetic analyses, differences in historical exploitation, differences in length composition within fleets, and availability of data sources. The STAR Panel discussed the potential for alternative stratifications such as north and south of Cape Mendocino depending on the results of future analyses of population structure.

Natural mortality remains the primary axis of uncertainty across assessment areas. Additional collection of otoliths from across the range of the stock and continued ageing of available otoliths may help reduce uncertainty in the future. In the relatively data-rich southern California model, steepness was estimated and uncertainties in both natural mortality and steepness were considered when determining alternative states of nature.

\FloatBarrier

\tagstructbegin{tag=H2}\tagmcbegin{tag=H2}

\hypertarget{decision-table-and-forecasts}{%
\subsection*{Decision Table and Forecasts}\label{decision-table-and-forecasts}}
\addcontentsline{toc}{subsection}{Decision Table and Forecasts}

\leavevmode\tagmcend\tagstructend

The forecasts of stock abundance and yield were developed using the post-STAR base model, with the forecast projections presented in Table \ref{tab:projectionES}. The total catches in 2021 and 2022 are set to the projected catch from the California Department of Fish and Wildlife (CDFW) by sector and model region, i.e., allocated north and south of $34^\circ 27^\prime N$ in California.

Uncertainty in the forecasts is based upon the three states of nature agreed upon at the STAR panel, reflecting three different natural mortality rates. The steepness parameter of the Beverton-Holt stock-recruit curve was fixed in the base model and in all of the forecasts. The northern California model is not data rich and while there is uncertainty in steepness, it was not well estimated in the base model when natural mortality was also estimated. The alternative states of nature maintain the female to male natural mortality rate ratio from the base model. To capture the 75\% interval around the negative log-likelihood, alternate states were identified within 0.66 negative log-likelihood points from the base model where female {\tagstructbegin{tag=Formula}\tagmcbegin{tag=Formula}\(M\)\leavevmode\tagmcend\tagstructend} = 0.0856 and male {\tagstructbegin{tag=Formula}\tagmcbegin{tag=Formula}\(M\)\leavevmode\tagmcend\tagstructend} = 0.0805. The high state of nature fixes female {\tagstructbegin{tag=Formula}\tagmcbegin{tag=Formula}\(M\)\leavevmode\tagmcend\tagstructend} = 0.0956 and male {\tagstructbegin{tag=Formula}\tagmcbegin{tag=Formula}\(M\)\leavevmode\tagmcend\tagstructend} = 0.08989. For the low state of nature, female {\tagstructbegin{tag=Formula}\tagmcbegin{tag=Formula}\(M\)\leavevmode\tagmcend\tagstructend} = 0.0769 and male {\tagstructbegin{tag=Formula}\tagmcbegin{tag=Formula}\(M\)\leavevmode\tagmcend\tagstructend} = 0.07231.

For reference, the base model predicted {\tagstructbegin{tag=Formula}\tagmcbegin{tag=Formula}\(\sigma\)\leavevmode\tagmcend\tagstructend} = 0.246. The buffers between the OFL and ABC were calculated assuming a category 2 stock, with {\tagstructbegin{tag=Formula}\tagmcbegin{tag=Formula}\(\sigma\)\leavevmode\tagmcend\tagstructend} = 1.0 and a {\tagstructbegin{tag=Formula}\tagmcbegin{tag=Formula}\(p^*\)\leavevmode\tagmcend\tagstructend} = 0.45. Alternative catch streams (rows in the table) include {\tagstructbegin{tag=Formula}\tagmcbegin{tag=Formula}\(\sigma\)\leavevmode\tagmcend\tagstructend} = 1.0 with a {\tagstructbegin{tag=Formula}\tagmcbegin{tag=Formula}\(p^*\)\leavevmode\tagmcend\tagstructend} = 0.4, and removals of long-term equilibrium catch with and without a buffer assuming {\tagstructbegin{tag=Formula}\tagmcbegin{tag=Formula}\(\sigma\)\leavevmode\tagmcend\tagstructend} = 1.0 with a {\tagstructbegin{tag=Formula}\tagmcbegin{tag=Formula}\(p^*\)\leavevmode\tagmcend\tagstructend} = 0.45. The buffer multiplier with {\tagstructbegin{tag=Formula}\tagmcbegin{tag=Formula}\(p^*\)\leavevmode\tagmcend\tagstructend} = 0.45 ranges from 0.874 in 2023 ramping to 0.803 in 2032.

Current forecasts based on the alternative states of nature and requested catch streams project that the stock will remain above the target threshold of 40\% in 2032 (Table \ref{tab:DecisionES}). In all of the scenarios of the low state of nature, the stock remains below the target threshold of 40\% until 2026 or 2027. The base model with the base catches results in an increasing stock over the period from 2023-2032. In all scenarios the catch significantly decreases from 2022 to 2023; projected catch in 2022 is 227 mt, and 2023 catches from the base model range from 118-139 mt. The base model includes a portion of the stock within the northern management unit (north of $40^\circ 10^\prime N$). An analysis based on the private/rental mode index through 2019 suggests that 4.44\% of the catches from this model should be apportioned to the northern management unit for vermilion rockfish.

The STAT cautions that the GMT projections for catches in 2021-2022 (22 mt per year) exceed the maximum sustainable yield according to both proxies ({\tagstructbegin{tag=Formula}\tagmcbegin{tag=Formula}\(B_{40\%}\)\leavevmode\tagmcend\tagstructend} and {\tagstructbegin{tag=Formula}\tagmcbegin{tag=Formula}\(SPR_{50\%}\)\leavevmode\tagmcend\tagstructend}) as well as the MSY value based on the estimated value of steepness (Table \ref{tab:referenceES}). The northern California stock is just above target biomass in 2021 (43\% of unfished spawning ouptut), so these catch levels are unlikely to result in significant stock declines over a short period of time. However, similar catch levels would exceed the overfishing limits (OFL) if carried forward for 2023 and beyond (Table \ref{tab:CombinedRefPtsES}), and would be unsustainable in the long term. Given recent and projected near-term exploitation levels, and especially if vermilion and sunset rockfish continue to be managed as part of the minor shelf rockfish complex, the STAT recommends regular monitoring of total mortality for these two species to avoid excessive stock depletion and potential loss of yield.

\begin{table}[H]

\caption{\label{tab:projectionES}Projections of potential OFLs (mt), ABCs (mt), estimated age 4+ biomass (mt), estimated spawning output ($10^6$ eggs) and fraction unfished, assuming default harvest control rule catches with p* = 0.45 and $\sigma = 1.0$.}
\centering
\begin{tabular}[t]{c>{\centering\arraybackslash}p{.8in}>{\centering\arraybackslash}p{.8in}>{\centering\arraybackslash}p{.8in}>{\centering\arraybackslash}p{.8in}>{\centering\arraybackslash}p{.8in}}
\toprule
Year & Predicted OFL & ABC Catch & Age 4+ Biomass & Spawning Output & Fraction Unfished\\
\midrule
\cellcolor{gray!6}{2021} & \cellcolor{gray!6}{} & \cellcolor{gray!6}{} & \cellcolor{gray!6}{3459.01} & \cellcolor{gray!6}{489.439} & \cellcolor{gray!6}{0.4273}\\
2022 &  &  & 3459.01 & 490.953 & 0.4287\\
\cellcolor{gray!6}{2023} & \cellcolor{gray!6}{154.226} & \cellcolor{gray!6}{134.794} & \cellcolor{gray!6}{3436.65} & \cellcolor{gray!6}{497.064} & \cellcolor{gray!6}{0.4340}\\
2024 & 157.800 & 136.497 & 3489.40 & 516.147 & 0.4507\\
\cellcolor{gray!6}{2025} & \cellcolor{gray!6}{159.485} & \cellcolor{gray!6}{136.679} & \cellcolor{gray!6}{3524.92} & \cellcolor{gray!6}{533.183} & \cellcolor{gray!6}{0.4655}\\
2026 & 159.856 & 135.718 & 3547.19 & 547.117 & 0.4777\\
\cellcolor{gray!6}{2027} & \cellcolor{gray!6}{159.445} & \cellcolor{gray!6}{134.093} & \cellcolor{gray!6}{3560.05} & \cellcolor{gray!6}{557.979} & \cellcolor{gray!6}{0.4872}\\
2028 & 158.663 & 132.167 & 3567.12 & 566.209 & 0.4944\\
\cellcolor{gray!6}{2029} & \cellcolor{gray!6}{157.790} & \cellcolor{gray!6}{130.334} & \cellcolor{gray!6}{3570.85} & \cellcolor{gray!6}{572.335} & \cellcolor{gray!6}{0.4997}\\
2030 & 156.983 & 128.412 & 3572.83 & 576.840 & 0.5037\\
\cellcolor{gray!6}{2031} & \cellcolor{gray!6}{156.332} & \cellcolor{gray!6}{126.629} & \cellcolor{gray!6}{3574.39} & \cellcolor{gray!6}{580.187} & \cellcolor{gray!6}{0.5066}\\
2032 & 155.866 & 125.160 & 3576.26 & 582.729 & 0.5088\\
\bottomrule
\end{tabular}
\end{table}

\FloatBarrier

\newpage

\begin{table}

\caption{\label{tab:DecisionES}Decision table summarizing 12-year projections 
                (2021 to 2032) for vermilion rockfish based on three alternative 
                states of nature spanning quantiles of spawning output in 
                2021.  Columns range over low, medium, and high state of 
                nature, and rows range over different assumptions of total 
                catch levels corresponding to the forecast catches from 
                each state of nature.  Catches in 2021 and 2022 are fixed 
                at catches provided by the CDFW.}
\centering
\resizebox{\linewidth}{!}{
\begin{tabular}[t]{>{\centering\arraybackslash}p{1in}|>{}c|>{}c|>{}c|>{\centering\arraybackslash}p{.8in}>{\centering\arraybackslash}p{.8in}|>{\centering\arraybackslash}p{.8in}>{\centering\arraybackslash}p{.8in}|>{\centering\arraybackslash}p{.8in}>{\centering\arraybackslash}p{.8in}}
\toprule
\multicolumn{4}{c}{ } & \multicolumn{2}{c}{Low Productivity} & \multicolumn{2}{c}{Base Model} & \multicolumn{2}{c}{High Productivity} \\
\cmidrule(l{3pt}r{3pt}){5-6} \cmidrule(l{3pt}r{3pt}){7-8} \cmidrule(l{3pt}r{3pt}){9-10}
\multicolumn{4}{c}{ } & \multicolumn{2}{c}{Female M = 0.0769} & \multicolumn{2}{c}{Female M = 0.0856} & \multicolumn{2}{c}{Female M = 0.0956} \\
\multicolumn{4}{c}{ } & \multicolumn{2}{c}{Male M = 0.0723} & \multicolumn{2}{c}{Male M = 0.0805} & \multicolumn{2}{c}{Male M = 0.0899} \\
\multicolumn{4}{c}{ } & \multicolumn{2}{c}{NLL = 1031.36} & \multicolumn{2}{c}{NLL = 1030.7} & \multicolumn{2}{c}{NLL = 1031.36} \\
\cmidrule(l{3pt}r{3pt}){5-6} \cmidrule(l{3pt}r{3pt}){7-8} \cmidrule(l{3pt}r{3pt}){9-10}
  & Year & Buffer & Catch (mt) & Spawning Output & Fraction Unfished & Spawning Output & Fraction Unfished & Spawning Output & Fraction Unfished\\
\midrule
 & 2021 & 1.000 & 227 & 437 & 0.362 & 489 & 0.427 & 554 & 0.506\\

 & 2022 & 1.000 & 227 & 435 & 0.361 & 491 & 0.429 & 558 & 0.510\\

 & 2023 & 0.874 & 135 & 438 & 0.363 & 497 & 0.434 & 568 & 0.519\\

 & 2024 & 0.865 & 136 & 453 & 0.376 & 516 & 0.451 & 591 & 0.540\\

 & 2025 & 0.857 & 137 & 467 & 0.387 & 533 & 0.466 & 612 & 0.559\\

 & 2026 & 0.849 & 136 & 477 & 0.396 & 547 & 0.478 & 629 & 0.575\\

 & 2027 & 0.841 & 134 & 485 & 0.402 & 558 & 0.487 & 642 & 0.587\\

 & 2028 & 0.833 & 132 & 491 & 0.407 & 566 & 0.494 & 652 & 0.595\\

 & 2029 & 0.826 & 130 & 496 & 0.411 & 572 & 0.500 & 658 & 0.602\\

 & 2030 & 0.818 & 128 & 499 & 0.414 & 577 & 0.504 & 663 & 0.606\\

 & 2031 & 0.810 & 127 & 502 & 0.416 & 580 & 0.507 & 666 & 0.608\\

\multirow{-12}{1in}{\centering\arraybackslash $p^\ast = 0.45,  \sigma = 1$} & 2032 & 0.803 & 125 & 505 & 0.418 & 583 & 0.509 & 667 & 0.610\\
\cmidrule{1-10}
 & 2021 & 1.000 & 227 & 437 & 0.362 & 489 & 0.427 & 554 & 0.506\\

 & 2022 & 1.000 & 227 & 435 & 0.361 & 491 & 0.429 & 558 & 0.510\\

 & 2023 & 0.762 & 118 & 438 & 0.363 & 497 & 0.434 & 568 & 0.519\\

 & 2024 & 0.747 & 118 & 456 & 0.378 & 519 & 0.453 & 593 & 0.542\\

 & 2025 & 0.733 & 118 & 472 & 0.392 & 539 & 0.470 & 616 & 0.563\\

 & 2026 & 0.719 & 117 & 487 & 0.404 & 556 & 0.485 & 636 & 0.581\\

 & 2027 & 0.706 & 115 & 499 & 0.414 & 570 & 0.498 & 652 & 0.595\\

 & 2028 & 0.693 & 113 & 509 & 0.422 & 581 & 0.508 & 664 & 0.607\\

 & 2029 & 0.680 & 111 & 518 & 0.429 & 591 & 0.516 & 674 & 0.615\\

 & 2030 & 0.667 & 108 & 525 & 0.436 & 599 & 0.523 & 681 & 0.622\\

 & 2031 & 0.654 & 106 & 533 & 0.442 & 606 & 0.529 & 686 & 0.627\\

\multirow{-12}{1in}{\centering\arraybackslash $p^\ast = 0.40,  \sigma = 1$} & 2032 & 0.642 & 105 & 539 & 0.447 & 612 & 0.534 & 691 & 0.631\\
\cmidrule{1-10}
 & 2021 & 1.000 & 227 & 437 & 0.362 & 489 & 0.427 & 554 & 0.506\\

 & 2022 & 1.000 & 227 & 435 & 0.361 & 491 & 0.429 & 558 & 0.510\\

 & 2023 & 1.000 & 139 & 438 & 0.363 & 497 & 0.434 & 568 & 0.519\\

 & 2024 & 1.000 & 139 & 453 & 0.376 & 516 & 0.451 & 590 & 0.539\\

 & 2025 & 1.000 & 139 & 467 & 0.387 & 533 & 0.465 & 610 & 0.558\\

 & 2026 & 1.000 & 139 & 477 & 0.396 & 546 & 0.477 & 627 & 0.573\\

 & 2027 & 1.000 & 139 & 485 & 0.402 & 557 & 0.486 & 639 & 0.584\\

 & 2028 & 1.000 & 139 & 491 & 0.407 & 564 & 0.493 & 647 & 0.591\\

 & 2029 & 1.000 & 139 & 495 & 0.410 & 569 & 0.497 & 652 & 0.596\\

 & 2030 & 1.000 & 139 & 497 & 0.412 & 572 & 0.499 & 654 & 0.598\\

 & 2031 & 1.000 & 139 & 98 & 0.413 & 573 & 0.500 & 655 & 0.598\\

\multirow{-12}{1in}{\centering\arraybackslash Long-term Equil. Yield (MSY proxy, $SPR_{50\%}$), no buffer} & 2032 & 1.000 & 139 & 499 & 0.414 & 573 & 0.501 & 654 & 0.597\\
\cmidrule{1-10}
 & 2021 & 1.000 & 227 & 437 & 0.362 & 489 & 0.427 & 554 & 0.506\\

 & 2022 & 1.000 & 227 & 435 & 0.361 & 491 & 0.429 & 558 & 0.510\\

 & 2023 & 0.874 & 122 & 438 & 0.363 & 497 & 0.434 & 568 & 0.519\\

 & 2024 & 0.865 & 120 & 456 & 0.378 & 518 & 0.453 & 593 & 0.542\\

 & 2025 & 0.857 & 119 & 472 & 0.392 & 538 & 0.470 & 616 & 0.563\\

 & 2026 & 0.849 & 118 & 486 & 0.403 & 555 & 0.485 & 635 & 0.580\\

 & 2027 & 0.841 & 117 & 498 & 0.413 & 569 & 0.497 & 651 & 0.595\\

 & 2028 & 0.833 & 116 & 508 & 0.421 & 580 & 0.507 & 663 & 0.606\\

 & 2029 & 0.826 & 116 & 516 & 0.428 & 589 & 0.515 & 672 & 0.614\\

 & 2030 & 0.818 & 115 & 522 & 0.433 & 596 & 0.521 & 678 & 0.620\\

 & 2031 & 0.810 & 114 & 528 & 0.438 & 602 & 0.526 & 682 & 0.624\\

\multirow{-12}{1in}{\centering\arraybackslash Long-term Equil. Yield (MSY proxy, $SPR_{50\%}$), with buffer} & 2032 & 0.803 & 113 & 533 & 0.442 & 606 & 0.529 & 685 & 0.626\\
\bottomrule
\end{tabular}}
\end{table}

\FloatBarrier

\tagstructbegin{tag=H2}\tagmcbegin{tag=H2}

\hypertarget{research-and-data-needs}{%
\subsection*{Research and Data Needs}\label{research-and-data-needs}}
\addcontentsline{toc}{subsection}{Research and Data Needs}

\leavevmode\tagmcend\tagstructend

The following are high priority research and data needs for this assessment. Additional details for each topic can be found in the full assessment.

We recommend the following research be conducted before the next assessment:

\tagstructbegin{tag=L}

\begin{itemize}
\item
  Develop a coastwide hook-and-line survey to provide indices of abundance and associated biological sampling providing representative data in untrawlable habitats.
\item
  Examine the available tools more fully in cases when a survey's footprint is abruptly changed as a result of management action. These tools may include (but are not limited to), treating the ``new'' and ``old'' surveys as completely separate (aka breaking the survey), using selectivity blocks, or spatial/temporal modeling approaches. This avenue is important for many fishery-independent and -dependent indices, as they are subjected to numerous spatial management changes which in turn can affect the veracity of the data collected. Additional efforts are needed to investigate how fishery selectivity changes with management changes and how best to address the effects of management changes on length composition and indices.
\item
  Expansion of the California Collaborative Fisheries Research Project into deeper depths outside and inside MPAs and to other closed areas to encompass the full depth distribution of vermilion and sunset rockfish or other shallow shelf rockfish species would provide valuable data for future assessments.
\item
  Conduct additional investigations to resolve uncertainties in historical catch reconstructions would improve estimates of the scale of assessments and provide more representative removal estimates.
\item
  Explore appropriate methods of including catches as numbers of fish vs.~biomass.
\item
  Connectedness of this stock with southern California (south of Point Conception) is an unresolved uncertainty as outlined in the STAT report and elsewhere in this report. Further studies on larval/juvenile/adult movement via tagging or other methods are warranted. Additionally population substructure investigations, particularly north and south of Cape Mendocino are also recommended.
\item
  Development of a more comprehensive fishery-independent index is a priority for this region. This could involve expansion of the CCFRP across depths and latitudes or expansion of the NWFSC hook-and-line survey northward.
\end{itemize}

\tagstructend

\pagebreak
\setlength{\parskip}{5mm plus1mm minus1mm}
\pagenumbering{arabic}
\setcounter{page}{1}
\renewcommand{\thefigure}{\arabic{figure}}
\renewcommand{\thetable}{\arabic{table}}
\setcounter{table}{0}
\setcounter{figure}{0}

\tagstructbegin{tag=H1}\tagmcbegin{tag=H1}

\hypertarget{introduction}{%
\section{Introduction}\label{introduction}}

\leavevmode\tagmcend\tagstructend

Note to readers: Text in this section is the same in both California vermilion rockfish assessment documents.

\tagstructbegin{tag=H2}\tagmcbegin{tag=H2}

\hypertarget{basic-information-and-life-history}{%
\subsection{Basic Information and Life History}\label{basic-information-and-life-history}}

\leavevmode\tagmcend\tagstructend

\emph{Note: Prior to the identification of sunset rockfish as a separate species {\tagstructbegin{tag=Reference}\tagmcbegin{tag=Reference}(Hyde, J.R.; Kimbrell, C. A.; Budrick, J. E.; Lynn, E. A.; Vetter 2008)\leavevmode\tagmcend\tagstructend}, historical studies of ``vermilion'' rockfish, particularly those conducted south of Point Conception ($34^\circ 27^\prime N$), California, could have included a mixture of both species. Also, many current studies and data sets (e.g., landing statistics) do not distinguish between the species. In this document, we refer simply to ``vermilion rockfish'' when no species-specific information is available.}

Vermilion rockfish (\emph{Sebastes miniatus}) range from Prince William Sound, Alaska, to central Baja California at depths of 6 m to 436 m {\tagstructbegin{tag=Reference}\tagmcbegin{tag=Reference}(Love et al. 2002)\leavevmode\tagmcend\tagstructend}. However, they are most commonly found from central Oregon to Punta Baja, Mexico {\tagstructbegin{tag=Reference}\tagmcbegin{tag=Reference}(Hyde and Vetter 2009)\leavevmode\tagmcend\tagstructend} at depths of 50 m to 150 m {\tagstructbegin{tag=Reference}\tagmcbegin{tag=Reference}(Hyde and Vetter 2009)\leavevmode\tagmcend\tagstructend}. Hyde and Vetter {\tagstructbegin{tag=Reference}\tagmcbegin{tag=Reference}(2009)\leavevmode\tagmcend\tagstructend} describe vermilion rockfish as residents of shallower depths (\textless100 m) than their sibling species, sunset rockfish (\emph{Sebastes crocotulus}). Adult fish tend to cluster on high relief rocky outcrops {\tagstructbegin{tag=Reference}\tagmcbegin{tag=Reference}(Love et al. 2002)\leavevmode\tagmcend\tagstructend} and kelp forests {\tagstructbegin{tag=Reference}\tagmcbegin{tag=Reference}(Hyde and Vetter 2009)\leavevmode\tagmcend\tagstructend}. North of Point Conception, California, some adults reside in shallower water, living in caves and cracks {\tagstructbegin{tag=Reference}\tagmcbegin{tag=Reference}(Love et al. 2002)\leavevmode\tagmcend\tagstructend}. Vermilion rockfish have shown high site fidelity {\tagstructbegin{tag=Reference}\tagmcbegin{tag=Reference}(Hannah and Rankin 2011 (only tagged one vermilion rockfish), Lea et al. 1999)\leavevmode\tagmcend\tagstructend}, and low to average larval dispersal distance {\tagstructbegin{tag=Reference}\tagmcbegin{tag=Reference}(Hyde and Vetter 2009)\leavevmode\tagmcend\tagstructend}. Lowe et al. {\tagstructbegin{tag=Reference}\tagmcbegin{tag=Reference}(2009)\leavevmode\tagmcend\tagstructend} suggested that vermilion rockfish have a lower site fidelity than previously believed, but acknowledged that their observations of movements to different depths may have been due to differences in depth distribution between the species. Vermilion rockfish have been aged to over 80 years, but few fish have been aged above 60 years, with females growing larger than their male counterparts. Fifty percent of females are mature at 5 years and about 37 cm, with males likely maturing at shorter lengths than females {\tagstructbegin{tag=Reference}\tagmcbegin{tag=Reference}(Love et al. 2002)\leavevmode\tagmcend\tagstructend}.

Vermilion rockfish are viviparous, and females produce an estimated 63,000 to 2,600,000 eggs per brood, with larger fish releasing a substantially larger number of larvae. In southern California, vermilion rockfish larvae are released between July and March. In central and northern California, this release occurs in September, December, and April-June {\tagstructbegin{tag=Reference}\tagmcbegin{tag=Reference}(Love et al. 2002)\leavevmode\tagmcend\tagstructend}. Hyde and Vetter {\tagstructbegin{tag=Reference}\tagmcbegin{tag=Reference}(2009)\leavevmode\tagmcend\tagstructend} suggest that low larval dispersal may be due to weak poleward flow of nearshore waters corresponding with peak vermilion rockfish larval release.

Young-of-the-year vermilion rockfish settle out of the water column during two primary recruitment periods per year, first from February to April and a second from August to October, and settlement has been observed in May off southern California {\tagstructbegin{tag=Reference}\tagmcbegin{tag=Reference}(Love et al. 2002)\leavevmode\tagmcend\tagstructend}. Young-of-the-year vermilion and sunset rockfish are both mottled brown with areas of black, and older juveniles turn a mottled orange or red color {\tagstructbegin{tag=Reference}\tagmcbegin{tag=Reference}(Love et al. 2012)\leavevmode\tagmcend\tagstructend}. Larvae measure about 4.3 mm and juvenile fish are found in depths of 6-36 m, living near sand and structure. After two months, juveniles travel deeper and live on low relief rocky outcrops and other structures {\tagstructbegin{tag=Reference}\tagmcbegin{tag=Reference}(Love et al. 2002)\leavevmode\tagmcend\tagstructend}.

Adult vermilion rockfish predominantly eat smaller fish, though sometimes they pursue euphausiids and other various macroplankton {\tagstructbegin{tag=Reference}\tagmcbegin{tag=Reference}(Phillips 1964)\leavevmode\tagmcend\tagstructend}. Love et al. {\tagstructbegin{tag=Reference}\tagmcbegin{tag=Reference}(2002)\leavevmode\tagmcend\tagstructend} noted their diet includes octopuses, salps, shrimps, and pelagic red crabs.

\emph{Population Structure and Multi-species Assessment Considerations}

This assessment represents the aggregate population dynamics of the cryptic species pair vermilion rockfish and sunset rockfish. Hyde {\tagstructbegin{tag=Reference}\tagmcbegin{tag=Reference}(2007)\leavevmode\tagmcend\tagstructend} examined seven mitochondrial and two nuclear genes, which upon analysis suggested three species within the subgenus \emph{Rosicola}. Hyde et al. {\tagstructbegin{tag=Reference}\tagmcbegin{tag=Reference}(2008)\leavevmode\tagmcend\tagstructend} described sunset rockfish as a distinct species noting depth separation of the adult populations of the two species using nine microsatellite loci. Adult sunset rockfish are mainly distributed at depths greater than 50 fm (100 m) and are predominantly located south of Point Conception ($34^\circ 27^\prime N$). Hyde and Vetter {\tagstructbegin{tag=Reference}\tagmcbegin{tag=Reference}(2009)\leavevmode\tagmcend\tagstructend} and Budrick {\tagstructbegin{tag=Reference}\tagmcbegin{tag=Reference}(2016)\leavevmode\tagmcend\tagstructend} identified species using mtDNA assays and microsatellite loci, respectively.

Vermilion and sunset rockfish are morphologically very similar, with color being the most commonly cited differentiating feature. Hyde and Vetter {\tagstructbegin{tag=Reference}\tagmcbegin{tag=Reference}(2009)\leavevmode\tagmcend\tagstructend} noted differences in three of six morphological parameters examined, but none of them can readily be used for field identification.

In all historical and current recreational and commercial catches, sunset and vermilion rockfish are both recorded as vermilion rockfish. Future studies, such as the one described below will provide data needed to compare biological parameters between the two species as well as habitats and distributions.

\emph{Ongoing Population Structure Research (Provided by John Harms, NWFSC)}

A group of researchers from the NWFSC and SWFSC is collaborating on a project to genotype tissue specimens collected from the vermilion and sunset rockfish cryptic pair captured during the West Coast Groundfish Bottom Trawl (WCGBT) Survey and the Southern California Shelf Rockfish Hook-and-Line Survey for the years 2004 - 2019. Funding for this project was obtained through the Saltonstall-Kennedy program for fiscal year 2020 through a proposal led by representatives from Pacific States Marine Fisheries Commission and the commercial passenger fishing vessel industry in southern California.

After combining with specimens obtained through other collection efforts along the West Coast, approximately 25,000 tissue specimens will be analyzed. Some earlier efforts to separate this cryptic pair to species used mitochondrial DNA (mtDNA) markers. However, due to a one-way mitochondrial introgression from the vermilion rockfish genome into the sunset rockfish genome (or incomplete lineage sorting), a portion of the sunset rockfish population contains mitochondrial DNA sequences consistent with vermilion rockfish resulting in incorrect species assignments for these introgressed individuals during the prior research project.

Once the collected specimens have been genotyped, any species-specific differences in spatial and depth distribution, size composition, weight-length relationships, and other biological characteristics will be identified. Using previously collected otoliths and ovaries, the demographics of the two species including age and growth and reproductive biology parameters such as length and age at 50\% maturity and the prevalence of skip spawning will be explored and compared. These new genotyping results will be combined with data from the prior mtDNA work to evaluate whether introgressed (hybrid) sunset rockfish represent a biologically intermediate subform of the species complex. The effort also proposes to develop and test the efficacy of models to predict the relative proportion of the two species based upon explanatory variables including latitude, depth, species of co-occurrence, oceanographic parameters, habitat descriptors and/or other information. The anticipated completion of the genotyping of all specimens is approximately December 2021 with provision of final results by the end of FY 2022.

This research is aimed at providing information to support the successful stock assessment of this commercially and recreationally valuable cryptic species pair and is responsive to any data gaps identified by the assessment community. If successful, this research, conducted in close communication with stock assessors, may also assist the PFMC in establishing best practices for the assessment and management of cryptic species complexes. Though this project will only focus on nominal vermilion rockfish specimens collected through the 2019 survey field season, it may be advisable that tissue specimens collected aboard fishery-independent surveys as well as through fishery-dependent programs continue to be genotyped on an ongoing basis to support continued and timely monitoring of this economically and ecologically important species complex.

\tagstructbegin{tag=H2}\tagmcbegin{tag=H2}

\hypertarget{map}{%
\subsection{Map}\label{map}}

\leavevmode\tagmcend\tagstructend

A map showing the scope of the two California vermilion rockfish assessments and depicting a boundary at Point Conception ($34^\circ 27^\prime N$) that separates the two assessments is provided as Figure \ref{fig:assess-area}. The northern California model is bounded in the north by the California/Oregon border ($42^\circ 00^\prime N$) and the southern California model is bounded by the U.S./ Mexico border in the south (Figure \ref{fig:assess-area}). Cape Mendocino ($40^\circ 10^\prime N$) is also noted as it is a management boundary for the Pacific Fishery Management Council (PFMC) ``minor shelf rockfish'' stock complex.

\tagstructbegin{tag=H2}\tagmcbegin{tag=H2}

\hypertarget{ecosystem-considerations-1}{%
\subsection{Ecosystem Considerations}\label{ecosystem-considerations-1}}

\leavevmode\tagmcend\tagstructend

This stock assessment does not explicitly incorporate trophic interactions, habitat factors (other than as they inform relative abundance indices) or environmental factors into the assessment model, but a brief description of likely or potential ecosystem considerations are provided below.

Vermilion/sunset rockfish are described as feeding on a wide range of both pelagic and benthic prey items, including forage fish species such as anchovies and mesopelagic fishes, squid, krill and octopus, as well as sporadically abundant pelagic organisms such as pyrosomes, salps and pelagic red crabs {\tagstructbegin{tag=Reference}\tagmcbegin{tag=Reference}(Phillips 1964, Love et al. 2002)\leavevmode\tagmcend\tagstructend}. Interestingly, other rockfishes (either juvenile or adult stages) have not been documented as prey for vermilion rockfish, as they have been for other large \emph{Sebastes} species such as cowcod, bocaccio, and yelloweye rockfish. For the latter species, the idea of ``cultivation effects,'' in which adults crop down forage species that are potential competitors/predators of their own juveniles {\tagstructbegin{tag=Reference}\tagmcbegin{tag=Reference}(Walters and Kitchell 2001)\leavevmode\tagmcend\tagstructend}, has been suggested by {\tagstructbegin{tag=Reference}\tagmcbegin{tag=Reference}Baskett et al. (2006)\leavevmode\tagmcend\tagstructend}. For example, Baskett et al. {\tagstructbegin{tag=Reference}\tagmcbegin{tag=Reference}(2006)\leavevmode\tagmcend\tagstructend} found that in such scenarios there could be alternative stable states in which either the overfished species or the smaller prey species could dominate. While the sparse diet data for vermilion/sunset rockfish do not suggest such a process for this species complex, food habits data for vermilion/sunset are not robust, and the larger community processes on these rocky reef communities may also influence productivity and community composition regardless of the direct predation interactions. Pelagic and benthic juvenile vermilion and sunset rockfish are likely preyed upon by the same wide range of predators that prey on juveniles and adults of other rockfish species, including seabirds, piscivorous fishes, and marine mammals.

As with most other rockfish and groundfish in the California Current, recruitment, or cohort (year-class) strength appears to be highly variable for the vermilion/sunset rockfish complex, with only a modest apparent relationship to estimated levels of spawning output. Oceanographic and ecosystem factors are widely recognized to be key drivers of recruitment variability for most species of groundfish, as well as most elements of California Current food webs. Empirical estimates of recruitment from pelagic juvenile rockfish surveys have been used to inform incoming year class strength for some of these stocks, however vermilion/sunset rockfish are rarely encountered in these surveys. Specifically, only 47 of nearly 300,000 total juvenile \emph{Sebastes} encountered in juvenile surveys since 2001 were identified as vermilion/sunset rockfish {\tagstructbegin{tag=Reference}\tagmcbegin{tag=Reference}(Field et al. 2021)\leavevmode\tagmcend\tagstructend}. Despite this, the results here suggest that at least a reasonable fraction of recruitment variability for sunset and vermilion rockfish is shared with other rockfish and groundfish stocks throughout the California Current, many of which also had strong year classes in 1984, 1999 and 2015-2016. Previous studies have demonstrated that large-scale oceanographic drivers, such as the relative transport of subarctic waters (typically indicated by relative sea level) tend to relate to a substantial fraction of overall groundfish recruitment trends and ecosystem productivity {\tagstructbegin{tag=Reference}\tagmcbegin{tag=Reference}Schroeder et al. (2019)\leavevmode\tagmcend\tagstructend}. Although it is feasible that ecosystem factors, the results of pre-recruit surveys for co-occurring species, or the results of other groundfish assessments might ultimately be used to forecast recruitment for more data-limited stocks such as vermilion and sunset rockfish, as suggested by {\tagstructbegin{tag=Reference}\tagmcbegin{tag=Reference}(Thorson and Ward 2014)\leavevmode\tagmcend\tagstructend}, such approaches would require more development and evaluation. Consequently, environmental factors are not explicitly considered in this assessment.

\FloatBarrier

\tagstructbegin{tag=H2}\tagmcbegin{tag=H2}

\hypertarget{historical-and-current-fishery-information}{%
\subsection{Historical and Current Fishery Information}\label{historical-and-current-fishery-information}}

\leavevmode\tagmcend\tagstructend

\emph{Commercial Fishery}\\
The commercial groundfish fishery off California developed in the late 19th century and consisted mainly of hook and line gear types (Figure \ref{fig:catch}). At the end of the 19th century, total rockfish landings were estimated to be between 2,000 to 3,500 tons statewide, with slightly over half of the catch during this period coming from waters south of Point Conception, and most of the remaining catch from central California ports (particularly San Francisco and Monterey). Catches declined through the 1930s as a result of the rapid expansion of the California sardine fishery, which tended to be more profitable {\tagstructbegin{tag=Reference}\tagmcbegin{tag=Reference}(Love et al. 2002)\leavevmode\tagmcend\tagstructend}. The rockfish trawl fishery rapidly expanded into California in the early 1940s, after the introduction of the `balloon trawl,' and when the United States became involved in World War II and wartime shortage of red meat created an increased demand for other sources of protein {\tagstructbegin{tag=Reference}\tagmcbegin{tag=Reference}(Harry and Morgan 1961, Alverson et al. 1964, Lenarz 1987)\leavevmode\tagmcend\tagstructend}. Trawl landings have been restricted in most of southern California for decades {\tagstructbegin{tag=Reference}\tagmcbegin{tag=Reference}(Frey 1971)\leavevmode\tagmcend\tagstructend}, and trawl gear north of Point Conception has not recently been a major component of the landings for vermilion rockfish, with the highest reported landings in the 1970s. The commercial setnet fishery has never been a large component of the vermilion rockfish landings and has essentially been non-existent for vermilion rockfish since 2002 when the state of California prohibited setnet gear in 60 fm or less. The largest net landings for vermilion rockfish were in the 1980s.

Vermilion rockfish have been landed in the commercial live-fish fishery that developed off the coast of California in the 1990s, but have not been a major target of that fishery due to their susceptibility to barotrauma. The fraction of the total catch from the live fish fleet is small, concentrated in northern California, and included in the commercial hook-and-line fleet in the northern California assessment models. The STAT also learned that vermilion rockfish landed dead (due to barotrauma) from a commercial trip landing live fish, remain valuable and may be sold dead. Separation of catch and size compositions for the live and dead catch is therefore less informative and was not pursued further.

Miller et al. {\tagstructbegin{tag=Reference}\tagmcbegin{tag=Reference}(2014)\leavevmode\tagmcend\tagstructend} described the spatial and temporal development of the California commercial groundfish fishery based on historical CDFW fish ticket and block summary data. They analyzed a spatially-explicit database of landings in California dating back to 1933, finding that groundfish fishing effort has shifted from shallow, coastal areas to deeper depths, greater distances from port, and in areas of more inclement weather over time. That general result was also found with limited data from recreational fisheries. Sampling of commercial species compositions in Southern California began in 1983, a time when the groundfish fleet was already fishing in deeper depths. Both historical reconstructions used these data to represent species compositions of total rockfish catch during earlier periods of the fishery. As a result, the reconstructions may overestimate the percentage of deep-water species in earlier fisheries that operated closer to port and in shallower depths.

\emph{Recreational Fishery}\\
Vermilion rockfish are a targeted species in California's recreational fishery and have always ranked high in terms of catch among rockfish species, both in the party/charter boat and private/rental sectors. The Commercial Passenger Fishing Vessel (CPFV; aka `party' and `charter' boat) fleet began circa 1919 in California, although recreational fishing effort for fishes other than Tunas, other gamefish, and salmon was minimal until about 1930. The CPFV fleet numbered about 200 vessels in 1939 ({\tagstructbegin{tag=Reference}\tagmcbegin{tag=Reference}(Croker 1940)\leavevmode\tagmcend\tagstructend}, cited in Young {\tagstructbegin{tag=Reference}\tagmcbegin{tag=Reference}(1969)\leavevmode\tagmcend\tagstructend}). After a hiatus in most operations during WWII, the fleet increased to about 590 vessels by 1953, then declined to approximately 256 vessels around 1963.

Onboard surveys of CPFV vessels in southern California ranked vermilion rockfish as the fifth and third most common rockfish species in the mid-1970s and mid-1980s, respectively {\tagstructbegin{tag=Reference}\tagmcbegin{tag=Reference}(Ally et al. 1991, Collins and Crooke n.d.)\leavevmode\tagmcend\tagstructend}. Onboard CPFV observers in central California saw vermilion rockfish in over 27\% of all observed drifts over the period 1987-1998, making vermilion rockfish fifth among rockfish species in terms of encounter rates per drift {\tagstructbegin{tag=Reference}\tagmcbegin{tag=Reference}(Monk et al. 2016)\leavevmode\tagmcend\tagstructend}

In southern California, harvest of vermilion rockfish from recreational fisheries, as a percentage of the total vermilion rockfish harvest, varied considerably from 1980 to 2000. After 2000, largely due to reduced commercial access to shelf habitat, recreational fisheries accounted for almost all the vermilion rockfish harvest in southern California, with relatively minor contributions from the commercial fleets. Similar patterns occurred north of Point Conception, with the majority of vermilion rockfish landings coming to ports in San Luis Obispo county.

\tagstructbegin{tag=H2}\tagmcbegin{tag=H2}

\hypertarget{summary-of-management-history}{%
\subsection{Summary of Management History}\label{summary-of-management-history}}

\leavevmode\tagmcend\tagstructend

Prior to the adoption of the Pacific Coast Groundfish Fishery Management Plan (FMP) in 1982, vermilion rockfish were managed through a regulatory process that included the California Department of Fish and Wildlife (CDFW) along with either the California State Legislature or the Fish and Game Commission (FGC) depending on the sector (recreation or commercial) and fishery. With implementation of the Pacific Coast Groundfish FMP, vermilion rockfish came under the management authority of the Pacific Fishery Management Council (PFMC), and were managed as part of the \emph{Sebastes} complex. Because neither species had undergone rigorous stock assessment and did not compose a large fraction of the landings they were classified and managed as part of ``Remaining Rockfish'' under the larger heading of ``Other Rockfish'' {\tagstructbegin{tag=Reference}\tagmcbegin{tag=Reference}(Pacific Fishery Management Council 2002, 2004)\leavevmode\tagmcend\tagstructend}.

Since the early 1980s a number of federal regulatory measures have been used to manage the commercial rockfish fishery including cumulative trip limits (generally for two- month periods) and seasons. Starting in 1994 the commercial groundfish fishery sector was divided into two components: limited entry and open access with specific regulations designed for each component. Other regulatory actions for the general rockfish categories have included area closures, gear restrictions, and cumulative bimonthly trip limits set for the four different commercial sectors: limited entry fixed gear, limited entry trawl, open access trawl, and open access non-trawl. Harvest guidelines are also used to regulate the annual harvest for both the recreational and commercial sectors.

In 2000, changes in the PFMC's rockfish management structure resulted in the discontinued use of the \emph{Sebastes} complex, and was replaced with three species groups: nearshore, shelf, and slope rockfishes (January 4, 2000; 65 FR 221). Vermilion rockfish are managed in aggregate with other species in the minor shelf rockfish group, which is further divided into management areas north and south of Cape Mendocino, California ($40^\circ 10^\prime N$).

Since the enactment of California's Marine Life Management Act (MLMA), the Pacific Fishery Management Council and state of California developed and adopted various management specifications including seasonal and area closures (e.g.~the CCAs; a closure of Cordell Banks to specific fishing), depth restrictions, gear restrictions, and bag limits to regulate the recreational fishery. Commercial fisheries were regulated through the use of license and permit regulations, finfish trap permits, gear restrictions, seasonal and area closures (e.g.~the RCAs and CCAs; a closure of Cordell Banks to specific fishing), depth restrictions, trip limits, and minimum size limits {\tagstructbegin{tag=Reference}\tagmcbegin{tag=Reference}(Wilson-Vandenberg et al. 2014)\leavevmode\tagmcend\tagstructend}.

\textbf{Management of Recreational Fisheries}

In March 1984 California adopted a general 20 aggregate daily bag limit that included a sub-bag limit of 10 fish for any given species. Significant regulatory changes in California's recreational sector began with a change from unlimited number of hooks and lines allowed prior to 2000 to no more than three hooks and one line per angler in 2000. Since 2001, the limit has been no more than two hooks and one line per angler and there is not a size limit on vermilion rockfish in the recreational fishery. Beginning January 1, 2021, the CDFW enacted a five-fish sub-bag limit for vermilion rockfish in the recreational fishery.

California also began spatial management, including area closures, and depth restrictions for the recreational fleet in 2000. In general, the recreational season north of Point Conception extends from April to December, and south of Point Conception from March to December. North of Point Conception vermilion rockfish in California are most commonly landed from Monterey to Morro Bay, where the maximum depth open to recreational fishing was between 30 and 40 fathoms until 2017. In 2017, the depth restrictions were eased by 10 fathoms, opening up 40-50 fm depths along the central California coast that had not been open consistently since 2002. In both 2017 and 2018, the deepest 10 fathoms was closed prior to the prescribed season in December due to high by-catch rates of yelloweye rockfish, which remains in an overfished status and is undergoing rebuilding. A full history of the recreational regulations relating to the spatial management of the fleet can be found in the Appendix.

\textbf{Cowcod Conservation Areas (CCA)} In 2001, two area closures {\tagstructbegin{tag=Link}\tagmcbegin{tag=Link}\href{https://nrm.dfg.ca.gov/FileHandler.ashx?DocumentID=36132\&inline}{``Cowcod Conservation Areas''}\leavevmode\tagmcend\tagstructend} were implemented to reduce fishing mortality of cowcod, originally prohibiting bottom-fishing deeper than 20 fm. Effective 2019, retention of nearshore and shelf rockfish (excluding cowcod) is allowed in depths shallower than 40 fm. The larger of the two areas (CCA West) is a 4200 square mile area west of Santa Catalina and San Clemente Islands. A smaller area (CCA East) is about 40 miles offshore of San Diego, and covers about 100 square miles.

\textbf{Rockfish Conservation Areas (RCA)} In 2002 the PFMC established trawl- and non-trawl area closures known as the Rockfish Conservation Areas. These closed areas are gear-specific, and have seasonally changing boundaries to help reduce fishing mortality.

\tagstructbegin{tag=H2}\tagmcbegin{tag=H2}

\hypertarget{management-performance-1}{%
\subsection{Management Performance}\label{management-performance-1}}

\leavevmode\tagmcend\tagstructend

The contribution of vermilion rockfish to the minor shelf rockfish Overfishing Limit (OFL) is currently derived from the data-poor Depletion-Based Stock Reduction Analysis {\tagstructbegin{tag=Reference}\tagmcbegin{tag=Reference}(Dick and MacCall 2010)\leavevmode\tagmcend\tagstructend}. A 2005 vermilion rockfish stock assessment was not accepted for use in management and a 2013 data-moderate assessment was not reviewed by the STAR panel due to insufficient time.

Total mortality for vermilion rockfish was obtained from the Groundfish Expanded Mortality Multiyear {\tagstructbegin{tag=Link}\tagmcbegin{tag=Link}\href{https://www.nwfsc.noaa.gov/data/api/v1/source/observer.gemm_fact/selection.xlsx}{GEMM}\leavevmode\tagmcend\tagstructend} report {\tagstructbegin{tag=Reference}\tagmcbegin{tag=Reference}(Somers et al. 2020)\leavevmode\tagmcend\tagstructend}. The coastwide management of the shelf rockfish complex is split at Cape Mendocino ($40^\circ 10^\prime N$). Therefore, the northern California vermilion rockfish model contains a portion of the management area from Cape Mendocino ($40^\circ 10^\prime N$) to the California-Oregon border ($42^\circ 00^\prime N$). The southern California vermilion rockfish model contains the area within the southern management area (south of $40^\circ 10^\prime N$) that is south of Point Conception ($34^\circ 27^\prime N$).

The total mortality of the shelf rockfish complex has been above the OFL in all years (2011-2019) north of $40^\circ 10^\prime N$, and above the OFL south of $40^\circ 10^\prime N$ from 2015-2019. Total mortality estimates from the NMFS NWFSC are not yet available for 2020 (Table \ref{tab:managementES}). Vermilion rockfish total mortality was on average 59\% (range 55\%-66\%) of the total shelf rockfish south of $40^\circ 10^\prime N$ total mortality from 2011-2016. Vermilion rockfish decreased from 21\% to 4\% of the total contribution to the shelf rockfish complex north of $40^\circ 10^\prime N$ from 2011-2019 with a noticeable decline from 16\% to 6\% from 2016 to 2017.

\tagstructbegin{tag=H2}\tagmcbegin{tag=H2}

\hypertarget{foreign-fisheries}{%
\subsection{Foreign Fisheries}\label{foreign-fisheries}}

\leavevmode\tagmcend\tagstructend

\emph{Sebastes} spp. are not in the Fisheries National Chart (FNC, database containing species status) maintained by the Mexican Government, i.e., they are not commercially harvested in the northwest Mexican Pacific Ocean (E.M. Bojórquez, Centro de Investigaciones Biológicas del Noroeste, S.C., personal communication). Dr.~ Bojórquez also reached out to colleagues at the {\tagstructbegin{tag=Link}\tagmcbegin{tag=Link}\href{https://www.gob.mx/inapesca}{Fisheries National Institute}\leavevmode\tagmcend\tagstructend} who reported that vermilion rockfish are occasionally caught in the sport fishery in Ensenada City. However, there are no data available on vermilion rockfish fisheries off the coast of Mexico. Catches in Mexican waters by US fleets are not included in this assessment.

\FloatBarrier

\tagstructbegin{tag=H1}\tagmcbegin{tag=H1}

\hypertarget{data}{%
\section{Data}\label{data}}

\leavevmode\tagmcend\tagstructend

The STAT presented proposed analyses and data sources for the 2021 vermilion rockfish assessment to the Council advisory bodies in Novemeber 2020, and again during the PFMC Pre-Assessment Workshop for 2021 Vermilion/Sunset Rockfish and Lingcod Stock Assessments, hosted virtually on March 29, 2021. Topics addressed included progress on research priorities, data sources and types, stock structure, fleet structure, key model parameters (e.g.~natural mortality), and potential challenges. Descriptions of each data source included in the model (Figure \ref{fig:data-plot}) and sources that were explored, but not included are included within this section.

\tagstructbegin{tag=H2}\tagmcbegin{tag=H2}

\hypertarget{fishery-dependent-data}{%
\subsection{Fishery-Dependent Data}\label{fishery-dependent-data}}

\leavevmode\tagmcend\tagstructend

A complete summary of estimated vermilion rockfish removals by each fleet in the commercial and recreational sectors modeled in this assessment is provided in Table \ref{tab:landings}. The data sources for landings varied by each fleet and a summary of each data source and the time period for which it was used is in Table \ref{tab:catch-source}. The commercial landings are in metric tons (mt) and the recreational landings are in numbers of fish (thousands of fish). Data and methods used to derive these estimates are described in this section.

\tagstructbegin{tag=H3}\tagmcbegin{tag=H3}

\hypertarget{commercial-landings-and-discards}{%
\subsubsection{Commercial Landings and Discards}\label{commercial-landings-and-discards}}

\leavevmode\tagmcend\tagstructend

\emph{Commercial Landings Prior to 1916}

For landings estimates prior to 1916, we based our reconstruction on the total rockfish catches reported in a summary of early California fisheries landings by Sette and Fielder {\tagstructbegin{tag=Reference}\tagmcbegin{tag=Reference}(1927)\leavevmode\tagmcend\tagstructend} for the years 1888, 1892, 1895, 1899, 1904, 1908 and 1915. No rockfish were reported for 1888. We assumed no catches prior to 1875 and interpolated the catches between 0 mt and the 1892 catches (total of 834 tons) as reported. Similarly, catches between the reported years were interpolated assuming a straight linear trend between the years reported. We used a ratio-estimator derived from the catch reconstruction fraction of vermililion rockfish rockfish in total rockfish landings for the 1916 to 1919 period (the ratio for a comparable five year period was nearly identical). We apportioned the catches north and south of Point Conception based on ratio estimators that used the same assumptions used to apportion catches in the reconstruction time period (1916-1968). The catch reconstruction estimates indicated that vermililion rockfish made up slightly under 1\% of the total rockfish catches during the early (1916-1919) time period, although the estimates indicate a slightly larger fraction (1.5\%) of total catches south of Point Conception relative to the fraction of total catches to the north (0.9\%). However, it is likely that the reconstruction is overestimating the fraction of smaller and/or more deeply distributed species relative to larger, shallower species as the reconstruction is based on the species composition data collected from market category samples in the late 1970s and early 1980s. The fishery has been shown to have progressed over time from a shallower, more nearshore distribution of effort to one in which deeper and more offshore waters were targeted {\tagstructbegin{tag=Reference}\tagmcbegin{tag=Reference}(Miller et al. 2014)\leavevmode\tagmcend\tagstructend}. The notion that vermililion rockfish catches may have been greater is also consistent with the recognition by Roedel {\tagstructbegin{tag=Reference}\tagmcbegin{tag=Reference}(1948)\leavevmode\tagmcend\tagstructend} that during the 1930s and 1940s vermililion rockfish were ``One of the more important commercial species, it is one of three leading species in southern California.'' However, by the time of that report, vermililion rockfish represented five to eight percent of the southern California catch, based on Ralston et al. {\tagstructbegin{tag=Reference}\tagmcbegin{tag=Reference}(2010)\leavevmode\tagmcend\tagstructend}, much more than at the beginning of the time series. This uncertainty is investigated more deeply in the model uncertainty and sensitivity section. Future efforts to improve historical catch reconstructions by accounting for the shift in effort over time to deeper waters should continue to be flagged as a research need.

\emph{Commercial Landings, 1916-2020}

For commercial landings prior to 1969, we queried the SWFSC catch reconstruction database for estimates from the California Catch Reconstruction {\tagstructbegin{tag=Reference}\tagmcbegin{tag=Reference}(Ralston and MacFarlane 2010)\leavevmode\tagmcend\tagstructend}. Landings in this database are divided into trawl, `non-trawl,' and `unknown' gear categories. Regions 7 and 8 as defined by Ralston et al. {\tagstructbegin{tag=Reference}\tagmcbegin{tag=Reference}(2010)\leavevmode\tagmcend\tagstructend} were assigned to southern California. Region 6 in Ralston et al.~includes Santa Barbara County (mainly south of Point Conception), plus some major ports in San Luis Obispo County (north of Point Conception). To allocate catches from Region 6 to the areas north and south of Point Conception, we followed an approach used by Dick et al. {\tagstructbegin{tag=Reference}\tagmcbegin{tag=Reference}(2007)\leavevmode\tagmcend\tagstructend} for the assessment of cowcod. Specifically, port-specific landings of total rockfish from the CDFW Fish Bulletin series were used to determine the annual fraction of landings in Region 6 that was south of Point Conception (Table \ref{tab:com-allocate}). Rockfish landings at that time were not reported at the species level. Although the use of total rockfish landings to partition catch in Region 6 is not ideal, we see this as the best available option in the absence of port-specific species composition data.

Years with no data were imputed using ratio estimates from adjacent years. Annual catches from unknown locations (Region 0) and unknown gear types were allocated proportional to the catches from known regions and gears. Catches from known regions, but unknown gears, were allocated proportional to catches by known gears within the same region. In this way, total annual removals in California were kept consistent with those reported by Ralston et al. {\tagstructbegin{tag=Reference}\tagmcbegin{tag=Reference}(2010)\leavevmode\tagmcend\tagstructend}, and assigned to the assessment areas north and south of Point Conception, and either trawl or `non-trawl' gear types. Since hook-and-line gears catch the majority of commercially-caught vermilion rockfish, we assigned estimated catch in the `non-trawl' category to the hook and line fleet in the assessment model.

In September 2005, the California Cooperative Groundfish Survey (CCGS) incorporated newly acquired commercial landings statistics from 1969-77 into the CALCOM database. The data consisted of landing receipts (``fish tickets''), including mixed species categories for rockfish. In order to assign rockfish landings to individual species, the earliest available species composition samples were applied to the fish ticket data by port, gear, and quarter. These `ratio estimator' landings are coded (internally) as market category 977 in the CALCOM database, and are used in this and past assessments as the best available landings for the time period 1969-1977 for all port complexes. Since commercial port sampling south of Point Conception started later, ratio estimates were used in some southern California port complexes through 1983. See Appendix A of Dick et al. {\tagstructbegin{tag=Reference}\tagmcbegin{tag=Reference}(2007)\leavevmode\tagmcend\tagstructend} and Pearson et al. {\tagstructbegin{tag=Reference}\tagmcbegin{tag=Reference}(2008)\leavevmode\tagmcend\tagstructend}(pp.~8 and 15-16) for further details.

Commercial catches from 1978-present were pulled from the CALCOM database, which is stratified using an identical design as the pre-1978 data described above and ensured consistency of the port complex and gear groupings over the entire time series (1969-2020). Although available strata definitions within PacFIN do not match the design of the California commercial catch expansion {\tagstructbegin{tag=Reference}\tagmcbegin{tag=Reference}(Pearson and Erwin 1997)\leavevmode\tagmcend\tagstructend}, the STAT was able to manually aggregate data from PacFIN to almost exactly match the CALCOM estimates (Figure \ref{fig:calcom-pacfin}). The STAT recommends that port complex and gear group definitions used to expand California commercial catch estimates be incorporated into PacFIN lookup tables to facilitate future comparisons, ensure consistency between the two systems, and help identify potential errors.

\emph{Commercial length and age composition data}

Biological data (lengths) from the commercial fisheries that landed vermilion rockfish were extracted from CALCOM. The CALCOM length composition data were ``expanded'' (catch-weighted by stratum, then aggregated by region, gear group, and year) to better represent the size composition of the landed catch. The length composition is available in Figure \ref{fig:len-data-COM-HKL} for the commercial hook-and-line fleet, Figure \ref{fig:len-data-COM-TWL} for the commercial trawl fleet and Figure \ref{fig:len-data-COM-NET} for the commercial net fleet. Input sample sizes for commercial length compositions were based on the number of port samples and are in Tables \ref{tab:length-sample-size} and \ref{tab:length-inputN}. Length compositions with fewer than 30 measured fish in a region/gear/year combination were not included in the model likelihood.

Commercial discard length compositions from WCGOP were provided on 17 Nov 2020 by Andi Stephens (NWFSC). Only 224 vermililion rockfish were measured statewide from 2004-2018. The sparse discard length composition data were not considered for use in the model as discarded catch is a small fraction of the overall commercial landings.

Otoliths collected from commercial fisheries north of Point Conception were provided by the Pacific States Fisheries Commission and aged, but not used in the assessment due to low annual sample sizes.

\tagstructbegin{tag=H3}\tagmcbegin{tag=H3}

\hypertarget{recreational-landings-and-discard}{%
\subsubsection{Recreational Landings and Discard}\label{recreational-landings-and-discard}}

\leavevmode\tagmcend\tagstructend

\emph{Recreational Landings, 1928-1980}

Recreational catch estimates prior to 1981 were based on the Ralston et al. {\tagstructbegin{tag=Reference}\tagmcbegin{tag=Reference}(2010)\leavevmode\tagmcend\tagstructend} catch reconstruction, which estimated catches by mode (CPFV and private vessel modes, where the latter included any shore-based catches) and estimated catches separately north and south of Point Conception. Party/Charter (PC mode) catches of all rockfish were based on logbook data (which do not report rockfish to the species level), scaled by compliance estimates, while total recreational catches from private/rental vessels (PR mode) catches were based on a combination of the relative catch rates observed in the PC fleet and a linear ramp between catch estimates in the early 1960s and those in the early 1980s (as described in Ralston et al. {\tagstructbegin{tag=Reference}\tagmcbegin{tag=Reference}(2010)\leavevmode\tagmcend\tagstructend}). The species composition of rockfish catches was estimated using a combination of the 1980s MRFS data as well as limited PC mode species composition data from onboard observer programs in the late 1970s (south of Point Conception) and dockside recreational creel surveys in the late 1950s and early 1960s (north of Point Conception). Vermilion (and sunset) rockfish have long been recognized as an important target of recreational fishers south of Point Conception, as well as those in the Morro Bay region, although they are less frequently encountered in recreational fisheries further north. As noted in Ralston et al. {\tagstructbegin{tag=Reference}\tagmcbegin{tag=Reference}(2010)\leavevmode\tagmcend\tagstructend} the catch reconstruction effort was intended to be an ``iterative and multistage process,'' and there is considerable room for improvements in both the commercial and the recreational catch reconstruction estimates.

\emph{Marine Recreational Fishery Statistics Survey (MRFSS), 1980-2003}

MRFSS estimates of California recreational landings from 1980-1989 and 1993-2003 were downloaded from the Recreational Fisheries Information Network ({\tagstructbegin{tag=Link}\tagmcbegin{tag=Link}\href{https://www.recfin.org/}{RecFIN}\leavevmode\tagmcend\tagstructend}). The MRFSS survey design included stratification by species (sunset rockfish were not recognized at the time), subregion (northern and southern California), 2-month `wave,' water area (e.g.~within or beyond three miles from shore), and fishing mode (party/charter (PC) and private/rental (PR) boats, plus various shore modes). The PC mode includes the Commercial Passenger Fishing Vessel fleet (CPFV).

Some known issues with the MRFSS estimates include 1) missing or imprecise estimates of catch in weight for some strata that reported catch in numbers, 2) a change in the spatial definition of California subregions after 1989, and 3) a hiatus in sampling from 1990-1992 (all modes) and also 1993-1995 in the party/charter mode north of Point Conception. The STAT attempted to address each of these issues, as described below. CRFS estimates from 2004 were also included in the MRFSS analysis, as they were not available on the current RecFIN website but are included with the MRFSS catch estimate tables.

The MRFSS estimated catch in numbers of fish and converted these to catch in weight using estimates of average fish weight {[}kg{]} from the same stratum. When a stratum contained an estimate of catch in numbers but was missing an average weight, the estimate of catch in weight for that stratum was omitted (or sometimes assigned a zero value) in the database. To correct these errors, the STAT first identified strata with positive catch in numbers but missing or zero values for catch in weight. Catch in weight for these strata was then estimated by imputing a value of average weight based on the mean of the reported average weights in the same year and subregion, which had a greater influence on average weight than boat mode (Figure \ref{fig:rec-avg-weights}). The effect of this data imputation was relatively minor for vermilion rockfish overall (\textasciitilde1\% increase in total catch by weight, 1980-2004). However, 70\% of missing catch in weight occurred over the years 2001-2004, with differences in individual year/mode/subregion combinations sometimes exceeding 10-20\%.

MRFSS catch estimates for California were spatially stratified into two subregions, ``Southern California'' (subregion 1) and ``Northern California'' (subregion 2). During the 1990-1992 statewide hiatus in sampling, the definitions of these two subregions changed. Specifically, San Luis Obispo (SLO) County was included in the southern region prior to the hiatus (i.e.~1980-1989) {\tagstructbegin{tag=Reference}\tagmcbegin{tag=Reference}(Witzig et al. 1992, Karpov et al. 1995)\leavevmode\tagmcend\tagstructend}, but moved to the northern subregion starting in 1993. In order to create a definition of spatial strata that is consistent and comparable over time, and one that is consistently divided near Point Conception, the STAT examined estimates of catch in numbers from a separate study {\tagstructbegin{tag=Reference}\tagmcbegin{tag=Reference}(Albin and Karpov 1993)\leavevmode\tagmcend\tagstructend} that used a finer spatial resolution in the northern subregion (including SLO County). Over the period 1981-1986, numbers of vermilion rockfish landed in SLO County were found to be roughly equal to the numbers of vermilion rockfish landed in all California counties north of SLO County (Table \ref{tab:albin-allocate}). Therefore, to approximate catches north and south of Point Conception from 1980-1989, the STAT reduced the `southern' subregion annual catch (which included SLO County) from 1980-1989 by an amount equal to the northern subregion catch during the same period, and doubled the northern subregion catch. On average, this `moves' the estimated SLO County catch from the southern region to the northern region from 1980-1989, creating a spatially consistent time series of landings over the entire time series.

Ultimately, the STAT chose to use recreational catch in numbers rather than catch in weight for the California assessment models. Since data from Albin {\tagstructbegin{tag=Reference}\tagmcbegin{tag=Reference}(1993)\leavevmode\tagmcend\tagstructend} were only available as catch in numbers, the ratios used to partition SLO County catch may not be consistent if applied to catch in weight due to differences in average weight between regions (Figure \ref{fig:rec-avg-weights}). Also, because missing weight estimates were concentrated over the period 2001-2004 rather than being spread over the entire time series, the method used to impute weights could have a greater influence on short-term stock dynamics.

As noted above, MRFSS sampling was halted from 1990-1992 due to funding issues. The survey resumed in 1993 in all modes, except for the PC boat mode which resumed in 1996 for counties north of Santa Barbara County. To produce catch estimates for the missing subregion/mode/year combinations, we used linear interpolation. Shore modes were a minor component of the vermilion rockfish catch and therefore combined with catches from the private (PR) boat mode into a single fleet. Specifically, catches were aggregated by subregion (adjusted as described above), year, and mode, and endpoints for the interpolations were defined as 2-year averages to reduce the effects of interannual variability in catch on interpolated estimates.

The MRFSS did not collect data on the size composition of discarded fish (except in the program's last year, 2003), although recent CRFS sampling shows that the mean size of discarded fish is smaller than retained catch. Since catch type ``B1'' is an angler-reported mixture of dead discards and landed fish which were unavailable to the sampler, the true size composition of B1 fish is unknown. To determine the effect of alternative assumptions about the size composition of discarded fish, the STAT separated B1 fish into a separate fleet in the model. This allowed us to apply discard size composition data from the more recent CRFS survey, and compare the result to a model that assumes B1 catch has the same size composition as the examined catch. Results are described in the model sensitivity section. Since the ratio of B1 catch to total catch (A+B1) was highly variable among years, an average B1/(A+B1) ratio was estimated for each subregion and boat mode. These average discard ratios were applied to the annual estimates of total catch to estimate annual discarded catch prior to 2005.

MRFSS estimates of catch and discard (1000s of fish) after adjustment for changes in subregion definition and sampling gaps are shown in Table \ref{tab:landings}.

\emph{California Recreational Fisheries Survey (CRFS), 2004-2020}

Estimates of recreational landings and discard since 2004 have been produced by the CRFS. This survey improves upon the MRFSS sampling design, employing higher sampling rates and producing estimates with finer spatial and temporal resolution. The CRFS also employs onboard CPFV observers, providing spatially referenced, drift-level estimates of catch and discard for a subset of anglers on observed groundfish trips, as well as length composition data for discarded catch. These data are extremely valuable to stock assessment (see the CRFS Onboard Index of Abundance Index for further details).

CRFS mortality estimates for the period 2005-2020 were queried from {\tagstructbegin{tag=Link}\tagmcbegin{tag=Link}\href{www.recfin.org}{RecFIN}\leavevmode\tagmcend\tagstructend}. Reported estimates were aggregated into subregion (north and south of Point Conception) and boat mode (PC and PR), and filtered to exclude fish caught in Mexican waters. Shore modes were a minor component of the recreational catch and were combined with the PR mode.

\emph{Discard mortality rates}

Total recreational mortality estimates provided to RecFIN are adjusted using species- and depth-specific discard mortality rates. The discard mortality rates for vermilion rockfish that were endorsed by the SSC and adopted by the PFMC in March 2017 are 20\% for 0-10 fm, 34\% for 10-20 fm, 50\% for 20-30 fm, and 100\% for greater than 30 fm.

Similar to the MRFSS data, CRFS discard estimates were treated as a separate fleet to evaluate the effect of alternative size composition assumptions on model results. Estimates of retained and released dead fish (in numbers) by subregion and mode are available from the RecFIN website, and these were used in the model. Other than combining PR and shore modes, the estimates described above were used without modification.

\emph{Recreational length composition data}

Length compositions were provided from the following sources:

There are no available recreational length composition data available for 2020 north of Point Conception in RecFIN and sparse sampling was confirmed by J. Budrick (CDFW, pers. comm.). Data collected during the Miller and Gotshall study was also used by Karpov (1995) to compare MRFSS and historical estimates. Some sections of the assessment refer to the Miller and Gotshall dataset as ``Karpov'' data.

\tagstructbegin{tag=L}

\begin{itemize}
\item
  \tagstructbegin{tag=LI}\tagstructbegin{tag=LBody}\tagmcbegin{tag=P}

  Recreational party/charter mode (PC)

  \tagmcend\tagstructend\tagstructend

  \tagstructbegin{tag=L}

  \begin{itemize}
  \item
    \tagstructbegin{tag=LI}\tagstructbegin{tag=LBody}\tagmcbegin{tag=P}

    \tagstructbegin{tag=LI}\tagstructbegin{tag=LBody}\tagmcbegin{tag=P}

    Miller and Gotshall dockside PC survey (1959-1960)

    \tagmcend\tagstructend\tagstructend

    \tagmcend\tagstructend\tagstructend
  \item
    \tagstructbegin{tag=LI}\tagstructbegin{tag=LBody}\tagmcbegin{tag=P}

    \tagstructbegin{tag=LI}\tagstructbegin{tag=LBody}\tagmcbegin{tag=P}

    PC samples collected by commercial port samples (1978-1979)

    \tagmcend\tagstructend\tagstructend

    \tagmcend\tagstructend\tagstructend
  \item
    \tagstructbegin{tag=LI}\tagstructbegin{tag=LBody}\tagmcbegin{tag=P}

    \tagstructbegin{tag=LI}\tagstructbegin{tag=LBody}\tagmcbegin{tag=P}

    MRFSS dockside PC survey (1980-2003)

    \tagmcend\tagstructend\tagstructend

    \tagmcend\tagstructend\tagstructend
  \item
    \tagstructbegin{tag=LI}\tagstructbegin{tag=LBody}\tagmcbegin{tag=P}

    \tagstructbegin{tag=LI}\tagstructbegin{tag=LBody}\tagmcbegin{tag=P}

    CRFS dockside PC survey (2004-2019)

    \tagmcend\tagstructend\tagstructend

    \tagmcend\tagstructend\tagstructend
  \item
    \tagstructbegin{tag=LI}\tagstructbegin{tag=LBody}\tagmcbegin{tag=P}

    \tagstructbegin{tag=LI}\tagstructbegin{tag=LBody}\tagmcbegin{tag=P}

    CRFS onboard (discard only) and dockside (retained only surveys 2004-2019)

    \tagmcend\tagstructend\tagstructend

    \tagmcend\tagstructend\tagstructend
  \item
    \tagstructbegin{tag=LI}\tagstructbegin{tag=LBody}\tagmcbegin{tag=P}

    \tagstructbegin{tag=LI}\tagstructbegin{tag=LBody}\tagmcbegin{tag=P}

    Deb Wilson-Vandenberg onboard CPFV survey (1988-1998)

    \tagmcend\tagstructend\tagstructend

    \tagmcend\tagstructend\tagstructend
  \end{itemize}

  \tagstructend
\item
  \tagstructbegin{tag=LI}\tagstructbegin{tag=LBody}\tagmcbegin{tag=P}

  Recreational private/rental mode (PR)

  \tagmcend\tagstructend\tagstructend

  \tagstructbegin{tag=L}

  \begin{itemize}
  \item
    \tagstructbegin{tag=LI}\tagstructbegin{tag=LBody}\tagmcbegin{tag=P}

    \tagstructbegin{tag=LI}\tagstructbegin{tag=LBody}\tagmcbegin{tag=P}

    Miller and Gotshall dockside PR survey (1959)

    \tagmcend\tagstructend\tagstructend

    \tagmcend\tagstructend\tagstructend
  \item
    \tagstructbegin{tag=LI}\tagstructbegin{tag=LBody}\tagmcbegin{tag=P}

    \tagstructbegin{tag=LI}\tagstructbegin{tag=LBody}\tagmcbegin{tag=P}

    MRFSS dockside PR (1980-2003)

    \tagmcend\tagstructend\tagstructend

    \tagmcend\tagstructend\tagstructend
  \item
    \tagstructbegin{tag=LI}\tagstructbegin{tag=LBody}\tagmcbegin{tag=P}

    \tagstructbegin{tag=LI}\tagstructbegin{tag=LBody}\tagmcbegin{tag=P}

    CRFS dockside PR (2004-2019)

    \tagmcend\tagstructend\tagstructend

    \tagmcend\tagstructend\tagstructend
  \end{itemize}

  \tagstructend
\end{itemize}

\tagstructend

The number of available fish by year and fleet as well as the method we used to calculate initial sample sizes are in Tables \ref{tab:length-sample-size} and \ref{tab:length-inputN}. Length composition data can be found in Figure \ref{fig:len-data-REC-PC} for the recreational PC retained fleet and Figure \ref{fig:len-data-REC-PC-DIS} for the recreational PC discard fleet, Figure \ref{fig:len-data-REC-PR} for the recreational PR fleet, and Figure \ref{fig:len-data-DWV-ONBOARD} for the Deb Wilson-Vandenberg CPFV onboard survey.

\emph{Recreational age composition data}

There are no recreational age composition data available for vermilion rockfish from California state sampling programs. Otoliths are available from SWFSC collaborative study with Cal Poly to investigate the precision of back-calculating whole fish length from filleted fish in the CPFV fleet. These otoliths were not aged for this assessment.

\emph{Recreational indices of abundance}

A number of indices of abundance were explored for the recreational fleet (Figure \ref{fig:cpueall}), noting there were limited recreational index data from 2020 due to COVID-19. Discarded catch is available from onboard observer surveys, but was not included in indices. The STAT considered developing separate indices for discards, but sample sizes were not large enough to warrant modeling. The CDFW CPFV logbook data were not considered as an index of abundance due to the fact that vermilion rockfish may not be accurately reported to the species level. Indices developed for the assessment include:

\tagstructbegin{tag=L}

\begin{itemize}
\item
  \tagstructbegin{tag=LI}\tagstructbegin{tag=LBody}\tagmcbegin{tag=P}

  MRFSS era dockside survey of the PC fleet (1980-1999)

  \tagmcend\tagstructend\tagstructend
\item
  \tagstructbegin{tag=LI}\tagstructbegin{tag=LBody}\tagmcbegin{tag=P}

  Deb Wilson Vandenberg's CPFV onboard observer survey (1988-1998)

  \tagmcend\tagstructend\tagstructend
\item
  \tagstructbegin{tag=LI}\tagstructbegin{tag=LBody}\tagmcbegin{tag=P}

  CDFW/Cal Poly CPFV onboard observer index (1999-2019)

  \tagmcend\tagstructend\tagstructend
\item
  \tagstructbegin{tag=LI}\tagstructbegin{tag=LBody}\tagmcbegin{tag=P}

  CRFS PR1 sites dockside survey (2004-2019)

  \tagmcend\tagstructend\tagstructend
\end{itemize}

\tagstructend

\tagstructbegin{tag=H2}\tagmcbegin{tag=H2}

\hypertarget{fishery-independent-data}{%
\subsection{Fishery-Independent Data}\label{fishery-independent-data}}

\leavevmode\tagmcend\tagstructend

\tagstructbegin{tag=H3}\tagmcbegin{tag=H3}

\hypertarget{nwfsc-west-coast-groundfish-bottom-trawl-survey}{%
\subsubsection{NWFSC West Coast Groundfish Bottom Trawl Survey}\label{nwfsc-west-coast-groundfish-bottom-trawl-survey}}

\leavevmode\tagmcend\tagstructend

The West Coast Groundfish Bottom Trawl Survey (WCGBTS) is based on a random-grid design; covering the coastal waters from a depth of 55-1,280 m {\tagstructbegin{tag=Reference}\tagmcbegin{tag=Reference}(Keller et al. 2017)\leavevmode\tagmcend\tagstructend}. This design generally uses four industry-chartered vessels per year assigned to a roughly equal number of randomly selected grid cells and divided into two `passes' off the coast of Washington, Oregon, and California. Two vessels fish from north to south during each pass between late May to early October. This design therefore incorporates both vessel-to-vessel differences in catchability, as well as variance associated with selecting a relatively small number (approximately 700) of possible cells from a very large set of possible cells spread from the Mexican to the Canadian borders.

Vermilion rockfish are strongly associated with rocky habitat, i.e., untrawlable habitat, but can be found over soft bottom, especially as juveniles. This survey spans the entire West Coast and provided data for both the northern and southern California assessments. However, this survey does not sample most rocky habitats, nor does the survey conduct sampling within the Cowcod Conservation Areas (CCAs) or the California state Marine Protected Area (MPA) network.

\textbf{Available Data}

\emph{Age and Length Data.} Vermilion rockfish are not found in high abundance in this survey, and in most cases lengths for the entire catch were available, i.e., few enough individuals were caught that all were measured. The assessment north of Point Conception includes 467 ages, which is the majority of the vermilion rockfish with available length information (587 total). South of Point Conception, 1,283 of the 1,962 vermilion rockfish observed and measured were also aged (Table \ref{tab:age-sample-size}). The length compositions by year of vermilion rockfish from the WCGBT survey are shown in Figure \ref{fig:len-data-NWFSC-TWL}.

\emph{Maturity samples.} Maturity samples were analyzed by by Melissa Head (NWFSC) and a description of the results is in the section on biological data.

\emph{Index of abundance.} The index was considered, but not used in the pre-STAR base model. VAST-WestCoast was explored for standardization of the WCBTS data both north and south of Point Conception. Unfortunately, results were uncertain given the small number of tows that observed vermilion rockfish. Truncating the spatial distribution of the survey to less than 300 m, which only eliminated a small handful of positive tows, did not decrease the uncertainty such that spatially-explicit parameters were estimable. Model convergence was more of an issue south of Point Conception rather than north of the break. Changing the distributional assumptions of the positive model or changing to a tweedie-like distribution that combines the two models did not increase the likelihood that the model could estimate spatially-explicit parameters. It was decided that a non-spatial model, which is more easily accomplished outside of the VAST framework would be best for all areas where the survey samples this species. Future research could investigate correlation structures between areas and if shared information across small regions of overlap would stabilize parameter estimation.

The STAT also developed a delta-glm model for each area (north and south of Point Conception). Full details of the final index are in the Appendix, including sample sizes, model selection criteria, and model diagnostics.

\tagstructbegin{tag=H3}\tagmcbegin{tag=H3}

\hypertarget{j.-abrams-thesis-data}{%
\subsubsection{J. Abrams thesis data}\label{j.-abrams-thesis-data}}

\leavevmode\tagmcend\tagstructend

For his master's thesis work at Humboldt State University, Jeff Abrams conducted fishery-independent hook-and-line surveys in 2010 and 2011 off of California's North Coast {\tagstructbegin{tag=Reference}\tagmcbegin{tag=Reference}(Abrams 2014)\leavevmode\tagmcend\tagstructend}. Sites were randomly sampled from areas of known rocky habitat within six depth by distance-from-port strata out of three ports: Crescent City Harbor, Trinidad Bay and Noyo River Harbor. The otoliths collected as part of this study are valuable stock assessments of recreationally-important groundifsh species that are often lacking biological samples, especially from the North Coast. This collection resides at the SWFSC Santa Cruz lab.

\textbf{Available Data}

\emph{Age and Length Data.} All 81 vermilion rockfish collected during the survey were aged and represent the most northern biological samples in the northern California model. The available length compositions for males and females, 2010-2011, are shown in Figure \ref{fig:len-data-ABRAMS-RESEARCH}.

\tagstructbegin{tag=H3}\tagmcbegin{tag=H3}

\hypertarget{swfsc-groundfish-ecology-cruises}{%
\subsubsection{SWFSC Groundfish Ecology Cruises}\label{swfsc-groundfish-ecology-cruises}}

\leavevmode\tagmcend\tagstructend

Don Pearson (SWFSC, retired) conducted a series of groundfish surveys (hook-and-line and trawl) from 2003 - 2005 along the coast of California. Surveys were conducted onboard chartered commercial vessels and NOAA research vessels.

Even though samples were collected via multiple gear types, the the majority were collected using commercial hook-and-line hear, and data from all gears were combined for use in the assessment.

\textbf{Available Data}

\emph{Age and Length Data.} A total of 229 vermilion rockfish otoliths were aged from this survey from samples in 2004-2005. The length composition includes 355 vermilion rockfish from these surveys (Figure \ref{fig:len-data-SWFSC-GF-ECOL}).

\tagstructbegin{tag=H3}\tagmcbegin{tag=H3}

\hypertarget{california-collaborative-fisheries-research-project}{%
\subsubsection{California Collaborative Fisheries Research Project}\label{california-collaborative-fisheries-research-project}}

\leavevmode\tagmcend\tagstructend

Since 2007, the California Collaborative Fisheries Research Program (CCFRP) has monitored several areas in California to evaluate the performance of MPAs and understand nearshore fish populations {\tagstructbegin{tag=Reference}\tagmcbegin{tag=Reference}(Wendt and Starr 2009, Starr et al. 2015)\leavevmode\tagmcend\tagstructend}. In 2017, the survey expanded beyond the four MPAs in central California (Año Nuevo, Point Lobos, Point Buchon, and Piedras Blancas) to include the entire California coast. Fish are collected by volunteer anglers aboard CPFVs guided by one of the following academic institutions based on proximity to fishing location: Humboldt State University; Bodega Marine Laboratories; Moss Landing Marine Laboratories; Cal Poly San Luis Obispo; University of California, Santa Barbara; and Scripps Institution of Oceanography.

Surveys consist of fishing with hook-and-line gear for 30-45 minutes within randomly chosen 500 by 500 m grid cells within and outside MPAs. Prior to 2017, all fish were measured for length and released or descended to depth; since then, some have been retained for biological collections including otoliths and fin clips. This is the only long-term fisheries-independent data series that spans the entire California coast (although coastwide coverage is limited to recent years) and provides relative abundance and demographic data on fish stocks within California's network of MPAs.

\textbf{Available Data}

\emph{Age and Length Data.} A total of 48 otoliths from the CCFRP survey were available, but not included in the assessment model due to annual sample sizes of fewer than in the CCFRP survey is in Figure \ref{fig:len-data-CCFRP}.

\emph{Index of Abundance} The index of abundance in the pre-STAR base model is based on a Bayesian negative binomial model, and the posterior predictions were weighted with the assumption that 20\% of the available habitat within California state waters (0-3 nm) is within MPAs (Figure \ref{fig:cpueall}). The SWFSC has worked extensively on quantifying area of rocky habitat from high resolution bathymetric data collected as part of the Seafloor Mapping Program. This method of habitat area quantification has been used in a number of stock assessments to weight indices of abundance since 2013. This is the first time the habitat data are utilized to weight an inside/outside MPA effect within an index of abundance. Full details on the observed data, model selection and modeling methods can be found in the Appendices. Although it was not used in the assessment, the details related to the index of abundance are retained in the document for future reference.

\tagstructbegin{tag=H2}\tagmcbegin{tag=H2}

\hypertarget{additional-considered-data-sources}{%
\subsection{Additional Considered Data Sources}\label{additional-considered-data-sources}}

\leavevmode\tagmcend\tagstructend

The STAT considered the following data sources, but found that vermilion rockfish were not well sampled and no further analysis was conducted.

\emph{NWFSC Triennial Survey}

The Triennial Survey was first conducted by the Alaska Fisheries Science Center in 1977, and the survey continued until 2004 {\tagstructbegin{tag=Reference}\tagmcbegin{tag=Reference}(Dark and Wilkins 1994)\leavevmode\tagmcend\tagstructend}. Its basic design was a series of equally-spaced east-to-west transects across the continental shelf from which searches for tows in a specific depth range were initiated. The survey design changed slightly over time. In general, all of the surveys were conducted in the mid summer through early fall. The 1977 survey was conducted from early July through late September. The surveys from 1980 through 1989 were conducted from mid-July to late September. The 1992 survey was conducted from mid July through early October. The 1995 survey was conducted from early June through late August. The 1998 survey was conducted from early June through early August. Finally, the 2001 and 2004 surveys were conducted from May to July.

Haul depths ranged from 91-457 m during the 1977 survey with no hauls shallower than 91 m. Due to haul performance issues and truncated sampling with respect to depth, the data from 1977 were omitted from this analysis. The surveys in 1980, 1983, and 1986 covered the US West Coast south to 36.8\textdegree N latitude and a depth range of 55-366 m. The surveys in 1989 and 1992 covered the same depth range but extended the southern range to $40^\circ 10^\prime N$ (near Point Conception). From 1995 through 2004, the surveys covered the depth range 55-500 m and surveyed south to $40^\circ 10^\prime N$. In 2004, the final year of the Triennial Survey series, the NWFSC Fishery Resource Analysis and Monitoring Division (FRAM) conducted the survey following similar protocols to earlier years.

\emph{Alaska Fisheries Science Center Slope Survey}

The Alaska Fisheries Science Center Slope Survey operated during the months of October to November aboard the R/V \emph{Miller Freeman}. Partial survey coverage of the US West Coast occurred during the years 1988-1996 and complete coverage (north of 34\textdegree 30\textquotesingle S) during the years 1997 and 1999-2001. Typically, only these four years that are seen as complete surveys are included in assessments.

\emph{Partnership for Interdisciplinary Studies of Coastal Oceans}

The Partnership for Interdisciplinary Studies of Coastal Oceans, {\tagstructbegin{tag=Link}\tagmcbegin{tag=Link}\href{http://www.piscoweb.org/kelp-forest-study}{PISCO-UCSC}\leavevmode\tagmcend\tagstructend}, conducts a number of surveys to monitor the kelp forests, one of which is a subtidal fish survey. PISCO has monitored fish population in the 0-20 m depth range as part of the Marine Life Protection Act (MLPA) since 1998. Paired sites inside and outside MPAs are surveyed to monitor the long-term dynamics of the kelp forest ecosystem and provide insight into the effect of MPAs on kelp forest species. PISCO conducts the fish surveys from late July through September. At each site, benthic, midwater, and canopy scuba transects are conducted at 5, 10, 15, and 20 m depth. All divers are trained in species identification. Along each 30 m transect, divers enumerate all identifiable non-cryptic fish, and estimate total length to the nearest centimeter. PISCO surveys are conducted by the University of California Santa Cruz (UCSC) in central California and the University of California Santa Barbara in southern California.

\emph{California Cooperative Oceanic Fisheries Investigations}

The California Cooperative Oceanic Fisheries Investigations (CalCOFI) survey began in 1951 and conducts quarterly cruises off southern and central California, collecting a suite of hydrographic and biological data at fixed stations and while underway; ichthyoplankon sampling with a paired bongo started in 1978. Data on larval abundance from the CalCOFI Ichthyoplankton survey have been used in stock assessments of several species, including bocaccio, cowcod and shortbelly rockfish. Although the long-term dataset is limited to a subset of species for which morphological identification of larvae has been possible, recent research has been successful at identifying a broader range of species based on genetic identification of larvae {\tagstructbegin{tag=Reference}\tagmcbegin{tag=Reference}(Thompson et al. 2016)\leavevmode\tagmcend\tagstructend}. Vermilion rockfish cannot be identified morphologically in the ichthyoplankton samples. Of more than 20,000 larvae identified in the 1998-2013 time period, only nine were vermilion rockfish. Consequently, the data are insufficient at this time to use to inform relative abundance, although Thompson et al. {\tagstructbegin{tag=Reference}\tagmcbegin{tag=Reference}(2017)\leavevmode\tagmcend\tagstructend} do provide several relative abundance time series for other taxa, and future efforts may lead to better taxonomic resolution of historical or future collections.

\emph{Rockfish Recruitment and Ecosystem Survey}

Since 1983, the SWFSC has conducted an annual midwater trawl survey for pelagic juvenile rockfish and other groundfish in the Central California region of the California Current (Ralston et al. {\tagstructbegin{tag=Reference}\tagmcbegin{tag=Reference}(2013)\leavevmode\tagmcend\tagstructend} and references therein). Due to concerns about mesoscale abundance patterns and a need for greater spatial representation in the data, including some apparent strong differences in spatial distribution patterns in the early 2000s {\tagstructbegin{tag=Reference}\tagmcbegin{tag=Reference}(Hastie and Ralston 2007, Ralston et al. 2013)\leavevmode\tagmcend\tagstructend}, this survey was expanded to a broader spatial scale in the 2001-2004 period, and since 2004 most years have coastwide data from a combination of SWFSC, NWFSC and Cooperative Research surveys (see Field et al. {\tagstructbegin{tag=Reference}\tagmcbegin{tag=Reference}(2021)\leavevmode\tagmcend\tagstructend} for more complete details regarding coastwide pre-recruit data, and Sakuma et al. {\tagstructbegin{tag=Reference}\tagmcbegin{tag=Reference}(2016)\leavevmode\tagmcend\tagstructend} and Friedman et al. {\tagstructbegin{tag=Reference}\tagmcbegin{tag=Reference}(2018)\leavevmode\tagmcend\tagstructend} for additional details and alternative applications of survey data). Only 47 of nearly 300,000 total juvenile \emph{Sebastes} encountered in the juvenile surveys since 2001 were identified as vermillion or sunset rockfish {\tagstructbegin{tag=Reference}\tagmcbegin{tag=Reference}(Field et al. 2021)\leavevmode\tagmcend\tagstructend}. Despite this, the assessment results suggest that at least a reasonable fraction of recruitment variability for sunset and vermillion rockfish is shared with other rockfish and groundfish stocks throughout the California Current, many of which also had strong year classes in 1984, 1999 and 2015, and future investigations could lead to the development of multispecies-based recruitment indicators that could be helpful for future assessments.

\tagstructbegin{tag=H2}\tagmcbegin{tag=H2}

\hypertarget{biology}{%
\subsection{Biology}\label{biology}}

\leavevmode\tagmcend\tagstructend

\tagstructbegin{tag=H3}\tagmcbegin{tag=H3}

\hypertarget{ageing-precision-and-bias}{%
\subsubsection{Ageing Precision and Bias}\label{ageing-precision-and-bias}}

\leavevmode\tagmcend\tagstructend

Uncertainty in ageing error was estimated using a collection of 357 vermilion rockfish otoliths with two age reads between the NWFSC (reader 1, B. Kamikawa) and the SWFSC (reader 2, D. Watters) (Figure \ref{fig:reader1reader2}). Age-composition data used in the model were from a number of sources described above. The same readers aged otoliths for both California vermilion rockfish stock assessment models. Age reader 1 read all of the otoliths for the southern model and both readers read otoliths for the northern California model. In addition to the otoliths from these two regions, the same two readers aged fish for a Committee of Age Reading Experts (CARE) exchange among four ageing labs, initiated by the SWFSC.

Ageing error was estimated using publicly available software {\tagstructbegin{tag=Reference}\tagmcbegin{tag=Reference}(Thorson et al. 2012)\leavevmode\tagmcend\tagstructend}. Reader 1 who was more experienced, was assumed to be unbiased. The {\tagstructbegin{tag=Formula}\tagmcbegin{tag=Formula}\(\Delta AIC\)\leavevmode\tagmcend\tagstructend} among the top three models was less than two. The best fitting model selected curvilinear bias for reader 1 and curvilinear standard deviation for both readers. An analysis of ageing error after removing one fish aged at 88 by reader 1 and 78 by reader 2 selected the model with reader 2 as unbiased and curvilinear standard deviation (Figure \ref{fig:oldfish}). The reading of the oldest aged fish falls within the 95\% confidence interval using this model (Figure \ref{fig:truereads}). The latter model was selected for use in the assessment and the distribution of true age and observed age is in Figure \ref{fig:ageerror}.

The resulting estimates of ageing error indicated a standard deviation in age readings increasing from 0.001 years at age 0 to a standard deviation of 2.37 years at age 70, the first year of the plus group in the assessment model.

\tagstructbegin{tag=H3}\tagmcbegin{tag=H3}

\hypertarget{maturity}{%
\subsubsection{Maturity}\label{maturity}}

\leavevmode\tagmcend\tagstructend

Maturity at length of nominal vermilion rockfish was previously studied by Wyllie Echeverria {\tagstructbegin{tag=Reference}\tagmcbegin{tag=Reference}(1987)\leavevmode\tagmcend\tagstructend} from fish collected off central California. She found that 50\% of females sampled were mature by 37 cm total length, and 50\% of males were mature by 38 cm total length. Love et al. {\tagstructbegin{tag=Reference}\tagmcbegin{tag=Reference}(1990)\leavevmode\tagmcend\tagstructend} reported 37 cm total length for female size at 50\% maturity, based on fish collected in southern California. Phillips {\tagstructbegin{tag=Reference}\tagmcbegin{tag=Reference}(1964)\leavevmode\tagmcend\tagstructend} reported a size at 50\% maturity of 13 inches (33 cm) total length, although the sampling location of the fish used to determine maturity for that study was not specified within California.

For the current assessment, Melissa Head (NWFSC, pers. comm.) determined maturity for 545 female vermilion rockfish caught by recent fishery-independent surveys. Two types of maturity determinations were provided, `biological maturity' and `functional maturity.' The former category includes ``juveniles exhibiting dummy runs (early vitellogenesis or yolk granules present in a small proportion of oocytes, some in early stages of cellular decay) and skip spawners (adults foregoing spawning in a given year)'' (M. Head, pers. comm.), while the latter excludes such cases. A logistic regression was fit to the functional maturity determination as a function of fork length (Figure \ref{fig:functional-maturity}), estimating length at 50\% maturity at 38.4 cm, with a slope of -0.312, based on the parameterization in Stock Synthesis. The samples available from areas north of Point Conception were smaller fish and did not allow for estimates of separate maturity curves. Both California vermilion rockfish assessments assumed the same maturity ogive (Figure \ref{fig:maturity}).

\tagstructbegin{tag=H3}\tagmcbegin{tag=H3}

\hypertarget{fecundity}{%
\subsubsection{Fecundity}\label{fecundity}}

\leavevmode\tagmcend\tagstructend

Phillips {\tagstructbegin{tag=Reference}\tagmcbegin{tag=Reference}(1964)\leavevmode\tagmcend\tagstructend} reported fecundity for nominal ``vermilion'' rockfish collected in waters off California. Based on a sample of 12 fish ranging in size from 315-550 mm total length, he reported the minimum and maximum number of eggs as 63,300 and 1,625,600 per female, respectively. Love et al. {\tagstructbegin{tag=Reference}\tagmcbegin{tag=Reference}(1990)\leavevmode\tagmcend\tagstructend} estimated fecundity of fish in southern California, and reported an allometric fecundity - length relationship (eggs vs.~total length, cm) with an exponent of 5.02, suggesting a significant increase in weight-specific fecundity with female size given a roughly cubic weight-length relationship. Dick et al. {\tagstructbegin{tag=Reference}\tagmcbegin{tag=Reference}(2017b)\leavevmode\tagmcend\tagstructend} conducted a meta-analysis of \emph{Sebastes} fecundity-length relationships. Insufficient data were available to model the subgenus \emph{Rosicola}, but the predictive distribution of the fecundity-length exponent for the genus as whole centered around a value of four, supporting a general pattern of increasing weight-specific fecundity among the \emph{Sebastes}. Analyses to date have not examined size-dependent changes in brood frequency for vermilion or sunset rockfish, i.e.~current fecundity estimates represent brood fecundity.

For this assessment, new observations of fecundity at length were supplied by S. Beyer (UCSC / SWFSC, pers. comm.). These data were combined with digitized historical data sets used by Dick et al. {\tagstructbegin{tag=Reference}\tagmcbegin{tag=Reference}(2017b)\leavevmode\tagmcend\tagstructend} to estimate a new fecundity-length relationship (Figure \ref{fig:fitted-fecundity}). The relationship between fecundity (millions of eggs) and fork length (cm) estimated from these data and used in the assessment was {\tagstructbegin{tag=Formula}\tagmcbegin{tag=Formula}\(F = 2.8e^{-9}L^{4.97}\)\leavevmode\tagmcend\tagstructend}

The resulting relationship between fecundity by female weight (kg) is illustrated in Figure \ref{fig:fecundity}, with spawning output at age (the product of maturity and fecundity) in Figure \ref{fig:spawnage}.

\tagstructbegin{tag=H3}\tagmcbegin{tag=H3}

\hypertarget{natural-mortality}{%
\subsubsection{Natural Mortality}\label{natural-mortality}}

\leavevmode\tagmcend\tagstructend

Natural mortality was not directly measured, so life-history based empirical relationships were used. The Natural Mortality Tool {\tagstructbegin{tag=Link}\tagmcbegin{tag=Link}\href{https://github.com/shcaba/Natural-Mortality-Tool}{NMT}\leavevmode\tagmcend\tagstructend}, a Shiny-based graphical user interface allowing for the application of a variety of natural mortality estimators based on measures such as longevity, size, age and growth, and maturity, was used to obtain estimates of natural mortality. The NMT currently provides 19 options, including the Hamel {\tagstructbegin{tag=Reference}\tagmcbegin{tag=Reference}(2015)\leavevmode\tagmcend\tagstructend} method, which is a corrected form of the Then et al. {\tagstructbegin{tag=Reference}\tagmcbegin{tag=Reference}(2018)\leavevmode\tagmcend\tagstructend} functional regression model and is a commonly applied method for West Coast groundfish. The NMT also allows for the construction of a natural mortality prior weighted across methods by the user.

The STATs for the four vermilion rockfish assessment models all used the same prior for natural mortality across models. We assumed the age of 54 years to represent the practical longevity (i.e., 90\% of the commonly seen maximum age of 60) for both females and males, though the absolute oldest age in Oregon was \textgreater60 years. In California, fish aged at 80+ were encountered. Empirical {\tagstructbegin{tag=Formula}\tagmcbegin{tag=Formula}\(M\)\leavevmode\tagmcend\tagstructend} estimators using the von Bertalanffy growth parameters were also considered, but they produced unreasonably high estimates (2-3 times higher than the longevity estimates). This is likely explained by the fact that vermilion rockfish have protracted longevity at {\tagstructbegin{tag=Formula}\tagmcbegin{tag=Formula}\(L_{\infty}\)\leavevmode\tagmcend\tagstructend}. Additionally, the FishLife {\tagstructbegin{tag=Reference}\tagmcbegin{tag=Reference}(Thorson and Barnett 2017)\leavevmode\tagmcend\tagstructend} estimate was included, though, given the source of FishLife data is FishBase, there is a good chance the estimates of {\tagstructbegin{tag=Formula}\tagmcbegin{tag=Formula}\(M\)\leavevmode\tagmcend\tagstructend} are also from methods using longevity, though the actual source of longevity in FishLife was unknown. Both California vermilion rockfish assessments used the Hamel prior {\tagstructbegin{tag=Reference}\tagmcbegin{tag=Reference}(2015)\leavevmode\tagmcend\tagstructend}, which is defined as a lognormal with log-scale mean = {\tagstructbegin{tag=Formula}\tagmcbegin{tag=Formula}\(ln\frac{5.4}{A_{max}}\)\leavevmode\tagmcend\tagstructend} and SE = 0.438. Using a maximum age of 54 the point estimate and median of the prior is 0.1, which is used as a prior on {\tagstructbegin{tag=Formula}\tagmcbegin{tag=Formula}\(M\)\leavevmode\tagmcend\tagstructend} in the assessment model. We also explore sensitivity to these assumptions of natural mortality through likelihood profiling.

\tagstructbegin{tag=H3}\tagmcbegin{tag=H3}

\hypertarget{sex-ratio}{%
\subsubsection{Sex Ratio}\label{sex-ratio}}

\leavevmode\tagmcend\tagstructend

The sex ratio at birth was assumed to be 50:50 and plots of the sex ratio by year for data with sex-specific CAAL data are available in Figure \ref{fig:sexratio-ABRAMS-RESEARCH-1}, Figure \ref{fig:sexratio-SWFSC-GF-ECOL-2}, and Figure \ref{fig:sexratio-NWFSC-TWL-3} along with 75\% intervals calculated as Jeffreys intervals based on adjusted input sample sizes from Francis weighting {\tagstructbegin{tag=Reference}\tagmcbegin{tag=Reference}(Brown et al. 2001)\leavevmode\tagmcend\tagstructend}. The WCGBTS provided the majority of age data to the assessment and no clear patterns can be seen in the sex ratios. For years with fish larger than 50 cm, the sex ratio is skewed towards females, which is consistent with the observation that females grow larger than males, on average.

\tagstructbegin{tag=H3}\tagmcbegin{tag=H3}

\hypertarget{weight-length-relationship}{%
\subsubsection{Weight-Length Relationship}\label{weight-length-relationship}}

\leavevmode\tagmcend\tagstructend

In California, the weight(kg)-length(cm) relationship for vermilion rockfish was estimated external to the model using biological data available from fishery-independent data sources including the NWFSC hook-and-line survey and the WCGBTS. The estimated weight-length was assumed the same for males and females: {\tagstructbegin{tag=Formula}\tagmcbegin{tag=Formula}\(W\)\leavevmode\tagmcend\tagstructend}=1.744e-05{\tagstructbegin{tag=Formula}\tagmcbegin{tag=Formula}\(L\)\leavevmode\tagmcend\tagstructend}\textsuperscript{3} (Figure \ref{fig:weightlength}).

\tagstructbegin{tag=H3}\tagmcbegin{tag=H3}

\hypertarget{environmental-or-ecosystem-data}{%
\subsubsection{Environmental or Ecosystem Data}\label{environmental-or-ecosystem-data}}

\leavevmode\tagmcend\tagstructend

As noted in Section 1.3, ecosystem data were not explicitly used in this assessment.

\tagstructbegin{tag=H1}\tagmcbegin{tag=H1}

\hypertarget{assessment-model-description}{%
\section{Assessment Model Description}\label{assessment-model-description}}

\leavevmode\tagmcend\tagstructend

\tagstructbegin{tag=H2}\tagmcbegin{tag=H2}

\hypertarget{history-of-modeling-approaches}{%
\subsection{History of Modeling Approaches}\label{history-of-modeling-approaches}}

\leavevmode\tagmcend\tagstructend

Current yield estimates for vermilion rockfish were estimated for the entire West Coast using Depletion-Based Stock Reduction Analysis (DB-SRA) {\tagstructbegin{tag=Reference}\tagmcbegin{tag=Reference}(Dick and MacCall 2010)\leavevmode\tagmcend\tagstructend}. Average catch in 2008-2009 was 136.3 mt, and the median OFL in 2010 was 314.3 mt with a 28\% probability that recent catch exceeded the OFL in 2010 {\tagstructbegin{tag=Reference}\tagmcbegin{tag=Reference}(Dick and MacCall 2010)\leavevmode\tagmcend\tagstructend}.

A 2005 assessment was not accepted for management. From the September 2005 {\tagstructbegin{tag=Link}\tagmcbegin{tag=Link}\href{https://www.pcouncil.org/documents/2005/09/f-groundfish-management-september-2005.pdf/}{Briefing Book}\leavevmode\tagmcend\tagstructend}: ``The SSC considers the assessment to be best available science, but at this stage does not endorse the results as being suitable for setting OYs.'' A 2013 data moderate assessment was prepared, but not reviewed. From the Pacific Coast Groundfish Stock Assessment Review (STAR) Panel Report for Data-Moderate Assessments (2013): ``There was insufficient time during the review to evaluate all the assessments originally requested by the Council. Assessments for vermilion/sunset rockfish (\emph{Sebastes miniatus} and \emph{Sebastes crocotulus}) and yellowtail rockfish (south of $40^\circ 10^\prime N$) were not presented by the Stock Assessment Team (STAT).''

\tagstructbegin{tag=H3}\tagmcbegin{tag=H3}

\hypertarget{most-recent-star-panel-and-ssc-recommendations}{%
\subsubsection{Most Recent STAR Panel and SSC Recommendations}\label{most-recent-star-panel-and-ssc-recommendations}}

\leavevmode\tagmcend\tagstructend

The 2005 STAR panel report compiled recommendations specific to vermilion rockfish, and also generic rockfish recommendations The generic rockfish recommendation are not presented here. The 2005 assessment was not accepted for management by the PFMC.

\textbf{Vermilion Rockfish Recommendations}

Investigation into the species composition of nominal vermilion rockfish is needed. It is not clear that separate assessments for the northern and southern areas are warranted for vermilion rockfish. Although there were differences in the estimated magnitude and timing of recruitment events, the estimated stock trends were similar in both areas. Pooling of data from northern and southern areas may permit a more robust assessment model to be obtained.

\emph{2021 STAT Response.} Since the 2005 assessment, vermilion rockfish were speciated to vermilion and sunset rockfishes {\tagstructbegin{tag=Reference}\tagmcbegin{tag=Reference}(Hyde and Vetter 2009)\leavevmode\tagmcend\tagstructend}. Sunset rockfish are more common south of Pt. Conception ($34^\circ 27^\prime N$) and historical catches and length distributions between the two areas are different. The STAT discussed this at the Pre-Assessment Workshop and all participants agreed that modeling the areas separately was an appropriate decision.

\tagstructbegin{tag=H3}\tagmcbegin{tag=H3}

\hypertarget{response-to-star-panel-requests}{%
\subsubsection{Response to STAR Panel Requests}\label{response-to-star-panel-requests}}

\leavevmode\tagmcend\tagstructend

For the STAT responses to the STAR panel requests see the STAR panel report available on the PFMC's website.

\tagstructbegin{tag=H2}\tagmcbegin{tag=H2}

\hypertarget{model-specifications}{%
\subsection{Model Specifications}\label{model-specifications}}

\leavevmode\tagmcend\tagstructend

A decision was made by the STAT after discussions with the Pacific Fishery Management Council's Groundfish Management Team and Groundfish Advisory Panel to model the areas north and south of Point Conception independently for a number of reasons. These included a discussion of the evidence supporting higher densities of sunset rockfish south of Point Conception and the general decline in vermilion rockfish density as latitude increases. The preliminary exploration of length data also suggested that the size composition of landed fish north and south of Point Conception differed in a number of fleets. The STAT maintained consistency across the two models when the data supported the decisions, i.e., maintaining the same recreational and commercial fleet structures and sharing biological data from the more data-rich southern assessment.

The structure of the California models north and south of Point Conception are very similar. Population dynamics in both regions operate on an annual time step and are initialized from an unfished equilibrium condition in 1875. Sex-specific age and length structure is modeled from age 0 (recruitment age) to an accumulator age (plus group) of 70, with 1-cm population length bins ranging from 6-70 cm. Length data bins are 2-cm wide, and range from 8-70 cm in the south and 10-70 cm in the north. Expected recruitment is assumed to follow a Beverton-Holt function of spawning output, with lognormally-distributed recruitment deviations. Growth (male and female) is modeled using the Schnute parameterization of von Bertalanffy growth, with two estimated lengths (ages 0 and 30) and a growth rate coefficient (\emph{k}). The major differences between the two models are the availability of fishery-independent data sources that are region-specific, and the parameterization of male growth and mortality parameters (details below).

The models in both regions are conditioned on catches from the commercial and recreational sectors. The commercial sector is divided into three fleets (hook-and-line, trawl, and net gears). Landings from minor commercial gears were a negligible component of the total harvest and were combined with the hook-and-line fleet. The recreational sector was divided into four fleets according to boat mode (party/charter or private/rental) and catch type (retained or discarded). This follows the same practice as a number of other recent rockfish stock assessments, where the ability to accurately estimate a retention curve is complicated by depth-dependent discard mortality rates.

Vermilion rockfish is a desirable species and discards are a small component of total fishing mortality in both the commercial and recreational sectors. The commercial catches do not include dead discards, which were estimated to be a small percent of the overall landings in both areas (averaging 7.4 mt coastwide since 2015, although increasing since 2017). In addition, there were very few observations available from WCGOP (fewer than 250 fish statewide). The size distribution of recreational discards from the CDFW and Cal Poly onboard observer programs represented larger fish from periods when the recreational shelf rockfish fishery closed versus smaller fish discarded when the fishery was open. Fish discarded during trips when vermilion rockfish were prohibited were removed from the recreational PC discard fleet length composition.

The northern California model is fit to four fishery-dependent indices of relative abundance: 1) MRFSS CPFV dockside, 2) two onboard observer surveys, and 3) CRFS PR1 dockside. The MRFSS CPFV dockside index is assumed to be proportional to changes in the abundance of fish vulnerable to the recreational party/charter fleet (retained fish only). The onboard observer indices represent the same fleet (rec party/charter), but indices change in abundance during recent years. The onboard indices are specified as separate ``survey'' fleets in the model because they overlaps in time with the MRFSS dockside time series. Both the MRFSS and onboard indices use the recreational party/charter fleet's selectivity curve to define vulnerable size classes. The CRFS PR1 dockside index is linked to the recreational private/rental fleet (retained fish), and uses the same selectivity curve. Recreational length measurements are included as marginal length compositions (proportions at length, sexes combined) by year starting in 1980 for both the PC and PR modes. Fishery-dependent length composition data are also included for the three commercial fleets. Age structures from the commercial fleets were also sparse and not considered for the northern California assessment.

Fishery-independent data sources in the northern California model are organized into four fleets. Data from the CCFRP survey were used to create an index of relative abundance and marginal length compositions by year. The WCGBTS is the second fishery-independent data source in the northern model. An abundance index was developed for the WCGBT survey, but ultimately rejected due to high interannual variability, sparse data, and imprecise estimates. However, trawl survey conditional-age-at-length data and associated marginal length comps, both by sex and year, were retained in the model. Additional age and length composition data came from the SWFSC groundfish ecology cruises and Jeff Abrams thesis collections (see data section for additional information).

\textbf{Changes from the pre-STAR base model to the post-STAR base model}\\
During the STAR panel two changes to the base model were agreed STAT proposed a modifications to the pre-STAR base that included 1) a time block on the CCFRP index in 2017 after the survey was expanded from central California to the entire California coast, as explored in Request 1 of the STAR panel report, 2) CCFRP length compositions re-weighted to reflect the weighting used in the index, and 3) removal of 2020 from the PR dockside index of abundance due to sampling constraints during COVID. The final base model was approved by both the STAT and the Panel.

\tagstructbegin{tag=H3}\tagmcbegin{tag=H3}

\hypertarget{additional-specifications}{%
\subsubsection{Additional Specifications}\label{additional-specifications}}

\leavevmode\tagmcend\tagstructend

Selectivity was specified using the double normal parameterization within Stock Synthesis for all fleets. Selectivity parameters were estimated for the commercial hook-and-line fleet, commercial trawl fleet, and the commercial net fleet, as well as for the recreational PC fleet, recreational PC discard fleet and the recreational PR fleet. There were no length data available for the recreational PR discard fleet, and it mirrors the selectivity of the recreational PC discard fleet. Selectivity for the recreational PC onboard index of abundance is mirrored to the recreational PC fleet as they share the same length composition. The Abrams dataset was too sparse to estimate its own selectivity parameters and was only collected over a two year period. The Abrams research selectivity mirrors the commercial hook-and-line fleet. The STAT explored mirroring this data set to the recreational PC fleet, but the length composition was more representative of the commercial fleet. Hook-and-line gear was the dominant collection method for the SWFSC groundfish ecology survey and is mirrored to the commercial hook-and-line fleet.

Selectivity parameters were estimated for the CCFRP fleet. Note that the post-STAR base model includes a time block on selectivity for CCFRP in 2017 when the survey expanded statewide.

The length composition sample sizes for some years and fleets was small, and observations may not be representative of the total catch. Years with insufficient data were excluded from the likelihood, and initial sample sizes (prior to data weighting) for length composition data were set equal to a proxy such as the number or trips, hauls, or sampling events (as described in Tables \ref{tab:length-sample-size} and \ref{tab:length-inputN}).

\tagstructbegin{tag=H3}\tagmcbegin{tag=H3}

\hypertarget{modeling-platform-and-structure}{%
\subsubsection{Modeling Platform and Structure}\label{modeling-platform-and-structure}}

\leavevmode\tagmcend\tagstructend

The assessment was conducted using Stock Synthesis (SS) version 3.30.17.00 developed by Dr.~Richard Methot {\tagstructbegin{tag=Reference}\tagmcbegin{tag=Reference}(Methot and Wetzel 2013)\leavevmode\tagmcend\tagstructend}. The R package {\tagstructbegin{tag=Link}\tagmcbegin{tag=Link}\href{https://github.com/r4ss/r4ss}{r4ss}\leavevmode\tagmcend\tagstructend}, version 1.38.0, along with R version 4.0.1 were used to investigate and plot model fits.

Electronic SS model input files including the data, control, starter, and forecast files can be found on the {\tagstructbegin{tag=Link}\tagmcbegin{tag=Link}\href{https://www.pcouncil.org/groundfish/stock-assessments/}{PFMC's website}\leavevmode\tagmcend\tagstructend}.

\tagstructbegin{tag=H3}\tagmcbegin{tag=H3}

\hypertarget{model-parameters}{%
\subsubsection{Model Parameters}\label{model-parameters}}

\leavevmode\tagmcend\tagstructend

The population dynamics model has many parameters, some estimated using the available data in the assessment and some fixed at values either determined external to the assessment or informed by the available data. Estimated and fixed parameter values, including associated properties (bounds, priors, asymptotic standard errors), are in Table \ref{tab:paramstable}.

A total of 118 parameters were estimated in the base model, including recruitment deviations. Time-invariant growth parameters (Brody growth coefficient, lengths at age 0 and age 30, and CV old/young) using the Schnute parameterization of the von Bertalanffy growth function were estimated for each sex. The CV of the distribution of length-at-age, CV(L), in the base model is estimated at the lower and upper ages specified in the Schnute parameterization of von Bertalanffy growth, and a linear interpolation between these 2 parameters is a function of age. This choice was based on visual inspection of the relationship between CV(L) and age, by sex, using the NWFSC hook-and-line survey data (Figure \ref{fig:cv-length-age}). Natural mortality was estimated for both females and males, and informed by a prior distribution. Selectivity varied by fleet, and was assumed to be either asymptotic or domed for retained fleets, and forced to be domed for discard fleets with initial and final selectivity fixed at zero. Most selectivity parameters were assumed to be time-invariant, except time blocks were used to capture changes in selectivity associated with regulatory changes in 2002 and 2017 (see regulations section). Recruitment deviates were estimated in the base model from 1970 -- 2020. Initial (unfished equilibrium) recruitment was also estimated. An extra standard deviations were estimated for the PR mode abundance index, as the externally estimated CVs were small due to extremely large sample sizes (1000s of trips).

\tagstructbegin{tag=H3}\tagmcbegin{tag=H3}

\hypertarget{priors}{%
\subsubsection{Priors}\label{priors}}

\leavevmode\tagmcend\tagstructend

The Thorson-Dorn rockfish prior (developed for use West Coast rockfish assessments) conducted by James Thorson (personal communication, NWFSC, NOAA) and reviewed and endorsed by the Scientific and Statistical Committee (SSC) in 2017, has been a primary source of information on steepness for rockfish. This approach, however, was subsequently rejected for future analysis in 2019 when the new meta-analysis resulted in a mean value of approximately 0.95. In the absence of a new method for generating a prior for steepness the default approach reverts to the previously endorsed method, the 2017 prior for steepness ({\tagstructbegin{tag=Formula}\tagmcbegin{tag=Formula}\(h\)\leavevmode\tagmcend\tagstructend}; beta distribution with {\tagstructbegin{tag=Formula}\tagmcbegin{tag=Formula}\(\mu\)\leavevmode\tagmcend\tagstructend}=0.72 and {\tagstructbegin{tag=Formula}\tagmcbegin{tag=Formula}\(\sigma\)\leavevmode\tagmcend\tagstructend}=0.16) is retained.

A prior for natural mortality was developed using the method of Hamel {\tagstructbegin{tag=Reference}\tagmcbegin{tag=Reference}(2015)\leavevmode\tagmcend\tagstructend}. The STAT examined the distribution of ages from the NWFSC hook-and-line survey and found that roughly 99.9\% of otoliths aged were in the mid-50s or younger. Therefore an approximate maximum age of 54 was selected, giving a median estimate of 0.1 {\tagstructbegin{tag=Formula}\tagmcbegin{tag=Formula}\(yr^{-1}\)\leavevmode\tagmcend\tagstructend} for the prior. The STAT notes that the recommended log-scale standard deviation of 0.438 for the prior makes it only weakly informative, so small changes to the prior's median value do not affect estimates of \emph{M} and other assessment results.

\tagstructbegin{tag=H3}\tagmcbegin{tag=H3}

\hypertarget{data-weighting}{%
\subsubsection{Data Weighting}\label{data-weighting}}

\leavevmode\tagmcend\tagstructend

Length composition and conditional-age-at-length (CAAL) composition sample sizes for the base model were tuned by the ``Francis method,'' based on equation TA1.8 in Francis {\tagstructbegin{tag=Reference}\tagmcbegin{tag=Reference}(2011)\leavevmode\tagmcend\tagstructend}, and implemented in the r4ss package (Table \ref{tab:data-weights}).

As outlined in the Best Practices, a sensitivity run was conducted with length and conditional-age-at-length (CAAL) compositions were re-weighted using the McAllister-Ianelli harmonic mean method {\tagstructbegin{tag=Reference}\tagmcbegin{tag=Reference}(McAllister, Murdoch K.; Ianelli 1997)\leavevmode\tagmcend\tagstructend}. See the model sensitivity section for a comparison of the Francis and McAllister-Ianelli results. Additionally, weighting using the Dirichlet-Multinomial likelihood, that includes and estimable parameter (theta) that scales the input sample size, was explored. However, all estimates of the ratio of {\tagstructbegin{tag=Formula}\tagmcbegin{tag=Formula}\(\theta/(1+\theta)\)\leavevmode\tagmcend\tagstructend} were greater than 0.99, which indicates the models is trying to tune the sample sizes unchanged. Given this result, the STAT chose not to further explore the Dirichlet-Multinomial data weighting. As a note, there is a bug in SS Version 3.30.16.00 that prevents the number of estimated weights from being larger than the number of fleets. This was fixed in SS Version 3.30.16.01 and this version was only used for exploration of the Dirichlet-Multinomial data weighting.

\tagstructbegin{tag=H3}\tagmcbegin{tag=H3}

\hypertarget{key-assumptions-and-structural-choices}{%
\subsubsection{Key Assumptions and Structural Choices}\label{key-assumptions-and-structural-choices}}

\leavevmode\tagmcend\tagstructend

The STAT used sensitivity analyses to evaluate robustness of the pre-STAR base models to key assumptions and structural choices. The major structural choices in both California assessments were 1) the use of a single, stationary, and closed population model to describe the aggregate population dynamics and biological parameters of the cryptic species pair in each region, 2) density-dependence entirely characterized by a Beverton-Holt stock recruitment relationship, 3) that natural mortality rates can be adequately estimated from available data, and 4) time blocks based on major regulatory changes adequately characterize changes in size-selectivity of fishing gear over time. The catch histories of vermilion and sunset rockfishes are inseparable at this time, making estimation of species-specific fishing mortality rates impossible. Ongoing research may shed light on this issue, and help improve our understanding of potential differences between the species (e.g., vital rates) that could influence estimates of stock productivity and sustainable yield.

\tagstructbegin{tag=H3}\tagmcbegin{tag=H3}

\hypertarget{convergence}{%
\subsubsection{Convergence}\label{convergence}}

\leavevmode\tagmcend\tagstructend

Model convergence was examined by starting the minimization algorithm from dispersed values of the maximum likelihood estimates to determine if the model found a better minimum. ``Jitter'' is an option in SS that generates random starting values from a normal distribution logistically transformed into each parameter's range {\tagstructbegin{tag=Reference}\tagmcbegin{tag=Reference}(Methot, R. D. et al. 2020)\leavevmode\tagmcend\tagstructend}. This was repeated 100 times and none of the runs converged to a lower negative log likelihood in the post-STAR base model (Figure \ref{fig:jitter}). The model did not experience convergence issues, e.g., final gradient was below 0.0001, when reasonable starting values were used and there were no difficulties in inverting the Hessian to obtain estimates of variability.

\tagstructbegin{tag=H1}\tagmcbegin{tag=H1}

\hypertarget{assessment-results}{%
\section{Assessment Results}\label{assessment-results}}

\leavevmode\tagmcend\tagstructend

The base model parameter estimates along with approximate asymptotic standard errors are shown in Table \ref{tab:paramstable}. The full r4ss plotting output is available in the supplementary material on the Council's website.

\tagstructbegin{tag=H2}\tagmcbegin{tag=H2}

\hypertarget{fixed-parameters}{%
\subsection{Fixed parameters}\label{fixed-parameters}}

\leavevmode\tagmcend\tagstructend

The following parameters were fixed in the post-STAR base model:\\
* {\tagstructbegin{tag=Formula}\tagmcbegin{tag=Formula}\(h\)\leavevmode\tagmcend\tagstructend} controlling the steepness of the stock-recruit relationship * Selectivity parameters estimated at the bounds during model exploration

\tagstructbegin{tag=H2}\tagmcbegin{tag=H2}

\hypertarget{parameter-estimates}{%
\subsection{Parameter Estimates}\label{parameter-estimates}}

\leavevmode\tagmcend\tagstructend

The base model has a total of 118 estimated parameters (Table \ref{tab:paramstable}) that are described in more detail in the following sections:

\tagstructbegin{tag=H2}\tagmcbegin{tag=H2}

\hypertarget{growth-estimation}{%
\subsection{Growth Estimation}\label{growth-estimation}}

\leavevmode\tagmcend\tagstructend

The northern California base model estimated reasonable growth parameters for {\tagstructbegin{tag=Formula}\tagmcbegin{tag=Formula}\(k\)\leavevmode\tagmcend\tagstructend} and lengths at age 0 and age 30. Internal estimates of growth were estimated directly for both females and males, and not as an offset. An offset for male growth was explored, but the CV of length at {\tagstructbegin{tag=Formula}\tagmcbegin{tag=Formula}\(L_{age=30}\)\leavevmode\tagmcend\tagstructend} was estimated around 2-3\%. When the male CV was fixed to equal the female CV, the assessment model shrunk the CV of females at {\tagstructbegin{tag=Formula}\tagmcbegin{tag=Formula}\(L_{age=30}\)\leavevmode\tagmcend\tagstructend}, which the STAT did not find reasonable. Therefore, the male CV at {\tagstructbegin{tag=Formula}\tagmcbegin{tag=Formula}\(L_{age=30}\)\leavevmode\tagmcend\tagstructend} was fixed at the CV estimated for females of 0.07 (Figure \ref{fig:fittedgrowth}).

The direct estimation of male {\tagstructbegin{tag=Formula}\tagmcbegin{tag=Formula}\(L_{age=0}=6.02\)\leavevmode\tagmcend\tagstructend} cm was reasonable compared to female {\tagstructbegin{tag=Formula}\tagmcbegin{tag=Formula}\(L_{age=0}=7.8\)\leavevmode\tagmcend\tagstructend}. While {\tagstructbegin{tag=Formula}\tagmcbegin{tag=Formula}\(k\)\leavevmode\tagmcend\tagstructend} was estimated larger for males (0.19) than females (0.15), female {\tagstructbegin{tag=Formula}\tagmcbegin{tag=Formula}\(L_{age=30}\)\leavevmode\tagmcend\tagstructend} of 55 cm was larger than males at 49 cm. These results are consistent with other studies that have looked at sex-specific growth in vermilion rockfish.

Estimates of the vonBertalanffy parameters transformed from the Schnute parameterization used by SS are below. In both parameterizations of the growth equation, the {\tagstructbegin{tag=Formula}\tagmcbegin{tag=Formula}\(k\)\leavevmode\tagmcend\tagstructend} parameter has the same definition.

\begin{centering}

Females $L_{\infty}$ = 55.8 cm; $k$ = 0.147; $t_0$ = -0.99

Males $L_{\infty}$ = 50 cm; $k$ = 0.199; $t_0$ = -0.63

\end{centering}

\vspace{0.5cm}

\tagstructbegin{tag=H2}\tagmcbegin{tag=H2}

\hypertarget{natural-mortality-estimation}{%
\subsection{Natural Mortality Estimation}\label{natural-mortality-estimation}}

\leavevmode\tagmcend\tagstructend

The northern California model directly estimated male growth and natural mortality (\emph{M}) parameters due to the above-mentioned issue with estimation of CVs for length at age 30 when using the offset parameterization. Female natural mortality was estimated at 0.09 (SE = 0.0083) and male natural mortality was estimated at 0.08 (SE = 0.0088), which the STAT considered reasonable given that observed maximum ages for both sexes are similar. Latitudinal gradients in natural mortality have been estimated for many species of rockfish, which is consistent with the higher estimates of {\tagstructbegin{tag=Formula}\tagmcbegin{tag=Formula}\(M\)\leavevmode\tagmcend\tagstructend} in the southern model.

\tagstructbegin{tag=H2}\tagmcbegin{tag=H2}

\hypertarget{fits-to-age-composition}{%
\subsection{Fits to Age Composition}\label{fits-to-age-composition}}

\leavevmode\tagmcend\tagstructend

The following plots show fits to the conditional age-at-length composition for each fleet/survey:\\
- Abrams research survey: Figures \ref{fig:comp_condAALfit_residsflt11mkt0}, \ref{fig:comp_condAALfit_data_weighting_TA1.8_condAgeABRAMS_RESEARCH} and \ref{fig:comp_condAALfit_Andre_plotsflt11mkt0}\\
- WCGBT survey: Figures \ref{fig:comp_condAALfit_residsflt9mkt0_page1} - \ref{fig:comp_condAALfit_residsflt9mkt0_page3}, \ref{fig:comp_condAALfit_data_weighting_TA1.8_condAgeNWFSC_TWL}, and \ref{fig:comp_condAALfit_Andre_plotsflt9mkt0_page1} - \ref{fig:comp_condAALfit_Andre_plotsflt9mkt0_page4}\\
- SWFSC groundfish ecology survey: Figures \ref{fig:comp_condAALfit_residsflt12mkt0}, \ref{fig:comp_condAALfit_data_weighting_TA1.8_condAgeSWFSC_GF_ECOL} and \ref{fig:comp_condAALfit_Andre_plotsflt12mkt0}

Fits to the conditional age-at-length data sets seemed reasonable, with no evidence of strong residual patterns. The largest residuals were mainly associated with the infrequently encountered, oldest individuals. The model was able to reproduce interannual changes in mean age for the NWFSC trawl and Abrams thesis research data, but tended to slightly overestimate average age (by about 1 year) in three of the four years for the SWFSC groundfish ecology data set.

\tagstructbegin{tag=H2}\tagmcbegin{tag=H2}

\hypertarget{estimated-selectivity-and-fits-to-length-composition}{%
\subsection{Estimated Selectivity and Fits to Length Composition}\label{estimated-selectivity-and-fits-to-length-composition}}

\leavevmode\tagmcend\tagstructend

Fits to the time-aggregated length comps were best for the commercial, recreational, and CCFRP fleets Figure (\ref{fig:lenfits-all}). The following plots show estimated selectivity (when not mirrored, Figures \ref{fig:selex-age-all} - \ref{fig:endyr-selex-CCFRP}) and fits to the length composition (Figures \ref{fig:lenfits-all} - \ref{fig:mean-len-fit-CCFRP}) for each fleet/survey:

\tagstructbegin{tag=L}

\begin{itemize}
\item
  \tagstructbegin{tag=LI}\tagstructbegin{tag=LBody}\tagmcbegin{tag=P}

  Commercial hook-and-line: Figures \ref{fig:endyr-selex-COM-HKL}, \ref{fig:len-pearson-COM-HKL}, and \ref{fig:mean-len-fit-COM-HKL}

  \tagmcend\tagstructend\tagstructend
\item
  \tagstructbegin{tag=LI}\tagstructbegin{tag=LBody}\tagmcbegin{tag=P}

  Commercial trawl: Figures \ref{fig:endyr-selex-COM-TWL}, \ref{fig:len-pearson-COM-TWL}, and \ref{fig:mean-len-fit-COM-TWL}

  \tagmcend\tagstructend\tagstructend
\item
  \tagstructbegin{tag=LI}\tagstructbegin{tag=LBody}\tagmcbegin{tag=P}

  Commercial net: Figures \ref{fig:endyr-selex-COM-NET}, \ref{fig:len-pearson-COM-NET}, and \ref{fig:mean-len-fit-COM-NET}

  \tagmcend\tagstructend\tagstructend
\item
  \tagstructbegin{tag=LI}\tagstructbegin{tag=LBody}\tagmcbegin{tag=P}

  Recreational retained PC: Figures \ref{fig:sel03_len_timevary_surf_flt4sex1}, \ref{fig:endyr-selex-REC-PC}, \ref{fig:len-pearson-REC-PC}, and \ref{fig:mean-len-fit-REC-PC}

  \tagmcend\tagstructend\tagstructend
\item
  \tagstructbegin{tag=LI}\tagstructbegin{tag=LBody}\tagmcbegin{tag=P}

  Recreational discard PC: Figures \ref{fig:endyr-selex-REC-PC-DIS}, \ref{fig:len-pearson-REC-PC-DIS} and \ref{fig:mean-len-fit-REC-PC-DIS}

  \tagmcend\tagstructend\tagstructend
\item
  \tagstructbegin{tag=LI}\tagstructbegin{tag=LBody}\tagmcbegin{tag=P}

  Recreational retained PR: Figures \ref{fig:sel03_len_timevary_surf_flt6sex1}, \ref{fig:endyr-selex-REC-PR}, \ref{fig:len-pearson-REC-PR} and \ref{fig:mean-len-fit-REC-PR}

  \tagmcend\tagstructend\tagstructend
\item
  \tagstructbegin{tag=LI}\tagstructbegin{tag=LBody}\tagmcbegin{tag=P}

  Deb Wilson-Vandenberg onboard CPFV survey: Figure \ref{fig:len-pearson-DWV-ONBOARD} and \ref{fig:mean-len-fit-DWV-ONBOARD}

  \tagmcend\tagstructend\tagstructend
\item
  \tagstructbegin{tag=LI}\tagstructbegin{tag=LBody}\tagmcbegin{tag=P}

  WCGBT survey: Figures \ref{fig:endyr-selex-NWFSC-TWL}, \ref{fig:endyr-selex-NWFSC-TWL}, \ref{fig:len-pearson-NWFSC-TWL} and \ref{fig:mean-len-fit-NWFSC-TWL}

  \tagmcend\tagstructend\tagstructend
\item
  \tagstructbegin{tag=LI}\tagstructbegin{tag=LBody}\tagmcbegin{tag=P}

  Abrams research survey: Figures \ref{fig:len-pearson-ABRAMS-RESEARCH} and \ref{fig:mean-len-fit-ABRAMS-RESEARCH}

  \tagmcend\tagstructend\tagstructend
\item
  \tagstructbegin{tag=LI}\tagstructbegin{tag=LBody}\tagmcbegin{tag=P}

  SWFSC groundfish ecology survey: Figures \ref{fig:len-pearson-SWFSC-GF-ECOL} and \ref{fig:mean-len-fit-SWFSC-GF-ECOL}

  \tagmcend\tagstructend\tagstructend
\item
  \tagstructbegin{tag=LI}\tagstructbegin{tag=LBody}\tagmcbegin{tag=P}

  CCFRP: Figures \ref{fig:endyr-selex-CCFRP}, \ref{fig:len-pearson-CCFRP} and Figure \ref{fig:mean-len-fit-CCFRP}

  \tagmcend\tagstructend\tagstructend
\end{itemize}

\tagstructend

The WCGBTS does not sample primary adult habitat types, resulting in a length frequency distribution that appears bimodal and is difficult for the model to reproduce (Figure \ref{fig:lenfits-all}), and also reflected in the descending limb of the selectivity around 0.42 for any fish larger than 20 cm (Figure \ref{fig:endyr-selex-NWFSC-TWL}). Fits to the two short-term research fleets (SWFSC groundfish ecology surveys and Abrams thesis research) are adequate, but ultimately based on mirrored selectivity curves (details in selectivity section, below). Fits to length composition data from the recreational fleets show evidence of modal progressions due to strong year classes (e.g., the 1984-1985 year classes entering the fishery a few years later), do not show evidence of strong patterns in the pearson residual plots, and the model is able to track associated changes in mean length over time.

The Abrams research fleet and SWFSC groundfish ecology selectivity curves were mirrored to the commercial hook-and-line fleet, which was fixed to an asymptotic selectivity (Figure \ref{fig:endyr-selex-COM-HKL}). Initial estimation of the commercial trawl selectivity resulted in large terminal estimates with large standard errors. Therefore, it was fixed to be asymptotic and the peak size and the ascending width were estimated (Figure \ref{fig:endyr-selex-COM-TWL}). The historic net fishery selectivity was estimated with four parameters including a descending limb; fish were fully selected by around 43 cm and by around 47 cm selectivity decreased to 0.6 (Figure \ref{fig:endyr-selex-COM-NET}).

All three selectivities for recreational fleets were estimated as domed selectivities (Figures \ref{fig:endyr-selex-REC-PC}, \ref{fig:endyr-selex-REC-PC-DIS}, \ref{fig:endyr-selex-REC-PR}). A number of regulatory changes in the recreational fleets prompted selectivity time blocks, and both recreational retained fleets estimated a large length at full selectivity in the later time block, from 2017-2020, which is consistent with relaxation of depth restrictions. Peak selectivity of the discard fleet is around 20 cm, and once fish reach 40 cm, they are no longer selected by the discard fleet. There is no size limit on vermilion rockfish and this is assumed to represent angler preference.

The CCFRP fishes within 150 ft of water to reduce barotrauma-induced mortality and does not observe the larger fish that are seen in the recreational fleets fishing in deeper waters (especially once depth restrictions were relaxed in 2017) and further offshore (Figure \ref{fig:endyr-selex-CCFRP}). The estimated length composition fits vary by year, with the early years expecting larger fish than were observed and the more recent years (2015-2020) expecting smaller fish, likely from the larger 2013-2014 year classes.

\tagstructbegin{tag=H2}\tagmcbegin{tag=H2}

\hypertarget{fits-to-indices}{%
\subsection{Fits to Indices}\label{fits-to-indices}}

\leavevmode\tagmcend\tagstructend

The following plots show log-scale fits to the indices and residuals by fleet/survey:

\tagstructbegin{tag=L}

\begin{itemize}
\item
  \tagstructbegin{tag=LI}\tagstructbegin{tag=LBody}\tagmcbegin{tag=P}

  MRFSS dockside PC survey: Figures \ref{fig:log-cpue-REC-PC} and \ref{fig:cpue-resid-REC-PC}

  \tagmcend\tagstructend\tagstructend
\item
  \tagstructbegin{tag=LI}\tagstructbegin{tag=LBody}\tagmcbegin{tag=P}

  CDFW dockside PR survey: Figures \ref{fig:log-cpue-REC-PR} and \ref{fig:cpue-resid-REC-PR}

  \tagmcend\tagstructend\tagstructend
\item
  \tagstructbegin{tag=LI}\tagstructbegin{tag=LBody}\tagmcbegin{tag=P}

  Deb Wilson-Vandenberg onboard CPFV survey: Figures \ref{fig:log-cpue-DWV-ONBOARD} and \ref{fig:cpue-resid-DWV-ONBOARD}\\

  \tagmcend\tagstructend\tagstructend
\item
  \tagstructbegin{tag=LI}\tagstructbegin{tag=LBody}\tagmcbegin{tag=P}

  CDFW/Cal Poly onboard CPFV survey: Figures \ref{fig:log-cpue-REC-PC-ONBOARD} and \ref{fig:cpue-resid-REC-PC-ONBOARD}

  \tagmcend\tagstructend\tagstructend
\item
  \tagstructbegin{tag=LI}\tagstructbegin{tag=LBody}\tagmcbegin{tag=P}

  CCFRP survey: Figures \ref{fig:log-cpue-CCFRP} and \ref{fig:cpue-resid-CCFRP}

  \tagmcend\tagstructend\tagstructend
\end{itemize}

\tagstructend

Fits to the indices vary in quality. Three of the four recreational indices represented the PC fleet, covering the years 1980-2019. The MRFSS era dockside interview index was fit reasonably well except for the first few years of the index (Figures \ref{fig:log-cpue-REC-PC}). The Deb Wilson-Vandenberg onboard survey from 1988-1998 was fairly flat and uninformative, and the model was not able to reproduce a spike in 1990 when an increased CPUE was estimated in the standardized index (Figures \ref{fig:log-cpue-DWV-ONBOARD} and \ref{fig:cpue-resid-DWV-ONBOARD}). However, it is unclear whether the increase in CPUE for one year is an artifact of the data or represents an actual short-term increase in abundance. The CDFW and Cal Poly onboard index, which now contains 21 years of data provided an uninformative fit to the data. Both the beginning (1999-2001) and ending (2017-2019) years represent time periods when the fishery had access to deeper water. Even with selectivity time blocks for these periods, the index was not fit to the decrease in observed CPUE from 1999-2001 or the slight increase from 2017-2019. There is some pattern in the residuals with groups of alternating positive and negative years (Figures \ref{fig:log-cpue-REC-PC-ONBOARD} and \ref{fig:cpue-resid-REC-PC-ONBOARD}). The recreational PR index was not well fit, even with additional error added within the model and residual patterns also indicate a poor fit (Figures \ref{fig:log-cpue-REC-PR} and \ref{fig:cpue-resid-REC-PR}). Residuals for the first part of the survey are positive and negative for the second half of the survey. The only fishery-independent index, CCFRP, fit reasonably well to the increasing trend from 2016-2020 (Figures \ref{fig:log-cpue-CCFRP} and \ref{fig:cpue-resid-CCFRP}). All of the indices (recreational and CCFRP) indicated an increasing trend from 2008 to 2010 and then a decrease with lows in all indices in 2013 that was not fit in any of the indices. CCFRP is the only index sampling within the MPAs, and starting in 2017 the index represents the entire coast north of Point Conception.

\tagstructbegin{tag=H2}\tagmcbegin{tag=H2}

\hypertarget{derived-quantities}{%
\subsection{Derived Quantities}\label{derived-quantities}}

\leavevmode\tagmcend\tagstructend

Spawning output north of Point Conception declined rapidly throughout the 1970s, 1980s, and 1990s to a level below the Minimum Stock Size Threshold (MSST), but catches decreased enough in the late 1990s and 2000s for the stock to reach a stable level of spawning output (Table \ref{tab:timeseries}, Figure \ref{fig:depl}). Stock size is estimated to have been at the lowest level during the 1990s, but has since increased, in part due to strong recruitments in 1999 and 2016. The stock is estimated to have been below the management target of B40\% since the early 1980s (Figure \ref{fig:depl}), recently returning to levels near the target biomass. Relative exploitation rates ({\tagstructbegin{tag=Formula}\tagmcbegin{tag=Formula}\(\frac{1-SPR}{1-SPR_{50\%}}\)\leavevmode\tagmcend\tagstructend}) increased through time, exceeding target levels from the 1970s through the 1990s. Exploitation over the past decade has fluctuated around target levels (Figure \ref{fig:1-spr}), with most catches landed by the recreational sector.

Vermilion rockfish spawning output in northern California was estimated to be 4489 million eggs in 2021 (95\% asymptotic interval: 263 - 716 million eggs) or 43\% (95\% asymptotic interval: 25\% - 61\%) of unfished spawning output in 2021(``depletion,'' Table \ref{tab:ssbES}) and Figure \ref{fig:depl}). In 2021, vermilion rockfish biomass north of Point Conception is estimated to be near the target biomass level, while experiencing fishing intensity around the SPR fishing intensity target (Figure \ref{fig:phase}). The equilibrium yield curve is shifted left, as expected from the fixed Beverton-Holt steepness parameter h=0.72 (Figures \ref{fig:yield2} and \ref{fig:yield3}). Harvest rates in northern California were near target in 2020, but above target in the three previous years (Figure \ref{fig:phase}).

\tagstructbegin{tag=H2}\tagmcbegin{tag=H2}

\hypertarget{recruitment-deviations}{%
\subsection{Recruitment Deviations}\label{recruitment-deviations}}

\leavevmode\tagmcend\tagstructend

Model estimates of recruitment for the 1970-2018 period indicated an extended period of below-average recruitment throughout the 1970s and early 1980s, which is generally inconsistent with strong recruitment trends for many years in the 1970s and in 1980 for other species of rockfish. As this time period is only weakly informed by length data, this may indicate some potential for model misspecification. Major recruitments (strong year classes) estimated in the northern California model include strong 1984, 1985, 1999, and 2016 year classes (Figures \ref{fig:bh-curve}, \ref{fig:bh-resids}, \ref{fig:recruits}, and \ref{fig:recdevs}). These are consistent with estimates of strong year classes in other rockfish stock assessments, nearly all of which suggest very high recruitment in 1999. More recently, strong year classes have been estimated in 2016 for widow rockfish, sablefish and Pacific hake, observations that are consistent with observations of high juvenile rockfish abundance in the California Current during the 2015-2016 large marine heatwave {\tagstructbegin{tag=Reference}\tagmcbegin{tag=Reference}(Schroeder et al. 2019)\leavevmode\tagmcend\tagstructend}. Due to ageing error, years adjacent to strong (or weak) cohorts are sometimes estimated as having similar deviations.

\tagstructbegin{tag=H2}\tagmcbegin{tag=H2}

\hypertarget{reference-points-1}\)\leavevmode\tagmcend\tagstructend} reference harvest rate. The spawning output equivalent to 40\% of the unfished level ({\tagstructbegin{tag=Formula}\tagmcbegin{tag=Formula}\(SB_{40\%}\)\leavevmode\tagmcend\tagstructend}) was 458 million eggs.

The 2020 spawning biomass relative to unfished equilibrium spawning biomass is just below the target of 40\% of unfished levels (Figure \ref{fig:depl}). The relative fishing intensity, {\tagstructbegin{tag=Formula}\tagmcbegin{tag=Formula}\((1-SPR)/(1-SPR_{50\%})\)\leavevmode\tagmcend\tagstructend}, was near the management target in 2020, and has fluctuated around the target level for the past decade (Figure \ref{fig:1-spr} and \ref{fig:phase}).

Table \ref{tab:referenceES} shows the full suite of estimated reference points for the base model and Figures \ref{fig:yield2} and \ref{fig:yield3} show the equilibrium yield curve and net production based on a steepness value fixed at 0.72.

\tagstructbegin{tag=H1}\tagmcbegin{tag=H1}

\hypertarget{assessment-model-diagnostics}{%
\section{Assessment Model Diagnostics}\label{assessment-model-diagnostics}}

\leavevmode\tagmcend\tagstructend

\tagstructbegin{tag=H2}\tagmcbegin{tag=H2}

\hypertarget{sensitivity-to-assumptions-data-and-weighting}{%
\subsection{Sensitivity to Assumptions, Data, and Weighting}\label{sensitivity-to-assumptions-data-and-weighting}}

\leavevmode\tagmcend\tagstructend

All sensitivities in this section use the \textbf{pre-STAR} base model.

To better understand how data from individual fishery sectors or scientific surveys affected assessment results, we excluded data sets from the likelihood, one fleet at a time (referred to here as a ``drop-one'' analysis). ``Fleet'' in this sense refers to either a fishing fleet or a survey ``fleet.'' To do this, we set ``lambdas'' (multipliers for each likelihood component) equal to zero. This is equivalent to removing the data from the model. When composition data were excluded, the selectivity parameters for that fleet were fixed at the base model estimates to standardize the size and age composition of harvested fish. When abundance indices were excluded, relevant catchability and `extraSE' parameters associated with the index were not estimated. Composition data weights for the remaining fleets were kept consistent with the base model values. Results from all the `drop-one' runs were compared to the base model using time series plots and tables containing likelihood components, parameter estimates and derived quantities.

Drop-one analysis of the northern California assessment revealed slightly larger variability in spawning output trends relative to the south, but all runs were still within the range of uncertainty estimated by the base model (Figure \ref{fig:drop-spawnbio}). Removal of most fleets had little effect on terminal stock status, with best estimates in the vicinity of target biomass levels (Figure \ref{fig:drop-bratio}). An exception was removal of the REC\_PC fleet, which caused the best estimate of terminal depletion to drop just above the minimum stock size threshold. This suggests that the REC\_PC data sets, together, favor a less-depleted stock relative to data from the other fleets. The strength of the 2016 year class is sensitive to the removal of fleet-specific data sets (Figure \ref{fig:drop-recdev}). Removal of the REC\_PR fleet produces the largest estimates of 2016 cohort size, and removal of the NWFSC\_TWL fleet estimates a 2016 deviation that is less than half as large (but still positive). Uncertainty in the strength of this recent year class should be taken into consideration for short-term forecasts of stock abundance and yield. Changes in likelihoods, parameter estimates and derived quantities are recorded in Table \ref{tab:drop-one}. Comparison of likelihoods among drop-one scenarios should be treated with caution due to changes in the data sets that were fit in each model run.

\tagstructbegin{tag=H3}\tagmcbegin{tag=H3}

\hypertarget{sensitivity-to-catch-uncertainty}{%
\subsubsection{Sensitivity to Catch Uncertainty}\label{sensitivity-to-catch-uncertainty}}

\leavevmode\tagmcend\tagstructend

To evaluate the influence of highly uncertain catch histories, the we both halved and doubled the historical catches in the pre-STAR base model. The historical catches tend to be far more uncertain than catches in the more recent and better documented era for rockfish (\emph{Sebastes} spp.), as historically most rockfish were landed in mixed stock market categories. However, relative to the somewhat elevated uncertainties described in the historical catch sensitivity analysis for southern California vermilion rockfish, there is less evidence for substantive bias in historical catches of vermilion rockfish north of Point Conception, where there is better evidence that vermilion rockfish made up a relatively modest fraction of the total catch.

The halving and doubling of historical catches in the northern model did lead to substantial differences in estimates of stock status (Figures \ref{fig:sens2-depl} and \ref{fig:sens2-spr}). A doubling of historical catches restuled in a stock slightly above the MSST, and a substantially larger predicted biomass when historical catches were halved. Equilibrium MSY estimates were also larger with greater historical catches (by approximately 32 tons) and reduced with lower historical catches. The relative change in model fit, as reflected by the total negative log likelihood, was modest, with a slightly improved fit in the lower historical catch scenario, and a slightly poorer fit in the higher historical catch scenario.

\tagstructbegin{tag=H3}\tagmcbegin{tag=H3}

\hypertarget{other-model-sensitivities}{%
\subsubsection{Other Model Sensitivities}\label{other-model-sensitivities}}

\leavevmode\tagmcend\tagstructend

Results from the \textbf{pre-STAR} base model were compared to several alternative model specifications, as described below.

\tagstructbegin{tag=L}

\begin{itemize}
\item
  \tagstructbegin{tag=LI}\tagstructbegin{tag=LBody}\tagmcbegin{tag=P}

  Estimate the Beverton-Holt steepness parameter (\emph{h}) rather than fixing it at the prior mean (\emph{h}=0.72); estimate uncertainty intervals for comparison to base

  \tagmcend\tagstructend\tagstructend
\item
  \tagstructbegin{tag=LI}\tagstructbegin{tag=LBody}\tagmcbegin{tag=P}

  Start recruitment deviations 5 years earlier than the base model configuration

  \tagmcend\tagstructend\tagstructend
\item
  \tagstructbegin{tag=LI}\tagstructbegin{tag=LBody}\tagmcbegin{tag=P}

  Start recruitment deviations 5 years later than the base model configuration

  \tagmcend\tagstructend\tagstructend
\item
  \tagstructbegin{tag=LI}\tagstructbegin{tag=LBody}\tagmcbegin{tag=P}

  Compare results based on the McAllister-Ianelli data weighting method (for composition data) to the Francis method used for the base model.

  \tagmcend\tagstructend\tagstructend
\item
  \tagstructbegin{tag=LI}\tagstructbegin{tag=LBody}\tagmcbegin{tag=P}

  Mirror the recreational discard fleets' selectivity curves to the corresponding retained fleets (PC or PR) rather than fitting to discard length comps as in the base model.

  \tagmcend\tagstructend\tagstructend
\end{itemize}

\tagstructend

Trends in spawning output for the northern California assessment model were generally robust to this set of sensitivities (Figure \ref{fig:sens1-spawnb}). Best estimates from all runs were within the estimated range of uncertainty for the base model. Steepness was estimated at a higher value than the prior mean (estimated at 0.94 vs.~fixed at 0.72). Similarly, stock status did not vary greatly among this set of sensitivity runs, with only a minor increase in 2021 relative spawning output when using McAllister-Ianelli weights and a slight decrease when estimating steepness (Figure \ref{fig:sens1-bratio}). The use of McAllister-Ianelli weights had the greatest impact on estimated recruitment deviations (Figure \ref{fig:sens1-recdevs}). This weighting method significantly reduced the magnitude of the 2016 year class, and generally increased the variance of the estimated deviations. The McAllister-Ianelli method gives greater weight to the composition data for this model (Table \ref{tab:data-weights}), and resulted in lower estimates of the male and female natural mortality rates (Table \ref{tab:sens1}).

During the STAR panel review, the STAT presented results from several sensitivity runs that were completed after distribution of the draft assessment document. All runs were conducted with the pre-STAR base models. These included:\\
- Fixing the natural mortality rate (M) to a value consistent with the observed maximum age when applying the Hamel prior (i.e., M = 5.4/80 = 0.07 {\tagstructbegin{tag=Formula}\tagmcbegin{tag=Formula}\(yr^{-1}\)\leavevmode\tagmcend\tagstructend})\\
- Assuming asymptotic (2-parameter) selectivity curves for all fleets, except for recreational discard and the NWFSC trawl survey)\\
- Estimating domed (4-parameter) selectivity curves for all fleets, but allowing for asymptotic shapes when supported by the data.\\
- Use of a 3-parameter, reparameterized Ricker stock-recruitment relationship instead of a standard, 2-parameter Beverton-Holt relationship.

The STAT compared several results from these runs to the pre-STAR base model, including time series of spawning output, relative spawning output, and recruitment deviations (Figures \ref{fig:sens3a}, \ref{fig:sens3b}, and \ref{fig:sens3c}). Negative log likelihoods (total and by data type), parameter estimates, and derived quantities were also examined relative to the pre-STAR base (Table \ref{tab:sens3}). Fixing {\tagstructbegin{tag=Formula}\tagmcbegin{tag=Formula}\(M\)\leavevmode\tagmcend\tagstructend} at 0.07 degraded the overall fit to the data, increasing the likelihood by about 3 points. The model with forced asymptotic selectivity estimated a slightly higher natural mortality rates (female {\tagstructbegin{tag=Formula}\tagmcbegin{tag=Formula}\(M\)\leavevmode\tagmcend\tagstructend} = 0.11, male {\tagstructbegin{tag=Formula}\tagmcbegin{tag=Formula}\(M\)\leavevmode\tagmcend\tagstructend} = 0.10) and had a total negative likelihood that was over 30 points higher than the pre-STAR base model. Estimating 4 selectivity parameters per fleet (excluding the discard and NWFSC trawl survey) produced results similar to the pre-base model, with a slightly larger population scale. Parameters from a reparameterized Ricker stock-recruitment relationship were estimable (Table \ref{tab:sens3}) with M fixed at base model values, but produced point results and a total likelihood that were generally consistent with the Beverton-Holt relationship assumed in the base model.

\tagstructbegin{tag=H2}\tagmcbegin{tag=H2}

\hypertarget{likelihood-profiles}{%
\subsection{Likelihood Profiles}\label{likelihood-profiles}}

\leavevmode\tagmcend\tagstructend

Likelihood profiles were conducted for natural mortality ({\tagstructbegin{tag=Formula}\tagmcbegin{tag=Formula}\(M\)\leavevmode\tagmcend\tagstructend}), steepness ({\tagstructbegin{tag=Formula}\tagmcbegin{tag=Formula}\(h\)\leavevmode\tagmcend\tagstructend}) and the log of R0 (unfished recruitment) by fixing these parameters across a range of values and continuing to estimate the remaining parameters assuming the base model framework. All models in this section use the \textbf{post-STAR} base models.

The profiles for natural mortality in the northern model (Figures \ref{fig:m-profile}, \ref{fig:m-depl}, \ref{fig:m-spawn}, \ref{fig:m-recruits}, and \ref{fig:m-recdevs}) suggest that this parameter is reasonably well informed between a range of approximately 0.07 and 0.11, a somewhat lower range of values than was estimated for the southern model. Interestingly, profiles for natural mortality in the north indicated that the length data were better fit by the model with a lower natural mortality rate, and the age data were better fit by a higher rate, a result that is in contrast to that observed in the southern model. Most of the indices were also better fit by a higher natural mortality rate, although the Deb Wilson-Vandenberg Index and the onboard observer index were better fit with lower natural mortality rates. As is typical, spawning output increased with decreasing natural mortality, such that the model estimate with {\tagstructbegin{tag=Formula}\tagmcbegin{tag=Formula}\(M\)\leavevmode\tagmcend\tagstructend} = 0.05 was 1.5 times that of the model estimate when {\tagstructbegin{tag=Formula}\tagmcbegin{tag=Formula}\(M\)\leavevmode\tagmcend\tagstructend} was assumed to be 0.12. Similarly, the model estimated depletion in 2020 was well above the target level, close to 70\%, for the M=0.12 model, while the ending depletion was just barely over the minimum stock size threshold of 25\% of the unfished level in the M=0.05 model (Figure \ref{fig:m-depl}).

A profile of steepness was conducted on values ranging from 0.30 to 0.90 in 0.10 increments. The likelihood profiles (by component, and by component and fleet) are shown as Figure \ref{fig:h-profile}, and the resulting model trajectories (spawning output, relative depletion, age-0 recruits, and recruitment deviations) are shown as Figures \ref{fig:h-spawn}, \ref{fig:h-depl}, \ref{fig:h-recruits}, and \ref{fig:h-recdevs}. The likelihood profiles show that the overall best fit to the data is associated with high steepness values, although the data were generally uninformative above steepness values of 0.5. Overall the length data were not very informative, and there was some odd jumps in the likelihood in some fleets at very low (0.3) steepness values. Similarly the age data were only marginally informative, and suggested higher steepness values in general. Most of the indices also suggested higher steepness values, particularly the NWFSC hook and line survey index, although the Rec PC index had a significantly better fit at lower steepness values. Predictably, spawning output scaled down with higher steepness values and up with lower values, however the estimate of stock status in 2021 was actually more optimistic with lower steepness values and more pessimistic with higher steepness values. The higher steepness runs were also more pessimistic with respect to historical (late 1980s through the early 2000s) stock status.

A profile on the log of unfished recruitment was conducted on values ranging from 5.7 to 6.4 (the base model estimate was 6.07), and is shown as Figure \ref{fig:r0-profile}. In general, age data was better fit by the model with higher values of R0, as was most of the survey data (as well as recruitment via likelihood penalties). The DWV\_onboard survey index and the Rec\_PR were both somewhat better fit by the lower R0 values. All length data were either better fit to lower R0 values or were uninformative. As with the southern model, spawning output is estimated to be greater with the higher R0 values, although the overall difference in scaling was minimal for early years. Also consistent with the southern model result, the northern model result was more pessimistic with the lower R0 values, with the low R0 values being below the MSST and higher R0 values suggesting that relative spawning output is well above target levels (Figures \ref{fig:r0-depl}, \ref{fig:r0-spawn}, \ref{fig:r0-recruits}, and \ref{fig:r0-recdevs}). As with the southern model, this is a result of corresponding model changes in the estimate of the natural mortality rate (which is estimated to be much lower in the low R0 model), such that the R0 profile is in many ways simply providing the same information as the profile on the natural mortality rate. Additional profiles in which M is fixed may be helpful in evaluating model performance.

\tagstructbegin{tag=H2}\tagmcbegin{tag=H2}

\hypertarget{retrospective-analysis}{%
\subsection{Retrospective Analysis}\label{retrospective-analysis}}

\leavevmode\tagmcend\tagstructend

All models in this section use the \textbf{post-STAR} base model.

A five year retrospective analysis was conducted on the northern base model by sequentially removing data, beginning with data from the year 2020. Figures \ref{fig:retro-spawnb}, \ref{fig:retro-bratio}, \ref{fig:retro-recruits}, and \ref{fig:retro-recdevs} show the estimated spawning output, the estimated depletion, the recruitment deviation estimates and the estimated fit to the CCFRP index (which was the index most affected by the analysis). The greatest impact of sequentially removing recent data was the declining estimate of the strength of the 2016 year class, a result similar to the southern model, as the length composition and index data that informed those year classes were removed. There was also a slightly lesser reduction in the strength of the 2013 and 2014 year classes. However, aside from a modest rescaling upwards of recruitment deviations, the spawning output and depletion estimates did not change by any significant measure, suggesting no concerning retrospective patterns (Table \ref{tab:retro}). Note that all composition data weights were held constant at the base model values during each run.

\tagstructbegin{tag=H2}\tagmcbegin{tag=H2}

\hypertarget{unresolved-problems-and-major-uncertainties-1}{%
\subsection{Unresolved Problems and Major Uncertainties}\label{unresolved-problems-and-major-uncertainties-1}}

\leavevmode\tagmcend\tagstructend

This assessment treats populations north and south of Point Conception as separate, but there is likely larval or juvenile dispersal, and potentially some adult movement, among these areas. Dispersal and movement rates are not well known.

The primary fishery-independent survey for West Coast groundfish, the NWFSC WCGBTS, does not sample rocky habitats where most vermilion rockfish are found, and thus does not provide a robust index of abundance. An alternative survey, the CCFRP hook-and-line survey, provides a good signal for vermilion rockfish, including relative abundance and demographic structure inside and outside a number of MPAs. In addition to not including data from closed areas, many of the fishery-dependent indices are noisy, and some are not particularly well fit, such as the recreational PR dockside index.

Age data are limited and consequently growth and natural mortality estimates are uncertain. There is an unusual pattern of tension among data sources inferred by the likelihood profiles, with age data suggesting a higher natural mortality rate and length data suggesting a lower \emph{M}. This is opposite the pattern seen in the southern assessment, and (very generally speaking) many other west coast groundfish assessments.

The model estimates a series of very low recruitment events throughout most of the 1970s, a period in which many other rockfish in this region experienced high levels of recruitment. Recruitment patterns in more recent years generally follow those for other stocks. It is possible that selectivity patterns changed, data are biased, model misspecification, or unknown ecosystem interactions could be responsible for this pattern.

\tagstructbegin{tag=H1}\tagmcbegin{tag=H1}

\hypertarget{harvest-projections-and-decision-tables}{%
\section{Harvest Projections and Decision Tables}\label{harvest-projections-and-decision-tables}}

\leavevmode\tagmcend\tagstructend

The forecasts of stock abundance and yield were developed using the post-STAR base model, with the forecast projections presented in Table \ref{tab:projectionES}. The total catches in 2021 and 2022 are set to the projected catch from the California Department of Fish and Wildlife (CDFW) by sector and model region, i.e., allocated north and south of $34^\circ 27^\prime N$ in California.

Uncertainty in the forecasts is based upon the three states of nature agreed upon at the STAR panel, reflecting three different natural mortality rates. The steepness parameter of the Beverton-Holt stock-recruit curve was fixed in the base model and in all of the forecasts. The northern California model is not data rich and while there is uncertainty in steepness, it was not well estimated in the base model when natural mortality was also estimated. The alternative states of nature maintain the female to male natural mortality rate ratio from the base model. To capture the 75\% interval around the negative log-likelihood, alternate states were identified within 0.66 negative log-likelihood points from the base model where female {\tagstructbegin{tag=Formula}\tagmcbegin{tag=Formula}\(M\)\leavevmode\tagmcend\tagstructend} = 0.0856 and male {\tagstructbegin{tag=Formula}\tagmcbegin{tag=Formula}\(M\)\leavevmode\tagmcend\tagstructend} = 0.0805. The high state of nature fixes female {\tagstructbegin{tag=Formula}\tagmcbegin{tag=Formula}\(M\)\leavevmode\tagmcend\tagstructend} = 0.0956 and male {\tagstructbegin{tag=Formula}\tagmcbegin{tag=Formula}\(M\)\leavevmode\tagmcend\tagstructend} = 0.08989. For the low state of nature, female {\tagstructbegin{tag=Formula}\tagmcbegin{tag=Formula}\(M\)\leavevmode\tagmcend\tagstructend} = 0.0769 and male {\tagstructbegin{tag=Formula}\tagmcbegin{tag=Formula}\(M\)\leavevmode\tagmcend\tagstructend} = 0.07231.

For reference, the base model predicted {\tagstructbegin{tag=Formula}\tagmcbegin{tag=Formula}\(\sigma\)\leavevmode\tagmcend\tagstructend} = 0.246. The buffers between the OFL and ABC were calculated assuming a category 2 stock, with {\tagstructbegin{tag=Formula}\tagmcbegin{tag=Formula}\(\sigma\)\leavevmode\tagmcend\tagstructend} = 1.0 and a {\tagstructbegin{tag=Formula}\tagmcbegin{tag=Formula}\(p^*\)\leavevmode\tagmcend\tagstructend} = 0.45. Alternative catch streams (rows in the table) include {\tagstructbegin{tag=Formula}\tagmcbegin{tag=Formula}\(\sigma\)\leavevmode\tagmcend\tagstructend} = 1.0 with a {\tagstructbegin{tag=Formula}\tagmcbegin{tag=Formula}\(p^*\)\leavevmode\tagmcend\tagstructend} = 0.4, and removals of long-term equilibrium catch with and without a buffer assuming {\tagstructbegin{tag=Formula}\tagmcbegin{tag=Formula}\(\sigma\)\leavevmode\tagmcend\tagstructend} = 1.0 with a {\tagstructbegin{tag=Formula}\tagmcbegin{tag=Formula}\(p^*\)\leavevmode\tagmcend\tagstructend} = 0.45. The buffer multiplier with {\tagstructbegin{tag=Formula}\tagmcbegin{tag=Formula}\(p^*\)\leavevmode\tagmcend\tagstructend} = 0.45 ranges from 0.874 in 2023 ramping to 0.803 in 2032.

Current forecasts based on the alternative states of nature and requested catch streams project that the stock will remain above the target threshold of 40\% in 2032 (Table \ref{tab:DecisionES}). In all of the scenarios of the low state of nature, the stock remains below the target threshold of 40\% until 2026 or 2027. The base model with the base catches results in an increasing stock over the period from 2023-2032. In all scenarios the catch significantly decreases from 2022 to 2023; projected catch in 2022 is 227 mt, and 2023 catches from the base model range from 118-139 mt. The base model includes a portion of the stock within the northern management unit (north of $40^\circ 10^\prime N$). An analysis based on the private/rental mode index through 2019 suggests that 4.44\% of the catches from this model should be apportioned to the northern management unit for vermilion rockfish.

The STAT cautions that the GMT projections for catches in 2021-2022 (22 mt per year) exceed the maximum sustainable yield according to both proxies ({\tagstructbegin{tag=Formula}\tagmcbegin{tag=Formula}\(B_{40\%}\)\leavevmode\tagmcend\tagstructend} and {\tagstructbegin{tag=Formula}\tagmcbegin{tag=Formula}\(SPR_{50\%}\)\leavevmode\tagmcend\tagstructend}) as well as the MSY value based on the estimated value of steepness (Table \ref{tab:referenceES}). The northern California stock is just above target biomass in 2021 (43\% of unfished spawning ouptut), so these catch levels are unlikely to result in significant stock declines over a short period of time. However, similar catch levels would exceed the overfishing limits (OFL) if carried forward for 2023 and beyond (Table \ref{tab:CombinedRefPtsES}), and would be unsustainable in the long term. Given recent and projected near-term exploitation levels, and especially if vermilion and sunset rockfish continue to be managed as part of the minor shelf rockfish complex, the STAT recommends regular monitoring of total mortality for these two species to avoid excessive stock depletion and potential loss of yield.

\tagstructbegin{tag=H2}\tagmcbegin{tag=H2}

\hypertarget{regional-management-and-spatial-management-considerations}{%
\subsection{Regional Management and Spatial Management Considerations}\label{regional-management-and-spatial-management-considerations}}

\leavevmode\tagmcend\tagstructend

Over the last several decades, spatially explicit management measures at both the state and federal/management council level have been implemented to achieve a wide range of marine resource and fishery management objectives. Depth restrictions to commercial and recreational fisheries in the Rockfish Conservation Areas (RCAs) and the Cowcod Conservation Areas (CCAs) are key among those, as are the suite of total and partial exclusion of commercial and recreational fishing activities in the California statewide network of Marine Protected Areas (MPAs). While the former are associated with explicit fisheries management objectives, the latter have a suite of ecological and economic objectives, most of which are not specific to, nor integrated across, the fisheries management arena. Despite this, both types of spatial management measures are expected to result in various biological, ecological, and socioeconomic effects within and adjacent to their boundaries. All of these effects have the potential to influence the nature and quality of the data used to inform stock assessments of species that reside in these areas, including vermilion rockfish.

Regardless of the management objective, spatial closures are expected to increase the spatial heterogeneity in abundance and size or age structure of fished stocks. This greater spatial variability can complicate the assumptions made in stock assessment models, particularly the assumption that the densities and demographic structure of assessed populations are relatively homogeneous, at least across predictable habitat types such as bathymetric gradients or substrate types {\tagstructbegin{tag=Reference}\tagmcbegin{tag=Reference}(Punt and Methot 2004, Field et al. 2006, Berger et al. 2017)\leavevmode\tagmcend\tagstructend}. Although a wide range of factors above and beyond spatial management measures can also lead to violations of those assumptions, and the challenge is intuitively less problematic for populations with high movement rates and/or high population turnover, the challenge can be particularly important for longer lived populations with lower movement rates. The challenge can best be summarized by the result that the more effective MPAs or other closed areas are at protecting populations within them, the more likely it is that traditional assessment approaches will be biased or more uncertain.

If the spatial closures also prevent fisheries independent surveys from evaluating the relative abundance and demographic structure of managed populations, the challenges in developing robust population models, and thus robust management advice, become even more severe. While spatially explicit assessment models provide a means of more explicitly addressing these challenges, such models are computationally intensive, require robust data from the specific areas being modeled, and may also require detailed information regarding movement and dispersal rates {\tagstructbegin{tag=Reference}\tagmcbegin{tag=Reference}(McGilliard et al. 2014, Berger et al. 2017, Cadrin 2020, Punt et al. 2020)\leavevmode\tagmcend\tagstructend}. Moreover, the complexity of these spatial models increases substantially if the size and location of closed areas changes over time, as many of the more ``fisheries management based'' closures (e.g., RCAs) have in California groundfish fisheries. Thus, such approaches may be less feasible for more data limited stocks, such as northern and southern vermilion rockfish, at least in the near term. However, the fact that both the northern and southern assessment models are informed by fishery-independent surveys that include habitats both inside and outside area closures provides some hope for greater recognition of spatial factors in future assessments.

\tagstructbegin{tag=H1}\tagmcbegin{tag=H1}

\hypertarget{research-and-data-needs-1}{%
\section{Research and Data Needs}\label{research-and-data-needs-1}}

\leavevmode\tagmcend\tagstructend

We recommend the following research be conducted before the next assessment:

\tagstructbegin{tag=L}

\begin{itemize}
\item
  Investigate the structure of complex and contribution of each species to the vermilion/sunset rockfish complex. Investigate possible spatial differences in biological parameters within a single species and also between the two species. Little biological data for south of Point Conception or north of Point Arena were available for this assessment and is needed to better under biological parameters.

  \tagstructbegin{tag=L}

  \begin{itemize}
  \item
    \tagstructbegin{tag=LI}\tagstructbegin{tag=LBody}\tagmcbegin{tag=P}

    \tagstructbegin{tag=LI}\tagstructbegin{tag=LBody}\tagmcbegin{tag=P}

    Conduct life history studies

    \tagmcend\tagstructend\tagstructend

    \tagmcend\tagstructend\tagstructend
  \item
    \tagstructbegin{tag=LI}\tagstructbegin{tag=LBody}\tagmcbegin{tag=P}

    \tagstructbegin{tag=LI}\tagstructbegin{tag=LBody}\tagmcbegin{tag=P}

    Conduct research to identify the proportion of each species in population and in catches

    \tagmcend\tagstructend\tagstructend

    \tagmcend\tagstructend\tagstructend
  \end{itemize}

  \tagstructend
\item
  Take a closer look at historical catch reconstructions and all other historical data sources.
\item
  Refine CCFRP survey index to look at alternative possible model structures, including a hierarchical structure and random effects. The CCFRP survey is the only fishery-independent survey available for nearshore rockfish sampling the nearshore rocky reef habitats. As of this assessment, only two years of coastwide data are available, and the index was limited to the site in central California that have been monitored since 2007.
\item
  Continue to investigate the most appropriate model structure for the NWFSC HL survey index. The NWFSC HL survey is the only long-term fishery-independent survey in rocky (untrawlable) habitat in the Southern California Bight. We also recommend evaluating how to structure the NWFSC Hook-and-Line survey index, given its expansion into the CCA, also independent analysis of information content in NWFSC Hook-and-Line survey. Increased spatiotemporal sampling around Point Conception would aid in identifying stock boundaries.
\item
  Utilize existing ROV survey data sources

  \tagstructbegin{tag=L}

  \begin{itemize}
  \item
    \tagstructbegin{tag=LI}\tagstructbegin{tag=LBody}\tagmcbegin{tag=P}

    \tagstructbegin{tag=LI}\tagstructbegin{tag=LBody}\tagmcbegin{tag=P}

    SWFSC Submersible Survey of the Cowcod Conservation Areas {\tagstructbegin{tag=Reference}\tagmcbegin{tag=Reference}(Yoklavich et al. 2007)\leavevmode\tagmcend\tagstructend}.

    \tagmcend\tagstructend\tagstructend

    \tagmcend\tagstructend\tagstructend
  \item
    \tagstructbegin{tag=LI}\tagstructbegin{tag=LBody}\tagmcbegin{tag=P}

    \tagstructbegin{tag=LI}\tagstructbegin{tag=LBody}\tagmcbegin{tag=P}

    This was a line-transect survey designed to estimate cowcod abundance in 2002 conducted from a submersible inside the CCAs. Originally, only cowcod were enumerated from the video footage. Over the last few years, the SWFSC has re-analyzed the video footage to enumerate other rockfish species.

    \tagmcend\tagstructend\tagstructend

    \tagmcend\tagstructend\tagstructend
  \item
    \tagstructbegin{tag=LI}\tagstructbegin{tag=LBody}\tagmcbegin{tag=P}

    \tagstructbegin{tag=LI}\tagstructbegin{tag=LBody}\tagmcbegin{tag=P}

    The SWFSC Fishery Resource Division (FRD) conducted a survey of potential cowcod habitat between Point Conception and the U.S. -- Mexico border from October through December of 2012 {\tagstructbegin{tag=Reference}\tagmcbegin{tag=Reference}(Stierhoff and Cutter 2013)\leavevmode\tagmcend\tagstructend}.

    \tagmcend\tagstructend\tagstructend

    \tagmcend\tagstructend\tagstructend
  \item
    \tagstructbegin{tag=LI}\tagstructbegin{tag=LBody}\tagmcbegin{tag=P}

    \tagstructbegin{tag=LI}\tagstructbegin{tag=LBody}\tagmcbegin{tag=P}

    SWFSC staff are submitting proposals to conduct an additional submersible survey in the Southern California Bight

    \tagmcend\tagstructend\tagstructend

    \tagmcend\tagstructend\tagstructend
  \item
    \tagstructbegin{tag=LI}\tagstructbegin{tag=LBody}\tagmcbegin{tag=P}

    \tagstructbegin{tag=LI}\tagstructbegin{tag=LBody}\tagmcbegin{tag=P}

    CDFW ROV survey data

    \tagmcend\tagstructend\tagstructend

    \tagmcend\tagstructend\tagstructend
  \end{itemize}

  \tagstructend
\item
  Collection of length and age data are recommended for both the commercial and recreational fisheries. Very little age data are available from either fishery for vermilion and sunset rockfish.
\item
  Investigate possible environmental drivers/co-variates for biological parameters, particularly for recruitment.
\item
  Resolve differences between CalCOM and PacFIN expanded length composition data sets.
\end{itemize}

\tagstructend

\clearpage

\tagstructbegin{tag=H1}\tagmcbegin{tag=H1}

\hypertarget{acknowledgments}{%
\section{Acknowledgments}\label{acknowledgments}}

\leavevmode\tagmcend\tagstructend

We gratefully acknowledge input and review from the STAR panel including John Budrick (STAR panel chair; California Department of Fish and Wildlife), Allan Hicks (International Pacific Halibut Commission), Matt Cieri (Center for Independent Experts), and Paul Medley (Center for Independent Experts). Thanks to the STAR panel advisors Gerry Richter (GAP), Melissa Mandrup (GMT), and John DeVore (PFMC) for their expertise and additional guidance during the STAR panel. Thanks to the SWFSC ages and the NWSFC CAP lab for reading vermilion rockfish otoliths as well as all of the labs that participated in the vermilion rockfish otolith CARE exchange. We thank Melissa Head (Northwest Fisheries Science Center NMFS, NOAA), for producing analyses on maturity, Sabrina Beyer (UCSC) for providing data on fecundity and John Harms and John Wallace for providing insight into the NWFSC hook-and-line survey and interpretation of available genetics and ongoing research of the cryptic species pair. Additionally, we thank all of the CCFRP program partners for conducting and providing the available data.

\clearpage
\floatplacement{table}{H}

\tagstructbegin{tag=H1}\tagmcbegin{tag=H1}

\hypertarget{tables}{%
\section{Tables}\label{tables}}

\leavevmode\tagmcend\tagstructend

\newpage

\begingroup\fontsize{10}{12}\selectfont

\tagstructbegin{tag=Table}\tagmcbegin{tag=Table}
\begin{longtable}[t]{>{\raggedleft\arraybackslash}p{1cm}>{\raggedleft\arraybackslash}p{1cm}>{\raggedleft\arraybackslash}p{1cm}>{\raggedleft\arraybackslash}p{1cm}>{\raggedleft\arraybackslash}p{1cm}>{\raggedleft\arraybackslash}p{1cm}>{\raggedleft\arraybackslash}p{1cm}>{\raggedleft\arraybackslash}p{1cm}>{\raggedleft\arraybackslash}p{1cm}>{\raggedleft\arraybackslash}p{1cm}}
\caption{\label{tab:landings}Landings of vermilion rockfish by fleet and year. All recreational fleet 
  landings are in numbers (thousands of fish) and commercial fleets in biomass 
  (mt). A description of the sources of the landings data are in the text and 
  the next table.}\\
\toprule
\rotatebox{45}{Year} & \rotatebox{45}{COM HKL(1)} & \rotatebox{45}{COM TWL(2)} & \rotatebox{45}{COM NET(3)} & \rotatebox{45}{REC PC(4)} & \rotatebox{45}{REC PC DIS(5)} & \rotatebox{45}{REC PR(6)} & \rotatebox{45}{REC PR DIS(7)} & \rotatebox{45}{Total commerical} & \rotatebox{45}{Total recreational}\\
\midrule
\endfirsthead
\caption[]{\label{tab:landings}Landings of vermilion rockfish by fleet and year. All recreational fle Landings of vermilion rockfish by fleet and year (continued).}\\
\toprule
\rotatebox{45}{Year} & \rotatebox{45}{COM HKL(1)} & \rotatebox{45}{COM TWL(2)} & \rotatebox{45}{COM NET(3)} & \rotatebox{45}{REC PC(4)} & \rotatebox{45}{REC PC DIS(5)} & \rotatebox{45}{REC PR(6)} & \rotatebox{45}{REC PR DIS(7)} & \rotatebox{45}{Total commerical} & \rotatebox{45}{Total recreational}\\
\midrule
\endhead

\endfoot
\bottomrule
\endlastfoot
1875 & \cellcolor[HTML]{CCCCCC}{0.240} &  & \cellcolor[HTML]{CCCCCC}{} &  & \cellcolor[HTML]{CCCCCC}{} &  & \cellcolor[HTML]{CCCCCC}{} & 0.240 & \\
1876 & \cellcolor[HTML]{CCCCCC}{0.481} &  & \cellcolor[HTML]{CCCCCC}{} &  & \cellcolor[HTML]{CCCCCC}{} &  & \cellcolor[HTML]{CCCCCC}{} & 0.481 & \\
1877 & \cellcolor[HTML]{CCCCCC}{0.721} &  & \cellcolor[HTML]{CCCCCC}{} &  & \cellcolor[HTML]{CCCCCC}{} &  & \cellcolor[HTML]{CCCCCC}{} & 0.721 & \\
1878 & \cellcolor[HTML]{CCCCCC}{0.961} &  & \cellcolor[HTML]{CCCCCC}{} &  & \cellcolor[HTML]{CCCCCC}{} &  & \cellcolor[HTML]{CCCCCC}{} & 0.961 & \\
1879 & \cellcolor[HTML]{CCCCCC}{1.201} &  & \cellcolor[HTML]{CCCCCC}{} &  & \cellcolor[HTML]{CCCCCC}{} &  & \cellcolor[HTML]{CCCCCC}{} & 1.201 & \\
1880 & \cellcolor[HTML]{CCCCCC}{1.442} &  & \cellcolor[HTML]{CCCCCC}{} &  & \cellcolor[HTML]{CCCCCC}{} &  & \cellcolor[HTML]{CCCCCC}{} & 1.442 & \\
1881 & \cellcolor[HTML]{CCCCCC}{1.682} &  & \cellcolor[HTML]{CCCCCC}{} &  & \cellcolor[HTML]{CCCCCC}{} &  & \cellcolor[HTML]{CCCCCC}{} & 1.682 & \\
1882 & \cellcolor[HTML]{CCCCCC}{1.922} &  & \cellcolor[HTML]{CCCCCC}{} &  & \cellcolor[HTML]{CCCCCC}{} &  & \cellcolor[HTML]{CCCCCC}{} & 1.922 & \\
1883 & \cellcolor[HTML]{CCCCCC}{2.163} &  & \cellcolor[HTML]{CCCCCC}{} &  & \cellcolor[HTML]{CCCCCC}{} &  & \cellcolor[HTML]{CCCCCC}{} & 2.163 & \\
1884 & \cellcolor[HTML]{CCCCCC}{2.403} &  & \cellcolor[HTML]{CCCCCC}{} &  & \cellcolor[HTML]{CCCCCC}{} &  & \cellcolor[HTML]{CCCCCC}{} & 2.403 & \\
1885 & \cellcolor[HTML]{CCCCCC}{2.643} &  & \cellcolor[HTML]{CCCCCC}{} &  & \cellcolor[HTML]{CCCCCC}{} &  & \cellcolor[HTML]{CCCCCC}{} & 2.643 & \\
1886 & \cellcolor[HTML]{CCCCCC}{2.884} &  & \cellcolor[HTML]{CCCCCC}{} &  & \cellcolor[HTML]{CCCCCC}{} &  & \cellcolor[HTML]{CCCCCC}{} & 2.884 & \\
1887 & \cellcolor[HTML]{CCCCCC}{3.124} &  & \cellcolor[HTML]{CCCCCC}{} &  & \cellcolor[HTML]{CCCCCC}{} &  & \cellcolor[HTML]{CCCCCC}{} & 3.124 & \\
1888 & \cellcolor[HTML]{CCCCCC}{3.364} &  & \cellcolor[HTML]{CCCCCC}{} &  & \cellcolor[HTML]{CCCCCC}{} &  & \cellcolor[HTML]{CCCCCC}{} & 3.364 & \\
1889 & \cellcolor[HTML]{CCCCCC}{3.604} &  & \cellcolor[HTML]{CCCCCC}{} &  & \cellcolor[HTML]{CCCCCC}{} &  & \cellcolor[HTML]{CCCCCC}{} & 3.604 & \\
1890 & \cellcolor[HTML]{CCCCCC}{3.845} &  & \cellcolor[HTML]{CCCCCC}{} &  & \cellcolor[HTML]{CCCCCC}{} &  & \cellcolor[HTML]{CCCCCC}{} & 3.845 & \\
1891 & \cellcolor[HTML]{CCCCCC}{4.085} &  & \cellcolor[HTML]{CCCCCC}{} &  & \cellcolor[HTML]{CCCCCC}{} &  & \cellcolor[HTML]{CCCCCC}{} & 4.085 & \\
1892 & \cellcolor[HTML]{CCCCCC}{4.325} &  & \cellcolor[HTML]{CCCCCC}{} &  & \cellcolor[HTML]{CCCCCC}{} &  & \cellcolor[HTML]{CCCCCC}{} & 4.325 & \\
1893 & \cellcolor[HTML]{CCCCCC}{4.082} &  & \cellcolor[HTML]{CCCCCC}{} &  & \cellcolor[HTML]{CCCCCC}{} &  & \cellcolor[HTML]{CCCCCC}{} & 4.082 & \\
1894 & \cellcolor[HTML]{CCCCCC}{3.839} &  & \cellcolor[HTML]{CCCCCC}{} &  & \cellcolor[HTML]{CCCCCC}{} &  & \cellcolor[HTML]{CCCCCC}{} & 3.839 & \\
1895 & \cellcolor[HTML]{CCCCCC}{3.596} &  & \cellcolor[HTML]{CCCCCC}{} &  & \cellcolor[HTML]{CCCCCC}{} &  & \cellcolor[HTML]{CCCCCC}{} & 3.596 & \\
1896 & \cellcolor[HTML]{CCCCCC}{3.396} &  & \cellcolor[HTML]{CCCCCC}{} &  & \cellcolor[HTML]{CCCCCC}{} &  & \cellcolor[HTML]{CCCCCC}{} & 3.396 & \\
1897 & \cellcolor[HTML]{CCCCCC}{3.195} &  & \cellcolor[HTML]{CCCCCC}{} &  & \cellcolor[HTML]{CCCCCC}{} &  & \cellcolor[HTML]{CCCCCC}{} & 3.195 & \\
1898 & \cellcolor[HTML]{CCCCCC}{2.995} &  & \cellcolor[HTML]{CCCCCC}{} &  & \cellcolor[HTML]{CCCCCC}{} &  & \cellcolor[HTML]{CCCCCC}{} & 2.995 & \\
1899 & \cellcolor[HTML]{CCCCCC}{2.794} &  & \cellcolor[HTML]{CCCCCC}{} &  & \cellcolor[HTML]{CCCCCC}{} &  & \cellcolor[HTML]{CCCCCC}{} & 2.794 & \\
1900 & \cellcolor[HTML]{CCCCCC}{3.091} &  & \cellcolor[HTML]{CCCCCC}{} &  & \cellcolor[HTML]{CCCCCC}{} &  & \cellcolor[HTML]{CCCCCC}{} & 3.091 & \\
1901 & \cellcolor[HTML]{CCCCCC}{3.389} &  & \cellcolor[HTML]{CCCCCC}{} &  & \cellcolor[HTML]{CCCCCC}{} &  & \cellcolor[HTML]{CCCCCC}{} & 3.389 & \\
1902 & \cellcolor[HTML]{CCCCCC}{3.686} &  & \cellcolor[HTML]{CCCCCC}{} &  & \cellcolor[HTML]{CCCCCC}{} &  & \cellcolor[HTML]{CCCCCC}{} & 3.686 & \\
1903 & \cellcolor[HTML]{CCCCCC}{3.983} &  & \cellcolor[HTML]{CCCCCC}{} &  & \cellcolor[HTML]{CCCCCC}{} &  & \cellcolor[HTML]{CCCCCC}{} & 3.983 & \\
1904 & \cellcolor[HTML]{CCCCCC}{4.281} &  & \cellcolor[HTML]{CCCCCC}{} &  & \cellcolor[HTML]{CCCCCC}{} &  & \cellcolor[HTML]{CCCCCC}{} & 4.281 & \\
1905 & \cellcolor[HTML]{CCCCCC}{4.574} &  & \cellcolor[HTML]{CCCCCC}{} &  & \cellcolor[HTML]{CCCCCC}{} &  & \cellcolor[HTML]{CCCCCC}{} & 4.574 & \\
1906 & \cellcolor[HTML]{CCCCCC}{4.867} &  & \cellcolor[HTML]{CCCCCC}{} &  & \cellcolor[HTML]{CCCCCC}{} &  & \cellcolor[HTML]{CCCCCC}{} & 4.867 & \\
1907 & \cellcolor[HTML]{CCCCCC}{5.161} &  & \cellcolor[HTML]{CCCCCC}{} &  & \cellcolor[HTML]{CCCCCC}{} &  & \cellcolor[HTML]{CCCCCC}{} & 5.161 & \\
1908 & \cellcolor[HTML]{CCCCCC}{5.454} &  & \cellcolor[HTML]{CCCCCC}{} &  & \cellcolor[HTML]{CCCCCC}{} &  & \cellcolor[HTML]{CCCCCC}{} & 5.454 & \\
1909 & \cellcolor[HTML]{CCCCCC}{6.137} &  & \cellcolor[HTML]{CCCCCC}{} &  & \cellcolor[HTML]{CCCCCC}{} &  & \cellcolor[HTML]{CCCCCC}{} & 6.137 & \\
1910 & \cellcolor[HTML]{CCCCCC}{6.820} &  & \cellcolor[HTML]{CCCCCC}{} &  & \cellcolor[HTML]{CCCCCC}{} &  & \cellcolor[HTML]{CCCCCC}{} & 6.820 & \\
1911 & \cellcolor[HTML]{CCCCCC}{7.504} &  & \cellcolor[HTML]{CCCCCC}{} &  & \cellcolor[HTML]{CCCCCC}{} &  & \cellcolor[HTML]{CCCCCC}{} & 7.504 & \\
1912 & \cellcolor[HTML]{CCCCCC}{8.187} &  & \cellcolor[HTML]{CCCCCC}{} &  & \cellcolor[HTML]{CCCCCC}{} &  & \cellcolor[HTML]{CCCCCC}{} & 8.187 & \\
1913 & \cellcolor[HTML]{CCCCCC}{8.870} &  & \cellcolor[HTML]{CCCCCC}{} &  & \cellcolor[HTML]{CCCCCC}{} &  & \cellcolor[HTML]{CCCCCC}{} & 8.870 & \\
1914 & \cellcolor[HTML]{CCCCCC}{9.553} &  & \cellcolor[HTML]{CCCCCC}{} &  & \cellcolor[HTML]{CCCCCC}{} &  & \cellcolor[HTML]{CCCCCC}{} & 9.553 & \\
1915 & \cellcolor[HTML]{CCCCCC}{10.236} &  & \cellcolor[HTML]{CCCCCC}{} &  & \cellcolor[HTML]{CCCCCC}{} &  & \cellcolor[HTML]{CCCCCC}{} & 10.236 & \\
1916 & \cellcolor[HTML]{CCCCCC}{11.401} & 0.078 & \cellcolor[HTML]{CCCCCC}{} &  & \cellcolor[HTML]{CCCCCC}{} &  & \cellcolor[HTML]{CCCCCC}{} & 11.479 & \\
1917 & \cellcolor[HTML]{CCCCCC}{18.423} & 0.121 & \cellcolor[HTML]{CCCCCC}{} &  & \cellcolor[HTML]{CCCCCC}{} &  & \cellcolor[HTML]{CCCCCC}{} & 18.544 & \\
1918 & \cellcolor[HTML]{CCCCCC}{17.339} & 0.141 & \cellcolor[HTML]{CCCCCC}{} &  & \cellcolor[HTML]{CCCCCC}{} &  & \cellcolor[HTML]{CCCCCC}{} & 17.480 & \\
1919 & \cellcolor[HTML]{CCCCCC}{10.214} & 0.098 & \cellcolor[HTML]{CCCCCC}{} &  & \cellcolor[HTML]{CCCCCC}{} &  & \cellcolor[HTML]{CCCCCC}{} & 10.312 & \\
1920 & \cellcolor[HTML]{CCCCCC}{11.101} & 0.100 & \cellcolor[HTML]{CCCCCC}{} &  & \cellcolor[HTML]{CCCCCC}{} &  & \cellcolor[HTML]{CCCCCC}{} & 11.201 & \\
1921 & \cellcolor[HTML]{CCCCCC}{9.792} & 0.083 & \cellcolor[HTML]{CCCCCC}{} &  & \cellcolor[HTML]{CCCCCC}{} &  & \cellcolor[HTML]{CCCCCC}{} & 9.875 & \\
1922 & \cellcolor[HTML]{CCCCCC}{9.512} & 0.072 & \cellcolor[HTML]{CCCCCC}{} &  & \cellcolor[HTML]{CCCCCC}{} &  & \cellcolor[HTML]{CCCCCC}{} & 9.584 & \\
1923 & \cellcolor[HTML]{CCCCCC}{12.425} & 0.077 & \cellcolor[HTML]{CCCCCC}{} &  & \cellcolor[HTML]{CCCCCC}{} &  & \cellcolor[HTML]{CCCCCC}{} & 12.502 & \\
1924 & \cellcolor[HTML]{CCCCCC}{16.444} & 0.044 & \cellcolor[HTML]{CCCCCC}{} &  & \cellcolor[HTML]{CCCCCC}{} &  & \cellcolor[HTML]{CCCCCC}{} & 16.488 & \\
1925 & \cellcolor[HTML]{CCCCCC}{18.531} & 0.038 & \cellcolor[HTML]{CCCCCC}{} &  & \cellcolor[HTML]{CCCCCC}{} &  & \cellcolor[HTML]{CCCCCC}{} & 18.569 & \\
1926 & \cellcolor[HTML]{CCCCCC}{22.856} & 0.108 & \cellcolor[HTML]{CCCCCC}{} &  & \cellcolor[HTML]{CCCCCC}{} &  & \cellcolor[HTML]{CCCCCC}{} & 22.964 & \\
1927 & \cellcolor[HTML]{CCCCCC}{19.433} & 0.199 & \cellcolor[HTML]{CCCCCC}{} &  & \cellcolor[HTML]{CCCCCC}{} &  & \cellcolor[HTML]{CCCCCC}{} & 19.632 & \\
1928 & \cellcolor[HTML]{CCCCCC}{16.629} & 0.237 & \cellcolor[HTML]{CCCCCC}{} & 0.968 & \cellcolor[HTML]{CCCCCC}{} & 0.129 & \cellcolor[HTML]{CCCCCC}{} & 16.866 & 1.097\\
1929 & \cellcolor[HTML]{CCCCCC}{16.687} & 0.415 & \cellcolor[HTML]{CCCCCC}{} & 1.936 & \cellcolor[HTML]{CCCCCC}{} & 0.258 & \cellcolor[HTML]{CCCCCC}{} & 17.102 & 2.194\\
1930 & \cellcolor[HTML]{CCCCCC}{17.743} & 0.351 & \cellcolor[HTML]{CCCCCC}{} & 2.225 & \cellcolor[HTML]{CCCCCC}{} & 0.297 & \cellcolor[HTML]{CCCCCC}{} & 18.094 & 2.522\\
1931 & \cellcolor[HTML]{CCCCCC}{6.901} & 0.372 & \cellcolor[HTML]{CCCCCC}{} & 2.967 & \cellcolor[HTML]{CCCCCC}{} & 0.396 & \cellcolor[HTML]{CCCCCC}{} & 7.273 & 3.363\\
1932 & \cellcolor[HTML]{CCCCCC}{29.852} & 0.400 & \cellcolor[HTML]{CCCCCC}{} & 3.709 & \cellcolor[HTML]{CCCCCC}{} & 0.494 & \cellcolor[HTML]{CCCCCC}{} & 30.252 & 4.203\\
1933 & \cellcolor[HTML]{CCCCCC}{4.689} & 0.662 & \cellcolor[HTML]{CCCCCC}{} & 4.450 & \cellcolor[HTML]{CCCCCC}{} & 0.593 & \cellcolor[HTML]{CCCCCC}{} & 5.351 & 5.043\\
1934 & \cellcolor[HTML]{CCCCCC}{15.902} & 0.538 & \cellcolor[HTML]{CCCCCC}{} & 5.192 & \cellcolor[HTML]{CCCCCC}{} & 0.692 & \cellcolor[HTML]{CCCCCC}{} & 16.440 & 5.884\\
1935 & \cellcolor[HTML]{CCCCCC}{22.085} & 0.516 & \cellcolor[HTML]{CCCCCC}{} & 5.934 & \cellcolor[HTML]{CCCCCC}{} & 0.791 & \cellcolor[HTML]{CCCCCC}{} & 22.601 & 6.725\\
1936 & \cellcolor[HTML]{CCCCCC}{22.880} & 0.361 & \cellcolor[HTML]{CCCCCC}{} & 6.676 & \cellcolor[HTML]{CCCCCC}{} & 0.890 & \cellcolor[HTML]{CCCCCC}{} & 23.241 & 7.566\\
1937 & \cellcolor[HTML]{CCCCCC}{23.651} & 0.642 & \cellcolor[HTML]{CCCCCC}{} & 7.913 & \cellcolor[HTML]{CCCCCC}{} & 1.055 & \cellcolor[HTML]{CCCCCC}{} & 24.293 & 8.968\\
1938 & \cellcolor[HTML]{CCCCCC}{15.579} & 0.683 & \cellcolor[HTML]{CCCCCC}{} & 7.783 & \cellcolor[HTML]{CCCCCC}{} & 1.037 & \cellcolor[HTML]{CCCCCC}{} & 16.262 & 8.820\\
1939 & \cellcolor[HTML]{CCCCCC}{15.410} & 0.964 & \cellcolor[HTML]{CCCCCC}{} & 6.806 & \cellcolor[HTML]{CCCCCC}{} & 0.907 & \cellcolor[HTML]{CCCCCC}{} & 16.374 & 7.713\\
1940 & \cellcolor[HTML]{CCCCCC}{19.535} & 0.711 & \cellcolor[HTML]{CCCCCC}{} & 9.802 & \cellcolor[HTML]{CCCCCC}{} & 1.307 & \cellcolor[HTML]{CCCCCC}{} & 20.246 & 11.109\\
1941 & \cellcolor[HTML]{CCCCCC}{22.414} & 0.575 & \cellcolor[HTML]{CCCCCC}{} & 9.059 & \cellcolor[HTML]{CCCCCC}{} & 1.208 & \cellcolor[HTML]{CCCCCC}{} & 22.989 & 10.267\\
1942 & \cellcolor[HTML]{CCCCCC}{8.129} & 0.205 & \cellcolor[HTML]{CCCCCC}{} & 4.812 & \cellcolor[HTML]{CCCCCC}{} & 0.641 & \cellcolor[HTML]{CCCCCC}{} & 8.334 & 5.453\\
1943 & \cellcolor[HTML]{CCCCCC}{8.953} & 2.123 & \cellcolor[HTML]{CCCCCC}{} & 4.602 & \cellcolor[HTML]{CCCCCC}{} & 0.613 & \cellcolor[HTML]{CCCCCC}{} & 11.076 & 5.215\\
1944 & \cellcolor[HTML]{CCCCCC}{11.685} & 6.952 & \cellcolor[HTML]{CCCCCC}{} & 3.779 & \cellcolor[HTML]{CCCCCC}{} & 0.504 & \cellcolor[HTML]{CCCCCC}{} & 18.637 & 4.283\\
1945 & \cellcolor[HTML]{CCCCCC}{25.632} & 13.081 & \cellcolor[HTML]{CCCCCC}{} & 5.038 & \cellcolor[HTML]{CCCCCC}{} & 0.672 & \cellcolor[HTML]{CCCCCC}{} & 38.713 & 5.710\\
1946 & \cellcolor[HTML]{CCCCCC}{26.460} & 9.621 & \cellcolor[HTML]{CCCCCC}{} & 8.672 & \cellcolor[HTML]{CCCCCC}{} & 1.156 & \cellcolor[HTML]{CCCCCC}{} & 36.081 & 9.828\\
1947 & \cellcolor[HTML]{CCCCCC}{8.080} & 6.436 & \cellcolor[HTML]{CCCCCC}{} & 6.860 & \cellcolor[HTML]{CCCCCC}{} & 0.914 & \cellcolor[HTML]{CCCCCC}{} & 14.516 & 7.774\\
1948 & \cellcolor[HTML]{CCCCCC}{15.948} & 4.601 & \cellcolor[HTML]{CCCCCC}{} & 13.691 & \cellcolor[HTML]{CCCCCC}{} & 1.825 & \cellcolor[HTML]{CCCCCC}{} & 20.549 & 15.516\\
1949 & \cellcolor[HTML]{CCCCCC}{11.403} & 3.289 & \cellcolor[HTML]{CCCCCC}{} & 17.744 & \cellcolor[HTML]{CCCCCC}{} & 2.365 & \cellcolor[HTML]{CCCCCC}{} & 14.692 & 20.109\\
1950 & \cellcolor[HTML]{CCCCCC}{20.761} & 3.513 & \cellcolor[HTML]{CCCCCC}{} & 21.624 & \cellcolor[HTML]{CCCCCC}{} & 2.883 & \cellcolor[HTML]{CCCCCC}{} & 24.274 & 24.507\\
1951 & \cellcolor[HTML]{CCCCCC}{50.232} & 8.231 & \cellcolor[HTML]{CCCCCC}{} & 24.697 & \cellcolor[HTML]{CCCCCC}{} & 4.297 & \cellcolor[HTML]{CCCCCC}{} & 58.463 & 28.994\\
1952 & \cellcolor[HTML]{CCCCCC}{25.828} & 6.312 & \cellcolor[HTML]{CCCCCC}{} & 21.489 & \cellcolor[HTML]{CCCCCC}{} & 3.739 & \cellcolor[HTML]{CCCCCC}{} & 32.140 & 25.228\\
1953 & \cellcolor[HTML]{CCCCCC}{12.207} & 6.873 & \cellcolor[HTML]{CCCCCC}{} & 18.300 & \cellcolor[HTML]{CCCCCC}{} & 3.184 & \cellcolor[HTML]{CCCCCC}{} & 19.080 & 21.484\\
1954 & \cellcolor[HTML]{CCCCCC}{15.970} & 11.603 & \cellcolor[HTML]{CCCCCC}{} & 22.750 & \cellcolor[HTML]{CCCCCC}{} & 3.958 & \cellcolor[HTML]{CCCCCC}{} & 27.573 & 26.708\\
1955 & \cellcolor[HTML]{CCCCCC}{12.291} & 33.959 & \cellcolor[HTML]{CCCCCC}{} & 27.121 & \cellcolor[HTML]{CCCCCC}{} & 4.719 & \cellcolor[HTML]{CCCCCC}{} & 46.250 & 31.840\\
1956 & \cellcolor[HTML]{CCCCCC}{18.960} & 13.532 & \cellcolor[HTML]{CCCCCC}{} & 30.283 & \cellcolor[HTML]{CCCCCC}{} & 5.269 & \cellcolor[HTML]{CCCCCC}{} & 32.492 & 35.552\\
1957 & \cellcolor[HTML]{CCCCCC}{22.741} & 16.798 & \cellcolor[HTML]{CCCCCC}{} & 33.760 & \cellcolor[HTML]{CCCCCC}{} & 5.140 & \cellcolor[HTML]{CCCCCC}{} & 39.539 & 38.900\\
1958 & \cellcolor[HTML]{CCCCCC}{29.220} & 17.213 & \cellcolor[HTML]{CCCCCC}{} & 48.275 & \cellcolor[HTML]{CCCCCC}{} & 8.438 & \cellcolor[HTML]{CCCCCC}{} & 46.433 & 56.713\\
1959 & \cellcolor[HTML]{CCCCCC}{13.554} & 11.745 & \cellcolor[HTML]{CCCCCC}{} & 43.326 & \cellcolor[HTML]{CCCCCC}{} & 7.054 & \cellcolor[HTML]{CCCCCC}{} & 25.299 & 50.380\\
1960 & \cellcolor[HTML]{CCCCCC}{14.338} & 12.582 & \cellcolor[HTML]{CCCCCC}{} & 28.609 & \cellcolor[HTML]{CCCCCC}{} & 5.462 & \cellcolor[HTML]{CCCCCC}{} & 26.920 & 34.071\\
1961 & \cellcolor[HTML]{CCCCCC}{13.131} & 9.555 & \cellcolor[HTML]{CCCCCC}{} & 21.774 & \cellcolor[HTML]{CCCCCC}{} & 4.127 & \cellcolor[HTML]{CCCCCC}{} & 22.686 & 25.901\\
1962 & \cellcolor[HTML]{CCCCCC}{15.768} & 8.377 & \cellcolor[HTML]{CCCCCC}{} & 27.549 & \cellcolor[HTML]{CCCCCC}{} & 6.925 & \cellcolor[HTML]{CCCCCC}{} & 24.145 & 34.474\\
1963 & \cellcolor[HTML]{CCCCCC}{22.174} & 10.624 & \cellcolor[HTML]{CCCCCC}{} & 25.949 & \cellcolor[HTML]{CCCCCC}{} & 9.265 & \cellcolor[HTML]{CCCCCC}{} & 32.798 & 35.214\\
1964 & \cellcolor[HTML]{CCCCCC}{18.474} & 6.040 & \cellcolor[HTML]{CCCCCC}{} & 20.317 & \cellcolor[HTML]{CCCCCC}{} & 10.292 & \cellcolor[HTML]{CCCCCC}{} & 24.514 & 30.609\\
1965 & \cellcolor[HTML]{CCCCCC}{18.676} & 6.526 & \cellcolor[HTML]{CCCCCC}{} & 31.612 & \cellcolor[HTML]{CCCCCC}{} & 14.553 & \cellcolor[HTML]{CCCCCC}{} & 25.202 & 46.165\\
1966 & \cellcolor[HTML]{CCCCCC}{23.120} & 6.256 & \cellcolor[HTML]{CCCCCC}{} & 30.545 & \cellcolor[HTML]{CCCCCC}{} & 17.182 & \cellcolor[HTML]{CCCCCC}{} & 29.376 & 47.727\\
1967 & \cellcolor[HTML]{CCCCCC}{33.441} & 8.818 & \cellcolor[HTML]{CCCCCC}{} & 23.824 & \cellcolor[HTML]{CCCCCC}{} & 19.300 & \cellcolor[HTML]{CCCCCC}{} & 42.259 & 43.124\\
1968 & \cellcolor[HTML]{CCCCCC}{15.824} & 9.849 & \cellcolor[HTML]{CCCCCC}{} & 30.533 & \cellcolor[HTML]{CCCCCC}{} & 21.820 & \cellcolor[HTML]{CCCCCC}{} & 25.673 & 52.353\\
1969 & \cellcolor[HTML]{CCCCCC}{18.957} & 14.557 & \cellcolor[HTML]{CCCCCC}{} & 29.109 & \cellcolor[HTML]{CCCCCC}{} & 24.365 & \cellcolor[HTML]{CCCCCC}{} & 33.514 & 53.474\\
1970 & \cellcolor[HTML]{CCCCCC}{38.364} & 17.340 & \cellcolor[HTML]{CCCCCC}{} & 29.796 & \cellcolor[HTML]{CCCCCC}{} & 28.201 & \cellcolor[HTML]{CCCCCC}{} & 55.704 & 57.997\\
1971 & \cellcolor[HTML]{CCCCCC}{44.496} & 14.437 & \cellcolor[HTML]{CCCCCC}{} & 27.202 & \cellcolor[HTML]{CCCCCC}{} & 28.446 & \cellcolor[HTML]{CCCCCC}{} & 58.933 & 55.648\\
1972 & \cellcolor[HTML]{CCCCCC}{50.334} & 21.573 & \cellcolor[HTML]{CCCCCC}{} & 38.963 & \cellcolor[HTML]{CCCCCC}{} & 32.994 & \cellcolor[HTML]{CCCCCC}{} & 71.907 & 71.957\\
1973 & \cellcolor[HTML]{CCCCCC}{64.800} & 24.939 & \cellcolor[HTML]{CCCCCC}{} & 51.944 & \cellcolor[HTML]{CCCCCC}{} & 37.561 & \cellcolor[HTML]{CCCCCC}{} & 89.739 & 89.505\\
1974 & \cellcolor[HTML]{CCCCCC}{86.226} & 26.708 & \cellcolor[HTML]{CCCCCC}{} & 46.594 & \cellcolor[HTML]{CCCCCC}{} & 40.375 & \cellcolor[HTML]{CCCCCC}{} & 112.934 & 86.969\\
1975 & \cellcolor[HTML]{CCCCCC}{69.642} & 34.322 & \cellcolor[HTML]{CCCCCC}{} & 45.448 & \cellcolor[HTML]{CCCCCC}{} & 42.019 & \cellcolor[HTML]{CCCCCC}{} & 103.964 & 87.467\\
1976 & \cellcolor[HTML]{CCCCCC}{93.143} & 41.767 & \cellcolor[HTML]{CCCCCC}{} & 47.441 & \cellcolor[HTML]{CCCCCC}{} & 45.682 & \cellcolor[HTML]{CCCCCC}{} & 134.910 & 93.123\\
1977 & \cellcolor[HTML]{CCCCCC}{80.740} & 36.816 & \cellcolor[HTML]{CCCCCC}{} & 36.765 & \cellcolor[HTML]{CCCCCC}{} & 46.300 & \cellcolor[HTML]{CCCCCC}{} & 117.556 & 83.065\\
1978 & \cellcolor[HTML]{CCCCCC}{147.227} & 24.299 & \cellcolor[HTML]{CCCCCC}{} & 28.583 & \cellcolor[HTML]{CCCCCC}{} & 46.998 & \cellcolor[HTML]{CCCCCC}{} & 171.526 & 75.581\\
1979 & \cellcolor[HTML]{CCCCCC}{207.084} & 24.304 & \cellcolor[HTML]{CCCCCC}{} & 39.460 & \cellcolor[HTML]{CCCCCC}{} & 50.204 & \cellcolor[HTML]{CCCCCC}{} & 231.388 & 89.664\\
1980 & \cellcolor[HTML]{CCCCCC}{156.720} & 51.306 & \cellcolor[HTML]{CCCCCC}{} & 38.464 & \cellcolor[HTML]{CCCCCC}{1.793} & 52.516 & \cellcolor[HTML]{CCCCCC}{0.675} & 208.026 & 93.448\\
1981 & \cellcolor[HTML]{CCCCCC}{143.396} & 17.770 & \cellcolor[HTML]{CCCCCC}{0.011} & 10.442 & \cellcolor[HTML]{CCCCCC}{0.648} & 16.625 & \cellcolor[HTML]{CCCCCC}{0.789} & 161.177 & 28.504\\
1982 & \cellcolor[HTML]{CCCCCC}{212.016} & 14.963 & \cellcolor[HTML]{CCCCCC}{0.002} & 37.100 & \cellcolor[HTML]{CCCCCC}{2.302} & 27.914 & \cellcolor[HTML]{CCCCCC}{1.325} & 226.981 & 68.641\\
1983 & \cellcolor[HTML]{CCCCCC}{42.209} & 26.913 & \cellcolor[HTML]{CCCCCC}{3.068} & 17.704 & \cellcolor[HTML]{CCCCCC}{1.098} & 28.437 & \cellcolor[HTML]{CCCCCC}{1.350} & 72.190 & 48.589\\
1984 & \cellcolor[HTML]{CCCCCC}{0.569} & 41.934 & \cellcolor[HTML]{CCCCCC}{5.704} & 13.350 & \cellcolor[HTML]{CCCCCC}{0.828} & 52.799 & \cellcolor[HTML]{CCCCCC}{2.506} & 48.207 & 69.483\\
1985 & \cellcolor[HTML]{CCCCCC}{0.823} & 42.702 & \cellcolor[HTML]{CCCCCC}{12.506} & 18.446 & \cellcolor[HTML]{CCCCCC}{1.144} & 40.652 & \cellcolor[HTML]{CCCCCC}{1.930} & 56.031 & 62.172\\
1986 & \cellcolor[HTML]{CCCCCC}{30.693} & 4.154 & \cellcolor[HTML]{CCCCCC}{31.418} & 10.063 & \cellcolor[HTML]{CCCCCC}{0.624} & 53.422 & \cellcolor[HTML]{CCCCCC}{2.536} & 66.265 & 66.645\\
1987 & \cellcolor[HTML]{CCCCCC}{29.064} & 44.472 & \cellcolor[HTML]{CCCCCC}{65.800} & 50.415 & \cellcolor[HTML]{CCCCCC}{3.128} & 55.851 & \cellcolor[HTML]{CCCCCC}{2.651} & 139.336 & 112.045\\
1988 & \cellcolor[HTML]{CCCCCC}{55.941} & 21.304 & \cellcolor[HTML]{CCCCCC}{49.169} & 77.850 & \cellcolor[HTML]{CCCCCC}{4.830} & 107.405 & \cellcolor[HTML]{CCCCCC}{5.098} & 126.414 & 195.183\\
1989 & \cellcolor[HTML]{CCCCCC}{34.448} & 2.654 & \cellcolor[HTML]{CCCCCC}{6.120} & 16.690 & \cellcolor[HTML]{CCCCCC}{1.036} & 68.454 & \cellcolor[HTML]{CCCCCC}{3.249} & 43.222 & 89.429\\
1990 & \cellcolor[HTML]{CCCCCC}{61.399} & 1.485 & \cellcolor[HTML]{CCCCCC}{60.728} & 45.553 & \cellcolor[HTML]{CCCCCC}{2.826} & 91.362 & \cellcolor[HTML]{CCCCCC}{4.337} & 123.612 & 144.078\\
1991 & \cellcolor[HTML]{CCCCCC}{126.397} & 0.561 & \cellcolor[HTML]{CCCCCC}{13.817} & 43.835 & \cellcolor[HTML]{CCCCCC}{2.720} & 94.794 & \cellcolor[HTML]{CCCCCC}{4.499} & 140.775 & 145.848\\
1992 & \cellcolor[HTML]{CCCCCC}{103.910} & 10.272 & \cellcolor[HTML]{CCCCCC}{0.328} & 42.118 & \cellcolor[HTML]{CCCCCC}{2.613} & 98.227 & \cellcolor[HTML]{CCCCCC}{4.662} & 114.510 & 147.620\\
1993 & \cellcolor[HTML]{CCCCCC}{150.587} & 21.365 & \cellcolor[HTML]{CCCCCC}{19.802} & 40.401 & \cellcolor[HTML]{CCCCCC}{2.507} & 125.643 & \cellcolor[HTML]{CCCCCC}{5.964} & 191.754 & 174.515\\
1994 & \cellcolor[HTML]{CCCCCC}{85.276} & 14.598 & \cellcolor[HTML]{CCCCCC}{10.992} & 38.684 & \cellcolor[HTML]{CCCCCC}{2.400} & 77.675 & \cellcolor[HTML]{CCCCCC}{3.687} & 110.866 & 122.446\\
1995 & \cellcolor[HTML]{CCCCCC}{49.647} & 15.606 & \cellcolor[HTML]{CCCCCC}{10.994} & 36.967 & \cellcolor[HTML]{CCCCCC}{2.294} & 44.727 & \cellcolor[HTML]{CCCCCC}{2.123} & 76.247 & 86.111\\
1996 & \cellcolor[HTML]{CCCCCC}{63.809} & 10.343 & \cellcolor[HTML]{CCCCCC}{9.314} & 24.106 & \cellcolor[HTML]{CCCCCC}{1.496} & 23.840 & \cellcolor[HTML]{CCCCCC}{1.132} & 83.466 & 50.574\\
1997 & \cellcolor[HTML]{CCCCCC}{64.022} & 13.592 & \cellcolor[HTML]{CCCCCC}{7.143} & 46.393 & \cellcolor[HTML]{CCCCCC}{2.878} & 25.198 & \cellcolor[HTML]{CCCCCC}{1.196} & 84.757 & 75.665\\
1998 & \cellcolor[HTML]{CCCCCC}{44.000} & 27.858 & \cellcolor[HTML]{CCCCCC}{6.333} & 15.063 & \cellcolor[HTML]{CCCCCC}{0.935} & 31.625 & \cellcolor[HTML]{CCCCCC}{1.501} & 78.191 & 49.124\\
1999 & \cellcolor[HTML]{CCCCCC}{34.011} & 8.816 & \cellcolor[HTML]{CCCCCC}{0.012} & 34.743 & \cellcolor[HTML]{CCCCCC}{2.156} & 50.273 & \cellcolor[HTML]{CCCCCC}{2.386} & 42.839 & 89.558\\
2000 & \cellcolor[HTML]{CCCCCC}{12.629} & 0.460 & \cellcolor[HTML]{CCCCCC}{0.017} & 35.685 & \cellcolor[HTML]{CCCCCC}{2.214} & 42.306 & \cellcolor[HTML]{CCCCCC}{2.008} & 13.106 & 82.213\\
2001 & \cellcolor[HTML]{CCCCCC}{11.286} & 2.752 & \cellcolor[HTML]{CCCCCC}{0.103} & 27.561 & \cellcolor[HTML]{CCCCCC}{1.710} & 28.825 & \cellcolor[HTML]{CCCCCC}{1.368} & 14.141 & 59.464\\
2002 & \cellcolor[HTML]{CCCCCC}{6.487} & 0.160 & \cellcolor[HTML]{CCCCCC}{} & 16.260 & \cellcolor[HTML]{CCCCCC}{1.009} & 49.219 & \cellcolor[HTML]{CCCCCC}{2.336} & 6.647 & 68.824\\
2003 & \cellcolor[HTML]{CCCCCC}{5.829} & 0.176 & \cellcolor[HTML]{CCCCCC}{} & 20.646 & \cellcolor[HTML]{CCCCCC}{1.281} & 148.420 & \cellcolor[HTML]{CCCCCC}{7.045} & 6.005 & 177.392\\
2004 & \cellcolor[HTML]{CCCCCC}{10.123} & 0.154 & \cellcolor[HTML]{CCCCCC}{} & 36.496 & \cellcolor[HTML]{CCCCCC}{2.264} & 36.835 & \cellcolor[HTML]{CCCCCC}{1.748} & 10.277 & 77.343\\
2005 & \cellcolor[HTML]{CCCCCC}{11.476} & 0.090 & \cellcolor[HTML]{CCCCCC}{0.007} & 94.576 & \cellcolor[HTML]{CCCCCC}{1.242} & 46.584 & \cellcolor[HTML]{CCCCCC}{1.067} & 11.573 & 143.469\\
2006 & \cellcolor[HTML]{CCCCCC}{12.101} & 0.001 & \cellcolor[HTML]{CCCCCC}{} & 59.900 & \cellcolor[HTML]{CCCCCC}{0.029} & 56.800 & \cellcolor[HTML]{CCCCCC}{1.474} & 12.102 & 118.203\\
2007 & \cellcolor[HTML]{CCCCCC}{13.314} &  & \cellcolor[HTML]{CCCCCC}{} & 45.292 & \cellcolor[HTML]{CCCCCC}{0.096} & 46.542 & \cellcolor[HTML]{CCCCCC}{0.779} & 13.314 & 92.709\\
2008 & \cellcolor[HTML]{CCCCCC}{9.778} & 0.164 & \cellcolor[HTML]{CCCCCC}{} & 17.789 & \cellcolor[HTML]{CCCCCC}{0.063} & 29.531 & \cellcolor[HTML]{CCCCCC}{2.392} & 9.942 & 49.775\\
2009 & \cellcolor[HTML]{CCCCCC}{7.058} & 0.029 & \cellcolor[HTML]{CCCCCC}{} & 23.383 & \cellcolor[HTML]{CCCCCC}{1.247} & 40.157 & \cellcolor[HTML]{CCCCCC}{1.430} & 7.087 & 66.217\\
2010 & \cellcolor[HTML]{CCCCCC}{6.939} & 0.010 & \cellcolor[HTML]{CCCCCC}{} & 52.499 & \cellcolor[HTML]{CCCCCC}{0.620} & 29.975 & \cellcolor[HTML]{CCCCCC}{1.284} & 6.949 & 84.378\\
2011 & \cellcolor[HTML]{CCCCCC}{10.047} &  & \cellcolor[HTML]{CCCCCC}{} & 38.094 & \cellcolor[HTML]{CCCCCC}{1.250} & 40.223 & \cellcolor[HTML]{CCCCCC}{0.618} & 10.047 & 80.185\\
2012 & \cellcolor[HTML]{CCCCCC}{9.400} & 0.006 & \cellcolor[HTML]{CCCCCC}{} & 35.352 & \cellcolor[HTML]{CCCCCC}{0.853} & 35.099 & \cellcolor[HTML]{CCCCCC}{0.798} & 9.406 & 72.102\\
2013 & \cellcolor[HTML]{CCCCCC}{13.845} & 0.005 & \cellcolor[HTML]{CCCCCC}{} & 20.502 & \cellcolor[HTML]{CCCCCC}{0.402} & 35.159 & \cellcolor[HTML]{CCCCCC}{0.409} & 13.850 & 56.472\\
2014 & \cellcolor[HTML]{CCCCCC}{14.139} & 0.015 & \cellcolor[HTML]{CCCCCC}{0.023} & 19.670 & \cellcolor[HTML]{CCCCCC}{0.163} & 35.302 & \cellcolor[HTML]{CCCCCC}{0.833} & 14.177 & 55.968\\
2015 & \cellcolor[HTML]{CCCCCC}{18.172} & 0.410 & \cellcolor[HTML]{CCCCCC}{0.010} & 36.213 & \cellcolor[HTML]{CCCCCC}{0.457} & 52.952 & \cellcolor[HTML]{CCCCCC}{0.945} & 18.592 & 90.567\\
2016 & \cellcolor[HTML]{CCCCCC}{13.271} & 0.094 & \cellcolor[HTML]{CCCCCC}{} & 34.281 & \cellcolor[HTML]{CCCCCC}{0.786} & 48.712 & \cellcolor[HTML]{CCCCCC}{1.071} & 13.365 & 84.850\\
2017 & \cellcolor[HTML]{CCCCCC}{14.226} & 0.062 & \cellcolor[HTML]{CCCCCC}{0.002} & 62.594 & \cellcolor[HTML]{CCCCCC}{1.528} & 42.352 & \cellcolor[HTML]{CCCCCC}{1.110} & 14.290 & 107.584\\
2018 & \cellcolor[HTML]{CCCCCC}{19.041} & 0.619 & \cellcolor[HTML]{CCCCCC}{} & 60.220 & \cellcolor[HTML]{CCCCCC}{0.734} & 53.136 & \cellcolor[HTML]{CCCCCC}{1.250} & 19.660 & 115.340\\
2019 & \cellcolor[HTML]{CCCCCC}{19.593} & 0.039 & \cellcolor[HTML]{CCCCCC}{} & 64.423 & \cellcolor[HTML]{CCCCCC}{1.118} & 69.455 & \cellcolor[HTML]{CCCCCC}{2.273} & 19.632 & 137.269\\
2020 & \cellcolor[HTML]{CCCCCC}{19.930} & 0.017 & \cellcolor[HTML]{CCCCCC}{} & 39.824 & \cellcolor[HTML]{CCCCCC}{0.531} & 49.817 & \cellcolor[HTML]{CCCCCC}{0.882} & 19.947 & 91.054\\*
\end{longtable}
\leavevmode\tagmcend\tagstructend\par
\endgroup{}

\vspace{2in}

\begin{landscape}\begin{table}

\caption{\label{tab:catch-source}Sources of landings for the commercial and recreational fleets.  
  The interpolated values were interpolated by J. Field (SWFSC). The reconstruction 
  refers to the commercial and recreational catch recontructions in Ralston et al. 
  (2010). Detailed descriptions of the sources are in the text.}
\centering
\begin{tabular}[t]{>{\raggedright\arraybackslash}p{3cm}lllll}
\toprule
Fleet & Interpolated & Reconstruction & CALCOM & MRFSS & CRFS\\
\midrule
\cellcolor{gray!6}{COM\_HKL\_1} & \cellcolor{gray!6}{1875-1915} & \cellcolor{gray!6}{1916-1968} & \cellcolor{gray!6}{1978-2020} & \cellcolor{gray!6}{} & \cellcolor{gray!6}{}\\
COM\_TWL\_2 &  & 1916-1968 & 1978-2020 &  & \\
\cellcolor{gray!6}{COM\_NET\_3} & \cellcolor{gray!6}{} & \cellcolor{gray!6}{} & \cellcolor{gray!6}{1981-2017} & \cellcolor{gray!6}{} & \cellcolor{gray!6}{}\\
REC\_PC\_4 &  & 1928-1980 &  & 1981-2003 & 2004-2020\\
\cellcolor{gray!6}{REC\_PC\_DIS\_5} & \cellcolor{gray!6}{} & \cellcolor{gray!6}{} & \cellcolor{gray!6}{} & \cellcolor{gray!6}{1980-2003} & \cellcolor{gray!6}{2004-2020}\\
REC\_PR\_6 &  & 1928-1980 &  & 1981-2003 & 2004-2020\\
\cellcolor{gray!6}{REC\_PR\_DIS\_7} & \cellcolor{gray!6}{} & \cellcolor{gray!6}{} & \cellcolor{gray!6}{} & \cellcolor{gray!6}{1980-2003} & \cellcolor{gray!6}{2004-2020}\\
\bottomrule
\end{tabular}
\end{table}
\end{landscape}

\begin{landscape}\begingroup\fontsize{10}{12}\selectfont

\tagstructbegin{tag=Table}\tagmcbegin{tag=Table}
\begin{longtable}[t]{>{\raggedleft\arraybackslash}p{1.2cm}>{\raggedleft\arraybackslash}p{1.2cm}>{\raggedleft\arraybackslash}p{1.2cm}>{\raggedleft\arraybackslash}p{1.2cm}>{\raggedleft\arraybackslash}p{1.2cm}>{\raggedleft\arraybackslash}p{1.5cm}>{\raggedleft\arraybackslash}p{1.5cm}>{\raggedleft\arraybackslash}p{1.5cm}>{\raggedright\arraybackslash}p{1.5cm}>{\raggedleft\arraybackslash}p{1.5cm}>{\raggedright\arraybackslash}p{1.5cm}>{\raggedleft\arraybackslash}p{1.5cm}}
\caption{\label{tab:com-allocate}Re-apportionment of the Ralson et al. (2010) commercial catch 
  reconstruction to north and south of Point Conception. San Luis Obispo county 
  landings were assigned to southern California.}\\
\toprule
\multicolumn{1}{c}{ } & \multicolumn{1}{c}{FB 105} & \multicolumn{3}{c}{NMFS ERD live-access server} & \multicolumn{1}{c}{ } & \multicolumn{2}{c}{Major SLO Ports} & \multicolumn{4}{c}{ } \\
\cmidrule(l{3pt}r{3pt}){2-2} \cmidrule(l{3pt}r{3pt}){3-5} \cmidrule(l{3pt}r{3pt}){7-8}
\multicolumn{1}{>{\centering\arraybackslash}p{1.2cm}}{Year} & \multicolumn{1}{>{\centering\arraybackslash}p{1.2cm}}{Southern} & \multicolumn{1}{>{\centering\arraybackslash}p{1.2cm}}{San Diego} & \multicolumn{1}{>{\centering\arraybackslash}p{1.2cm}}{Los Angeles} & \multicolumn{1}{>{\centering\arraybackslash}p{1.2cm}}{Santa Barbara} & \multicolumn{1}{>{\centering\arraybackslash}p{1.5cm}}{Foreign catch landed in U.S.} & \multicolumn{1}{>{\centering\arraybackslash}p{1.5cm}}{Morro Bay} & \multicolumn{1}{>{\centering\arraybackslash}p{1.5cm}}{Avila} & \multicolumn{1}{>{\centering\arraybackslash}p{1.5cm}}{Source of SLO catch} & \multicolumn{1}{>{\centering\arraybackslash}p{1.5cm}}{Adjusted Santa Barbara} & \multicolumn{1}{>{\centering\arraybackslash}p{1.5cm}}{Ratio years} & \multicolumn{1}{>{\centering\arraybackslash}p{1.5cm}}{Percent Area 6 So. of Pt. Conc}\\
\midrule
\endfirsthead
\caption[]{\label{tab:com-allocate}Re-apportionment of the Ralson et al. (2010) commercial cat Re-apportionment of the commercial catch reconstruction (continued).}\\
\toprule
\multicolumn{1}{c}{ } & \multicolumn{1}{c}{FB 105} & \multicolumn{3}{c}{NMFS ERD live-access server} & \multicolumn{1}{c}{ } & \multicolumn{2}{c}{Major SLO Ports} & \multicolumn{4}{c}{ } \\
\cmidrule(l{3pt}r{3pt}){2-2} \cmidrule(l{3pt}r{3pt}){3-5} \cmidrule(l{3pt}r{3pt}){7-8}
\multicolumn{1}{>{\centering\arraybackslash}p{1.2cm}}{Year} & \multicolumn{1}{>{\centering\arraybackslash}p{1.2cm}}{Southern} & \multicolumn{1}{>{\centering\arraybackslash}p{1.2cm}}{San Diego} & \multicolumn{1}{>{\centering\arraybackslash}p{1.2cm}}{Los Angeles} & \multicolumn{1}{>{\centering\arraybackslash}p{1.2cm}}{Santa Barbara} & \multicolumn{1}{>{\centering\arraybackslash}p{1.5cm}}{Foreign catch landed in U.S.} & \multicolumn{1}{>{\centering\arraybackslash}p{1.5cm}}{Morro Bay} & \multicolumn{1}{>{\centering\arraybackslash}p{1.5cm}}{Avila} & \multicolumn{1}{>{\centering\arraybackslash}p{1.5cm}}{Source of SLO catch} & \multicolumn{1}{>{\centering\arraybackslash}p{1.5cm}}{Adjusted Santa Barbara} & \multicolumn{1}{>{\centering\arraybackslash}p{1.5cm}}{Ratio years} & \multicolumn{1}{>{\centering\arraybackslash}p{1.5cm}}{Percent Area 6 So. of Pt. Conc}\\
\midrule
\endhead

\endfoot
\bottomrule
\endlastfoot
\cellcolor{gray!6}{1916} & \cellcolor{gray!6}{966.622} & \cellcolor{gray!6}{330.180} & \cellcolor{gray!6}{620.062} & \cellcolor{gray!6}{} & \cellcolor{gray!6}{7.111} & \cellcolor{gray!6}{} & \cellcolor{gray!6}{} & \cellcolor{gray!6}{ratio} & \cellcolor{gray!6}{9.269} & \cellcolor{gray!6}{1928-33} & \cellcolor{gray!6}{0.328}\\
1917 & 1559.699 & 532.764 & 1000.505 &  & 11.474 &  &  & ratio & 14.956 & 1928-33 & 0.328\\
\cellcolor{gray!6}{1918} & \cellcolor{gray!6}{1422.288} & \cellcolor{gray!6}{485.827} & \cellcolor{gray!6}{912.360} & \cellcolor{gray!6}{} & \cellcolor{gray!6}{10.463} & \cellcolor{gray!6}{} & \cellcolor{gray!6}{} & \cellcolor{gray!6}{ratio} & \cellcolor{gray!6}{13.638} & \cellcolor{gray!6}{1928-33} & \cellcolor{gray!6}{0.328}\\
1919 & 850.462 & 290.502 & 545.548 &  & 6.257 &  &  & ratio & 8.155 & 1928-33 & 0.328\\
\cellcolor{gray!6}{1920} & \cellcolor{gray!6}{923.717} & \cellcolor{gray!6}{315.525} & \cellcolor{gray!6}{592.540} & \cellcolor{gray!6}{} & \cellcolor{gray!6}{6.796} & \cellcolor{gray!6}{} & \cellcolor{gray!6}{} & \cellcolor{gray!6}{ratio} & \cellcolor{gray!6}{8.857} & \cellcolor{gray!6}{1928-33} & \cellcolor{gray!6}{0.328}\\
\addlinespace
1921 & 806.935 & 275.634 & 517.627 &  & 5.936 &  &  & ratio & 7.738 & 1928-33 & 0.328\\
\cellcolor{gray!6}{1922} & \cellcolor{gray!6}{793.996} & \cellcolor{gray!6}{271.214} & \cellcolor{gray!6}{509.327} & \cellcolor{gray!6}{} & \cellcolor{gray!6}{5.841} & \cellcolor{gray!6}{} & \cellcolor{gray!6}{} & \cellcolor{gray!6}{ratio} & \cellcolor{gray!6}{7.614} & \cellcolor{gray!6}{1928-33} & \cellcolor{gray!6}{0.328}\\
1923 & 1063.847 & 363.390 & 682.429 &  & 7.826 &  &  & ratio & 10.201 & 1928-33 & 0.328\\
\cellcolor{gray!6}{1924} & \cellcolor{gray!6}{1426.244} & \cellcolor{gray!6}{487.178} & \cellcolor{gray!6}{914.897} & \cellcolor{gray!6}{} & \cellcolor{gray!6}{10.492} & \cellcolor{gray!6}{} & \cellcolor{gray!6}{} & \cellcolor{gray!6}{ratio} & \cellcolor{gray!6}{13.676} & \cellcolor{gray!6}{1928-33} & \cellcolor{gray!6}{0.328}\\
1925 & 1564.436 & 534.382 & 1003.544 &  & 11.509 &  &  & ratio & 15.001 & 1928-33 & 0.328\\
\addlinespace
\cellcolor{gray!6}{1926} & \cellcolor{gray!6}{1941.864} & \cellcolor{gray!6}{663.304} & \cellcolor{gray!6}{1245.654} & \cellcolor{gray!6}{} & \cellcolor{gray!6}{14.286} & \cellcolor{gray!6}{} & \cellcolor{gray!6}{} & \cellcolor{gray!6}{ratio} & \cellcolor{gray!6}{18.620} & \cellcolor{gray!6}{1928-33} & \cellcolor{gray!6}{0.328}\\
1927 & 1611.490 & 550.455 & 1033.728 &  & 11.855 &  &  & ratio & 15.452 & 1928-33 & 0.328\\
\cellcolor{gray!6}{1928} & \cellcolor{gray!6}{1373.499} & \cellcolor{gray!6}{554.760} & \cellcolor{gray!6}{769.848} & \cellcolor{gray!6}{46.650} & \cellcolor{gray!6}{2.240} & \cellcolor{gray!6}{17.445} & \cellcolor{gray!6}{13.895} & \cellcolor{gray!6}{ratio} & \cellcolor{gray!6}{15.310} & \cellcolor{gray!6}{1949-51} & \cellcolor{gray!6}{0.328}\\
1929 & 1389.528 & 641.799 & 687.264 & 44.600 & 15.864 & 16.678 & 13.285 & ratio & 14.637 & 1949-51 & 0.328\\
\cellcolor{gray!6}{1930} & \cellcolor{gray!6}{1415.632} & \cellcolor{gray!6}{477.907} & \cellcolor{gray!6}{906.133} & \cellcolor{gray!6}{21.152} & \cellcolor{gray!6}{10.439} & \cellcolor{gray!6}{7.910} & \cellcolor{gray!6}{6.300} & \cellcolor{gray!6}{ratio} & \cellcolor{gray!6}{6.942} & \cellcolor{gray!6}{1949-51} & \cellcolor{gray!6}{0.328}\\
\addlinespace
1931 & 1617.811 & 400.302 & 1182.352 & 30.906 & 4.252 & 11.557 & 9.206 & ratio & 10.143 & 1949-51 & 0.328\\
\cellcolor{gray!6}{1932} & \cellcolor{gray!6}{1135.482} & \cellcolor{gray!6}{298.471} & \cellcolor{gray!6}{797.365} & \cellcolor{gray!6}{34.762} & \cellcolor{gray!6}{4.885} & \cellcolor{gray!6}{12.999} & \cellcolor{gray!6}{10.354} & \cellcolor{gray!6}{ratio} & \cellcolor{gray!6}{11.408} & \cellcolor{gray!6}{1949-51} & \cellcolor{gray!6}{0.328}\\
1933 & 907.472 & 252.635 & 588.304 & 46.540 & 19.993 & 17.404 & 13.863 & ratio & 15.274 & 1949-51 & 0.328\\
\cellcolor{gray!6}{1934} & \cellcolor{gray!6}{857.005} & \cellcolor{gray!6}{129.533} & \cellcolor{gray!6}{510.376} & \cellcolor{gray!6}{127.600} & \cellcolor{gray!6}{89.495} & \cellcolor{gray!6}{47.716} & \cellcolor{gray!6}{38.007} & \cellcolor{gray!6}{ratio} & \cellcolor{gray!6}{41.877} & \cellcolor{gray!6}{1949-51} & \cellcolor{gray!6}{0.328}\\
1935 & 741.225 & 77.847 & 373.921 & 177.653 & 111.805 & 66.434 & 52.916 & ratio & 58.303 & 1949-51 & 0.328\\
\addlinespace
\cellcolor{gray!6}{1936} & \cellcolor{gray!6}{424.053} & \cellcolor{gray!6}{69.717} & \cellcolor{gray!6}{122.803} & \cellcolor{gray!6}{181.882} & \cellcolor{gray!6}{49.651} & \cellcolor{gray!6}{68.015} & \cellcolor{gray!6}{54.176} & \cellcolor{gray!6}{ratio} & \cellcolor{gray!6}{59.691} & \cellcolor{gray!6}{1949-51} & \cellcolor{gray!6}{0.328}\\
1937 & 460.648 & 65.181 & 156.838 & 166.262 & 72.367 & 62.174 & 49.523 & ratio & 54.565 & 1949-51 & 0.328\\
\cellcolor{gray!6}{1938} & \cellcolor{gray!6}{309.183} & \cellcolor{gray!6}{33.821} & \cellcolor{gray!6}{126.044} & \cellcolor{gray!6}{72.755} & \cellcolor{gray!6}{76.562} & \cellcolor{gray!6}{27.207} & \cellcolor{gray!6}{21.671} & \cellcolor{gray!6}{ratio} & \cellcolor{gray!6}{23.877} & \cellcolor{gray!6}{1949-51} & \cellcolor{gray!6}{0.328}\\
1939 & 389.656 & 92.008 & 140.829 & 91.190 & 65.629 & 34.101 & 27.162 & ratio & 29.927 & 1949-51 & 0.328\\
\cellcolor{gray!6}{1940} & \cellcolor{gray!6}{396.317} & \cellcolor{gray!6}{66.629} & \cellcolor{gray!6}{153.114} & \cellcolor{gray!6}{136.399} & \cellcolor{gray!6}{40.176} & \cellcolor{gray!6}{51.007} & \cellcolor{gray!6}{40.628} & \cellcolor{gray!6}{ratio} & \cellcolor{gray!6}{44.764} & \cellcolor{gray!6}{1949-51} & \cellcolor{gray!6}{0.328}\\
\addlinespace
1941 & 470.112 & 42.149 & 202.954 & 131.567 & 93.442 & 49.200 & 39.189 & ratio & 43.179 & 1949-51 & 0.328\\
\cellcolor{gray!6}{1942} & \cellcolor{gray!6}{192.964} & \cellcolor{gray!6}{10.126} & \cellcolor{gray!6}{74.461} & \cellcolor{gray!6}{38.266} & \cellcolor{gray!6}{70.112} & \cellcolor{gray!6}{14.310} & \cellcolor{gray!6}{11.398} & \cellcolor{gray!6}{ratio} & \cellcolor{gray!6}{12.558} & \cellcolor{gray!6}{1949-51} & \cellcolor{gray!6}{0.328}\\
1943 & 226.429 & 5.169 & 89.074 & 38.614 & 93.572 & 14.440 & 11.502 & ratio & 12.673 & 1949-51 & 0.328\\
\cellcolor{gray!6}{1944} & \cellcolor{gray!6}{43.382} & \cellcolor{gray!6}{4.630} & \cellcolor{gray!6}{10.338} & \cellcolor{gray!6}{22.144} & \cellcolor{gray!6}{6.270} & \cellcolor{gray!6}{8.281} & \cellcolor{gray!6}{6.596} & \cellcolor{gray!6}{ratio} & \cellcolor{gray!6}{7.267} & \cellcolor{gray!6}{1949-51} & \cellcolor{gray!6}{0.328}\\
1945 & 92.924 & 4.558 & 26.967 & 44.949 & 16.450 & 16.809 & 13.388 & ratio & 14.752 & 1949-51 & 0.328\\
\addlinespace
\cellcolor{gray!6}{1946} & \cellcolor{gray!6}{161.187} & \cellcolor{gray!6}{8.714} & \cellcolor{gray!6}{79.597} & \cellcolor{gray!6}{48.777} & \cellcolor{gray!6}{24.098} & \cellcolor{gray!6}{18.240} & \cellcolor{gray!6}{14.529} & \cellcolor{gray!6}{ratio} & \cellcolor{gray!6}{16.008} & \cellcolor{gray!6}{1949-51} & \cellcolor{gray!6}{0.328}\\
1947 & 185.457 & 8.786 & 131.603 & 26.850 & 18.218 & 10.041 & 7.998 & ratio & 8.812 & 1949-51 & 0.328\\
\cellcolor{gray!6}{1948} & \cellcolor{gray!6}{287.675} & \cellcolor{gray!6}{24.117} & \cellcolor{gray!6}{200.075} & \cellcolor{gray!6}{36.114} & \cellcolor{gray!6}{27.369} & \cellcolor{gray!6}{13.505} & \cellcolor{gray!6}{10.757} & \cellcolor{gray!6}{ratio} & \cellcolor{gray!6}{11.852} & \cellcolor{gray!6}{1949-51} & \cellcolor{gray!6}{0.328}\\
1949 & 412.088 & 36.639 & 258.883 & 61.876 & 54.690 & 20.622 & 22.953 & FB 80 & 18.301 &  & 0.296\\
\cellcolor{gray!6}{1950} & \cellcolor{gray!6}{427.871} & \cellcolor{gray!6}{33.670} & \cellcolor{gray!6}{294.001} & \cellcolor{gray!6}{85.959} & \cellcolor{gray!6}{14.241} & \cellcolor{gray!6}{41.230} & \cellcolor{gray!6}{28.680} & \cellcolor{gray!6}{FB 86} & \cellcolor{gray!6}{16.049} & \cellcolor{gray!6}{} & \cellcolor{gray!6}{0.187}\\
\addlinespace
1951 & 470.814 & 14.547 & 328.925 & 121.629 & 5.713 & 38.915 & 28.630 & FB 89 & 54.084 &  & 0.445\\
\cellcolor{gray!6}{1952} & \cellcolor{gray!6}{366.255} & \cellcolor{gray!6}{9.471} & \cellcolor{gray!6}{218.591} & \cellcolor{gray!6}{108.149} & \cellcolor{gray!6}{30.044} & \cellcolor{gray!6}{32.526} & \cellcolor{gray!6}{25.907} & \cellcolor{gray!6}{FB 95, ratio} & \cellcolor{gray!6}{49.716} & \cellcolor{gray!6}{1949-51} & \cellcolor{gray!6}{0.460}\\
1953 & 298.737 & 14.706 & 179.438 & 88.656 & 15.937 & 56.383 & 4.399 & FB 102, ratio & 27.874 & 1954-57 & 0.314\\
\cellcolor{gray!6}{1954} & \cellcolor{gray!6}{583.020} & \cellcolor{gray!6}{14.098} & \cellcolor{gray!6}{247.222} & \cellcolor{gray!6}{263.088} & \cellcolor{gray!6}{58.612} & \cellcolor{gray!6}{183.912} & \cellcolor{gray!6}{43.299} & \cellcolor{gray!6}{FB 102} & \cellcolor{gray!6}{35.877} & \cellcolor{gray!6}{} & \cellcolor{gray!6}{0.136}\\
1955 & 1810.387 & 48.451 & 199.073 & 1532.343 & 30.520 & 1393.824 & 119.727 & FB 105 & 18.791 &  & 0.012\\
\addlinespace
\cellcolor{gray!6}{1956} & \cellcolor{gray!6}{1481.432} & \cellcolor{gray!6}{35.073} & \cellcolor{gray!6}{257.455} & \cellcolor{gray!6}{1168.674} & \cellcolor{gray!6}{20.230} & \cellcolor{gray!6}{1026.897} & \cellcolor{gray!6}{69.943} & \cellcolor{gray!6}{FB 105} & \cellcolor{gray!6}{71.835} & \cellcolor{gray!6}{} & \cellcolor{gray!6}{0.061}\\
1957 &  & 32.080 & 227.864 & 1522.506 &  & 1298.195 & 71.549 & FB 108 & 152.763 &  & 0.100\\
\cellcolor{gray!6}{1958} & \cellcolor{gray!6}{} & \cellcolor{gray!6}{141.032} & \cellcolor{gray!6}{228.887} & \cellcolor{gray!6}{1425.890} & \cellcolor{gray!6}{} & \cellcolor{gray!6}{1136.077} & \cellcolor{gray!6}{88.642} & \cellcolor{gray!6}{FB 108, ratio} & \cellcolor{gray!6}{201.171} & \cellcolor{gray!6}{1954-57} & \cellcolor{gray!6}{0.141}\\
1959 &  & 94.833 & 264.463 & 670.998 &  & 470.075 & 36.678 & FB 111, ratio & 164.245 & 1954-57 & 0.245\\
\cellcolor{gray!6}{1960} & \cellcolor{gray!6}{} & \cellcolor{gray!6}{89.909} & \cellcolor{gray!6}{238.784} & \cellcolor{gray!6}{1280.674} & \cellcolor{gray!6}{} & \cellcolor{gray!6}{910.701} & \cellcolor{gray!6}{71.057} & \cellcolor{gray!6}{FB 117, ratio} & \cellcolor{gray!6}{298.916} & \cellcolor{gray!6}{1954-57} & \cellcolor{gray!6}{0.233}\\
\addlinespace
1961 &  & 98.523 & 174.942 & 1052.766 &  & 550.967 & 42.989 & FB 121, ratio & 458.809 & 1954-57 & 0.436\\
\cellcolor{gray!6}{1962} & \cellcolor{gray!6}{} & \cellcolor{gray!6}{70.086} & \cellcolor{gray!6}{172.422} & \cellcolor{gray!6}{916.793} & \cellcolor{gray!6}{} & \cellcolor{gray!6}{602.720} & \cellcolor{gray!6}{56.922} & \cellcolor{gray!6}{FB 125} & \cellcolor{gray!6}{257.151} & \cellcolor{gray!6}{} & \cellcolor{gray!6}{0.280}\\
1963 &  & 112.154 & 220.538 & 1180.383 &  & 652.240 & 230.784 & FB 129 & 297.359 &  & 0.252\\
\cellcolor{gray!6}{1964} & \cellcolor{gray!6}{} & \cellcolor{gray!6}{87.014} & \cellcolor{gray!6}{207.471} & \cellcolor{gray!6}{718.626} & \cellcolor{gray!6}{} & \cellcolor{gray!6}{467.924} & \cellcolor{gray!6}{114.139} & \cellcolor{gray!6}{FB 132} & \cellcolor{gray!6}{136.564} & \cellcolor{gray!6}{} & \cellcolor{gray!6}{0.190}\\
1965 &  & 132.791 & 248.713 & 786.035 &  & 453.991 & 40.039 & FB 135 & 292.005 &  & 0.371\\
\addlinespace
\cellcolor{gray!6}{1966} & \cellcolor{gray!6}{} & \cellcolor{gray!6}{136.442} & \cellcolor{gray!6}{226.385} & \cellcolor{gray!6}{1026.923} & \cellcolor{gray!6}{} & \cellcolor{gray!6}{666.109} & \cellcolor{gray!6}{82.682} & \cellcolor{gray!6}{FB 138} & \cellcolor{gray!6}{278.132} & \cellcolor{gray!6}{} & \cellcolor{gray!6}{0.271}\\
1967 &  & 167.066 & 250.557 & 1313.093 &  & 721.161 & 96.735 & FB 144 & 495.197 &  & 0.377\\
\cellcolor{gray!6}{1968} & \cellcolor{gray!6}{} & \cellcolor{gray!6}{126.059} & \cellcolor{gray!6}{242.670} & \cellcolor{gray!6}{1187.506} & \cellcolor{gray!6}{} & \cellcolor{gray!6}{612.312} & \cellcolor{gray!6}{34.805} & \cellcolor{gray!6}{FB 149} & \cellcolor{gray!6}{540.388} & \cellcolor{gray!6}{} & \cellcolor{gray!6}{0.455}\\*
\end{longtable}
\leavevmode\tagmcend\tagstructend\par
\endgroup{}
\end{landscape}

\begingroup\fontsize{10}{12}\selectfont

\tagstructbegin{tag=Table}\tagmcbegin{tag=Table}
\begin{longtable}[t]{>{\raggedright\arraybackslash}p{2cm}>{\raggedright\arraybackslash}p{1cm}>{\raggedright\arraybackslash}p{4cm}>{\raggedleft\arraybackslash}p{2cm}>{\raggedleft\arraybackslash}p{3cm}>{\raggedleft\arraybackslash}p{1cm}}
\caption{\label{tab:length-sample-size}Samples sizes of length composition data by year.}\\
\toprule
Source & Year & Fleet(\#) & Number fish & Sample size & Trips\\
\midrule
\endfirsthead
\caption[]{\label{tab:length-sample-size}Samples sizes of length composition data by year. \textit{(continued)}}\\
\toprule
Source & Year & Fleet(\#) & Number fish & Sample size & Trips\\
\midrule
\endhead

\endfoot
\bottomrule
\endlastfoot
CALCOM & 1978 & COM\_HKL(1) & 25 & 1 & 1.00\\
CALCOM & 1979 & COM\_HKL(1) & 464 & 14 & 14.00\\
CALCOM & 1980 & COM\_HKL(1) & 770 & 19 & 19.00\\
CALCOM & 1981 & COM\_HKL(1) & 898 & 23 & 23.00\\
CALCOM & 1982 & COM\_HKL(1) & 407 & 10 & 10.00\\
CALCOM & 1983 & COM\_HKL(1) & 89 & 3 & 3.00\\
CALCOM & 1986 & COM\_HKL(1) & 17 & 1 & 1.00\\
CALCOM & 1990 & COM\_HKL(1) & 10 & 1 & 1.00\\
CALCOM & 1991 & COM\_HKL(1) & 70 & 4 & 4.00\\
CALCOM & 1992 & COM\_HKL(1) & 219 & 15 & 15.00\\
CALCOM & 1993 & COM\_HKL(1) & 924 & 50 & 49.00\\
CALCOM & 1994 & COM\_HKL(1) & 309 & 20 & 20.00\\
CALCOM & 1995 & COM\_HKL(1) & 163 & 10 & 10.00\\
CALCOM & 1996 & COM\_HKL(1) & 394 & 23 & 23.00\\
CALCOM & 1997 & COM\_HKL(1) & 289 & 14 & 14.00\\
CALCOM & 1998 & COM\_HKL(1) & 203 & 9 & 9.00\\
CALCOM & 1999 & COM\_HKL(1) & 264 & 16 & 16.00\\
CALCOM & 2000 & COM\_HKL(1) & 15 & 1 & 1.00\\
CALCOM & 2001 & COM\_HKL(1) & 20 & 2 & 2.00\\
CALCOM & 2002 & COM\_HKL(1) & 28 & 2 & 2.00\\
CALCOM & 2005 & COM\_HKL(1) & 34 & 4 & 3.00\\
CALCOM & 2006 & COM\_HKL(1) & 68 & 4 & 4.00\\
CALCOM & 2007 & COM\_HKL(1) & 74 & 4 & 4.00\\
CALCOM & 2008 & COM\_HKL(1) & 22 & 3 & 2.00\\
CALCOM & 2009 & COM\_HKL(1) & 45 & 4 & 4.00\\
CALCOM & 2011 & COM\_HKL(1) & 22 & 1 & 1.00\\
CALCOM & 2012 & COM\_HKL(1) & 12 & 1 & 1.00\\
CALCOM & 2013 & COM\_HKL(1) & 12 & 1 & 1.00\\
CALCOM & 2014 & COM\_HKL(1) & 116 & 9 & 9.00\\
CALCOM & 2015 & COM\_HKL(1) & 29 & 2 & 2.00\\
CALCOM & 2016 & COM\_HKL(1) & 15 & 1 & 1.00\\
CALCOM & 2017 & COM\_HKL(1) & 45 & 4 & 4.00\\
CALCOM & 2018 & COM\_HKL(1) & 11 & 1 & 1.00\\
CALCOM & 2019 & COM\_HKL(1) & 108 & 6 & 6.00\\
CALCOM & 2020 & COM\_HKL(1) & 61 & 4 & 4.00\\
CALCOM & 1979 & COM\_TWL(2) & 14 & 1 & 1.00\\
CALCOM & 1983 & COM\_TWL(2) & 22 & 2 & 2.00\\
CALCOM & 1984 & COM\_TWL(2) & 76 & 5 & 5.00\\
CALCOM & 1985 & COM\_TWL(2) & 18 & 1 & 1.00\\
CALCOM & 1987 & COM\_TWL(2) & 13 & 1 & 1.00\\
CALCOM & 1992 & COM\_TWL(2) & 13 & 1 & 1.00\\
CALCOM & 1993 & COM\_TWL(2) & 35 & 3 & 3.00\\
CALCOM & 1994 & COM\_TWL(2) & 12 & 1 & 1.00\\
CALCOM & 1996 & COM\_TWL(2) & 44 & 2 & 2.00\\
CALCOM & 1997 & COM\_TWL(2) & 42 & 3 & 3.00\\
CALCOM & 1999 & COM\_TWL(2) & 21 & 1 & 1.00\\
CALCOM & 2015 & COM\_TWL(2) & 18 & 1 & 1.00\\
CALCOM & 2016 & COM\_TWL(2) & 15 & 1 & 1.00\\
CALCOM & 2017 & COM\_TWL(2) & 26 & 1 & 1.00\\
CALCOM & 2018 & COM\_TWL(2) & 47 & 2 & 2.00\\
CALCOM & 1987 & COM\_NET(3) & 28 & 2 & 2.00\\
CALCOM & 1988 & COM\_NET(3) & 21 & 1 & 1.00\\
CALCOM & 1990 & COM\_NET(3) & 110 & 7 & 7.00\\
CALCOM & 1993 & COM\_NET(3) & 66 & 3 & 3.00\\
CALCOM & 1994 & COM\_NET(3) & 42 & 2 & 2.00\\
CALCOM & 1995 & COM\_NET(3) & 80 & 6 & 6.00\\
CALCOM & 1996 & COM\_NET(3) & 36 & 2 & 2.00\\
CALCOM & 1997 & COM\_NET(3) & 34 & 2 & 2.00\\
CALCOM & 1998 & COM\_NET(3) & 70 & 3 & 3.00\\
KARPOV & 1959 & REC\_PC(4) & 506 &  & \\
KARPOV & 1960 & REC\_PC(4) & 1042 &  & \\
SWFSC & 1978 & REC\_PC(4) & 30 & 26 & \\
SWFSC & 1979 & REC\_PC(4) & 82 & 31 & \\
MRFSS & 1980 & REC\_PC(4) & 73 &  & 51.00\\
MRFSS & 1981 & REC\_PC(4) & 33 &  & 27.00\\
MRFSS & 1982 & REC\_PC(4) & 37 &  & 34.00\\
MRFSS & 1983 & REC\_PC(4) & 37 &  & 30.00\\
MRFSS & 1984 & REC\_PC(4) & 86 &  & 62.00\\
MRFSS & 1985 & REC\_PC(4) & 139 &  & 93.00\\
MRFSS & 1986 & REC\_PC(4) & 127 &  & 84.00\\
MRFSS & 1987 & REC\_PC(4) & 223 &  & 73.00\\
MRFSS & 1988 & REC\_PC(4) & 154 &  & 89.00\\
MRFSS & 1989 & REC\_PC(4) & 234 &  & 94.00\\
MRFSS & 1993 & REC\_PC(4) & 59 &  & 40.00\\
MRFSS & 1994 & REC\_PC(4) & 81 &  & 45.00\\
MRFSS & 1995 & REC\_PC(4) & 88 &  & 65.00\\
MRFSS & 1996 & REC\_PC(4) & 315 &  & 208.00\\
MRFSS & 1997 & REC\_PC(4) & 1209 &  & 156.00\\
MRFSS & 1998 & REC\_PC(4) & 210 &  & 91.00\\
MRFSS & 1999 & REC\_PC(4) & 571 &  & 295.00\\
MRFSS & 2000 & REC\_PC(4) & 129 &  & 81.00\\
MRFSS & 2001 & REC\_PC(4) & 200 &  & 134.00\\
MRFSS & 2002 & REC\_PC(4) & 378 &  & 235.00\\
MRFSS & 2003 & REC\_PC(4) & 577 &  & 361.00\\
CRFS & 2004 & REC\_PC(4) & 995 &  & 176.00\\
CRFS & 2005 & REC\_PC(4) & 1627 &  & 288.00\\
CRFS & 2006 & REC\_PC(4) & 1444 &  & 256.00\\
CRFS & 2007 & REC\_PC(4) & 1805 &  & 319.00\\
CRFS & 2008 & REC\_PC(4) & 690 &  & 122.00\\
CRFS & 2009 & REC\_PC(4) & 884 &  & 156.00\\
CRFS & 2010 & REC\_PC(4) & 1630 &  & 288.00\\
CRFS & 2011 & REC\_PC(4) & 1426 &  & 252.00\\
CRFS & 2012 & REC\_PC(4) & 1234 &  & 218.00\\
CRFS & 2013 & REC\_PC(4) & 917 &  & 162.00\\
CRFS & 2014 & REC\_PC(4) & 563 &  & 159.00\\
CRFS & 2015 & REC\_PC(4) & 734 &  & 190.00\\
CRFS & 2016 & REC\_PC(4) & 742 &  & 166.00\\
CRFS & 2017 & REC\_PC(4) & 1082 &  & 175.00\\
CRFS & 2018 & REC\_PC(4) & 1190 &  & 150.00\\
CRFS & 2019 & REC\_PC(4) & 1357 &  & 163.00\\
CRFS & 2003 & REC\_PC\_DIS(5) & 38 &  & \\
CRFS & 2004 & REC\_PC\_DIS(5) & 78 &  & \\
CRFS & 2005 & REC\_PC\_DIS(5) & 67 &  & \\
CRFS & 2006 & REC\_PC\_DIS(5) & 49 &  & \\
CRFS & 2007 & REC\_PC\_DIS(5) & 9 &  & \\
CRFS & 2008 & REC\_PC\_DIS(5) & 9 &  & \\
CRFS & 2009 & REC\_PC\_DIS(5) & 40 &  & \\
CRFS & 2010 & REC\_PC\_DIS(5) & 70 &  & \\
CRFS & 2011 & REC\_PC\_DIS(5) & 13 &  & \\
CRFS & 2012 & REC\_PC\_DIS(5) & 6 &  & \\
CRFS & 2013 & REC\_PC\_DIS(5) & 6 &  & \\
CRFS & 2014 & REC\_PC\_DIS(5) & 7 &  & \\
CRFS & 2015 & REC\_PC\_DIS(5) & 6 &  & \\
CRFS & 2016 & REC\_PC\_DIS(5) & 5 &  & \\
CRFS & 2017 & REC\_PC\_DIS(5) & 6 &  & \\
CRFS & 2018 & REC\_PC\_DIS(5) & 2 &  & \\
CRFS & 2019 & REC\_PC\_DIS(5) & 13 &  & \\
KARPOV & 1959 & REC\_PR(6) & 499 &  & \\
MRFSS & 1980 & REC\_PR(6) & 89 &  & 62.00\\
MRFSS & 1981 & REC\_PR(6) & 55 &  & 36.00\\
MRFSS & 1982 & REC\_PR(6) & 109 &  & 65.00\\
MRFSS & 1983 & REC\_PR(6) & 83 &  & 60.00\\
MRFSS & 1984 & REC\_PR(6) & 176 &  & 117.00\\
MRFSS & 1985 & REC\_PR(6) & 137 &  & 93.00\\
MRFSS & 1986 & REC\_PR(6) & 158 &  & 102.00\\
MRFSS & 1987 & REC\_PR(6) & 97 &  & 45.00\\
MRFSS & 1988 & REC\_PR(6) & 79 &  & 46.00\\
MRFSS & 1989 & REC\_PR(6) & 94 &  & 51.00\\
MRFSS & 1993 & REC\_PR(6) & 510 &  & 269.00\\
MRFSS & 1994 & REC\_PR(6) & 285 &  & 147.00\\
MRFSS & 1995 & REC\_PR(6) & 152 &  & 85.00\\
MRFSS & 1996 & REC\_PR(6) & 119 &  & 73.00\\
MRFSS & 1997 & REC\_PR(6) & 92 &  & 50.00\\
MRFSS & 1998 & REC\_PR(6) & 124 &  & 79.00\\
MRFSS & 1999 & REC\_PR(6) & 255 &  & 135.00\\
MRFSS & 2000 & REC\_PR(6) & 197 &  & 101.00\\
MRFSS & 2001 & REC\_PR(6) & 71 &  & 45.00\\
MRFSS & 2002 & REC\_PR(6) & 240 &  & 126.00\\
MRFSS & 2003 & REC\_PR(6) & 494 &  & 187.00\\
CRFS & 2004 & REC\_PR(6) & 2098 &  & 371.00\\
CRFS & 2005 & REC\_PR(6) & 4068 &  & 1784.21\\
CRFS & 2006 & REC\_PR(6) & 5036 &  & 2208.77\\
CRFS & 2007 & REC\_PR(6) & 3889 &  & 1705.70\\
CRFS & 2008 & REC\_PR(6) & 2600 &  & 1140.35\\
CRFS & 2009 & REC\_PR(6) & 1994 &  & 874.56\\
CRFS & 2010 & REC\_PR(6) & 1938 &  & 850.00\\
CRFS & 2011 & REC\_PR(6) & 2210 &  & 969.30\\
CRFS & 2012 & REC\_PR(6) & 1917 &  & 840.79\\
CRFS & 2013 & REC\_PR(6) & 2409 &  & 1056.58\\
CRFS & 2014 & REC\_PR(6) & 2117 &  & 1058.00\\
CRFS & 2015 & REC\_PR(6) & 3492 &  & 1620.00\\
CRFS & 2016 & REC\_PR(6) & 3315 &  & 1406.00\\
CRFS & 2017 & REC\_PR(6) & 2963 &  & 1384.00\\
CRFS & 2018 & REC\_PR(6) & 3225 &  & 1350.00\\
CRFS & 2019 & REC\_PR(6) & 3426 &  & 1329.00\\
CDFW & 1988 & DWV\_ONBOARD(8) & 674 &  & 100.00\\
CDFW & 1989 & DWV\_ONBOARD(8) & 1274 &  & 134.00\\
CDFW & 1990 & DWV\_ONBOARD(8) & 583 &  & 48.00\\
CDFW & 1991 & DWV\_ONBOARD(8) & 388 &  & 62.00\\
CDFW & 1992 & DWV\_ONBOARD(8) & 1173 &  & 145.00\\
CDFW & 1993 & DWV\_ONBOARD(8) & 1079 &  & 162.00\\
CDFW & 1994 & DWV\_ONBOARD(8) & 753 &  & 112.00\\
CDFW & 1995 & DWV\_ONBOARD(8) & 964 &  & 147.00\\
CDFW & 1996 & DWV\_ONBOARD(8) & 582 &  & 137.00\\
CDFW & 1997 & DWV\_ONBOARD(8) & 1278 &  & 177.00\\
CDFW & 1998 & DWV\_ONBOARD(8) & 662 &  & 118.00\\
NWFSC & 2003 & NWFSC\_TWL(9) & 21 & 4 & \\
NWFSC & 2004 & NWFSC\_TWL(9) & 6 & 4 & \\
NWFSC & 2005 & NWFSC\_TWL(9) & 7 & 2 & \\
NWFSC & 2006 & NWFSC\_TWL(9) & 18 & 4 & \\
NWFSC & 2007 & NWFSC\_TWL(9) & 1 & 1 & \\
NWFSC & 2008 & NWFSC\_TWL(9) & 37 & 14 & \\
NWFSC & 2009 & NWFSC\_TWL(9) & 208 & 21 & \\
NWFSC & 2010 & NWFSC\_TWL(9) & 33 & 12 & \\
NWFSC & 2011 & NWFSC\_TWL(9) & 3 & 2 & \\
NWFSC & 2012 & NWFSC\_TWL(9) & 40 & 4 & \\
NWFSC & 2013 & NWFSC\_TWL(9) & 50 & 9 & \\
NWFSC & 2014 & NWFSC\_TWL(9) & 17 & 12 & \\
NWFSC & 2015 & NWFSC\_TWL(9) & 5 & 5 & \\
NWFSC & 2016 & NWFSC\_TWL(9) & 9 & 9 & \\
NWFSC & 2017 & NWFSC\_TWL(9) & 64 & 12 & \\
NWFSC & 2018 & NWFSC\_TWL(9) & 26 & 7 & \\
NWFSC & 2019 & NWFSC\_TWL(9) & 42 & 7 & \\
J\_ABRAMS & 2010 & ABRAMS\_RESEARCH(11) & 25 &  & \\
J\_ABRAMS & 2011 & ABRAMS\_RESEARCH(11) & 56 &  & \\
SWFSC & 2002 & SWFSC\_GF\_ECOL(12) & 71 & 13 & \\
SWFSC & 2003 & SWFSC\_GF\_ECOL(12) & 110 & 22 & \\
SWFSC & 2004 & SWFSC\_GF\_ECOL(12) & 118 & 18 & \\
SWFSC & 2005 & SWFSC\_GF\_ECOL(12) & 25 & 9 & \\
SWFSC & 2010 & SWFSC\_GF\_ECOL(12) & 12 & 3 & \\
SWFSC & 2016 & SWFSC\_GF\_ECOL(12) & 19 & 3 & \\
CCFRP & 2007 & CCFRP(13) & 140 & 57 & \\
CCFRP & 2008 & CCFRP(13) & 230 & 74 & \\
CCFRP & 2009 & CCFRP(13) & 226 & 65 & \\
CCFRP & 2010 & CCFRP(13) & 320 & 86 & \\
CCFRP & 2011 & CCFRP(13) & 282 & 75 & \\
CCFRP & 2012 & CCFRP(13) & 294 & 90 & \\
CCFRP & 2013 & CCFRP(13) & 172 & 73 & \\
CCFRP & 2014 & CCFRP(13) & 272 & 92 & \\
CCFRP & 2015 & CCFRP(13) & 168 & 56 & \\
CCFRP & 2016 & CCFRP(13) & 387 & 87 & \\
CCFRP & 2017 & CCFRP(13) & 366 & 107 & \\
CCFRP & 2018 & CCFRP(13) & 482 & 115 & \\
CCFRP & 2019 & CCFRP(13) & 558 & 130 & \\
CCFRP & 2020 & CCFRP(13) & 447 & 111 & \\*
\end{longtable}
\leavevmode\tagmcend\tagstructend\par
\endgroup{}

\FloatBarrier

\begin{table}

\caption{\label{tab:length-inputN}Basis for initial input samples sizes by fleet and years for the length composition data in the table above.}
\centering
\fontsize{10}{12}\selectfont
\begin{tabular}[t]{lllr}
\toprule
Source & Fleet No. & Initial Sample Size Basis & Years\\
\midrule
\cellcolor{gray!6}{CALCOM} & \cellcolor{gray!6}{1} & \cellcolor{gray!6}{N\_SAMPLES, YEARS WITH <30 FISH EXCLUDED} & \cellcolor{gray!6}{1978-2020}\\
CALCOM & 2 & N\_SAMPLES, YEARS WITH <30 FISH EXCLUDED & 1979-2018\\
\cellcolor{gray!6}{CALCOM} & \cellcolor{gray!6}{3} & \cellcolor{gray!6}{N\_SAMPLES, YEARS WITH <30 FISH EXCLUDED} & \cellcolor{gray!6}{1987-1998}\\
CRFS & 4 & N\_TRIPS & 2014-2019\\
\cellcolor{gray!6}{CRFS} & \cellcolor{gray!6}{4} & \cellcolor{gray!6}{N\_TRIPS ESTIMATED FROM AVG. FISH/TRIP} & \cellcolor{gray!6}{2004-2013}\\
KARPOV & 4 & N\_FISH / 10 & 1959-1960\\
\cellcolor{gray!6}{MRFSS} & \cellcolor{gray!6}{4} & \cellcolor{gray!6}{N\_TRIPS ESTIMATED FROM B. SOPER ALGORITHM} & \cellcolor{gray!6}{1980-2003}\\
SWFSC & 4 & N\_SAMPLES & 1978-1979\\
\cellcolor{gray!6}{CRFS} & \cellcolor{gray!6}{5} & \cellcolor{gray!6}{N\_FISH, YEARS WITH <10 FISH EXCLUDED} & \cellcolor{gray!6}{2003-2019}\\
CRFS & 6 & N\_TRIPS & 2014-2019\\
\cellcolor{gray!6}{CRFS} & \cellcolor{gray!6}{6} & \cellcolor{gray!6}{N\_TRIPS ESTIMATED FROM AVG. FISH/TRIP} & \cellcolor{gray!6}{2004-2013}\\
KARPOV & 6 & N\_FISH / 10 & 1959-1959\\
\cellcolor{gray!6}{MRFSS} & \cellcolor{gray!6}{6} & \cellcolor{gray!6}{N\_TRIPS ESTIMATED FROM B. SOPER ALGORITHM} & \cellcolor{gray!6}{1980-2003}\\
CDFW & 8 & N\_TRIPS (UNIQUE ASSIGNMENT NUMBERS) & 1988-1998\\
\cellcolor{gray!6}{NWFSC} & \cellcolor{gray!6}{9} & \cellcolor{gray!6}{EFFECTIVE N  BASED ON STEWART \& HAMEL (2014)} & \cellcolor{gray!6}{2003-2019}\\
J\_ABRAMS & 11 & N\_FISH & 2010-2011\\
\cellcolor{gray!6}{SWFSC} & \cellcolor{gray!6}{12} & \cellcolor{gray!6}{N\_SAMPLES (NUMBER OF HAULS)} & \cellcolor{gray!6}{2002-2016}\\
CCFRP & 13 & N\_SAMPLES (UNIQUE ID.CELL.PER.TRIP) & 2007-2020\\
\bottomrule
\end{tabular}
\end{table}

\FloatBarrier

\begin{table}[H]

\caption{\label{tab:albin-allocate}Estimated ratio of SLO catch (in numbers) to catch in California 
  counties north of SLO from Albin et al. (1993).}
\centering
\fontsize{10}{12}\selectfont
\begin{tabular}[t]{lrlrrlr}
\toprule
Species & Year & Area & Estimate & SE & CV & SLO/(Total-SLO)\\
\midrule
Vermilion & 1981 & San\_Luis\_Obispo & 16 & 9 & 58 & 1.7777778\\
Vermilion & 1981 & Total & 25 & 10 & 39 & \\
Vermilion & 1982 & San\_Luis\_Obispo & 12 & 5 & 46 & 0.6315789\\
Vermilion & 1982 & Total & 31 & 8 & 27 & \\
Vermilion & 1983 & San\_Luis\_Obispo & 17 & 12 & 67 & 1.1333333\\
Vermilion & 1983 & Total & 32 & 12 & 38 & \\
Vermilion & 1984 & San\_Luis\_Obispo & 30 & 27 & 91 & 1.0714286\\
Vermilion & 1984 & Total & 58 & 28 & 49 & \\
Vermilion & 1985 & San\_Luis\_Obispo & 15 & 8 & 54 & 0.7142857\\
Vermilion & 1985 & Total & 36 & 10 & 27 & \\
Vermilion & 1986 & San\_Luis\_Obispo & 23 & 13 & 56 & 1.0454545\\
Vermilion & 1986 & Total & 45 & 14 & 30 & \\
 &  &  &  &  & Average & 1.0623098\\
 &  &  &  &  & Catch-weighted Avg. & 1.0360910\\
\bottomrule
\end{tabular}
\end{table}

\tagstructbegin{tag=Table}\tagmcbegin{tag=Table}
\begin{longtable}[t]{>{\raggedright\arraybackslash}p{3cm}>{\raggedleft\arraybackslash}p{1cm}>{\raggedright\arraybackslash}p{5cm}>{\raggedleft\arraybackslash}p{2cm}}
\caption{\label{tab:age-sample-size}Samples sizes of conditional age-at-length data by year.}\\
\toprule
Source & Year & Fleet(\#) & Number of fish\\
\midrule
\endfirsthead
\caption[]{\label{tab:age-sample-size}Samples sizes of conditional age-at-length data by year. \textit{(continued)}}\\
\toprule
Source & Year & Fleet(\#) & Number of fish\\
\midrule
\endhead

\endfoot
\bottomrule
\endlastfoot
NWFSC & 2004 & NWFSC\_TWL(9) & 6\\
NWFSC & 2005 & NWFSC\_TWL(9) & 7\\
NWFSC & 2006 & NWFSC\_TWL(9) & 18\\
NWFSC & 2007 & NWFSC\_TWL(9) & 1\\
NWFSC & 2008 & NWFSC\_TWL(9) & 37\\
NWFSC & 2009 & NWFSC\_TWL(9) & 111\\
NWFSC & 2010 & NWFSC\_TWL(9) & 33\\
NWFSC & 2011 & NWFSC\_TWL(9) & 2\\
NWFSC & 2012 & NWFSC\_TWL(9) & 40\\
NWFSC & 2013 & NWFSC\_TWL(9) & 50\\
NWFSC & 2014 & NWFSC\_TWL(9) & 16\\
NWFSC & 2015 & NWFSC\_TWL(9) & 5\\
NWFSC & 2016 & NWFSC\_TWL(9) & 9\\
NWFSC & 2017 & NWFSC\_TWL(9) & 64\\
NWFSC & 2018 & NWFSC\_TWL(9) & 26\\
NWFSC & 2019 & NWFSC\_TWL(9) & 42\\
J\_ABRAMS & 2010 & ABRAMS\_RESEARCH(11) & 25\\
J\_ABRAMS & 2011 & ABRAMS\_RESEARCH(11) & 56\\
SWFSC & 2002 & SWFSC\_GF\_ECOL(12) & 44\\
SWFSC & 2003 & SWFSC\_GF\_ECOL(12) & 58\\
SWFSC & 2004 & SWFSC\_GF\_ECOL(12) & 108\\
SWFSC & 2005 & SWFSC\_GF\_ECOL(12) & 19\\*
\end{longtable}
\leavevmode\tagmcend\tagstructend\par

\begin{landscape}\begingroup\fontsize{8}{10}\selectfont

\tagstructbegin{tag=Table}\tagmcbegin{tag=Table}
\begin{longtable}[t]{lllllll}
\caption{\label{tab:paramstable}List of parameters used in the base model, including estimated values and standard deviations (SD), bounds (minimum and maximum), estimation phase (negative values not estimated), status (indicates if parameters are near bounds), and prior type information (mean and SD).}\\
\toprule
Parameter & Value & Phase & Bounds & Status & SD & Prior (Exp.Val, SD)\\
\midrule
\endfirsthead
\caption[]{\label{tab:paramstable}List of parameters used in the base model, including estimated values and standard deviations (SD), bounds (minimum and maximum), estimation phase (negative values not estimated), status (indicates if parameters are near bounds), and prior type information (mean and SD). \textit{(continued)}}\\
\toprule
Parameter & Value & Phase & Bounds & Status & SD & Prior (Exp.Val, SD)\\
\midrule
\endhead

\endfoot
\bottomrule
\endlastfoot
NatM uniform Fem GP 1 & 0.086 & 2 & (0.001, 0.4) & OK & 0.0082872 & Log Norm (-2.3026, 0.438)\\
L at Amin Fem GP 1 & 7.592 & 2 & (1, 15) & OK & 0.6175560 & None\\
L at Amax Fem GP 1 & 55.184 & 2 & (45, 65) & OK & 0.6960840 & None\\
VonBert K Fem GP 1 & 0.147 & 2 & (0.05, 0.25) & OK & 0.0079037 & None\\
CV young Fem GP 1 & 0.099 & 2 & (0.01, 0.3) & OK & 0.0097941 & None\\
\addlinespace
CV old Fem GP 1 & 0.074 & 2 & (0.01, 0.3) & OK & 0.0094818 & None\\
Wtlen 1 Fem GP 1 & 0.000 & -2 & (1.744e-05, 1.744e-05) &  &  & None\\
Wtlen 2 Fem GP 1 & 2.995 & -2 & (1, 3) &  &  & None\\
Mat50\% Fem GP 1 & 38.400 & -2 & (38.4, 38.4) &  &  & None\\
Mat slope Fem GP 1 & -0.312 & -2 & (-0.4, -0.2) &  &  & None\\
\addlinespace
Eggs scalar Fem GP 1 & 0.000 & -2 & (0, 1) &  &  & None\\
Eggs exp len Fem GP 1 & 4.970 & -2 & (3, 6) &  &  & None\\
NatM uniform Mal GP 1 & 0.080 & 2 & (0.001, 0.4) & OK & 0.0087910 & Log Norm (-2.3026, 0.438)\\
L at Amin Mal GP 1 & 5.898 & 2 & (1, 15) & OK & 0.5581020 & None\\
L at Amax Mal GP 1 & 49.940 & 2 & (45, 65) & OK & 0.5173370 & None\\
\addlinespace
VonBert K Mal GP 1 & 0.199 & 2 & (0.05, 0.25) & OK & 0.0085385 & None\\
CV young Mal GP 1 & 0.077 & 2 & (0.01, 0.3) & OK & 0.0053649 & None\\
CV old Mal GP 1 & 0.070 & -2 & (0.01, 0.3) &  &  & None\\
Wtlen 1 Mal GP 1 & 0.000 & -2 & (0, 1) &  &  & None\\
Wtlen 2 Mal GP 1 & 2.995 & -2 & (1, 3) &  &  & None\\
\addlinespace
CohortGrowDev & 1.000 & -1 & (0.1, 10) &  &  & None\\
FracFemale GP 1 & 0.500 & -1 & (1e-06, 0.999999) &  &  & None\\
SR LN(R0) & 6.041 & 1 & (5, 8) & OK & 0.1471020 & None\\
SR BH steep & 0.720 & -2 & (0.201, 0.999) &  &  & Full Beta (0.72, 0.16)\\
SR sigmaR & 0.500 & -2 & (0, 2) &  &  & None\\
\addlinespace
SR regime & 0.000 & -2 & (-5, 5) &  &  & None\\
SR autocorr & 0.000 & -2 & (0, 0) &  &  & None\\
Main RecrDev 1970 & -0.134 & 4 & (-5, 5) & act & 0.4521630 & dev (NA, NA)\\
Main RecrDev 1971 & -0.215 & 4 & (-5, 5) & act & 0.4352040 & dev (NA, NA)\\
Main RecrDev 1972 & -0.327 & 4 & (-5, 5) & act & 0.4153930 & dev (NA, NA)\\
\addlinespace
Main RecrDev 1973 & -0.477 & 4 & (-5, 5) & act & 0.3937230 & dev (NA, NA)\\
Main RecrDev 1974 & -0.651 & 4 & (-5, 5) & act & 0.3731320 & dev (NA, NA)\\
Main RecrDev 1975 & -0.798 & 4 & (-5, 5) & act & 0.3566460 & dev (NA, NA)\\
Main RecrDev 1976 & -0.850 & 4 & (-5, 5) & act & 0.3464360 & dev (NA, NA)\\
Main RecrDev 1977 & -0.791 & 4 & (-5, 5) & act & 0.3461600 & dev (NA, NA)\\
\addlinespace
Main RecrDev 1978 & -0.590 & 4 & (-5, 5) & act & 0.3420630 & dev (NA, NA)\\
Main RecrDev 1979 & -0.521 & 4 & (-5, 5) & act & 0.3449060 & dev (NA, NA)\\
Main RecrDev 1980 & -0.473 & 4 & (-5, 5) & act & 0.3462160 & dev (NA, NA)\\
Main RecrDev 1981 & -0.316 & 4 & (-5, 5) & act & 0.3461750 & dev (NA, NA)\\
Main RecrDev 1982 & -0.053 & 4 & (-5, 5) & act & 0.3347880 & dev (NA, NA)\\
\addlinespace
Main RecrDev 1983 & -0.015 & 4 & (-5, 5) & act & 0.3737700 & dev (NA, NA)\\
Main RecrDev 1984 & 0.673 & 4 & (-5, 5) & act & 0.3220940 & dev (NA, NA)\\
Main RecrDev 1985 & 1.445 & 4 & (-5, 5) & act & 0.2003400 & dev (NA, NA)\\
Main RecrDev 1986 & -0.022 & 4 & (-5, 5) & act & 0.4117670 & dev (NA, NA)\\
Main RecrDev 1987 & -0.452 & 4 & (-5, 5) & act & 0.3692650 & dev (NA, NA)\\
\addlinespace
Main RecrDev 1988 & -0.401 & 4 & (-5, 5) & act & 0.3647980 & dev (NA, NA)\\
Main RecrDev 1989 & 0.073 & 4 & (-5, 5) & act & 0.3084210 & dev (NA, NA)\\
Main RecrDev 1990 & 0.235 & 4 & (-5, 5) & act & 0.2723070 & dev (NA, NA)\\
Main RecrDev 1991 & -0.021 & 4 & (-5, 5) & act & 0.3063500 & dev (NA, NA)\\
Main RecrDev 1992 & -0.010 & 4 & (-5, 5) & act & 0.3095030 & dev (NA, NA)\\
\addlinespace
Main RecrDev 1993 & 0.220 & 4 & (-5, 5) & act & 0.3014960 & dev (NA, NA)\\
Main RecrDev 1994 & 0.935 & 4 & (-5, 5) & act & 0.2268070 & dev (NA, NA)\\
Main RecrDev 1995 & 0.443 & 4 & (-5, 5) & act & 0.3528520 & dev (NA, NA)\\
Main RecrDev 1996 & 0.142 & 4 & (-5, 5) & act & 0.3822650 & dev (NA, NA)\\
Main RecrDev 1997 & 0.251 & 4 & (-5, 5) & act & 0.3901180 & dev (NA, NA)\\
\addlinespace
Main RecrDev 1998 & 0.566 & 4 & (-5, 5) & act & 0.3653510 & dev (NA, NA)\\
Main RecrDev 1999 & 1.139 & 4 & (-5, 5) & act & 0.2375850 & dev (NA, NA)\\
Main RecrDev 2000 & 0.476 & 4 & (-5, 5) & act & 0.3120040 & dev (NA, NA)\\
Main RecrDev 2001 & 0.198 & 4 & (-5, 5) & act & 0.2527550 & dev (NA, NA)\\
Main RecrDev 2002 & -0.621 & 4 & (-5, 5) & act & 0.2759890 & dev (NA, NA)\\
\addlinespace
Main RecrDev 2003 & -0.742 & 4 & (-5, 5) & act & 0.2616310 & dev (NA, NA)\\
Main RecrDev 2004 & -0.604 & 4 & (-5, 5) & act & 0.2766190 & dev (NA, NA)\\
Main RecrDev 2005 & -0.093 & 4 & (-5, 5) & act & 0.2425840 & dev (NA, NA)\\
Main RecrDev 2006 & -0.028 & 4 & (-5, 5) & act & 0.2511420 & dev (NA, NA)\\
Main RecrDev 2007 & 0.635 & 4 & (-5, 5) & act & 0.1721730 & dev (NA, NA)\\
\addlinespace
Main RecrDev 2008 & 0.398 & 4 & (-5, 5) & act & 0.2123430 & dev (NA, NA)\\
Main RecrDev 2009 & 0.692 & 4 & (-5, 5) & act & 0.1806370 & dev (NA, NA)\\
Main RecrDev 2010 & -0.298 & 4 & (-5, 5) & act & 0.3026240 & dev (NA, NA)\\
Main RecrDev 2011 & -0.397 & 4 & (-5, 5) & act & 0.2854350 & dev (NA, NA)\\
Main RecrDev 2012 & 0.196 & 4 & (-5, 5) & act & 0.2424730 & dev (NA, NA)\\
\addlinespace
Main RecrDev 2013 & 0.326 & 4 & (-5, 5) & act & 0.2785740 & dev (NA, NA)\\
Main RecrDev 2014 & 0.341 & 4 & (-5, 5) & act & 0.3003230 & dev (NA, NA)\\
Main RecrDev 2015 & -0.215 & 4 & (-5, 5) & act & 0.3681840 & dev (NA, NA)\\
Main RecrDev 2016 & 1.472 & 4 & (-5, 5) & act & 0.2595480 & dev (NA, NA)\\
Main RecrDev 2017 & -0.800 & 4 & (-5, 5) & act & 0.4491540 & dev (NA, NA)\\
\addlinespace
Main RecrDev 2018 & 0.048 & 4 & (-5, 5) & act & 0.4795330 & dev (NA, NA)\\
Main RecrDev 2019 & 0.003 & 4 & (-5, 5) & act & 0.4931920 & dev (NA, NA)\\
Main RecrDev 2020 & 0.009 & 4 & (-5, 5) & act & 0.4945830 & dev (NA, NA)\\
LnQ base REC PC(4) & -9.529 & -1 & (-15, 0) &  &  & None\\
LnQ base REC PR(6) & -7.818 & -1 & (-15, 0) &  &  & None\\
\addlinespace
Q extraSD REC PR(6) & 0.163 & 1 & (0, 0.5) & OK & 0.0431518 & None\\
LnQ base DWV ONBOARD(8) & -10.817 & -1 & (-15, 0) &  &  & None\\
LnQ base REC PC ONBOARD(10) & -10.665 & -1 & (-15, 0) &  &  & None\\
LnQ base CCFRP(13) & -8.617 & -1 & (-15, 0) &  &  & None\\
Size DblN peak COM HKL(1) & 46.184 & 2 & (30, 60) & OK & 1.8247300 & None\\
\addlinespace
Size DblN top logit COM HKL(1) & -9.000 & -2 & (-12, 0) &  &  & None\\
Size DblN ascend se COM HKL(1) & 4.731 & 2 & (2, 8) & OK & 0.2392430 & None\\
Size DblN descend se COM HKL(1) & 10.000 & -2 & (1, 10) &  &  & None\\
Size DblN start logit COM HKL(1) & -10.000 & -2 & (-11, -9) &  &  & None\\
Size DblN end logit COM HKL(1) & 10.000 & -2 & (-11, 11) &  &  & None\\
\addlinespace
Size DblN peak COM TWL(2) & 47.816 & 2 & (25, 65) & OK & 3.4986700 & None\\
Size DblN top logit COM TWL(2) & -9.000 & -2 & (-12, 0) &  &  & None\\
Size DblN ascend se COM TWL(2) & 4.153 & 2 & (0.05, 10) & OK & 0.6224990 & None\\
Size DblN descend se COM TWL(2) & 10.000 & -2 & (0.05, 10) &  &  & None\\
Size DblN start logit COM TWL(2) & -10.000 & -2 & (-11, -9) &  &  & None\\
\addlinespace
Size DblN end logit COM TWL(2) & 10.000 & -2 & (-10, 10) &  &  & None\\
Size DblN peak COM NET(3) & 42.706 & 2 & (25, 65) & OK & 2.6409800 & None\\
Size DblN top logit COM NET(3) & -9.000 & -2 & (-12, 0) &  &  & None\\
Size DblN ascend se COM NET(3) & 3.303 & 2 & (0.05, 10) & OK & 0.8105430 & None\\
Size DblN descend se COM NET(3) & 0.226 & 2 & (0.05, 10) & OK & 4.6011900 & None\\
\addlinespace
Size DblN start logit COM NET(3) & -10.000 & -2 & (-11, -9) &  &  & None\\
Size DblN end logit COM NET(3) & 0.332 & 2 & (-10, 10) & OK & 1.5862600 & None\\
Size DblN peak REC PC(4) & 46.582 & 2 & (20, 60) & OK & 2.6591600 & None\\
Size DblN top logit REC PC(4) & -9.000 & -2 & (-12, 0) &  &  & None\\
Size DblN ascend se REC PC(4) & 5.132 & 2 & (0.5, 8) & OK & 0.3535910 & None\\
\addlinespace
Size DblN descend se REC PC(4) & 4.037 & 2 & (1, 10) & OK & 0.9537990 & None\\
Size DblN start logit REC PC(4) & -10.000 & -2 & (-11, -9) &  &  & None\\
Size DblN end logit REC PC(4) & -10.000 & -2 & (-11, 11) &  &  & None\\
Size DblN peak REC PC DIS(5) & 18.728 & 2 & (10, 50) & OK & 1.5126900 & None\\
Size DblN top logit REC PC DIS(5) & -9.000 & -2 & (-10, 10) &  &  & None\\
\addlinespace
Size DblN ascend se REC PC DIS(5) & 1.740 & 2 & (1, 10) & OK & 0.9218750 & None\\
Size DblN descend se REC PC DIS(5) & 4.470 & 2 & (2, 8) & OK & 0.2767100 & None\\
Size DblN start logit REC PC DIS(5) & -10.000 & -2 & (-11, -9) &  &  & None\\
Size DblN end logit REC PC DIS(5) & -10.000 & -2 & (-11, -9) &  &  & None\\
Size DblN peak REC PR(6) & 40.022 & 2 & (10, 50) & OK & 2.2625100 & None\\
\addlinespace
Size DblN top logit REC PR(6) & -9.000 & -2 & (-12, 0) &  &  & None\\
Size DblN ascend se REC PR(6) & 4.416 & 2 & (1, 10) & OK & 0.3324090 & None\\
Size DblN descend se REC PR(6) & 5.522 & 2 & (1, 10) & OK & 0.5701240 & None\\
Size DblN start logit REC PR(6) & -10.000 & -2 & (-11, -9) &  &  & None\\
Size DblN end logit REC PR(6) & -10.000 & -2 & (-11, 11) &  &  & None\\
\addlinespace
Size DblN peak NWFSC TWL(9) & 16.300 & 2 & (10, 60) & OK & 0.8883540 & None\\
Size DblN top logit NWFSC TWL(9) & -9.000 & -2 & (-12, 0) &  &  & None\\
Size DblN ascend se NWFSC TWL(9) & 0.977 & 2 & (0.05, 10) & OK & 1.0706500 & None\\
Size DblN descend se NWFSC TWL(9) & 0.070 & -2 & (0.05, 10) &  &  & None\\
Size DblN start logit NWFSC TWL(9) & -10.000 & -2 & (-11, -9) &  &  & None\\
\addlinespace
Size DblN end logit NWFSC TWL(9) & -0.433 & 2 & (-10, 10) & OK & 1.0731400 & None\\
Size DblN peak CCFRP(13) & 40.886 & 2 & (20, 60) & OK & 1.8257500 & None\\
Size DblN top logit CCFRP(13) & -9.000 & -2 & (-12, 0) &  &  & None\\
Size DblN ascend se CCFRP(13) & 4.986 & 2 & (1, 8) & OK & 0.2970620 & None\\
Size DblN descend se CCFRP(13) & 4.716 & 2 & (1, 8) & OK & 0.5171640 & None\\
\addlinespace
Size DblN start logit CCFRP(13) & -10.000 & -2 & (-11, -9) &  &  & None\\
Size DblN end logit CCFRP(13) & -8.000 & -2 & (-10, 10) &  &  & None\\
Size DblN peak REC PC(4) BLK1repl 1875 & 34.896 & 3 & (20, 50) & OK & 1.4011300 & None\\
Size DblN peak REC PC(4) BLK1repl 2002 & 34.171 & 3 & (20, 50) & OK & 1.5347700 & None\\
Size DblN ascend se REC PC(4) BLK1repl 1875 & 4.236 & 3 & (1, 8) & OK & 0.2511310 & None\\
\addlinespace
Size DblN ascend se REC PC(4) BLK1repl 2002 & 4.224 & 3 & (1, 8) & OK & 0.3016690 & None\\
Size DblN descend se REC PC(4) BLK1repl 1875 & 4.416 & 3 & (0.05, 10) & OK & 0.6250990 & None\\
Size DblN descend se REC PC(4) BLK1repl 2002 & 5.390 & 3 & (0.05, 10) & OK & 0.4656160 & None\\
Size DblN end logit REC PC(4) BLK1repl 1875 & -0.589 & 3 & (-8, 9) & OK & 0.4286590 & None\\
Size DblN end logit REC PC(4) BLK1repl 2002 & -2.097 & 3 & (-8, 9) & OK & 1.2891700 & None\\
\addlinespace
Size DblN peak REC PR(6) BLK1repl 1875 & 34.377 & 3 & (20, 50) & OK & 1.2105900 & None\\
Size DblN peak REC PR(6) BLK1repl 2002 & 36.471 & 3 & (20, 50) & OK & 0.7783140 & None\\
Size DblN ascend se REC PR(6) BLK1repl 1875 & 4.259 & 3 & (0.05, 9) & OK & 0.2307990 & None\\
Size DblN ascend se REC PR(6) BLK1repl 2002 & 4.105 & 3 & (0.05, 9) & OK & 0.1512450 & None\\
Size DblN descend se REC PR(6) BLK1repl 1875 & 2.778 & 3 & (0.05, 10) & OK & 0.8692620 & None\\
\addlinespace
Size DblN descend se REC PR(6) BLK1repl 2002 & 5.253 & 3 & (0.05, 10) & OK & 0.3040040 & None\\
Size DblN end logit REC PR(6) BLK1repl 1875 & -0.398 & 3 & (-8, 9) & OK & 0.3136130 & None\\
Size DblN end logit REC PR(6) BLK1repl 2002 & -1.752 & 3 & (-8, 9) & OK & 0.7407420 & None\\
Size DblN peak CCFRP(13) BLK2repl 1875 & 35.330 & 3 & (20, 60) & OK & 1.0697500 & None\\
Size DblN ascend se CCFRP(13) BLK2repl 1875 & 4.745 & 3 & (1, 8) & OK & 0.1936110 & None\\
\addlinespace
Size DblN descend se CCFRP(13) BLK2repl 1875 & 4.082 & 3 & (1, 8) & OK & 0.2258800 & None\\
Size DblN end logit CCFRP(13) BLK2repl 1875 & -8.000 & -3 & (-10, 10) &  &  & None\\*
\end{longtable}
\leavevmode\tagmcend\tagstructend\par
\endgroup{}
\end{landscape}

\begin{landscape}\begingroup\fontsize{8}{10}\selectfont

\tagstructbegin{tag=Table}\tagmcbegin{tag=Table}
\begin{longtable}[t]{llrlrrrrrr}
\caption{\label{tab:data-weights}Suggested data-weighting for length and age composition data using the McAllister-Ianelli and Francis approaches, after five tuning iterations to the pre-STAR base model.}\\
\toprule
Method & Data Type & Fleet No. & Fleet Name & Iteration 1 & Iteration 2 & Iteration 3 & Iteration 4 & Iteration 5 & Base Model\\
\midrule
\endfirsthead
\caption[]{\label{tab:data-weights}Suggested data-weighting for length and age composition data using the McAllister-Ianelli and Francis approaches, after five tuning iterations to the pre-STAR base model. \textit{(continued)}}\\
\toprule
Method & Data Type & Fleet No. & Fleet Name & Iteration 1 & Iteration 2 & Iteration 3 & Iteration 4 & Iteration 5 & Base Model\\
\midrule
\endhead

\endfoot
\bottomrule
\endlastfoot
Francis & Length & 1 & COM\_HKL & 0.428 & 0.385 & 0.348 & 0.329 & 0.319 & 0.304\\
Francis & Length & 2 & COM\_TWL & 4.310 & 4.941 & 4.630 & 4.543 & 4.489 & 4.388\\
Francis & Length & 3 & COM\_NET & 0.596 & 0.615 & 0.603 & 0.598 & 0.595 & 0.590\\
Francis & Length & 4 & REC\_PC & 0.207 & 0.156 & 0.148 & 0.144 & 0.142 & 0.140\\
Francis & Length & 5 & REC\_PC\_DIS & 0.196 & 0.192 & 0.193 & 0.195 & 0.196 & 0.196\\
Francis & Length & 6 & REC\_PR & 0.166 & 0.113 & 0.103 & 0.097 & 0.095 & 0.093\\
Francis & Length & 8 & DWV\_ONBOARD & 0.260 & 0.222 & 0.220 & 0.218 & 0.217 & 0.215\\
Francis & Length & 9 & NWFSC\_TWL & 0.202 & 0.229 & 0.246 & 0.247 & 0.247 & 0.246\\
Francis & Length & 11 & ABRAMS\_RESEARCH & 0.961 & 1.408 & 1.729 & 1.918 & 2.018 & 2.161\\
Francis & Length & 12 & SWFSC\_GF\_ECOL & 0.511 & 0.480 & 0.456 & 0.438 & 0.427 & 0.411\\
Francis & Length & 13 & CCFRP & 0.120 & 0.057 & 0.049 & 0.047 & 0.047 & 0.046\\
Francis & Ages & 9 & NWFSC\_TWL & 1.110 & 0.565 & 0.585 & 0.578 & 0.575 & 0.577\\
Francis & Ages & 11 & ABRAMS\_RESEARCH & 1.060 & 0.922 & 0.893 & 0.880 & 0.876 & 0.869\\
Francis & Ages & 12 & SWFSC\_GF\_ECOL & 0.388 & 0.287 & 0.280 & 0.275 & 0.273 & 0.269\\
M-I & Length & 1 & COM\_HKL & 0.991 & 0.985 & 0.983 & 0.982 & 0.982 & \\
M-I & Length & 2 & COM\_TWL & 6.483 & 6.669 & 6.700 & 6.705 & 6.706 & \\
M-I & Length & 3 & COM\_NET & 3.619 & 3.689 & 3.691 & 3.692 & 3.692 & \\
M-I & Length & 4 & REC\_PC & 0.622 & 0.610 & 0.607 & 0.606 & 0.606 & \\
M-I & Length & 5 & REC\_PC\_DIS & 0.302 & 0.302 & 0.303 & 0.303 & 0.303 & \\
M-I & Length & 6 & REC\_PR & 0.208 & 0.185 & 0.180 & 0.179 & 0.179 & \\
M-I & Length & 8 & DWV\_ONBOARD & 1.007 & 1.283 & 1.330 & 1.336 & 1.337 & \\
M-I & Length & 9 & NWFSC\_TWL & 0.647 & 0.644 & 0.643 & 0.643 & 0.642 & \\
M-I & Length & 11 & ABRAMS\_RESEARCH & 0.812 & 0.791 & 0.783 & 0.781 & 0.780 & \\
M-I & Length & 12 & SWFSC\_GF\_ECOL & 1.897 & 2.036 & 2.055 & 2.058 & 2.059 & \\
M-I & Length & 13 & CCFRP & 0.792 & 0.861 & 0.886 & 0.894 & 0.896 & \\
M-I & Ages & 9 & NWFSC\_TWL & 0.425 & 0.408 & 0.408 & 0.409 & 0.409 & \\
M-I & Ages & 11 & ABRAMS\_RESEARCH & 0.521 & 0.524 & 0.524 & 0.524 & 0.524 & \\
M-I & Ages & 12 & SWFSC\_GF\_ECOL & 0.432 & 0.433 & 0.433 & 0.433 & 0.433 & \\*
\end{longtable}
\leavevmode\tagmcend\tagstructend\par
\endgroup{}
\end{landscape}

\begingroup\fontsize{8}{10}\selectfont

\tagstructbegin{tag=Table}\tagmcbegin{tag=Table}
\begin{longtable}[t]{>{\centering\arraybackslash}p{.3in}>{\centering\arraybackslash}p{.6in}>{\centering\arraybackslash}p{.6in}>{\centering\arraybackslash}p{.6in}>{\centering\arraybackslash}p{.6in}>{\centering\arraybackslash}p{.6in}>{\centering\arraybackslash}p{.6in}>{\centering\arraybackslash}p{.6in}>{\centering\arraybackslash}p{.6in}}
\caption{\label{tab:timeseries}Time series of population estimates from the base model.}\\
\toprule
Year & Total Biomass (mt) & Spawning Output ($10^6$ eggs) & Total Biomass 4+ (mt) & Fraction Unfished & Age-0 Recruits & Total Mortality (mt) & $\frac{1-SPR}{1-SPR_{50\%}}$ & Exploitation Rate\\
\midrule
\endfirsthead
\caption[]{\label{tab:timeseries}Time series of population estimates from the base model. \textit{(continued)}}\\
\toprule
Year & Total Biomass (mt) & Spawning Output ($10^6$ eggs) & Total Biomass 4+ (mt) & Fraction Unfished & Age-0 Recruits & Total Mortality (mt) & $\frac{1-SPR}{1-SPR_{50\%}}$ & Exploitation Rate\\
\midrule
\endhead

\endfoot
\bottomrule
\endlastfoot
1875 & 6457.95 & 1145.180 & 6341.79 & 1.000 & 420.189 & 0.240 & 0.001 & 0.000\\
1876 & 6457.72 & 1145.140 & 6341.55 & 1.000 & 420.188 & 0.481 & 0.003 & 0.000\\
1877 & 6457.27 & 1145.050 & 6341.10 & 1.000 & 420.184 & 0.721 & 0.004 & 0.000\\
1878 & 6456.60 & 1144.910 & 6340.43 & 1.000 & 420.180 & 0.961 & 0.005 & 0.000\\
1879 & 6455.74 & 1144.730 & 6339.57 & 1.000 & 420.173 & 1.201 & 0.007 & 0.000\\
1880 & 6454.68 & 1144.510 & 6338.52 & 0.999 & 420.165 & 1.442 & 0.008 & 0.000\\
1881 & 6453.45 & 1144.250 & 6337.28 & 0.999 & 420.156 & 1.682 & 0.010 & 0.000\\
1882 & 6452.04 & 1143.960 & 6335.88 & 0.999 & 420.146 & 1.922 & 0.011 & 0.000\\
1883 & 6450.48 & 1143.630 & 6334.32 & 0.999 & 420.134 & 2.163 & 0.012 & 0.000\\
1884 & 6448.77 & 1143.260 & 6332.61 & 0.998 & 420.121 & 2.403 & 0.014 & 0.000\\
1885 & 6446.92 & 1142.870 & 6330.77 & 0.998 & 420.107 & 2.643 & 0.015 & 0.000\\
1886 & 6444.94 & 1142.440 & 6328.79 & 0.998 & 420.091 & 2.884 & 0.016 & 0.000\\
1887 & 6442.83 & 1141.990 & 6326.68 & 0.997 & 420.075 & 3.124 & 0.018 & 0.000\\
1888 & 6440.61 & 1141.510 & 6324.47 & 0.997 & 420.058 & 3.364 & 0.019 & 0.001\\
1889 & 6438.28 & 1141.000 & 6322.14 & 0.996 & 420.040 & 3.604 & 0.020 & 0.001\\
1890 & 6435.84 & 1140.470 & 6319.71 & 0.996 & 420.021 & 3.845 & 0.022 & 0.001\\
1891 & 6433.32 & 1139.920 & 6317.19 & 0.995 & 420.001 & 4.085 & 0.023 & 0.001\\
1892 & 6430.70 & 1139.350 & 6314.57 & 0.995 & 419.980 & 4.325 & 0.025 & 0.001\\
1893 & 6428.00 & 1138.760 & 6311.88 & 0.994 & 419.959 & 4.082 & 0.023 & 0.001\\
1894 & 6425.68 & 1138.250 & 6309.57 & 0.994 & 419.941 & 3.839 & 0.022 & 0.001\\
1895 & 6423.74 & 1137.810 & 6307.63 & 0.994 & 419.925 & 3.596 & 0.020 & 0.001\\
1896 & 6422.16 & 1137.440 & 6306.05 & 0.993 & 419.912 & 3.396 & 0.019 & 0.001\\
1897 & 6420.87 & 1137.140 & 6304.77 & 0.993 & 419.901 & 3.195 & 0.018 & 0.001\\
1898 & 6419.87 & 1136.890 & 6303.77 & 0.993 & 419.892 & 2.995 & 0.017 & 0.000\\
1899 & 6419.13 & 1136.710 & 6303.04 & 0.993 & 419.885 & 2.794 & 0.016 & 0.000\\
1900 & 6418.64 & 1136.580 & 6302.55 & 0.992 & 419.880 & 3.091 & 0.018 & 0.000\\
1901 & 6417.91 & 1136.400 & 6301.83 & 0.992 & 419.874 & 3.389 & 0.019 & 0.001\\
1902 & 6416.94 & 1136.190 & 6300.86 & 0.992 & 419.866 & 3.686 & 0.021 & 0.001\\
1903 & 6415.74 & 1135.930 & 6299.66 & 0.992 & 419.857 & 3.983 & 0.023 & 0.001\\
1904 & 6414.32 & 1135.620 & 6298.24 & 0.992 & 419.846 & 4.281 & 0.024 & 0.001\\
1905 & 6412.69 & 1135.280 & 6296.62 & 0.991 & 419.833 & 4.574 & 0.026 & 0.001\\
1906 & 6410.87 & 1134.890 & 6294.79 & 0.991 & 419.819 & 4.867 & 0.028 & 0.001\\
1907 & 6408.86 & 1134.460 & 6292.79 & 0.991 & 419.804 & 5.161 & 0.029 & 0.001\\
1908 & 6406.67 & 1133.990 & 6290.60 & 0.990 & 419.787 & 5.454 & 0.031 & 0.001\\
1909 & 6404.32 & 1133.490 & 6288.26 & 0.990 & 419.768 & 6.137 & 0.035 & 0.001\\
1910 & 6401.44 & 1132.880 & 6285.38 & 0.989 & 419.746 & 6.820 & 0.039 & 0.001\\
1911 & 6398.05 & 1132.160 & 6281.99 & 0.989 & 419.720 & 7.504 & 0.043 & 0.001\\
1912 & 6394.18 & 1131.340 & 6278.13 & 0.988 & 419.690 & 8.187 & 0.046 & 0.001\\
1913 & 6389.85 & 1130.420 & 6273.81 & 0.987 & 419.657 & 8.870 & 0.050 & 0.001\\
1914 & 6385.10 & 1129.410 & 6269.07 & 0.986 & 419.620 & 9.553 & 0.054 & 0.002\\
1915 & 6379.94 & 1128.300 & 6263.92 & 0.985 & 419.579 & 10.236 & 0.058 & 0.002\\
1916 & 6374.41 & 1127.120 & 6258.39 & 0.984 & 419.536 & 11.479 & 0.065 & 0.002\\
1917 & 6367.97 & 1125.740 & 6251.97 & 0.983 & 419.485 & 18.544 & 0.103 & 0.003\\
1918 & 6355.07 & 1123.070 & 6239.08 & 0.981 & 419.387 & 17.480 & 0.098 & 0.003\\
1919 & 6343.82 & 1120.690 & 6227.84 & 0.979 & 419.298 & 10.312 & 0.059 & 0.002\\
1920 & 6340.07 & 1119.770 & 6224.11 & 0.978 & 419.264 & 11.201 & 0.064 & 0.002\\
1921 & 6335.77 & 1118.750 & 6219.84 & 0.977 & 419.227 & 9.875 & 0.056 & 0.002\\
1922 & 6333.05 & 1118.060 & 6217.13 & 0.976 & 419.201 & 9.584 & 0.055 & 0.002\\
1923 & 6330.82 & 1117.490 & 6214.92 & 0.976 & 419.180 & 12.502 & 0.071 & 0.002\\
1924 & 6325.96 & 1116.420 & 6210.07 & 0.975 & 419.140 & 16.488 & 0.093 & 0.003\\
1925 & 6317.54 & 1114.640 & 6201.65 & 0.973 & 419.073 & 18.569 & 0.104 & 0.003\\
1926 & 6307.53 & 1112.530 & 6191.65 & 0.971 & 418.994 & 22.964 & 0.128 & 0.004\\
1927 & 6293.79 & 1109.670 & 6177.92 & 0.969 & 418.886 & 19.632 & 0.110 & 0.003\\
1928 & 6283.95 & 1107.540 & 6168.10 & 0.967 & 418.806 & 18.350 & 0.105 & 0.003\\
1929 & 6275.80 & 1105.800 & 6159.98 & 0.966 & 418.740 & 20.069 & 0.116 & 0.003\\
1930 & 6266.32 & 1103.850 & 6150.52 & 0.964 & 418.666 & 21.504 & 0.124 & 0.003\\
1931 & 6255.85 & 1101.720 & 6140.08 & 0.962 & 418.584 & 11.819 & 0.072 & 0.002\\
1932 & 6255.12 & 1101.550 & 6139.37 & 0.962 & 418.578 & 35.932 & 0.201 & 0.006\\
1933 & 6231.08 & 1096.840 & 6115.36 & 0.958 & 418.397 & 12.164 & 0.077 & 0.002\\
1934 & 6230.87 & 1096.810 & 6115.16 & 0.958 & 418.396 & 24.388 & 0.145 & 0.004\\
1935 & 6218.79 & 1094.490 & 6103.10 & 0.956 & 418.306 & 31.681 & 0.184 & 0.005\\
1936 & 6200.04 & 1090.870 & 6084.39 & 0.953 & 418.165 & 33.450 & 0.195 & 0.005\\
1937 & 6180.24 & 1087.010 & 6064.60 & 0.949 & 418.015 & 36.386 & 0.212 & 0.006\\
1938 & 6158.28 & 1082.730 & 6042.67 & 0.945 & 417.847 & 28.148 & 0.170 & 0.005\\
1939 & 6145.17 & 1080.110 & 6029.60 & 0.943 & 417.743 & 26.763 & 0.161 & 0.004\\
1940 & 6134.13 & 1077.790 & 6018.60 & 0.941 & 417.651 & 35.201 & 0.210 & 0.006\\
1941 & 6115.25 & 1074.110 & 5999.77 & 0.938 & 417.504 & 36.802 & 0.218 & 0.006\\
1942 & 6095.66 & 1070.180 & 5980.20 & 0.935 & 417.346 & 15.666 & 0.099 & 0.003\\
1943 & 6097.81 & 1070.210 & 5982.37 & 0.935 & 417.347 & 18.086 & 0.112 & 0.003\\
1944 & 6098.00 & 1069.860 & 5982.60 & 0.934 & 417.333 & 24.392 & 0.144 & 0.004\\
1945 & 6092.61 & 1068.370 & 5977.24 & 0.933 & 417.273 & 46.379 & 0.259 & 0.008\\
1946 & 6066.56 & 1062.850 & 5951.19 & 0.928 & 417.049 & 49.264 & 0.280 & 0.008\\
1947 & 6038.62 & 1057.190 & 5923.27 & 0.923 & 416.817 & 24.937 & 0.153 & 0.004\\
1948 & 6035.50 & 1056.290 & 5920.19 & 0.922 & 416.780 & 41.344 & 0.250 & 0.007\\
1949 & 6016.16 & 1052.710 & 5900.92 & 0.919 & 416.632 & 41.637 & 0.258 & 0.007\\
1950 & 5996.61 & 1049.340 & 5881.42 & 0.916 & 416.492 & 57.103 & 0.341 & 0.010\\
1951 & 5962.08 & 1043.210 & 5846.92 & 0.911 & 416.234 & 97.256 & 0.523 & 0.017\\
1952 & 5889.33 & 1029.570 & 5774.23 & 0.899 & 415.652 & 65.834 & 0.388 & 0.011\\
1953 & 5849.64 & 1021.890 & 5734.59 & 0.892 & 415.319 & 47.731 & 0.297 & 0.008\\
1954 & 5829.43 & 1017.640 & 5714.47 & 0.889 & 415.132 & 63.144 & 0.379 & 0.011\\
1955 & 5795.16 & 1010.820 & 5680.34 & 0.883 & 414.830 & 88.583 & 0.498 & 0.016\\
1956 & 5737.76 & 999.421 & 5623.03 & 0.873 & 414.316 & 79.676 & 0.471 & 0.014\\
1957 & 5690.69 & 990.309 & 5576.04 & 0.865 & 413.897 & 91.083 & 0.527 & 0.016\\
1958 & 5634.15 & 979.363 & 5519.61 & 0.855 & 413.386 & 121.435 & 0.674 & 0.022\\
1959 & 5548.86 & 963.632 & 5434.50 & 0.841 & 412.633 & 91.794 & 0.556 & 0.017\\
1960 & 5495.17 & 953.454 & 5380.92 & 0.833 & 412.134 & 71.784 & 0.449 & 0.013\\
1961 & 5464.32 & 946.547 & 5350.20 & 0.827 & 411.789 & 56.711 & 0.367 & 0.011\\
1962 & 5450.96 & 942.553 & 5337.00 & 0.823 & 411.588 & 69.341 & 0.441 & 0.013\\
1963 & 5426.52 & 936.985 & 5312.70 & 0.818 & 411.306 & 78.880 & 0.489 & 0.015\\
1964 & 5394.49 & 929.991 & 5280.75 & 0.812 & 410.946 & 64.498 & 0.417 & 0.012\\
1965 & 5378.50 & 925.931 & 5264.82 & 0.809 & 410.735 & 85.459 & 0.536 & 0.016\\
1966 & 5342.24 & 918.924 & 5228.67 & 0.802 & 410.368 & 91.600 & 0.569 & 0.018\\
1967 & 5301.05 & 911.001 & 5187.58 & 0.796 & 409.946 & 98.364 & 0.596 & 0.019\\
1968 & 5255.18 & 901.738 & 5141.79 & 0.787 & 409.445 & 93.686 & 0.591 & 0.018\\
1969 & 5214.97 & 894.004 & 5101.69 & 0.781 & 409.019 & 102.856 & 0.635 & 0.020\\
1970 & 5166.83 & 884.693 & 5054.09 & 0.773 & 346.965 & 130.721 & 0.755 & 0.026\\
1971 & 5092.18 & 870.471 & 4981.02 & 0.760 & 317.689 & 130.677 & 0.760 & 0.026\\
1972 & 5016.08 & 856.438 & 4910.74 & 0.748 & 282.120 & 164.526 & 0.902 & 0.034\\
1973 & 4900.84 & 837.002 & 4809.39 & 0.731 & 241.104 & 205.261 & 1.053 & 0.043\\
1974 & 4736.19 & 810.859 & 4653.67 & 0.708 & 200.838 & 226.049 & 1.125 & 0.049\\
1975 & 4540.43 & 780.610 & 4468.21 & 0.682 & 171.651 & 219.141 & 1.136 & 0.049\\
1976 & 4338.58 & 751.211 & 4277.19 & 0.656 & 161.263 & 259.520 & 1.264 & 0.061\\
1977 & 4083.33 & 713.366 & 4031.27 & 0.623 & 169.023 & 230.890 & 1.232 & 0.057\\
1978 & 3844.81 & 678.980 & 3798.14 & 0.593 & 204.136 & 276.837 & 1.362 & 0.073\\
1979 & 3553.55 & 633.680 & 3507.25 & 0.553 & 215.419 & 358.233 & 1.551 & 0.102\\
1980 & 3181.04 & 571.196 & 3130.40 & 0.499 & 220.993 & 337.873 & 1.592 & 0.108\\
1981 & 2835.44 & 510.402 & 2777.92 & 0.446 & 252.470 & 199.414 & 1.308 & 0.072\\
1982 & 2640.15 & 472.735 & 2578.95 & 0.413 & 323.285 & 315.991 & 1.633 & 0.123\\
1983 & 2343.47 & 413.522 & 2277.67 & 0.361 & 326.166 & 132.770 & 1.243 & 0.058\\
1984 & 2249.88 & 389.422 & 2170.86 & 0.340 & 640.086 & 131.391 & 1.293 & 0.061\\
1985 & 2184.84 & 367.228 & 2082.72 & 0.321 & 1364.890 & 126.880 & 1.276 & 0.061\\
1986 & 2164.33 & 347.128 & 2029.06 & 0.303 & 310.434 & 136.868 & 1.338 & 0.067\\
1987 & 2207.34 & 327.946 & 1983.22 & 0.286 & 198.765 & 245.180 & 1.648 & 0.124\\
1988 & 2191.75 & 293.898 & 1931.71 & 0.257 & 202.897 & 286.600 & 1.733 & 0.148\\
1989 & 2148.83 & 262.807 & 2074.96 & 0.229 & 314.940 & 114.534 & 1.193 & 0.055\\
1990 & 2265.37 & 264.830 & 2206.98 & 0.231 & 371.258 & 247.955 & 1.614 & 0.112\\
1991 & 2230.42 & 259.567 & 2162.19 & 0.227 & 285.484 & 277.263 & 1.666 & 0.128\\
1992 & 2143.36 & 256.142 & 2053.77 & 0.224 & 287.513 & 259.934 & 1.666 & 0.127\\
1993 & 2052.02 & 254.118 & 1959.34 & 0.222 & 360.886 & 364.615 & 1.817 & 0.186\\
1994 & 1844.78 & 232.480 & 1762.71 & 0.203 & 715.994 & 231.089 & 1.687 & 0.131\\
1995 & 1765.66 & 224.142 & 1671.85 & 0.196 & 432.086 & 160.390 & 1.521 & 0.096\\
1996 & 1775.17 & 222.943 & 1646.89 & 0.195 & 319.144 & 131.419 & 1.394 & 0.080\\
1997 & 1835.33 & 224.090 & 1671.14 & 0.196 & 356.441 & 152.875 & 1.474 & 0.091\\
1998 & 1887.00 & 223.131 & 1778.34 & 0.195 & 487.581 & 120.203 & 1.311 & 0.068\\
1999 & 1981.63 & 227.935 & 1883.77 & 0.199 & 872.018 & 121.399 & 1.275 & 0.064\\
2000 & 2082.04 & 238.541 & 1962.76 & 0.208 & 456.099 & 87.959 & 1.019 & 0.045\\
2001 & 2236.81 & 257.370 & 2073.76 & 0.225 & 354.237 & 68.729 & 0.837 & 0.033\\
2002 & 2426.16 & 280.246 & 2234.09 & 0.245 & 160.435 & 75.965 & 0.804 & 0.034\\
2003 & 2608.49 & 304.085 & 2498.83 & 0.266 & 145.627 & 185.721 & 1.335 & 0.074\\
2004 & 2655.77 & 317.366 & 2579.41 & 0.277 & 169.288 & 86.888 & 0.822 & 0.034\\
2005 & 2768.19 & 343.800 & 2724.26 & 0.300 & 288.417 & 163.640 & 1.198 & 0.060\\
2006 & 2763.10 & 363.725 & 2717.26 & 0.318 & 312.414 & 148.195 & 1.139 & 0.055\\
2007 & 2743.71 & 382.850 & 2681.88 & 0.334 & 613.925 & 126.823 & 1.057 & 0.047\\
2008 & 2729.68 & 399.411 & 2640.45 & 0.349 & 489.007 & 71.262 & 0.731 & 0.027\\
2009 & 2780.98 & 417.505 & 2665.12 & 0.365 & 662.871 & 89.049 & 0.870 & 0.033\\
2010 & 2829.01 & 427.774 & 2671.22 & 0.374 & 247.817 & 105.254 & 0.988 & 0.039\\
2011 & 2883.10 & 431.973 & 2741.11 & 0.377 & 224.973 & 100.124 & 0.939 & 0.037\\
2012 & 2952.51 & 435.431 & 2813.22 & 0.380 & 407.824 & 87.083 & 0.826 & 0.031\\
2013 & 3032.45 & 442.395 & 2961.29 & 0.386 & 465.847 & 75.837 & 0.715 & 0.026\\
2014 & 3120.45 & 454.034 & 3037.34 & 0.396 & 475.537 & 77.220 & 0.701 & 0.025\\
2015 & 3205.37 & 469.146 & 3087.71 & 0.410 & 277.184 & 123.528 & 0.966 & 0.040\\
2016 & 3250.53 & 479.639 & 3118.04 & 0.419 & 1536.160 & 112.080 & 0.905 & 0.036\\
2017 & 3307.80 & 490.602 & 3173.25 & 0.428 & 162.592 & 165.560 & 1.108 & 0.052\\
2018 & 3359.55 & 490.707 & 3184.58 & 0.428 & 387.483 & 180.495 & 1.164 & 0.057\\
2019 & 3419.45 & 487.751 & 3135.42 & 0.426 & 372.609 & 204.445 & 1.248 & 0.065\\
2020 & 3461.79 & 482.178 & 3393.48 & 0.421 & 373.837 & 139.006 & 0.990 & 0.041\\
2021 & 3564.40 & 489.439 & 3459.01 & 0.427 & 371.777 & 148.994 & 1.000 & 0.043\\
2022 & 3642.30 & 501.884 & 3539.37 & 0.438 & 373.641 & 156.383 & 1.000 & 0.044\\
2023 & 3693.75 & 518.613 & 3590.72 & 0.453 & 376.032 & 141.065 & 0.916 & 0.039\\
2024 & 3741.37 & 538.451 & 3638.45 & 0.470 & 378.712 & 142.518 & 0.909 & 0.039\\
2025 & 3770.44 & 555.898 & 3666.95 & 0.485 & 380.939 & 142.328 & 0.904 & 0.039\\
2026 & 3785.58 & 569.855 & 3681.40 & 0.498 & 382.641 & 140.918 & 0.898 & 0.038\\
2027 & 3791.16 & 580.383 & 3686.31 & 0.507 & 383.879 & 138.819 & 0.892 & 0.038\\
2028 & 3790.77 & 587.989 & 3685.38 & 0.513 & 384.751 & 136.434 & 0.886 & 0.037\\
2029 & 3787.10 & 593.289 & 3681.30 & 0.518 & 385.348 & 134.186 & 0.881 & 0.036\\
2030 & 3781.90 & 596.847 & 3675.80 & 0.521 & 385.743 & 131.889 & 0.875 & 0.036\\
2031 & 3776.60 & 599.185 & 3670.30 & 0.523 & 386.001 & 129.775 & 0.870 & 0.035\\
2032 & 3771.99 & 600.698 & 3665.54 & 0.525 & 386.167 & 128.020 & 0.864 & 0.035\\*
\end{longtable}
\leavevmode\tagmcend\tagstructend\par
\endgroup{}

\FloatBarrier

\begin{landscape}\begin{table}

\caption{\label{tab:drop-one}Likelihood components, parameter estimates and derived quantities from the leave one out analysis of the pre-STAR base model. Continued in the next table.}
\centering
\begin{tabular}[t]{lrrrrrr}
\toprule
\multicolumn{2}{c}{ } & \multicolumn{5}{c}{Fleet Removed} \\
\cmidrule(l{3pt}r{3pt}){3-7}
Label & pre-STAR base & COM HKL & COM TWL & COM NET & REC PC & REC PC DIS\\
\midrule
N.Parms & 115.000 & 113.000 & 113.000 & 111.000 & 104.000 & 112.000\\
TOTAL & 910.571 & 886.431 & 887.894 & 903.309 & 853.962 & 887.799\\
Survey & -55.121 & -55.341 & -55.069 & -54.993 & -46.506 & -55.296\\
Length\_comp & 370.792 & 347.782 & 348.279 & 363.172 & 314.154 & 348.598\\
Age\_comp & 581.647 & 581.938 & 582.039 & 581.783 & 577.342 & 581.765\\
Recruitment & 13.129 & 11.972 & 12.545 & 13.221 & 8.811 & 12.612\\
Parm\_priors & 0.116 & 0.072 & 0.092 & 0.121 & 0.153 & 0.113\\
NatM\_uniform\_Fem\_GP\_1 & 0.089 & 0.091 & 0.090 & 0.088 & 0.087 & 0.089\\
L\_at\_Amin\_Fem\_GP\_1 & 7.786 & 7.754 & 7.788 & 7.773 & 7.941 & 7.798\\
L\_at\_Amax\_Fem\_GP\_1 & 55.383 & 55.347 & 55.414 & 55.315 & 55.480 & 55.359\\
VonBert\_K\_Fem\_GP\_1 & 0.145 & 0.146 & 0.145 & 0.146 & 0.143 & 0.146\\
CV\_young\_Fem\_GP\_1 & 0.096 & 0.096 & 0.096 & 0.096 & 0.097 & 0.095\\
CV\_old\_Fem\_GP\_1 & 0.075 & 0.076 & 0.076 & 0.075 & 0.069 & 0.075\\
NatM\_uniform\_Mal\_GP\_1 & 0.084 & 0.087 & 0.085 & 0.084 & 0.082 & 0.084\\
L\_at\_Amin\_Mal\_GP\_1 & 6.025 & 5.992 & 5.984 & 6.019 & 6.134 & 6.073\\
L\_at\_Amax\_Mal\_GP\_1 & 49.896 & 49.861 & 49.835 & 49.873 & 49.931 & 49.897\\
VonBert\_K\_Mal\_GP\_1 & 0.199 & 0.200 & 0.200 & 0.199 & 0.197 & 0.198\\
CV\_young\_Mal\_GP\_1 & 0.076 & 0.076 & 0.076 & 0.075 & 0.075 & 0.075\\
SR\_LN(R0) & 6.072 & 6.130 & 6.093 & 6.065 & 6.001 & 6.078\\
Q\_extraSD\_REC\_PR(6) & 0.184 & 0.185 & 0.182 & 0.185 & 0.165 & 0.184\\
Bratio\_2021 & 0.391 & 0.427 & 0.397 & 0.386 & 0.276 & 0.394\\
SSB\_unfished & 1114.670 & 1111.270 & 1100.150 & 1104.010 & 1056.010 & 1116.470\\
Totbio\_unfished & 6264.570 & 6302.200 & 6219.740 & 6240.070 & 6008.800 & 6279.770\\
Recr\_unfished & 433.531 & 459.429 & 442.736 & 430.575 & 403.752 & 436.042\\
Dead\_Catch\_SPR & 140.884 & 145.028 & 141.881 & 140.347 & 134.758 & 141.325\\
OFLCatch\_2023 & 168.686 & 184.428 & 171.247 & 166.356 & 127.785 & 170.296\\
\bottomrule
\end{tabular}
\end{table}
\end{landscape}

\begin{landscape}\begin{table}

\caption{\label{tab:drop-one2}Likelihood components from the additional leave one out analysis of the pre-STAR base model. The column name is the fleet removed from the model.}
\centering
\begin{tabular}[t]{>{\raggedright\arraybackslash}p{1.2in}>{\raggedleft\arraybackslash}p{.6in}>{\raggedleft\arraybackslash}p{.6in}>{\raggedleft\arraybackslash}p{.6in}>{\raggedleft\arraybackslash}p{.6in}>{\raggedleft\arraybackslash}p{.6in}>{\raggedleft\arraybackslash}p{.6in}>{\raggedleft\arraybackslash}p{.6in}}
\toprule
Label & REC PR & DWV ONBOARD & NWFSC TWL & REC ONBOARD & ABRAMS RESEARCH & SWFSC GF ECOL & CCFRP\\
\midrule
N.Parms & 103.000 & 115.000 & 112.000 & 115.000 & 115.000 & 115.000 & 112.000\\
TOTAL & 842.732 & 901.725 & 453.806 & 915.877 & 723.727 & 815.093 & 903.442\\
Survey & -41.896 & -45.539 & -53.087 & -49.159 & -55.803 & -55.071 & -47.274\\
Length\_comp & 296.000 & 355.090 & 317.016 & 370.117 & 298.109 & 352.340 & 357.127\\
Age\_comp & 573.202 & 582.738 & 185.839 & 581.878 & 468.925 & 504.841 & 584.196\\
Recruitment & 15.378 & 9.408 & 3.887 & 12.928 & 12.300 & 12.815 & 9.221\\
Parm\_priors & 0.041 & 0.020 & 0.146 & 0.105 & 0.189 & 0.160 & 0.165\\
NatM\_uniform\_Fem\_GP\_1 & 0.094 & 0.096 & 0.082 & 0.089 & 0.090 & 0.087 & 0.086\\
L\_at\_Amin\_Fem\_GP\_1 & 7.873 & 7.894 & 2.665 & 7.858 & 7.907 & 8.042 & 7.682\\
L\_at\_Amax\_Fem\_GP\_1 & 55.398 & 55.461 & 54.476 & 55.475 & 55.895 & 55.465 & 55.360\\
VonBert\_K\_Fem\_GP\_1 & 0.143 & 0.144 & 0.177 & 0.144 & 0.142 & 0.141 & 0.147\\
CV\_young\_Fem\_GP\_1 & 0.103 & 0.097 & 0.054 & 0.097 & 0.103 & 0.094 & 0.095\\
CV\_old\_Fem\_GP\_1 & 0.058 & 0.074 & 0.091 & 0.075 & 0.085 & 0.075 & 0.075\\
NatM\_uniform\_Mal\_GP\_1 & 0.090 & 0.092 & 0.087 & 0.085 & 0.078 & 0.082 & 0.082\\
L\_at\_Amin\_Mal\_GP\_1 & 6.125 & 5.987 & 5.661 & 6.075 & 6.293 & 6.198 & 6.015\\
L\_at\_Amax\_Mal\_GP\_1 & 49.932 & 49.806 & 48.895 & 49.911 & 51.341 & 49.814 & 49.986\\
VonBert\_K\_Mal\_GP\_1 & 0.197 & 0.200 & 0.223 & 0.198 & 0.186 & 0.196 & 0.198\\
CV\_young\_Mal\_GP\_1 & 0.076 & 0.076 & 0.050 & 0.076 & 0.082 & 0.076 & 0.075\\
SR\_LN(R0) & 6.167 & 6.261 & 6.054 & 6.065 & 6.022 & 6.048 & 5.998\\
Q\_extraSD\_REC\_PR(6) &  & 0.183 & 0.190 & 0.186 & 0.184 & 0.189 & 0.156\\
Bratio\_2021 & 0.461 & 0.475 & 0.350 & 0.367 & 0.387 & 0.374 & 0.381\\
SSB\_unfished & 1062.550 & 1138.850 & 1258.060 & 1097.050 & 1075.040 & 1116.830 & 1105.680\\
Totbio\_unfished & 6200.400 & 6546.320 & 6381.090 & 6147.110 & 6464.640 & 6307.160 & 6117.130\\
Recr\_unfished & 476.663 & 523.510 & 425.995 & 430.385 & 412.280 & 423.416 & 402.527\\
Dead\_Catch\_SPR & 146.545 & 156.489 & 141.986 & 139.694 & 140.130 & 139.100 & 130.307\\
OFLCatch\_2023 & 210.223 & 212.608 & 144.434 & 160.312 & 168.305 & 161.547 & 122.502\\
\bottomrule
\end{tabular}
\end{table}
\end{landscape}

\begin{landscape}\begin{table}

\caption{\label{tab:sens1}Likelihood components from additional sensitivity runs to estimating steepness, starting recruitment deviations in 1965 or 1975, McAllister-Ianelli data weighting and estimating discard selecitity for the pre-STAR base model.}
\centering
\resizebox{\linewidth}{!}{
\begin{tabular}[t]{lrrrrrr}
\toprule
Label & pre-STAR\_base & est\_h & dev\_1965 & dev\_1975 & M-I\_wgts & disc\_selex\\
\midrule
N.Parms & 115.000 & 116.000 & 120.000 & 110.000 & 115.000 & 112.000\\
TOTAL & 910.571 & 907.729 & 910.111 & 916.042 & 1459.840 & 887.696\\
Survey & -55.121 & -53.639 & -55.143 & -53.784 & -23.580 & -55.272\\
Length\_comp & 370.792 & 368.226 & 370.773 & 374.478 & 975.367 & 348.521\\
Age\_comp & 581.647 & 581.438 & 581.650 & 582.166 & 481.182 & 581.739\\
Recruitment & 13.129 & 10.601 & 12.711 & 13.134 & 25.981 & 12.583\\
Parm\_priors & 0.116 & 1.096 & 0.111 & 0.041 & 0.879 & 0.118\\
NatM\_uniform\_Fem\_GP\_1 & 0.089 & 0.071 & 0.089 & 0.094 & 0.072 & 0.089\\
L\_at\_Amin\_Fem\_GP\_1 & 7.786 & 7.816 & 7.787 & 7.723 & 6.787 & 7.798\\
L\_at\_Amax\_Fem\_GP\_1 & 55.383 & 55.380 & 55.382 & 55.411 & 54.076 & 55.357\\
VonBert\_K\_Fem\_GP\_1 & 0.145 & 0.145 & 0.145 & 0.146 & 0.164 & 0.146\\
CV\_young\_Fem\_GP\_1 & 0.096 & 0.096 & 0.096 & 0.097 & 0.103 & 0.095\\
CV\_old\_Fem\_GP\_1 & 0.075 & 0.075 & 0.075 & 0.075 & 0.099 & 0.076\\
NatM\_uniform\_Mal\_GP\_1 & 0.084 & 0.066 & 0.084 & 0.090 & 0.062 & 0.084\\
L\_at\_Amin\_Mal\_GP\_1 & 6.025 & 6.032 & 6.024 & 5.981 & 5.335 & 6.075\\
L\_at\_Amax\_Mal\_GP\_1 & 49.896 & 49.896 & 49.893 & 49.929 & 49.751 & 49.897\\
VonBert\_K\_Mal\_GP\_1 & 0.199 & 0.199 & 0.199 & 0.199 & 0.203 & 0.198\\
CV\_young\_Mal\_GP\_1 & 0.076 & 0.076 & 0.076 & 0.076 & 0.083 & 0.075\\
SR\_LN(R0) & 6.072 & 5.646 & 6.077 & 6.109 & 5.748 & 6.082\\
SR\_BH\_steep & 0.720 & 0.949 & 0.720 & 0.720 & 0.720 & 0.720\\
Q\_extraSD\_REC\_PR(6) & 0.184 & 0.204 & 0.184 & 0.170 & 0.291 & 0.184\\
Bratio\_2021 & 0.391 & 0.354 & 0.392 & 0.403 & 0.450 & 0.395\\
SSB\_unfished & 1114.670 & 1122.630 & 1112.830 & 1039.250 & 1171.900 & 1124.530\\
Totbio\_unfished & 6264.570 & 5886.500 & 6262.020 & 5915.060 & 6674.760 & 6332.760\\
Recr\_unfished & 433.531 & 283.081 & 435.707 & 450.010 & 313.684 & 437.870\\
Dead\_Catch\_SPR & 140.884 & 122.568 & 141.190 & 139.207 & 121.059 & 142.990\\
OFLCatch\_2023 & 168.686 & 134.232 & 169.016 & 166.763 & 122.969 & 172.726\\
\bottomrule
\end{tabular}}
\end{table}
\end{landscape}

\begin{table}

\caption{\label{tab:sens3}Likelihood components from additional sensitivity runs conducted after the draft document was submitted, and before the STAR panel. Descriptions of each run are in the text and all models are sensitivities using the pre-STAR base model.}
\centering
\resizebox{\linewidth}{!}{
\begin{tabular}[t]{lrrrrr}
\toprule
Label & base & M=0.07 & all\_2asymp & all\_4domed & ricker3p\\
\midrule
N.Parms & 115.000 & 113.000 & 102.000 & 120.000 & 115.000\\
TOTAL & 910.571 & 913.403 & 941.375 & 903.410 & 910.344\\
Survey & -55.121 & -54.362 & -47.772 & -56.047 & -55.068\\
Length\_comp & 370.792 & 367.419 & 391.825 & 366.576 & 371.412\\
Age\_comp & 581.647 & 583.641 & 583.745 & 580.208 & 581.194\\
Recruitment & 13.129 & 16.033 & 13.533 & 12.433 & 10.747\\
Parm\_priors & 0.116 & 0.663 & 0.040 & 0.231 & 2.050\\
NatM\_uniform\_Fem\_GP\_1 & 0.089 & 0.070 & 0.112 & 0.082 & 0.089\\
L\_at\_Amin\_Fem\_GP\_1 & 7.786 & 7.654 & 8.164 & 7.813 & 7.826\\
L\_at\_Amax\_Fem\_GP\_1 & 55.383 & 54.861 & 54.327 & 56.252 & 55.492\\
VonBert\_K\_Fem\_GP\_1 & 0.145 & 0.149 & 0.145 & 0.142 & 0.144\\
CV\_young\_Fem\_GP\_1 & 0.096 & 0.095 & 0.099 & 0.100 & 0.096\\
CV\_old\_Fem\_GP\_1 & 0.075 & 0.076 & 0.083 & 0.072 & 0.075\\
NatM\_uniform\_Mal\_GP\_1 & 0.084 & 0.070 & 0.104 & 0.080 & 0.084\\
L\_at\_Amin\_Mal\_GP\_1 & 6.025 & 5.875 & 6.145 & 6.192 & 6.066\\
L\_at\_Amax\_Mal\_GP\_1 & 49.896 & 49.672 & 49.159 & 50.399 & 49.958\\
VonBert\_K\_Mal\_GP\_1 & 0.199 & 0.203 & 0.202 & 0.193 & 0.198\\
CV\_young\_Mal\_GP\_1 & 0.076 & 0.076 & 0.076 & 0.076 & 0.075\\
SR\_LN(R0) & 6.072 & 5.808 & 6.351 & 6.000 & 6.155\\
SR\_BH\_steep & 0.720 & 0.720 & 0.720 & 0.720 & \\
Q\_extraSD\_REC\_PR(6) & 0.184 & 0.181 & 0.227 & 0.173 & 0.194\\
SR\_RkrPower\_steep &  &  &  &  & 0.756\\
SR\_RkrPower\_gamma &  &  &  &  & 2.091\\
Bratio\_2021 & 0.391 & 0.272 & 0.447 & 0.374 & 0.399\\
SSB\_unfished & 1114.670 & 1299.090 & 812.554 & 1296.250 & 1216.720\\
Totbio\_unfished & 6264.570 & 6638.840 & 5455.100 & 6635.630 & 6818.550\\
Recr\_unfished & 433.531 & 332.940 & 573.138 & 403.454 & 470.973\\
Dead\_Catch\_SPR & 140.884 & 126.330 & 149.426 & 138.704 & 141.940\\
OFLCatch\_2023 & 168.686 & 117.583 & 192.884 & 164.012 & 185.183\\
\bottomrule
\end{tabular}}
\end{table}

\begin{landscape}\begin{table}

\caption{\label{tab:retro}Likelihood components from the retrospective analysis removing one to five years of data of the pre-STAR base model.}
\centering
\begin{tabular}[t]{lrrrrrr}
\toprule
Label & base & retro-1 & retro-2 & retro-3 & retro-4 & retro-5\\
\midrule
N.Parms & 115.000 & 115.000 & 115.000 & 115.000 & 115.000 & 115.000\\
TOTAL & 910.571 & 906.994 & 867.262 & 822.588 & 788.220 & 777.922\\
Survey & -55.121 & -55.766 & -53.996 & -51.246 & -48.585 & -46.604\\
Length\_comp & 370.792 & 368.179 & 359.674 & 349.063 & 339.825 & 332.774\\
Age\_comp & 581.647 & 582.287 & 548.815 & 520.694 & 493.187 & 488.214\\
Recruitment & 13.129 & 12.155 & 12.593 & 3.888 & 3.656 & 3.391\\
Parm\_priors & 0.116 & 0.131 & 0.168 & 0.181 & 0.126 & 0.139\\
NatM\_uniform\_Fem\_GP\_1 & 0.089 & 0.088 & 0.086 & 0.085 & 0.088 & 0.086\\
L\_at\_Amin\_Fem\_GP\_1 & 7.786 & 7.755 & 7.631 & 8.810 & 8.088 & 7.758\\
L\_at\_Amax\_Fem\_GP\_1 & 55.383 & 55.333 & 55.203 & 55.622 & 55.859 & 55.750\\
VonBert\_K\_Fem\_GP\_1 & 0.145 & 0.146 & 0.149 & 0.142 & 0.142 & 0.144\\
CV\_young\_Fem\_GP\_1 & 0.096 & 0.096 & 0.099 & 0.075 & 0.076 & 0.079\\
CV\_old\_Fem\_GP\_1 & 0.075 & 0.075 & 0.076 & 0.081 & 0.082 & 0.080\\
NatM\_uniform\_Mal\_GP\_1 & 0.084 & 0.083 & 0.081 & 0.081 & 0.084 & 0.084\\
L\_at\_Amin\_Mal\_GP\_1 & 6.025 & 6.010 & 5.935 & 7.013 & 6.981 & 6.850\\
L\_at\_Amax\_Mal\_GP\_1 & 49.896 & 49.898 & 49.780 & 49.883 & 49.926 & 49.829\\
VonBert\_K\_Mal\_GP\_1 & 0.199 & 0.199 & 0.201 & 0.199 & 0.197 & 0.199\\
CV\_young\_Mal\_GP\_1 & 0.076 & 0.075 & 0.076 & 0.066 & 0.065 & 0.063\\
SR\_LN(R0) & 6.072 & 6.060 & 6.011 & 6.005 & 6.027 & 6.012\\
Q\_extraSD\_REC\_PR(6) & 0.184 & 0.163 & 0.155 & 0.160 & 0.166 & 0.165\\
Bratio\_2021 & 0.391 & 0.398 & 0.372 & 0.349 & 0.376 & 0.375\\
SSB\_unfished & 1114.670 & 1120.050 & 1102.160 & 1142.560 & 1115.090 & 1128.810\\
Totbio\_unfished & 6264.570 & 6284.820 & 6181.330 & 6280.630 & 6109.090 & 6102.100\\
Recr\_unfished & 433.531 & 428.271 & 407.856 & 405.568 & 414.364 & 408.450\\
Dead\_Catch\_SPR & 140.884 & 139.496 & 134.971 & 132.663 & 114.791 & 113.918\\
OFLCatch\_2023 & 168.686 & 161.477 & 150.481 & 114.336 & 105.421 & 101.855\\
\bottomrule
\end{tabular}
\end{table}
\end{landscape}

\clearpage

\tagstructbegin{tag=H1}\tagmcbegin{tag=H1}

\hypertarget{figures}{%
\section{Figures}\label{figures}}

\leavevmode\tagmcend\tagstructend

\FloatBarrier

\begin{figure}
\centering
\includegraphics[width=0.8\textwidth,height=0.8\textheight]{C:/Stock_Assessments/VRML_Assessment_2021/GitHub/Vermilion_2021/doc/figures/assess_area.png}
\caption{Map of the assssment area with the 3 nm California state water boundary. The northern California model includes areas from Point Conception to the California-Oregon border and the southern California assessment includes areas from Point Concpetion to the USA-Mexico border. The boundary at Cape Mendocino is a Pacific Fishery Management Council boundary for management of the stock complex, provided for reference.\label{fig:assess-area}}
\end{figure}

\begin{figure}
\centering
\includegraphics[width=1\textwidth,height=1\textheight]{C:/Stock_Assessments/VRML_Assessment_2021/Model_files/NCA/FINAL_base/Verm_NoCA_base_files_Day1_Request1_drop2020PR_weight_CCFRP_lengths_hessian - Copy/plots/catch2 landings stacked.png}
\caption{Catch histories by fleet used in the base model (Commercial hook-and-line = COM\_HKL, Commercial trawl = COM\_TWL, Commercial net = COM\_NET, Recreational party/charter retained = REC\_PC, Recreational private/rental retained = REC\_PR, Recreational party/charter dead discards = REC\_PC\_DIS, Recreational private/rental dead discards = REC\_PR\_DIS).\label{fig:catch}}
\end{figure}

\begin{figure}
\centering
\includegraphics[width=1\textwidth,height=1\textheight]{C:/Stock_Assessments/VRML_Assessment_2021/Model_files/NCA/FINAL_base/Verm_NoCA_base_files_Day1_Request1_drop2020PR_weight_CCFRP_lengths_hessian - Copy/plots/data_plot2.png}
\caption{Summary of data sources used in the base model. See the text for fleet descriptions.\label{fig:data-plot}}
\end{figure}

\begin{figure}
\centering
\includegraphics[width=1\textwidth,height=1\textheight]{C:/Stock_Assessments/VRML_Assessment_2021/GitHub/Vermilion_2021/doc/figures/compare_pacfin_calcom.png}
\caption{Comparison of total California landings from CALCOM and PacFIN.\label{fig:calcom-pacfin}}
\end{figure}

\FloatBarrier

\FloatBarrier

\begin{figure}
\centering
\includegraphics[width=1\textwidth,height=1\textheight]{C:/Stock_Assessments/VRML_Assessment_2021/Model_files/NCA/FINAL_base/Verm_NoCA_base_files_Day1_Request1_drop2020PR_weight_CCFRP_lengths_hessian - Copy/plots/comp_lendat_bubflt1mkt0_page2.png}
\caption{Length composition data from the commercial hook-and-line fishery.\label{fig:len-data-COM-HKL}}
\end{figure}

\begin{figure}
\centering
\includegraphics[width=1\textwidth,height=1\textheight]{C:/Stock_Assessments/VRML_Assessment_2021/Model_files/NCA/FINAL_base/Verm_NoCA_base_files_Day1_Request1_drop2020PR_weight_CCFRP_lengths_hessian - Copy/plots/comp_lendat_bubflt2mkt0.png}
\caption{Length composition data from the commercial trawl fishery.\label{fig:len-data-COM-TWL}}
\end{figure}

\begin{figure}
\centering
\includegraphics[width=1\textwidth,height=1\textheight]{C:/Stock_Assessments/VRML_Assessment_2021/Model_files/NCA/FINAL_base/Verm_NoCA_base_files_Day1_Request1_drop2020PR_weight_CCFRP_lengths_hessian - Copy/plots/comp_lendat_bubflt3mkt0.png}
\caption{Length composition data from the commercial net fishery.\label{fig:len-data-COM-NET}}
\end{figure}

\begin{figure}
\centering
\includegraphics[width=1\textwidth,height=1\textheight]{C:/Stock_Assessments/VRML_Assessment_2021/GitHub/Vermilion_2021/doc/figures/rec_avg_weight.png}
\caption{Average weights calculated from the recreational landings data on RecFIN.\label{fig:rec-avg-weights}}
\end{figure}

\begin{figure}
\centering
\includegraphics[width=1\textwidth,height=1\textheight]{C:/Stock_Assessments/VRML_Assessment_2021/Model_files/NCA/FINAL_base/Verm_NoCA_base_files_Day1_Request1_drop2020PR_weight_CCFRP_lengths_hessian - Copy/plots/comp_lendat_bubflt4mkt0_page3.png}
\caption{Length composition data from the recreational PC retained fishery.\label{fig:len-data-REC-PC}}
\end{figure}

\begin{figure}
\centering
\includegraphics[width=1\textwidth,height=1\textheight]{C:/Stock_Assessments/VRML_Assessment_2021/Model_files/NCA/FINAL_base/Verm_NoCA_base_files_Day1_Request1_drop2020PR_weight_CCFRP_lengths_hessian - Copy/plots/comp_lendat_bubflt5mkt0.png}
\caption{Length composition data from the recreational PC discard fishery.\label{fig:len-data-REC-PC-DIS}}
\end{figure}

\begin{figure}
\centering
\includegraphics[width=1\textwidth,height=1\textheight]{C:/Stock_Assessments/VRML_Assessment_2021/Model_files/NCA/FINAL_base/Verm_NoCA_base_files_Day1_Request1_drop2020PR_weight_CCFRP_lengths_hessian - Copy/plots/comp_lendat_bubflt6mkt0_page3.png}
\caption{Length composition data from the recreational PR retained fishery.\label{fig:len-data-REC-PR}}
\end{figure}

\begin{figure}
\centering
\includegraphics[width=1\textwidth,height=1\textheight]{C:/Stock_Assessments/VRML_Assessment_2021/Model_files/NCA/FINAL_base/Verm_NoCA_base_files_Day1_Request1_drop2020PR_weight_CCFRP_lengths_hessian - Copy/plots/comp_lendat_bubflt8mkt0.png}
\caption{Length composition data from the Deb Wilson-Vandenberg onboard survey.\label{fig:len-data-DWV-ONBOARD}}
\end{figure}

\begin{figure}
\centering
\includegraphics[width=1\textwidth,height=1\textheight]{C:/Stock_Assessments/VRML_Assessment_2021/Model_files/NCA/FINAL_base/Verm_NoCA_base_files_Day1_Request1_drop2020PR_weight_CCFRP_lengths_hessian - Copy//plots/index9_standcpueall.png}
\caption{Standardized indices overlaid. Each index is rescaled to have mean observation = 1.0.\label{fig:cpueall}}
\end{figure}

\begin{figure}
\centering
\includegraphics[width=1\textwidth,height=1\textheight]{C:/Stock_Assessments/VRML_Assessment_2021/Model_files/NCA/FINAL_base/Verm_NoCA_base_files_Day1_Request1_drop2020PR_weight_CCFRP_lengths_hessian - Copy/plots/comp_lendat_bubflt9mkt0_page2.png}
\caption{Length composition data from the West coast groundfish bottomfish trawl survey.\label{fig:len-data-NWFSC-TWL}}
\end{figure}

\begin{figure}
\centering
\includegraphics[width=1\textwidth,height=1\textheight]{C:/Stock_Assessments/VRML_Assessment_2021/Model_files/NCA/FINAL_base/Verm_NoCA_base_files_Day1_Request1_drop2020PR_weight_CCFRP_lengths_hessian - Copy/plots/comp_lendat_bubflt11mkt0.png}
\caption{Length composition data from the Abrams thesis research survey.\label{fig:len-data-ABRAMS-RESEARCH}}
\end{figure}

\begin{figure}
\centering
\includegraphics[width=1\textwidth,height=1\textheight]{C:/Stock_Assessments/VRML_Assessment_2021/Model_files/NCA/FINAL_base/Verm_NoCA_base_files_Day1_Request1_drop2020PR_weight_CCFRP_lengths_hessian - Copy/plots/comp_lendat_bubflt12mkt0.png}
\caption{Length composition data from the SWFSC groundfish ecology survey.\label{fig:len-data-SWFSC-GF-ECOL}}
\end{figure}

\begin{figure}
\centering
\includegraphics[width=1\textwidth,height=1\textheight]{C:/Stock_Assessments/VRML_Assessment_2021/Model_files/NCA/FINAL_base/Verm_NoCA_base_files_Day1_Request1_drop2020PR_weight_CCFRP_lengths_hessian - Copy/plots/comp_lendat_bubflt13mkt0.png}
\caption{Length composition data from the California Collaborative Fisheries Research Program survey.\label{fig:len-data-CCFRP}}
\end{figure}

\begin{figure}
\centering
\includegraphics[width=1\textwidth,height=1\textheight]{C:/Stock_Assessments/VRML_Assessment_2021/GitHub/Vermilion_2021/doc/figures/Reader 1 vs Reader 2.png}
\caption{Aging precision between initial and blind double reads for vermilion rockfish. Numbers in the bubbles are the sample sizes of otoliths cross-read.\label{fig:reader1reader2}}
\end{figure}

\begin{figure}
\centering
\includegraphics[width=1\textwidth,height=1\textheight]{C:/Stock_Assessments/VRML_Assessment_2021/GitHub/Vermilion_2021/doc/figures/oldfish.jpg}
\caption{Photograph of the \emph{oldest} aged fish used in the assessment with annuli marked by B. Kamikawa (NWFSC).\label{fig:oldfish}}
\end{figure}

\begin{figure}
\centering
\includegraphics[width=1\textwidth,height=1\textheight]{C:/Stock_Assessments/VRML_Assessment_2021/GitHub/Vermilion_2021/doc/figures/True vs Reads (by reader).png}
\caption{True versus predicted age for two current age readers at the NWFSC from the ageing error software with unbiased reads for reader 1 and curvilinear bias for reader 1 and curvilinear standard deviation for both readers.\label{fig:truereads}}
\end{figure}

\begin{figure}
\centering
\includegraphics[width=1\textwidth,height=1\textheight]{C:/Stock_Assessments/VRML_Assessment_2021/Model_files/NCA/FINAL_base/Verm_NoCA_base_files_Day1_Request1_drop2020PR_weight_CCFRP_lengths_hessian - Copy//plots/numbers10_ageerror_matrix_1.png}
\caption{Distribution of observed age at true age for ageing error type 1.\label{fig:ageerror}}
\end{figure}

\begin{figure}
\centering
\includegraphics[width=1\textwidth,height=1\textheight]{C:/Stock_Assessments/VRML_Assessment_2021/GitHub/Vermilion_2021/doc/figures/vermilion_functional_maturity.png}
\caption{Fitted logistic regression of estimated functional maturity as a function of fork length for vermilion rockfish.\label{fig:functional-maturity}}
\end{figure}

\begin{figure}
\centering
\includegraphics[width=1\textwidth,height=1\textheight]{C:/Stock_Assessments/VRML_Assessment_2021/Model_files/NCA/FINAL_base/Verm_NoCA_base_files_Day1_Request1_drop2020PR_weight_CCFRP_lengths_hessian - Copy//plots/bio6_maturity.png}
\caption{Maturity at length.\label{fig:maturity}}
\end{figure}

\begin{figure}
\centering
\includegraphics[width=1\textwidth,height=1\textheight]{C:/Stock_Assessments/VRML_Assessment_2021/GitHub/Vermilion_2021/doc/figures/fitted_fec_len.png}
\caption{Fitted fecundity as a function of weight from samples of vermilion rockfish.\label{fig:fitted-fecundity}}
\end{figure}

\begin{figure}
\centering
\includegraphics[width=1\textwidth,height=1\textheight]{C:/Stock_Assessments/VRML_Assessment_2021/Model_files/NCA/FINAL_base/Verm_NoCA_base_files_Day1_Request1_drop2020PR_weight_CCFRP_lengths_hessian - Copy//plots/bio8_fecundity_wt.png}
\caption{Fecundity as a function of weight.\label{fig:fecundity}}
\end{figure}

\begin{figure}
\centering
\includegraphics[width=1\textwidth,height=1\textheight]{C:/Stock_Assessments/VRML_Assessment_2021/Model_files/NCA/FINAL_base/Verm_NoCA_base_files_Day1_Request1_drop2020PR_weight_CCFRP_lengths_hessian - Copy/plots/bio11_spawningoutput_age.png}
\caption{Spawning output at age. This is the product of maturity and fecundity. When these processes are length-based they are converted into the age dimension using the matrix of length at age.\label{fig:spawnage}}
\end{figure}

\FloatBarrier

\begin{figure}
\centering
\includegraphics[width=1\textwidth,height=1\textheight]{C:/Stock_Assessments/VRML_Assessment_2021/Model_files/NCA/FINAL_base/Verm_NoCA_base_files_Day1_Request1_drop2020PR_weight_CCFRP_lengths_hessian - Copy//plots/sexratio_len_flt11mkt0.png}
\caption{Sex ratios for length comps, whole catch, Abrams thesis research survey. Observed sex ratios (points) with 75\% intervals (vertical lines) calculated as a Jeffreys interval based on the adjusted input sample size. The model expectation is shown in the purple line.\label{fig:sexratio-ABRAMS-RESEARCH-1}}
\end{figure}

\begin{figure}
\centering
\includegraphics[width=1\textwidth,height=1\textheight]{C:/Stock_Assessments/VRML_Assessment_2021/Model_files/NCA/FINAL_base/Verm_NoCA_base_files_Day1_Request1_drop2020PR_weight_CCFRP_lengths_hessian - Copy//plots/sexratio_len_flt12mkt0.png}
\caption{Sex ratios for length comps, whole catch, SWFSC groundfish ecology survey. Observed sex ratios (points) with 75\% intervals (vertical lines) calculated as a Jeffreys interval based on the adjusted input sample size. The model expectation is shown in the purple line.\label{fig:sexratio-SWFSC-GF-ECOL-2}}
\end{figure}

\begin{figure}
\centering
\includegraphics[width=1\textwidth,height=1\textheight]{C:/Stock_Assessments/VRML_Assessment_2021/Model_files/NCA/FINAL_base/Verm_NoCA_base_files_Day1_Request1_drop2020PR_weight_CCFRP_lengths_hessian - Copy//plots/sexratio_len_flt9mkt0_page1.png}
\caption{Sex ratios for length comps, whole catch, West coast groundfish bottomfish trawl survey. Observed sex ratios (points) with 75\% intervals (vertical lines) calculated as a Jeffreys interval based on the adjusted input sample size. The model expectation is shown in the purple line.\label{fig:sexratio-NWFSC-TWL-3}}
\end{figure}

\begin{figure}
\centering
\includegraphics[width=1\textwidth,height=1\textheight]{C:/Stock_Assessments/VRML_Assessment_2021/Model_files/NCA/FINAL_base/Verm_NoCA_base_files_Day1_Request1_drop2020PR_weight_CCFRP_lengths_hessian - Copy//plots/sexratio_len_flt9mkt0_page2.png}
\caption{Sex ratios for length comps, whole catch, West coast groundfish bottomfish trawl survey. Observed sex ratios (points) with 75\% intervals (vertical lines) calculated as a Jeffreys interval based on the adjusted input sample size. The model expectation is shown in the purple line.\label{fig:sexratio-NWFSC-TWL-4}}
\end{figure}

\FloatBarrier

\begin{figure}
\centering
\includegraphics[width=1\textwidth,height=1\textheight]{C:/Stock_Assessments/VRML_Assessment_2021/Model_files/NCA/FINAL_base/Verm_NoCA_base_files_Day1_Request1_drop2020PR_weight_CCFRP_lengths_hessian - Copy//plots/bio5_weightatsize.png}
\caption{Weight-length relationship.\label{fig:weightlength}}
\end{figure}

\FloatBarrier

\begin{figure}
\centering
\includegraphics[width=1\textwidth,height=1\textheight]{C:/Stock_Assessments/VRML_Assessment_2021/GitHub/Vermilion_2021/doc/figures/CV_of_Length_vs_Age.png}
\caption{Coefficient of variation of length versus age for vermilion rockfish from the NWFSC hook-and-line survey.\label{fig:cv-length-age}}
\end{figure}

\begin{figure}
\centering
\includegraphics[width=1\textwidth,height=1\textheight]{C:/Stock_Assessments/VRML_Assessment_2021/GitHub/Vermilion_2021/doc/figures/jitter_postSTAR_base_NCA.png}
\caption{Results from 100 jittered runs of the post-STAR base model. Missing values indicate the 43 runs that did not converge.\label{fig:jitter}}
\end{figure}

\begin{figure}
\centering
\includegraphics[width=1\textwidth,height=1\textheight]{C:/Stock_Assessments/VRML_Assessment_2021/Model_files/NCA/FINAL_base/Verm_NoCA_base_files_Day1_Request1_drop2020PR_weight_CCFRP_lengths_hessian - Copy//plots/bio1_sizeatage.png}
\caption{Length at age in the beginning of the year (or season) in the ending year of the model. Shaded area indicates 95\% distribution of length at age around estimated growth curve.\label{fig:fittedgrowth}}
\end{figure}

\FloatBarrier

\FloatBarrier

\begin{figure}
\centering
\includegraphics[width=1\textwidth,height=1\textheight]{C:/Stock_Assessments/VRML_Assessment_2021/Model_files/NCA/FINAL_base/Verm_NoCA_base_files_Day1_Request1_drop2020PR_weight_CCFRP_lengths_hessian - Copy/plots/comp_condAALfit_residsflt11mkt0.png}
\caption{Pearson residuals, whole catch, ABRAMS\_RESEARCH (max=11.91).\label{fig:comp_condAALfit_residsflt11mkt0}}
\end{figure}

\begin{figure}
\centering
\includegraphics[width=1\textwidth,height=1\textheight]{C:/Stock_Assessments/VRML_Assessment_2021/Model_files/NCA/FINAL_base/Verm_NoCA_base_files_Day1_Request1_drop2020PR_weight_CCFRP_lengths_hessian - Copy/plots/comp_condAALfit_data_weighting_TA1.8_condAgeABRAMS_RESEARCH.png}
\caption{Mean age from conditional data (aggregated across length bins) for ABRAMS\_RESEARCH with 95\% confidence intervals based on current samples sizes. Francis data weighting method TA1.8: thinner intervals (with capped ends) show result of further adjusting sample sizes based on suggested multiplier (with 95\% interval) for conditional age-at-length data from ABRAMS\_RESEARCH: 0.9942 (0.9942-Inf).\label{fig:comp_condAALfit_data_weighting_TA1.8_condAgeABRAMS_RESEARCH}}
\end{figure}

\begin{figure}
\centering
\includegraphics[width=1\textwidth,height=1\textheight]{C:/Stock_Assessments/VRML_Assessment_2021/Model_files/NCA/FINAL_base/Verm_NoCA_base_files_Day1_Request1_drop2020PR_weight_CCFRP_lengths_hessian - Copy/plots/comp_condAALfit_Andre_plotsflt11mkt0.png}
\caption{Conditional AAL plot, whole catch, ABRAMS\_RESEARCH These plots show mean age and std. dev. in conditional {\tagstructbegin{tag=Link}\tagmcbegin{tag=Link}\href{mailto:A@L}{\nolinkurl{A@L}}\leavevmode\tagmcend\tagstructend}. Left plots are mean {\tagstructbegin{tag=Link}\tagmcbegin{tag=Link}\href{mailto:A@L}{\nolinkurl{A@L}}\leavevmode\tagmcend\tagstructend} by size-class (obs. and exp.) with 90\% CIs based on adding 1.64 SE of mean to the data. Right plots in each pair are SE of mean {\tagstructbegin{tag=Link}\tagmcbegin{tag=Link}\href{mailto:A@L}{\nolinkurl{A@L}}\leavevmode\tagmcend\tagstructend} (obs. and exp.) with 90\% CIs based on the chi-square distribution.\label{fig:comp_condAALfit_Andre_plotsflt11mkt0}}
\end{figure}

\begin{figure}
\centering
\includegraphics[width=1\textwidth,height=1\textheight]{C:/Stock_Assessments/VRML_Assessment_2021/Model_files/NCA/FINAL_base/Verm_NoCA_base_files_Day1_Request1_drop2020PR_weight_CCFRP_lengths_hessian - Copy/plots/comp_condAALfit_residsflt9mkt0_page1.png}
\caption{Pearson residuals, whole catch, NWFSC\_TWL (max=43.61) (plot 1 of 3).\label{fig:comp_condAALfit_residsflt9mkt0_page1}}
\end{figure}

\begin{figure}
\centering
\includegraphics[width=1\textwidth,height=1\textheight]{C:/Stock_Assessments/VRML_Assessment_2021/Model_files/NCA/FINAL_base/Verm_NoCA_base_files_Day1_Request1_drop2020PR_weight_CCFRP_lengths_hessian - Copy/plots/comp_condAALfit_residsflt9mkt0_page2.png}
\caption{Pearson residuals, whole catch, NWFSC\_TWL (max=43.61) (plot 2 of 3).\label{fig:comp_condAALfit_residsflt9mkt0_page2}}
\end{figure}

\begin{figure}
\centering
\includegraphics[width=1\textwidth,height=1\textheight]{C:/Stock_Assessments/VRML_Assessment_2021/Model_files/NCA/FINAL_base/Verm_NoCA_base_files_Day1_Request1_drop2020PR_weight_CCFRP_lengths_hessian - Copy/plots/comp_condAALfit_residsflt9mkt0_page3.png}
\caption{Pearson residuals, whole catch, NWFSC\_TWL (max=43.61) (plot 3 of 3).\label{fig:comp_condAALfit_residsflt9mkt0_page3}}
\end{figure}

\begin{figure}
\centering
\includegraphics[width=1\textwidth,height=1\textheight]{C:/Stock_Assessments/VRML_Assessment_2021/Model_files/NCA/FINAL_base/Verm_NoCA_base_files_Day1_Request1_drop2020PR_weight_CCFRP_lengths_hessian - Copy/plots/comp_condAALfit_data_weighting_TA1.8_condAgeNWFSC_TWL.png}
\caption{Mean age from conditional data (aggregated across length bins) for NWFSC\_TWL with 95\% confidence intervals based on current samples sizes. Francis data weighting method TA1.8: thinner intervals (with capped ends) show result of further adjusting sample sizes based on suggested multiplier (with 95\% interval) for conditional age-at-length data from NWFSC\_TWL: 0.9764 (0.542-3.5311).\label{fig:comp_condAALfit_data_weighting_TA1.8_condAgeNWFSC_TWL}}
\end{figure}

\begin{figure}
\centering
\includegraphics[width=1\textwidth,height=1\textheight]{C:/Stock_Assessments/VRML_Assessment_2021/Model_files/NCA/FINAL_base/Verm_NoCA_base_files_Day1_Request1_drop2020PR_weight_CCFRP_lengths_hessian - Copy/plots/comp_condAALfit_Andre_plotsflt9mkt0_page1.png}
\caption{Conditional AAL plot, whole catch, NWFSC\_TWL (plot 1 of 4) These plots show mean age and std. dev. in conditional {\tagstructbegin{tag=Link}\tagmcbegin{tag=Link}\href{mailto:A@L}{\nolinkurl{A@L}}\leavevmode\tagmcend\tagstructend}. Left plots are mean {\tagstructbegin{tag=Link}\tagmcbegin{tag=Link}\href{mailto:A@L}{\nolinkurl{A@L}}\leavevmode\tagmcend\tagstructend} by size-class (obs. and exp.) with 90\% CIs based on adding 1.64 SE of mean to the data. Right plots in each pair are SE of mean {\tagstructbegin{tag=Link}\tagmcbegin{tag=Link}\href{mailto:A@L}{\nolinkurl{A@L}}\leavevmode\tagmcend\tagstructend} (obs. and exp.) with 90\% CIs based on the chi-square distribution.\label{fig:comp_condAALfit_Andre_plotsflt9mkt0_page1}}
\end{figure}

\begin{figure}
\centering
\includegraphics[width=1\textwidth,height=1\textheight]{C:/Stock_Assessments/VRML_Assessment_2021/Model_files/NCA/FINAL_base/Verm_NoCA_base_files_Day1_Request1_drop2020PR_weight_CCFRP_lengths_hessian - Copy/plots/comp_condAALfit_Andre_plotsflt9mkt0_page2.png}
\caption{Conditional AAL plot, whole catch, NWFSC\_TWL (plot 2 of 4).\label{fig:comp_condAALfit_Andre_plotsflt9mkt0_page2}}
\end{figure}

\begin{figure}
\centering
\includegraphics[width=1\textwidth,height=1\textheight]{C:/Stock_Assessments/VRML_Assessment_2021/Model_files/NCA/FINAL_base/Verm_NoCA_base_files_Day1_Request1_drop2020PR_weight_CCFRP_lengths_hessian - Copy/plots/comp_condAALfit_Andre_plotsflt9mkt0_page3.png}
\caption{Conditional AAL plot, whole catch, NWFSC\_TWL (plot 3 of 4).\label{fig:comp_condAALfit_Andre_plotsflt9mkt0_page3}}
\end{figure}

\begin{figure}
\centering
\includegraphics[width=1\textwidth,height=1\textheight]{C:/Stock_Assessments/VRML_Assessment_2021/Model_files/NCA/FINAL_base/Verm_NoCA_base_files_Day1_Request1_drop2020PR_weight_CCFRP_lengths_hessian - Copy/plots/comp_condAALfit_Andre_plotsflt9mkt0_page4.png}
\caption{Conditional AAL plot, whole catch, NWFSC\_TWL (plot 4 of 4).\label{fig:comp_condAALfit_Andre_plotsflt9mkt0_page4}}
\end{figure}

\begin{figure}
\centering
\includegraphics[width=1\textwidth,height=1\textheight]{C:/Stock_Assessments/VRML_Assessment_2021/Model_files/NCA/FINAL_base/Verm_NoCA_base_files_Day1_Request1_drop2020PR_weight_CCFRP_lengths_hessian - Copy/plots/comp_condAALfit_residsflt12mkt0.png}
\caption{Pearson residuals, whole catch, SWFSC\_GF\_ECOL (max=25.1).\label{fig:comp_condAALfit_residsflt12mkt0}}
\end{figure}

\begin{figure}
\centering
\includegraphics[width=1\textwidth,height=1\textheight]{C:/Stock_Assessments/VRML_Assessment_2021/Model_files/NCA/FINAL_base/Verm_NoCA_base_files_Day1_Request1_drop2020PR_weight_CCFRP_lengths_hessian - Copy/plots/comp_condAALfit_data_weighting_TA1.8_condAgeSWFSC_GF_ECOL.png}
\caption{Mean age from conditional data (aggregated across length bins) for SWFSC\_GF\_ECOL with 95\% confidence intervals based on current samples sizes. Francis data weighting method TA1.8: thinner intervals (with capped ends) show result of further adjusting sample sizes based on suggested multiplier (with 95\% interval) for conditional age-at-length data from SWFSC\_GF\_ECOL: 0.9874 (0.6839-10.5598).\label{fig:comp_condAALfit_data_weighting_TA1.8_condAgeSWFSC_GF_ECOL}}
\end{figure}

\begin{figure}
\centering
\includegraphics[width=1\textwidth,height=1\textheight]{C:/Stock_Assessments/VRML_Assessment_2021/Model_files/NCA/FINAL_base/Verm_NoCA_base_files_Day1_Request1_drop2020PR_weight_CCFRP_lengths_hessian - Copy/plots/comp_condAALfit_Andre_plotsflt12mkt0.png}
\caption{Conditional AAL plot, whole catch, SWFSC\_GF\_ECOL These plots show mean age and std. dev. in conditional {\tagstructbegin{tag=Link}\tagmcbegin{tag=Link}\href{mailto:A@L}{\nolinkurl{A@L}}\leavevmode\tagmcend\tagstructend}.Left plots are mean {\tagstructbegin{tag=Link}\tagmcbegin{tag=Link}\href{mailto:A@L}{\nolinkurl{A@L}}\leavevmode\tagmcend\tagstructend} by size-class (obs. and exp.) with 90\% CIs based on adding 1.64 SE of mean to the data.Right plots in each pair are SE of mean {\tagstructbegin{tag=Link}\tagmcbegin{tag=Link}\href{mailto:A@L}{\nolinkurl{A@L}}\leavevmode\tagmcend\tagstructend} (obs. and exp.) with 90\% CIs based on the chi-square distribution.\label{fig:comp_condAALfit_Andre_plotsflt12mkt0}}
\end{figure}

\clearpage
\FloatBarrier

\FloatBarrier

\begin{figure}
\centering
\includegraphics[width=1\textwidth,height=1\textheight]{C:/Stock_Assessments/VRML_Assessment_2021/Model_files/NCA/FINAL_base/Verm_NoCA_base_files_Day1_Request1_drop2020PR_weight_CCFRP_lengths_hessian - Copy//plots/sel02_multiple_fleets_age1.png}
\caption{Selectivity at age derived from selectivity at length for multiple fleets.\label{fig:selex-age-all}}
\end{figure}

\FloatBarrier

\begin{figure}
\centering
\includegraphics[width=1\textwidth,height=1\textheight]{C:/Stock_Assessments/VRML_Assessment_2021/Model_files/NCA/FINAL_base/Verm_NoCA_base_files_Day1_Request1_drop2020PR_weight_CCFRP_lengths_hessian - Copy//plots/sel01_multiple_fleets_length1.png}
\caption{End year selectivity at length by fleet/survey.\label{fig:selex-length-all}}
\end{figure}

\FloatBarrier

\begin{figure}
\centering
\includegraphics[width=1\textwidth,height=1\textheight]{C:/Stock_Assessments/VRML_Assessment_2021/Model_files/NCA/FINAL_base/Verm_NoCA_base_files_Day1_Request1_drop2020PR_weight_CCFRP_lengths_hessian - Copy//plots/sel03_len_timevary_surf_flt4sex1.png}
\caption{Surface plot of Female time-varying selectivity for REC\_PC.\label{fig:sel03_len_timevary_surf_flt4sex1}}
\end{figure}

\begin{figure}
\centering
\includegraphics[width=1\textwidth,height=1\textheight]{C:/Stock_Assessments/VRML_Assessment_2021/Model_files/NCA/FINAL_base/Verm_NoCA_base_files_Day1_Request1_drop2020PR_weight_CCFRP_lengths_hessian - Copy//plots/sel03_len_timevary_surf_flt6sex1.png}
\caption{Surface plot of Female time-varying selectivity for REC\_PR.\label{fig:sel03_len_timevary_surf_flt6sex1}}
\end{figure}

\FloatBarrier

\FloatBarrier

\begin{figure}
\centering
\includegraphics[width=1\textwidth,height=1\textheight]{C:/Stock_Assessments/VRML_Assessment_2021/Model_files/NCA/FINAL_base/Verm_NoCA_base_files_Day1_Request1_drop2020PR_weight_CCFRP_lengths_hessian - Copy//plots/sel09_len_flt1sex1.png}
\caption{Female ending year selectivity for the commercial hook-and-line fishery.\label{fig:endyr-selex-COM-HKL}}
\end{figure}

\begin{figure}
\centering
\includegraphics[width=1\textwidth,height=1\textheight]{C:/Stock_Assessments/VRML_Assessment_2021/Model_files/NCA/FINAL_base/Verm_NoCA_base_files_Day1_Request1_drop2020PR_weight_CCFRP_lengths_hessian - Copy//plots/sel09_len_flt2sex1.png}
\caption{Female ending year selectivity for the commercial trawl fishery.\label{fig:endyr-selex-COM-TWL}}
\end{figure}

\begin{figure}
\centering
\includegraphics[width=1\textwidth,height=1\textheight]{C:/Stock_Assessments/VRML_Assessment_2021/Model_files/NCA/FINAL_base/Verm_NoCA_base_files_Day1_Request1_drop2020PR_weight_CCFRP_lengths_hessian - Copy//plots/sel09_len_flt3sex1.png}
\caption{Female ending year selectivity for the commercial net fishery.\label{fig:endyr-selex-COM-NET}}
\end{figure}

\begin{figure}
\centering
\includegraphics[width=1\textwidth,height=1\textheight]{C:/Stock_Assessments/VRML_Assessment_2021/Model_files/NCA/FINAL_base/Verm_NoCA_base_files_Day1_Request1_drop2020PR_weight_CCFRP_lengths_hessian - Copy//plots/sel09_len_flt4sex1.png}
\caption{Female ending year selectivity for the recreational PC retained fishery.\label{fig:endyr-selex-REC-PC}}
\end{figure}

\begin{figure}
\centering
\includegraphics[width=1\textwidth,height=1\textheight]{C:/Stock_Assessments/VRML_Assessment_2021/Model_files/NCA/FINAL_base/Verm_NoCA_base_files_Day1_Request1_drop2020PR_weight_CCFRP_lengths_hessian - Copy//plots/sel09_len_flt5sex1.png}
\caption{Female ending year selectivity for the recreational PC discard fishery.\label{fig:endyr-selex-REC-PC-DIS}}
\end{figure}

\begin{figure}
\centering
\includegraphics[width=1\textwidth,height=1\textheight]{C:/Stock_Assessments/VRML_Assessment_2021/Model_files/NCA/FINAL_base/Verm_NoCA_base_files_Day1_Request1_drop2020PR_weight_CCFRP_lengths_hessian - Copy//plots/sel09_len_flt6sex1.png}
\caption{Female ending year selectivity for the recreational PR retained fishery.\label{fig:endyr-selex-REC-PR}}
\end{figure}

\begin{figure}
\centering
\includegraphics[width=1\textwidth,height=1\textheight]{C:/Stock_Assessments/VRML_Assessment_2021/Model_files/NCA/FINAL_base/Verm_NoCA_base_files_Day1_Request1_drop2020PR_weight_CCFRP_lengths_hessian - Copy//plots/sel09_len_flt9sex1.png}
\caption{Female ending year selectivity for the West coast groundfish bottomfish trawl survey.\label{fig:endyr-selex-NWFSC-TWL}}
\end{figure}

\begin{figure}
\centering
\includegraphics[width=1\textwidth,height=1\textheight]{C:/Stock_Assessments/VRML_Assessment_2021/Model_files/NCA/FINAL_base/Verm_NoCA_base_files_Day1_Request1_drop2020PR_weight_CCFRP_lengths_hessian - Copy//plots/sel09_len_flt13sex1.png}
\caption{Female ending year selectivity for the California Collaborative Fisheries Research Program survey.\label{fig:endyr-selex-CCFRP}}
\end{figure}

\FloatBarrier

\FloatBarrier

\begin{figure}
\centering
\includegraphics[width=1\textwidth,height=1\textheight]{C:/Stock_Assessments/VRML_Assessment_2021/Model_files/NCA/FINAL_base/Verm_NoCA_base_files_Day1_Request1_drop2020PR_weight_CCFRP_lengths_hessian - Copy//plots/comp_lenfit__aggregated_across_time.png}
\caption{Length comps, aggregated across time by fleet. Labels `retained' and `discard' indicate discarded or retained sampled for each fleet. Panels without this designation represent the whole catch.\label{fig:lenfits-all}}
\end{figure}

\FloatBarrier

\begin{figure}
\centering
\includegraphics[width=1\textwidth,height=1\textheight]{C:/Stock_Assessments/VRML_Assessment_2021/Model_files/NCA/FINAL_base/Verm_NoCA_base_files_Day1_Request1_drop2020PR_weight_CCFRP_lengths_hessian - Copy//plots/comp_lenfit_residsflt1mkt0_page2.png}
\caption{Pearson residuals for the commercial hook-and-line fishery. Closed bubbles are positive residuals (observed \textgreater{} expected) and open bubbles are negative residuals (observed \textless{} expected).\label{fig:len-pearson-COM-HKL}}
\end{figure}

\begin{figure}
\centering
\includegraphics[width=1\textwidth,height=1\textheight]{C:/Stock_Assessments/VRML_Assessment_2021/Model_files/NCA/FINAL_base/Verm_NoCA_base_files_Day1_Request1_drop2020PR_weight_CCFRP_lengths_hessian - Copy//plots/comp_lenfit_data_weighting_TA1.8_COM_HKL.png}
\caption{Mean length (cm) for REC\_PR with 95\% confidence intervals based on current samples sizes. Francis data weighting method TA1.8: thinner intervals (with capped ends) show result of further adjusting sample sizes based on suggested multiplier (with 95\% interval) for length data from the commercial hook-and-line fishery.\label{fig:mean-len-fit-COM-HKL}}
\end{figure}

\begin{figure}
\centering
\includegraphics[width=1\textwidth,height=1\textheight]{C:/Stock_Assessments/VRML_Assessment_2021/Model_files/NCA/FINAL_base/Verm_NoCA_base_files_Day1_Request1_drop2020PR_weight_CCFRP_lengths_hessian - Copy//plots/comp_lenfit_residsflt2mkt0.png}
\caption{Pearson residuals for the commercial trawl fishery. Closed bubbles are positive residuals (observed \textgreater{} expected) and open bubbles are negative residuals (observed \textless{} expected).\label{fig:len-pearson-COM-TWL}}
\end{figure}

\begin{figure}
\centering
\includegraphics[width=1\textwidth,height=1\textheight]{C:/Stock_Assessments/VRML_Assessment_2021/Model_files/NCA/FINAL_base/Verm_NoCA_base_files_Day1_Request1_drop2020PR_weight_CCFRP_lengths_hessian - Copy//plots/comp_lenfit_data_weighting_TA1.8_COM_TWL.png}
\caption{Mean length (cm) for REC\_PR with 95\% confidence intervals based on current samples sizes. Francis data weighting method TA1.8: thinner intervals (with capped ends) show result of further adjusting sample sizes based on suggested multiplier (with 95\% interval) for length data from the commercial trawl fishery.\label{fig:mean-len-fit-COM-TWL}}
\end{figure}

\begin{figure}
\centering
\includegraphics[width=1\textwidth,height=1\textheight]{C:/Stock_Assessments/VRML_Assessment_2021/Model_files/NCA/FINAL_base/Verm_NoCA_base_files_Day1_Request1_drop2020PR_weight_CCFRP_lengths_hessian - Copy//plots/comp_lenfit_residsflt3mkt0.png}
\caption{Pearson residuals for the commercial net fishery. Closed bubbles are positive residuals (observed \textgreater{} expected) and open bubbles are negative residuals (observed \textless{} expected).\label{fig:len-pearson-COM-NET}}
\end{figure}

\begin{figure}
\centering
\includegraphics[width=1\textwidth,height=1\textheight]{C:/Stock_Assessments/VRML_Assessment_2021/Model_files/NCA/FINAL_base/Verm_NoCA_base_files_Day1_Request1_drop2020PR_weight_CCFRP_lengths_hessian - Copy//plots/comp_lenfit_data_weighting_TA1.8_COM_NET.png}
\caption{Mean length (cm) for REC\_PR with 95\% confidence intervals based on current samples sizes. Francis data weighting method TA1.8: thinner intervals (with capped ends) show result of further adjusting sample sizes based on suggested multiplier (with 95\% interval) for length data from the commercial net fishery.\label{fig:mean-len-fit-COM-NET}}
\end{figure}

\begin{figure}
\centering
\includegraphics[width=1\textwidth,height=1\textheight]{C:/Stock_Assessments/VRML_Assessment_2021/Model_files/NCA/FINAL_base/Verm_NoCA_base_files_Day1_Request1_drop2020PR_weight_CCFRP_lengths_hessian - Copy//plots/comp_lenfit_residsflt4mkt0_page3.png}
\caption{Pearson residuals for the recreational PC retained fishery. Closed bubbles are positive residuals (observed \textgreater{} expected) and open bubbles are negative residuals (observed \textless{} expected).\label{fig:len-pearson-REC-PC}}
\end{figure}

\begin{figure}
\centering
\includegraphics[width=1\textwidth,height=1\textheight]{C:/Stock_Assessments/VRML_Assessment_2021/Model_files/NCA/FINAL_base/Verm_NoCA_base_files_Day1_Request1_drop2020PR_weight_CCFRP_lengths_hessian - Copy//plots/comp_lenfit_data_weighting_TA1.8_REC_PC.png}
\caption{Mean length (cm) for REC\_PR with 95\% confidence intervals based on current samples sizes. Francis data weighting method TA1.8: thinner intervals (with capped ends) show result of further adjusting sample sizes based on suggested multiplier (with 95\% interval) for length data from the recreational PC retained fishery.\label{fig:mean-len-fit-REC-PC}}
\end{figure}

\begin{figure}
\centering
\includegraphics[width=1\textwidth,height=1\textheight]{C:/Stock_Assessments/VRML_Assessment_2021/Model_files/NCA/FINAL_base/Verm_NoCA_base_files_Day1_Request1_drop2020PR_weight_CCFRP_lengths_hessian - Copy//plots/comp_lenfit_residsflt5mkt0.png}
\caption{Pearson residuals for the recreational PC discard fishery. Closed bubbles are positive residuals (observed \textgreater{} expected) and open bubbles are negative residuals (observed \textless{} expected).\label{fig:len-pearson-REC-PC-DIS}}
\end{figure}

\begin{figure}
\centering
\includegraphics[width=1\textwidth,height=1\textheight]{C:/Stock_Assessments/VRML_Assessment_2021/Model_files/NCA/FINAL_base/Verm_NoCA_base_files_Day1_Request1_drop2020PR_weight_CCFRP_lengths_hessian - Copy//plots/comp_lenfit_data_weighting_TA1.8_REC_PC_DIS.png}
\caption{Mean length (cm) for REC\_PR with 95\% confidence intervals based on current samples sizes. Francis data weighting method TA1.8: thinner intervals (with capped ends) show result of further adjusting sample sizes based on suggested multiplier (with 95\% interval) for length data from the recreational PC discard fishery.\label{fig:mean-len-fit-REC-PC-DIS}}
\end{figure}

\begin{figure}
\centering
\includegraphics[width=1\textwidth,height=1\textheight]{C:/Stock_Assessments/VRML_Assessment_2021/Model_files/NCA/FINAL_base/Verm_NoCA_base_files_Day1_Request1_drop2020PR_weight_CCFRP_lengths_hessian - Copy//plots/comp_lenfit_residsflt6mkt0_page3.png}
\caption{Pearson residuals for the recreational PR retained fishery. Closed bubbles are positive residuals (observed \textgreater{} expected) and open bubbles are negative residuals (observed \textless{} expected).\label{fig:len-pearson-REC-PR}}
\end{figure}

\begin{figure}
\centering
\includegraphics[width=1\textwidth,height=1\textheight]{C:/Stock_Assessments/VRML_Assessment_2021/Model_files/NCA/FINAL_base/Verm_NoCA_base_files_Day1_Request1_drop2020PR_weight_CCFRP_lengths_hessian - Copy//plots/comp_lenfit_data_weighting_TA1.8_REC_PR.png}
\caption{Mean length (cm) for REC\_PR with 95\% confidence intervals based on current samples sizes. Francis data weighting method TA1.8: thinner intervals (with capped ends) show result of further adjusting sample sizes based on suggested multiplier (with 95\% interval) for length data from the recreational PR retained fishery.\label{fig:mean-len-fit-REC-PR}}
\end{figure}

\begin{figure}
\centering
\includegraphics[width=1\textwidth,height=1\textheight]{C:/Stock_Assessments/VRML_Assessment_2021/Model_files/NCA/FINAL_base/Verm_NoCA_base_files_Day1_Request1_drop2020PR_weight_CCFRP_lengths_hessian - Copy//plots/comp_lenfit_residsflt8mkt0.png}
\caption{Pearson residuals for the Deb Wilson-Vandenberg onboard survey. Closed bubbles are positive residuals (observed \textgreater{} expected) and open bubbles are negative residuals (observed \textless{} expected).\label{fig:len-pearson-DWV-ONBOARD}}
\end{figure}

\begin{figure}
\centering
\includegraphics[width=1\textwidth,height=1\textheight]{C:/Stock_Assessments/VRML_Assessment_2021/Model_files/NCA/FINAL_base/Verm_NoCA_base_files_Day1_Request1_drop2020PR_weight_CCFRP_lengths_hessian - Copy//plots/comp_lenfit_data_weighting_TA1.8_DWV_ONBOARD.png}
\caption{Mean length (cm) for REC\_PR with 95\% confidence intervals based on current samples sizes. Francis data weighting method TA1.8: thinner intervals (with capped ends) show result of further adjusting sample sizes based on suggested multiplier (with 95\% interval) for length data from the Deb Wilson-Vandenberg onboard survey.\label{fig:mean-len-fit-DWV-ONBOARD}}
\end{figure}

\begin{figure}
\centering
\includegraphics[width=1\textwidth,height=1\textheight]{C:/Stock_Assessments/VRML_Assessment_2021/Model_files/NCA/FINAL_base/Verm_NoCA_base_files_Day1_Request1_drop2020PR_weight_CCFRP_lengths_hessian - Copy//plots/comp_lenfit_residsflt9mkt0_page2.png}
\caption{Pearson residuals for the West coast groundfish bottomfish trawl survey. Closed bubbles are positive residuals (observed \textgreater{} expected) and open bubbles are negative residuals (observed \textless{} expected).\label{fig:len-pearson-NWFSC-TWL}}
\end{figure}

\begin{figure}
\centering
\includegraphics[width=1\textwidth,height=1\textheight]{C:/Stock_Assessments/VRML_Assessment_2021/Model_files/NCA/FINAL_base/Verm_NoCA_base_files_Day1_Request1_drop2020PR_weight_CCFRP_lengths_hessian - Copy//plots/comp_lenfit_data_weighting_TA1.8_NWFSC_TWL.png}
\caption{Mean length (cm) for REC\_PR with 95\% confidence intervals based on current samples sizes. Francis data weighting method TA1.8: thinner intervals (with capped ends) show result of further adjusting sample sizes based on suggested multiplier (with 95\% interval) for length data from the West coast groundfish bottomfish trawl survey.\label{fig:mean-len-fit-NWFSC-TWL}}
\end{figure}

\begin{figure}
\centering
\includegraphics[width=1\textwidth,height=1\textheight]{C:/Stock_Assessments/VRML_Assessment_2021/Model_files/NCA/FINAL_base/Verm_NoCA_base_files_Day1_Request1_drop2020PR_weight_CCFRP_lengths_hessian - Copy//plots/comp_lenfit_residsflt11mkt0.png}
\caption{Pearson residuals for the Abrams thesis research survey. Closed bubbles are positive residuals (observed \textgreater{} expected) and open bubbles are negative residuals (observed \textless{} expected).\label{fig:len-pearson-ABRAMS-RESEARCH}}
\end{figure}

\begin{figure}
\centering
\includegraphics[width=1\textwidth,height=1\textheight]{C:/Stock_Assessments/VRML_Assessment_2021/Model_files/NCA/FINAL_base/Verm_NoCA_base_files_Day1_Request1_drop2020PR_weight_CCFRP_lengths_hessian - Copy//plots/comp_lenfit_data_weighting_TA1.8_ABRAMS_RESEARCH.png}
\caption{Mean length (cm) for REC\_PR with 95\% confidence intervals based on current samples sizes. Francis data weighting method TA1.8: thinner intervals (with capped ends) show result of further adjusting sample sizes based on suggested multiplier (with 95\% interval) for length data from the Abrams thesis research survey.\label{fig:mean-len-fit-ABRAMS-RESEARCH}}
\end{figure}

\begin{figure}
\centering
\includegraphics[width=1\textwidth,height=1\textheight]{C:/Stock_Assessments/VRML_Assessment_2021/Model_files/NCA/FINAL_base/Verm_NoCA_base_files_Day1_Request1_drop2020PR_weight_CCFRP_lengths_hessian - Copy//plots/comp_lenfit_residsflt12mkt0.png}
\caption{Pearson residuals for the SWFSC groundfish ecology survey. Closed bubbles are positive residuals (observed \textgreater{} expected) and open bubbles are negative residuals (observed \textless{} expected).\label{fig:len-pearson-SWFSC-GF-ECOL}}
\end{figure}

\begin{figure}
\centering
\includegraphics[width=1\textwidth,height=1\textheight]{C:/Stock_Assessments/VRML_Assessment_2021/Model_files/NCA/FINAL_base/Verm_NoCA_base_files_Day1_Request1_drop2020PR_weight_CCFRP_lengths_hessian - Copy//plots/comp_lenfit_data_weighting_TA1.8_SWFSC_GF_ECOL.png}
\caption{Mean length (cm) for REC\_PR with 95\% confidence intervals based on current samples sizes. Francis data weighting method TA1.8: thinner intervals (with capped ends) show result of further adjusting sample sizes based on suggested multiplier (with 95\% interval) for length data from the SWFSC groundfish ecology survey.\label{fig:mean-len-fit-SWFSC-GF-ECOL}}
\end{figure}

\begin{figure}
\centering
\includegraphics[width=1\textwidth,height=1\textheight]{C:/Stock_Assessments/VRML_Assessment_2021/Model_files/NCA/FINAL_base/Verm_NoCA_base_files_Day1_Request1_drop2020PR_weight_CCFRP_lengths_hessian - Copy//plots/comp_lenfit_residsflt13mkt0.png}
\caption{Pearson residuals for the California Collaborative Fisheries Research Program survey. Closed bubbles are positive residuals (observed \textgreater{} expected) and open bubbles are negative residuals (observed \textless{} expected).\label{fig:len-pearson-CCFRP}}
\end{figure}

\begin{figure}
\centering
\includegraphics[width=1\textwidth,height=1\textheight]{C:/Stock_Assessments/VRML_Assessment_2021/Model_files/NCA/FINAL_base/Verm_NoCA_base_files_Day1_Request1_drop2020PR_weight_CCFRP_lengths_hessian - Copy//plots/comp_lenfit_data_weighting_TA1.8_CCFRP.png}
\caption{Mean length (cm) for REC\_PR with 95\% confidence intervals based on current samples sizes. Francis data weighting method TA1.8: thinner intervals (with capped ends) show result of further adjusting sample sizes based on suggested multiplier (with 95\% interval) for length data from the California Collaborative Fisheries Research Program survey.\label{fig:mean-len-fit-CCFRP}}
\end{figure}

\FloatBarrier

\FloatBarrier

\begin{figure}
\centering
\includegraphics[width=1\textwidth,height=1\textheight]{C:/Stock_Assessments/VRML_Assessment_2021/Model_files/NCA/FINAL_base/Verm_NoCA_base_files_Day1_Request1_drop2020PR_weight_CCFRP_lengths_hessian - Copy//plots/index5_logcpuefit_REC_PC.png}
\caption{Fit to log index data on log scale for the recreational PC retained fishery. Lines indicate 95\% uncertainty interval around index values based on the model assumption of lognormal error. Thicker lines (if present) indicate input uncertainty before addition of estimated additional uncertainty parameter.\label{fig:log-cpue-REC-PC}}
\end{figure}

\begin{figure}
\centering
\includegraphics[width=1\textwidth,height=1\textheight]{C:/Stock_Assessments/VRML_Assessment_2021/Model_files/NCA/FINAL_base/Verm_NoCA_base_files_Day1_Request1_drop2020PR_weight_CCFRP_lengths_hessian - Copy//plots/index10_resids_SE_total_REC_PC.png}
\caption{Residuals of fit to index for the REC\_PC. Values are (log(Obs) - log(Exp))/SE where SE is the total standard error including any estimated additional uncertainty.\label{fig:cpue-resid-REC-PC}}
\end{figure}

\begin{figure}
\centering
\includegraphics[width=1\textwidth,height=1\textheight]{C:/Stock_Assessments/VRML_Assessment_2021/Model_files/NCA/FINAL_base/Verm_NoCA_base_files_Day1_Request1_drop2020PR_weight_CCFRP_lengths_hessian - Copy//plots/index5_logcpuefit_REC_PR.png}
\caption{Fit to log index data on log scale for the recreational PR retained fishery. Lines indicate 95\% uncertainty interval around index values based on the model assumption of lognormal error. Thicker lines (if present) indicate input uncertainty before addition of estimated additional uncertainty parameter.\label{fig:log-cpue-REC-PR}}
\end{figure}

\begin{figure}
\centering
\includegraphics[width=1\textwidth,height=1\textheight]{C:/Stock_Assessments/VRML_Assessment_2021/Model_files/NCA/FINAL_base/Verm_NoCA_base_files_Day1_Request1_drop2020PR_weight_CCFRP_lengths_hessian - Copy//plots/index10_resids_SE_total_REC_PR.png}
\caption{Residuals of fit to index for the REC\_PR. Values are (log(Obs) - log(Exp))/SE where SE is the total standard error including any estimated additional uncertainty.\label{fig:cpue-resid-REC-PR}}
\end{figure}

\begin{figure}
\centering
\includegraphics[width=1\textwidth,height=1\textheight]{C:/Stock_Assessments/VRML_Assessment_2021/Model_files/NCA/FINAL_base/Verm_NoCA_base_files_Day1_Request1_drop2020PR_weight_CCFRP_lengths_hessian - Copy//plots/index5_logcpuefit_DWV_ONBOARD.png}
\caption{Fit to log index data on log scale for the Deb Wilson-Vandenberg onboard survey. Lines indicate 95\% uncertainty interval around index values based on the model assumption of lognormal error. Thicker lines (if present) indicate input uncertainty before addition of estimated additional uncertainty parameter.\label{fig:log-cpue-DWV-ONBOARD}}
\end{figure}

\begin{figure}
\centering
\includegraphics[width=1\textwidth,height=1\textheight]{C:/Stock_Assessments/VRML_Assessment_2021/Model_files/NCA/FINAL_base/Verm_NoCA_base_files_Day1_Request1_drop2020PR_weight_CCFRP_lengths_hessian - Copy//plots/index10_resids_SE_total_DWV_ONBOARD.png}
\caption{Residuals of fit to index for the DWV\_ONBOARD. Values are (log(Obs) - log(Exp))/SE where SE is the total standard error including any estimated additional uncertainty.\label{fig:cpue-resid-DWV-ONBOARD}}
\end{figure}

\begin{figure}
\centering
\includegraphics[width=1\textwidth,height=1\textheight]{C:/Stock_Assessments/VRML_Assessment_2021/Model_files/NCA/FINAL_base/Verm_NoCA_base_files_Day1_Request1_drop2020PR_weight_CCFRP_lengths_hessian - Copy//plots/index5_logcpuefit_REC_PC_ONBOARD.png}
\caption{Fit to log index data on log scale for the recreational PC onboard survey. Lines indicate 95\% uncertainty interval around index values based on the model assumption of lognormal error. Thicker lines (if present) indicate input uncertainty before addition of estimated additional uncertainty parameter.\label{fig:log-cpue-REC-PC-ONBOARD}}
\end{figure}

\begin{figure}
\centering
\includegraphics[width=1\textwidth,height=1\textheight]{C:/Stock_Assessments/VRML_Assessment_2021/Model_files/NCA/FINAL_base/Verm_NoCA_base_files_Day1_Request1_drop2020PR_weight_CCFRP_lengths_hessian - Copy//plots/index10_resids_SE_total_REC_PC_ONBOARD.png}
\caption{Residuals of fit to index for the REC\_PC\_ONBOARD. Values are (log(Obs) - log(Exp))/SE where SE is the total standard error including any estimated additional uncertainty.\label{fig:cpue-resid-REC-PC-ONBOARD}}
\end{figure}

\begin{figure}
\centering
\includegraphics[width=1\textwidth,height=1\textheight]{C:/Stock_Assessments/VRML_Assessment_2021/Model_files/NCA/FINAL_base/Verm_NoCA_base_files_Day1_Request1_drop2020PR_weight_CCFRP_lengths_hessian - Copy//plots/index5_logcpuefit_CCFRP.png}
\caption{Fit to log index data on log scale for the California Collaborative Fisheries Research Program survey. Lines indicate 95\% uncertainty interval around index values based on the model assumption of lognormal error. Thicker lines (if present) indicate input uncertainty before addition of estimated additional uncertainty parameter.\label{fig:log-cpue-CCFRP}}
\end{figure}

\begin{figure}
\centering
\includegraphics[width=1\textwidth,height=1\textheight]{C:/Stock_Assessments/VRML_Assessment_2021/Model_files/NCA/FINAL_base/Verm_NoCA_base_files_Day1_Request1_drop2020PR_weight_CCFRP_lengths_hessian - Copy//plots/index10_resids_SE_total_CCFRP.png}
\caption{Residuals of fit to index for the CCFRP. Values are (log(Obs) - log(Exp))/SE where SE is the total standard error including any estimated additional uncertainty.\label{fig:cpue-resid-CCFRP}}
\end{figure}

\FloatBarrier

\begin{figure}
\centering
\includegraphics[width=1\textwidth,height=1\textheight]{C:/Stock_Assessments/VRML_Assessment_2021/Model_files/NCA/FINAL_base/Verm_NoCA_base_files_Day1_Request1_drop2020PR_weight_CCFRP_lengths_hessian - Copy//plots/ts7_Spawning_output_with_95_asymptotic_intervals_intervals.png}
\caption{Estimated time series of spawning output.\label{fig:spawnout}}
\end{figure}

\FloatBarrier

\begin{figure}
\centering
\includegraphics[width=1\textwidth,height=1\textheight]{C:/Stock_Assessments/VRML_Assessment_2021/Model_files/NCA/FINAL_base/Verm_NoCA_base_files_Day1_Request1_drop2020PR_weight_CCFRP_lengths_hessian - Copy//plots/ts9_Relative_spawning_output_intervals.png}
\caption{Estimated time series of relative spawning output.\label{fig:depl}}
\end{figure}

\FloatBarrier

\begin{figure}
\centering
\includegraphics[width=1\textwidth,height=1\textheight]{C:/Stock_Assessments/VRML_Assessment_2021/Model_files/NCA/FINAL_base/Verm_NoCA_base_files_Day1_Request1_drop2020PR_weight_CCFRP_lengths_hessian - Copy//plots/SPR3_ratiointerval.png}
\caption{Timeseries of SPR ratio: (1-SPR)/(1-SPR\_50\%).\label{fig:1-spr}}
\end{figure}

\begin{figure}
\centering
\includegraphics[width=1\textwidth,height=1\textheight]{C:/Stock_Assessments/VRML_Assessment_2021/Model_files/NCA/FINAL_base/Verm_NoCA_base_files_Day1_Request1_drop2020PR_weight_CCFRP_lengths_hessian - Copy//plots/yield2_yield_curve_with_refpoints.png}
\caption{Equilibrium yield curve for the base case model. Values are based on the 2020 fishery selectivity and with steepness fixed at 0.72.\label{fig:yield2}}
\end{figure}

\begin{figure}
\centering
\includegraphics[width=1\textwidth,height=1\textheight]{C:/Stock_Assessments/VRML_Assessment_2021/Model_files/NCA/FINAL_base/Verm_NoCA_base_files_Day1_Request1_drop2020PR_weight_CCFRP_lengths_hessian - Copy//plots/yield3_surplus_production.png}
\caption{Surplus production vs.~biomass plot.\label{fig:yield3}}
\end{figure}

\FloatBarrier

\begin{figure}
\centering
\includegraphics[width=1\textwidth,height=1\textheight]{C:/Stock_Assessments/VRML_Assessment_2021/Model_files/NCA/FINAL_base/Verm_NoCA_base_files_Day1_Request1_drop2020PR_weight_CCFRP_lengths_hessian - Copy//plots/SPR4_phase.png}
\caption{Phase plot of the relative biomass (also referred to as fraction unfished) versus the SPR ratio where each point represents the biomass ratio at the start of the year and the relative fishing intensity in that same year. Lines through the final point show the 95 percent intervals based on the asymptotic uncertainty for each dimension. The shaded ellipse is a 95 percent region which accounts for the estimated correlations between the biomass ratio and SPR ratio.\label{fig:phase}}
\end{figure}

\begin{figure}
\centering
\includegraphics[width=1\textwidth,height=1\textheight]{C:/Stock_Assessments/VRML_Assessment_2021/Model_files/NCA/FINAL_base/Verm_NoCA_base_files_Day1_Request1_drop2020PR_weight_CCFRP_lengths_hessian - Copy//plots/SR_curve2.png}
\caption{Stock-recruit curve with labels on first, last, and years with (log) deviations \textgreater{} 0.5. Point colors indicate year, with warmer colors indicating earlier years and cooler colors in showing later years.\label{fig:bh-curve}}
\end{figure}

\begin{figure}
\centering
\includegraphics[width=1\textwidth,height=1\textheight]{C:/Stock_Assessments/VRML_Assessment_2021/Model_files/NCA/FINAL_base/Verm_NoCA_base_files_Day1_Request1_drop2020PR_weight_CCFRP_lengths_hessian - Copy//plots/SR_resids.png}
\caption{Deviations around the stock-recruit curve. Labels are on first, last, and years with (log) deviations \textgreater{} 0.5. Point colors indicate year, with warmer colors indicating earlier years and cooler colors in showing later years.\label{fig:bh-resids}}
\end{figure}

\FloatBarrier

\begin{figure}
\centering
\includegraphics[width=1\textwidth,height=1\textheight]{C:/Stock_Assessments/VRML_Assessment_2021/Model_files/NCA/FINAL_base/Verm_NoCA_base_files_Day1_Request1_drop2020PR_weight_CCFRP_lengths_hessian - Copy//plots/ts11_Age-0_recruits_(1000s)_with_95_asymptotic_intervals.png}
\caption{Age-0 recruits (1,000s) with \textasciitilde95\% asymptotic intervals.\label{fig:recruits}}
\end{figure}

\begin{figure}
\centering
\includegraphics[width=1\textwidth,height=1\textheight]{C:/Stock_Assessments/VRML_Assessment_2021/Model_files/NCA/FINAL_base/Verm_NoCA_base_files_Day1_Request1_drop2020PR_weight_CCFRP_lengths_hessian - Copy//plots/recdevs2_withbars.png}
\caption{Estimated time series of recruitment deviations.\label{fig:recdevs}}
\end{figure}

\FloatBarrier

\begin{figure}
\centering
\includegraphics[width=1\textwidth,height=1\textheight]{C:/Stock_Assessments/VRML_Assessment_2021/GitHub/Vermilion_2021/doc/figures/drop1_spawnbio_NCA.png}
\caption{Change in the spawning output when a single fleet is removed from the model.\label{fig:drop-spawnbio}}
\end{figure}

\begin{figure}
\centering
\includegraphics[width=1\textwidth,height=1\textheight]{C:/Stock_Assessments/VRML_Assessment_2021/GitHub/Vermilion_2021/doc/figures/drop1_Bratio_NCA.png}
\caption{Change in the fraction of unfished biomass when a single fleet is removed from the model.\label{fig:drop-bratio}}
\end{figure}

\begin{figure}
\centering
\includegraphics[width=1\textwidth,height=1\textheight]{C:/Stock_Assessments/VRML_Assessment_2021/GitHub/Vermilion_2021/doc/figures/drop1_recdevs_NCA.png}
\caption{Change in the recruitment deviations when a single fleet is removed from the model.\label{fig:drop-recdev}}
\end{figure}

\FloatBarrier

\begin{figure}
\centering
\includegraphics[width=1\textwidth,height=1\textheight]{C:/Stock_Assessments/VRML_Assessment_2021/GitHub/Vermilion_2021/doc/figures/sens2_depl_NCA.png}
\caption{Change in depletion when historical catches are modified.\label{fig:sens2-depl}}
\end{figure}

\begin{figure}
\centering
\includegraphics[width=1\textwidth,height=1\textheight]{C:/Stock_Assessments/VRML_Assessment_2021/GitHub/Vermilion_2021/doc/figures/sens2_spr_NCA.png}
\caption{Change in the relative SPR when historical catches are modified.\label{fig:sens2-spr}}
\end{figure}

\FloatBarrier

\begin{figure}
\centering
\includegraphics[width=1\textwidth,height=1\textheight]{C:/Stock_Assessments/VRML_Assessment_2021/GitHub/Vermilion_2021/doc/figures/sens01_spawnbio_NCA.png}
\caption{Change in the trajectory of spawning output to a series of model sensitivity runs.\label{fig:sens1-spawnb}}
\end{figure}

\begin{figure}
\centering
\includegraphics[width=1\textwidth,height=1\textheight]{C:/Stock_Assessments/VRML_Assessment_2021/GitHub/Vermilion_2021/doc/figures/sens01_Bratio_NCA.png}
\caption{Change in the fraction of unfished biomass to a series of model sensitivity runs.\label{fig:sens1-bratio}}
\end{figure}

\begin{figure}
\centering
\includegraphics[width=1\textwidth,height=1\textheight]{C:/Stock_Assessments/VRML_Assessment_2021/GitHub/Vermilion_2021/doc/figures/sens01_recdevs_NCA.png}
\caption{Change in the recruitment deviations to a series of model sensitivity runs.\label{fig:sens1-recdevs}}
\end{figure}

\FloatBarrier

\begin{figure}
\centering
\includegraphics[width=1\textwidth,height=1\textheight]{C:/Stock_Assessments/VRML_Assessment_2021/GitHub/Vermilion_2021/doc/figures/sens3a_NCA.png}
\caption{Change in the trajectory of spawning output to a series of model sensitivity runs.\label{fig:sens3a}}
\end{figure}

\begin{figure}
\centering
\includegraphics[width=1\textwidth,height=1\textheight]{C:/Stock_Assessments/VRML_Assessment_2021/GitHub/Vermilion_2021/doc/figures/sens3b_NCA.png}
\caption{Change in the fraction of unfished biomass to a series of model sensitivity runs.\label{fig:sens3b}}
\end{figure}

\begin{figure}
\centering
\includegraphics[width=1\textwidth,height=1\textheight]{C:/Stock_Assessments/VRML_Assessment_2021/GitHub/Vermilion_2021/doc/figures/sens3c_NCA.png}
\caption{Change in the recruitment deviations to a series of model sensitivity runs.\label{fig:sens3c}}
\end{figure}

\FloatBarrier

\begin{figure}
\centering
\includegraphics[width=1\textwidth,height=1\textheight]{C:/Stock_Assessments/VRML_Assessment_2021/GitHub/Vermilion_2021/doc/figures/M_profile_4fig_NCA.png}
\caption{Likelihood profile across natural mortality values for each data type.\label{fig:m-profile}}
\end{figure}

\begin{figure}
\centering
\includegraphics[width=1\textwidth,height=1\textheight]{C:/Stock_Assessments/VRML_Assessment_2021/GitHub/Vermilion_2021/doc/figures/M_depl_NCA.png}
\caption{Trajectories of depletion across values of female natural mortality.\label{fig:m-depl}}
\end{figure}

\begin{figure}
\centering
\includegraphics[width=1\textwidth,height=1\textheight]{C:/Stock_Assessments/VRML_Assessment_2021/GitHub/Vermilion_2021/doc/figures/M_spawnout_NCA.png}
\caption{Trajectories of spawning output across values of female natural mortality.\label{fig:m-spawn}}
\end{figure}

\begin{figure}
\centering
\includegraphics[width=1\textwidth,height=1\textheight]{C:/Stock_Assessments/VRML_Assessment_2021/GitHub/Vermilion_2021/doc/figures/M_recruits_NCA.png}
\caption{Trajectories of age-0 recruits across values of female natural mortality.\label{fig:m-recruits}}
\end{figure}

\begin{figure}
\centering
\includegraphics[width=1\textwidth,height=1\textheight]{C:/Stock_Assessments/VRML_Assessment_2021/GitHub/Vermilion_2021/doc/figures/M_recdevs_NCA.png}
\caption{Trajectories of estimated recruitment deviations across values of female natural mortality.\label{fig:m-recdevs}}
\end{figure}

\begin{figure}
\centering
\includegraphics[width=1\textwidth,height=1\textheight]{C:/Stock_Assessments/VRML_Assessment_2021/GitHub/Vermilion_2021/doc/figures/h_profile_4fig_NCA.png}
\caption{Likelihood profile across steepness values for each data type.\label{fig:h-profile}}
\end{figure}

\begin{figure}
\centering
\includegraphics[width=1\textwidth,height=1\textheight]{C:/Stock_Assessments/VRML_Assessment_2021/GitHub/Vermilion_2021/doc/figures/h_depl_NCA.png}
\caption{Trajectories of depletion across values of steepness.\label{fig:h-depl}}
\end{figure}

\begin{figure}
\centering
\includegraphics[width=1\textwidth,height=1\textheight]{C:/Stock_Assessments/VRML_Assessment_2021/GitHub/Vermilion_2021/doc/figures/h_spawnout_NCA.png}
\caption{Trajectories of spawning output across values of steepness.\label{fig:h-spawn}}
\end{figure}

\begin{figure}
\centering
\includegraphics[width=1\textwidth,height=1\textheight]{C:/Stock_Assessments/VRML_Assessment_2021/GitHub/Vermilion_2021/doc/figures/h_recruits_NCA.png}
\caption{Trajectories of age-0 recruits across values of steepness.\label{fig:h-recruits}}
\end{figure}

\begin{figure}
\centering
\includegraphics[width=1\textwidth,height=1\textheight]{C:/Stock_Assessments/VRML_Assessment_2021/GitHub/Vermilion_2021/doc/figures/h_recdevs_NCA.png}
\caption{Trajectories of estimated recruitment deviations across values of steepness.\label{fig:h-recdevs}}
\end{figure}

\begin{figure}
\centering
\includegraphics[width=1\textwidth,height=1\textheight]{C:/Stock_Assessments/VRML_Assessment_2021/GitHub/Vermilion_2021/doc/figures/R0_profile_4fig_NCA.png}
\caption{Likelihood profile across R0 values for each data type.\label{fig:r0-profile}}
\end{figure}

\begin{figure}
\centering
\includegraphics[width=1\textwidth,height=1\textheight]{C:/Stock_Assessments/VRML_Assessment_2021/GitHub/Vermilion_2021/doc/figures/R0_depl_NCA.png}
\caption{Trajectories of depletion across values of R0.\label{fig:r0-depl}}
\end{figure}

\begin{figure}
\centering
\includegraphics[width=1\textwidth,height=1\textheight]{C:/Stock_Assessments/VRML_Assessment_2021/GitHub/Vermilion_2021/doc/figures/R0_spawnout_NCA.png}
\caption{Trajectories of spawning output across values of R0.\label{fig:r0-spawn}}
\end{figure}

\begin{figure}
\centering
\includegraphics[width=1\textwidth,height=1\textheight]{C:/Stock_Assessments/VRML_Assessment_2021/GitHub/Vermilion_2021/doc/figures/R0_recruits_NCA.png}
\caption{Trajectories of age-0 recruits across values of R0.\label{fig:r0-recruits}}
\end{figure}

\begin{figure}
\centering
\includegraphics[width=1\textwidth,height=1\textheight]{C:/Stock_Assessments/VRML_Assessment_2021/GitHub/Vermilion_2021/doc/figures/R0_recdevs_NCA.png}
\caption{Trajectories of estimated recruitment deviations across values of R0.\label{fig:r0-recdevs}}
\end{figure}

\begin{figure}
\centering
\includegraphics[width=1\textwidth,height=1\textheight]{C:/Stock_Assessments/VRML_Assessment_2021/GitHub/Vermilion_2021/doc/figures/retro_spawnout_NCA.png}
\caption{Change in the spawning output when the most recent 5 years of data area removed sequentially.\label{fig:retro-spawnb}}
\end{figure}

\begin{figure}
\centering
\includegraphics[width=1\textwidth,height=1\textheight]{C:/Stock_Assessments/VRML_Assessment_2021/GitHub/Vermilion_2021/doc/figures/retro_Bratio_NCA.png}
\caption{Change in the fraction of unfished biomass when the most recent 5 years of data area removed sequentially.\label{fig:retro-bratio}}
\end{figure}

\begin{figure}
\centering
\includegraphics[width=1\textwidth,height=1\textheight]{C:/Stock_Assessments/VRML_Assessment_2021/GitHub/Vermilion_2021/doc/figures/retro_recruits_NCA.png}
\caption{Trajectories of age-0 recruits when the most recent 5 years of data area removed sequentially.\label{fig:retro-recruits}}
\end{figure}

\begin{figure}
\centering
\includegraphics[width=1\textwidth,height=1\textheight]{C:/Stock_Assessments/VRML_Assessment_2021/GitHub/Vermilion_2021/doc/figures/retro_recdevs_NCA.png}
\caption{Change in the recruitment deviations when the most recent 5 years of data area removed sequentially.\label{fig:retro-recdevs}}
\end{figure}

\newpage

\tagstructbegin{tag=H1}\tagmcbegin{tag=H1}

\hypertarget{appendices}{%
\section*{Appendices}\label{appendices}}
\addcontentsline{toc}{section}{Appendices}

\leavevmode\tagmcend\tagstructend

\renewcommand{\thesubsection}{Appendix \Alph{subsection}}

\tagstructbegin{tag=H2}\tagmcbegin{tag=H2}

\hypertarget{append-fit}{%
\subsection{Detailed Fit to Length Composition Data}\label{append-fit}}

\leavevmode\tagmcend\tagstructend

\renewcommand{\thepage}{A\arabic{page}}
\renewcommand{\thefigure}{A\arabic{figure}}
\renewcommand{\thetable}{A\arabic{table}}
\setcounter{page}{1}
\setcounter{figure}{0}
\setcounter{table}{0}

\begin{figure}
\centering
\includegraphics[width=1\textwidth,height=1\textheight]{C:/Stock_Assessments/VRML_Assessment_2021/Model_files/NCA/FINAL_base/Verm_NoCA_base_files_Day1_Request1_drop2020PR_weight_CCFRP_lengths_hessian - Copy/plots/comp_lenfit_flt1mkt0_page1.png}
\caption{Length comps, whole catch, COM\_HKL (plot 1 of 2).`N adj.' is the input sample size after data-weighting adjustment. N eff. is the calculated effective sample size used in the McAllister-Ianelli tuning method.\label{fig:comp_lenfit_flt1mkt0_page1}}
\end{figure}

\begin{figure}
\centering
\includegraphics[width=1\textwidth,height=1\textheight]{C:/Stock_Assessments/VRML_Assessment_2021/Model_files/NCA/FINAL_base/Verm_NoCA_base_files_Day1_Request1_drop2020PR_weight_CCFRP_lengths_hessian - Copy/plots/comp_lenfit_flt1mkt0_page2.png}
\caption{Length comps, whole catch, COM\_HKL (plot 2 of 2).\label{fig:comp_lenfit_flt1mkt0_page2}}
\end{figure}

\begin{figure}
\centering
\includegraphics[width=1\textwidth,height=1\textheight]{C:/Stock_Assessments/VRML_Assessment_2021/Model_files/NCA/FINAL_base/Verm_NoCA_base_files_Day1_Request1_drop2020PR_weight_CCFRP_lengths_hessian - Copy/plots/comp_lenfit_flt2mkt0.png}
\caption{Length comps, whole catch, COM\_TWL.`N adj.' is the input sample size after data-weighting adjustment. N eff. is the calculated effective sample size used in the McAllister-Ianelli tuning method.\label{fig:comp_lenfit_flt2mkt0}}
\end{figure}

\begin{figure}
\centering
\includegraphics[width=1\textwidth,height=1\textheight]{C:/Stock_Assessments/VRML_Assessment_2021/Model_files/NCA/FINAL_base/Verm_NoCA_base_files_Day1_Request1_drop2020PR_weight_CCFRP_lengths_hessian - Copy/plots/comp_lenfit_flt3mkt0.png}
\caption{Length comps, whole catch, COM\_NET.`N adj.' is the input sample size after data-weighting adjustment. N eff. is the calculated effective sample size used in the McAllister-Ianelli tuning method.\label{fig:comp_lenfit_flt3mkt0}}
\end{figure}

\begin{figure}
\centering
\includegraphics[width=1\textwidth,height=1\textheight]{C:/Stock_Assessments/VRML_Assessment_2021/Model_files/NCA/FINAL_base/Verm_NoCA_base_files_Day1_Request1_drop2020PR_weight_CCFRP_lengths_hessian - Copy/plots/comp_lenfit_flt4mkt0_page1.png}
\caption{Length comps, whole catch, REC\_PC (plot 1 of 3).`N adj.' is the input sample size after data-weighting adjustment. N eff. is the calculated effective sample size used in the McAllister-Ianelli tuning method.\label{fig:comp_lenfit_flt4mkt0_page1}}
\end{figure}

\begin{figure}
\centering
\includegraphics[width=1\textwidth,height=1\textheight]{C:/Stock_Assessments/VRML_Assessment_2021/Model_files/NCA/FINAL_base/Verm_NoCA_base_files_Day1_Request1_drop2020PR_weight_CCFRP_lengths_hessian - Copy/plots/comp_lenfit_flt4mkt0_page2.png}
\caption{Length comps, whole catch, REC\_PC (plot 2 of 3).\label{fig:comp_lenfit_flt4mkt0_page2}}
\end{figure}

\begin{figure}
\centering
\includegraphics[width=1\textwidth,height=1\textheight]{C:/Stock_Assessments/VRML_Assessment_2021/Model_files/NCA/FINAL_base/Verm_NoCA_base_files_Day1_Request1_drop2020PR_weight_CCFRP_lengths_hessian - Copy/plots/comp_lenfit_flt4mkt0_page3.png}
\caption{Length comps, whole catch, REC\_PC (plot 3 of 3).\label{fig:comp_lenfit_flt4mkt0_page3}}
\end{figure}

\begin{figure}
\centering
\includegraphics[width=1\textwidth,height=1\textheight]{C:/Stock_Assessments/VRML_Assessment_2021/Model_files/NCA/FINAL_base/Verm_NoCA_base_files_Day1_Request1_drop2020PR_weight_CCFRP_lengths_hessian - Copy/plots/comp_lenfit_flt5mkt0.png}
\caption{Length comps, whole catch, REC\_PC\_DIS.`N adj.' is the input sample size after data-weighting adjustment. N eff. is the calculated effective sample size used in the McAllister-Ianelli tuning method.\label{fig:comp_lenfit_flt5mkt0}}
\end{figure}

\begin{figure}
\centering
\includegraphics[width=1\textwidth,height=1\textheight]{C:/Stock_Assessments/VRML_Assessment_2021/Model_files/NCA/FINAL_base/Verm_NoCA_base_files_Day1_Request1_drop2020PR_weight_CCFRP_lengths_hessian - Copy/plots/comp_lenfit_flt6mkt0_page1.png}
\caption{Length comps, whole catch, REC\_PR (plot 1 of 3).`N adj.' is the input sample size after data-weighting adjustment. N eff. is the calculated effective sample size used in the McAllister-Ianelli tuning method.\label{fig:comp_lenfit_flt6mkt0_page1}}
\end{figure}

\begin{figure}
\centering
\includegraphics[width=1\textwidth,height=1\textheight]{C:/Stock_Assessments/VRML_Assessment_2021/Model_files/NCA/FINAL_base/Verm_NoCA_base_files_Day1_Request1_drop2020PR_weight_CCFRP_lengths_hessian - Copy/plots/comp_lenfit_flt6mkt0_page2.png}
\caption{Length comps, whole catch, REC\_PR (plot 2 of 3).\label{fig:comp_lenfit_flt6mkt0_page2}}
\end{figure}

\begin{figure}
\centering
\includegraphics[width=1\textwidth,height=1\textheight]{C:/Stock_Assessments/VRML_Assessment_2021/Model_files/NCA/FINAL_base/Verm_NoCA_base_files_Day1_Request1_drop2020PR_weight_CCFRP_lengths_hessian - Copy/plots/comp_lenfit_flt6mkt0_page3.png}
\caption{Length comps, whole catch, REC\_PR (plot 3 of 3).\label{fig:comp_lenfit_flt6mkt0_page3}}
\end{figure}

\begin{figure}
\centering
\includegraphics[width=1\textwidth,height=1\textheight]{C:/Stock_Assessments/VRML_Assessment_2021/Model_files/NCA/FINAL_base/Verm_NoCA_base_files_Day1_Request1_drop2020PR_weight_CCFRP_lengths_hessian - Copy/plots/comp_lenfit_flt8mkt0.png}
\caption{Length comps, whole catch, DWV\_ONBOARD.`N adj.' is the input sample size after data-weighting adjustment. N eff. is the calculated effective sample size used in the McAllister-Ianelli tuning method.\label{fig:comp_lenfit_flt8mkt0}}
\end{figure}

\begin{figure}
\centering
\includegraphics[width=1\textwidth,height=1\textheight]{C:/Stock_Assessments/VRML_Assessment_2021/Model_files/NCA/FINAL_base/Verm_NoCA_base_files_Day1_Request1_drop2020PR_weight_CCFRP_lengths_hessian - Copy/plots/comp_lenfit_flt9mkt0_page1.png}
\caption{Length comps, whole catch, NWFSC\_TWL (plot 1 of 2).`N adj.' is the input sample size after data-weighting adjustment. N eff. is the calculated effective sample size used in the McAllister-Ianelli tuning method.\label{fig:comp_lenfit_flt9mkt0_page1}}
\end{figure}

\begin{figure}
\centering
\includegraphics[width=1\textwidth,height=1\textheight]{C:/Stock_Assessments/VRML_Assessment_2021/Model_files/NCA/FINAL_base/Verm_NoCA_base_files_Day1_Request1_drop2020PR_weight_CCFRP_lengths_hessian - Copy/plots/comp_lenfit_flt9mkt0_page2.png}
\caption{Length comps, whole catch, NWFSC\_TWL (plot 2 of 2).\label{fig:comp_lenfit_flt9mkt0_page2}}
\end{figure}

\begin{figure}
\centering
\includegraphics[width=1\textwidth,height=1\textheight]{C:/Stock_Assessments/VRML_Assessment_2021/Model_files/NCA/FINAL_base/Verm_NoCA_base_files_Day1_Request1_drop2020PR_weight_CCFRP_lengths_hessian - Copy/plots/comp_lenfit_flt11mkt0.png}
\caption{Length comps, whole catch, ABRAMS\_RESEARCH.`N adj.' is the input sample size after data-weighting adjustment. N eff. is the calculated effective sample size used in the McAllister-Ianelli tuning method.\label{fig:comp_lenfit_flt11mkt0}}
\end{figure}

\begin{figure}
\centering
\includegraphics[width=1\textwidth,height=1\textheight]{C:/Stock_Assessments/VRML_Assessment_2021/Model_files/NCA/FINAL_base/Verm_NoCA_base_files_Day1_Request1_drop2020PR_weight_CCFRP_lengths_hessian - Copy/plots/comp_lenfit_flt12mkt0.png}
\caption{Length comps, whole catch, SWFSC\_GF\_ECOL.`N adj.' is the input sample size after data-weighting adjustment. N eff. is the calculated effective sample size used in the McAllister-Ianelli tuning method.\label{fig:comp_lenfit_flt12mkt0}}
\end{figure}

\begin{figure}
\centering
\includegraphics[width=1\textwidth,height=1\textheight]{C:/Stock_Assessments/VRML_Assessment_2021/Model_files/NCA/FINAL_base/Verm_NoCA_base_files_Day1_Request1_drop2020PR_weight_CCFRP_lengths_hessian - Copy/plots/comp_lenfit_flt13mkt0.png}
\caption{Length comps, whole catch, CCFRP.`N adj.' is the input sample size after data-weighting adjustment. N eff. is the calculated effective sample size used in the McAllister-Ianelli tuning method.\label{fig:comp_lenfit_flt13mkt0}}
\end{figure}

\FloatBarrier

\tagstructbegin{tag=H2}\tagmcbegin{tag=H2}

\hypertarget{mrfss-index}{%
\subsection{MRFSS Dockside Index of Abundance}\label{mrfss-index}}

\leavevmode\tagmcend\tagstructend

\renewcommand{\thepage}{B\arabic{page}}
\renewcommand{\thefigure}{B\arabic{figure}}
\renewcommand{\thetable}{B\arabic{table}}
\setcounter{page}{1}
\setcounter{figure}{0}
\setcounter{table}{0}

\textbf{MRFSS Dockside CPFV Index, 1980-1999}

From 1980 to 2003 the MRFSS program conducted dockside intercept surveys of the recreational CPFV fishing fleet. No MRFSS CPUE data are available for the years 1990-1992, due to a hiatus in sampling related to funding issues. Sampling of California CPFVs north of Point Conception was further delayed, and CPFV samples in 1993 and 1994 are limited to San Luis Obispo County. For purposes of this assessment, the MRFSS time series was truncated at 1999 due to sampling overlap with the onboard observer program (i.e., the same observer samples the catch while onboard the vessel and also conducts the dockside intercept survey for the same vessel).

Each entry in the RecFIN Type 3 database corresponds to a single fish examined by a sampler at a particular survey site. Since only a subset of the catch may be sampled, each record also identifies the total number of that species possessed by the group of anglers being interviewed. The number of anglers and the hours fished are also recorded. The data, as they exist in RecFIN, do not indicate which records belong to the same boat trip. A description of the algorithms and process used to aggregate the RecFIN records to the trip level is outlined in the Supplemental Materials (``Identifying Trips in RecFIN'').

\textbf{MRFSS CPUE Index: Data Preparation, Filtering, and Sample Sizes}

Trips recorded with a primary area fished in Mexico or in bays, e.g., San Francisco Bay, were excluded before any filtering on species composition. For indices representing only north of Point Conception, the years 1993-1994 were excluded due to limited spatial coverage.

The Stephens-MacCall {\tagstructbegin{tag=Reference}\tagmcbegin{tag=Reference}(2004)\leavevmode\tagmcend\tagstructend} filtering approach was used to predict the probability of catching vermilion rockfish, based on the species composition of the sampler examined catch in a given trip. Prior to applying the Stephens-MacCall filter, we identified potentially informative predictor species, i.e., species with sufficient sample sizes and temporal coverage (present in at least 5\% of all trips) to inform the binomial model. The remaining 25 species all co-occurred with vermilion rockfish in at least one trip and were retained for the Stephens-MacCall logistic regression. Coefficients from the Stephens-MacCall analysis (a binomial GLM) are positive for species that are more likely to co-occur with vermilion rockfish, and negative for species that are less likely to be caught with vermilion rockfish (Figure \ref{fig:fig-sm-mrfss}). The top five species with high probability of co-occurrence with vermilion rockfish include gopher, flag, copper, canary, and starry rockfishes, all of which are associated with rocky reef and kelp habitats. The five species with the lowest probability of co-occurrence were chinook salmon, widow and greenspotted rockfishes, chub mackerel and rosy rockfish.

While the filter is useful in identifying co-occurring or non-occurring species assuming all effort was exerted in pursuit of a single target, the targeting of more than one species or species complex (``mixed trips'') can result in co-occurrence of species in the catch that do not truly co-occur in terms of habitat associations informative for an index of abundance. Stephens and MacCall {\tagstructbegin{tag=Reference}\tagmcbegin{tag=Reference}(2004)\leavevmode\tagmcend\tagstructend} recommended including all trips above a threshold where the false negatives and false positives are equally balanced. However, this does not have any biological relevance and for this data set, and we assume that if a vermilion rockfish was landed, the anglers fished in appropriate habitat, especially given vermilion rockfish is strongly associated with rocky habitat.

Stephens and MacCall {\tagstructbegin{tag=Reference}\tagmcbegin{tag=Reference}(2004)\leavevmode\tagmcend\tagstructend} proposed filtering (excluding) trips from the index standardization based on a criterion of balancing the number of false positives and false negatives. False positives (FP) are trips that are predicted to catch a vermilion rockfish based on the species composition of the catch, but did not. False negatives (FN) are trips that were not predicted to catch a vermilion rockfish, given the catch composition, but caught at least one. The Stephens-MacCall filtering method identified the probability of occurrence at which the rate of ``false positives'' equals ``false negatives'' of 0.35. The trips selected using this criteria were compared to an alternative method including all the ``false positive'' trips, regardless of the probability of encountering vermilion rockfish. This assumes that if vermilion rockfish were caught, the anglers must have fished in appropriate habitat during the trip. The catch included in this index is ``sampler-examined'' and the samplers are well trained in species identification.

The threshold probability that balances FP and FN excludes 1182 trips that did not catch a vermilion rockfish (52\% of the trips), and 188 trips (8\% of the data) that caught a vermilion rockfish. We retained the latter set of trips (FN), assuming that catching a vermilion rockfish indicates that a non-negligible fraction of the fishing effort occurred in habitat where vermilion rockfish occur. Only ``true negatives'' (the 1182 trips that neither caught vermilion rockfish, nor were predicted to catch them by the model) were excluded from the index standardization. The final dataset selected included 1083 trips, 70\% of which encountered vermilion rockfish. Sample sizes by the factors selected to model are in Tables \ref{tab:tab-region-mrfss} and \ref{tab:tab-year-mrfss}.

\textbf{MRFSS CPUE Index: Model Selection, Fits, and Diagnostics}

Initial exploration of negative binomial models for this dataset proved to be ill-fitting. The proportion of zeroes predicted by the Bayesian negative binomial models were different enough from the fraction of zeroes in the raw data, that a negative binomial model was not considered for model selection. We modeled catch per angler hour (CPUE; number of fish per angler hour) with a Bayesian delta-GLM model. Models incorporating temporal (year, 2-month waves) and geographic (region and primary area fished (inshore \textless3 nm, offshore \textgreater3 nm) factors were evaluated. For assessments north of Point Conception, two regions were defined based on counties, 1) Del Norte to Santa Cruz (``N'') and 2) Monterey to San Luis Obispo (``C''). For assessment models south of Point Conception, the region represents individual counties. Note that Santa Barbara county spans north and south of Point Conception, but all accessible fishing ports in Santa Barbara county are south of Point Conception and vessels rarely (if ever) transit Point Conception. Indices with a year and area interaction were not considered in model selection; trends in the average CPUE by region were similar in the filtered data set (Figure \ref{fig:fig-areacpue-mrfss}).

The positive observations were modeled with a Lognormal distribution that was selected over a Gamma model by a {\tagstructbegin{tag=Formula}\tagmcbegin{tag=Formula}\(\Delta AIC\)\leavevmode\tagmcend\tagstructend} of 62.35, and supported by Q-Q plots of the positive observations fit to both distributions (Figure \ref{fig:fig-dist-fits-mrfss}). The delta-GLM method allows selection of differing linear predictors between the binomial and positive models. Based on AIC values from maximum likelihood fits, a main effects model including YEAR and SubRegion was fit for the binomial model and a main effects model including YEAR and SubRegion and AREA X was fit for the Lognormal model (Table \ref{tab:tab-model-select-mrfss}). Models were fit using the ``rstanarm'' R package (version 2.21.1). Posterior predictive checks of the Bayesian model fit for the binomial model and the positive model were all reasonable (Figures \ref{fig:fig-posterior-mean-mrfss} and \ref{fig:fig-posterior-sd-mrfss}). The binomial model generated data sets with the proportion zeros similar to the 30\% zeroes in the observed data (Figure \ref{fig:fig-propzero-mrfss}) and the predicted marginal effects from both the binomial and Lognormal models can be found in (Figures \ref{fig:fig-Dbin-marginal-mrfss} and \ref{fig:fig-Dpos-marginal-mrfss}). The final index (Table \ref{tab:tab-index-mrfss}) represents a similar trend to the arithmetic mean of the annual CPUE (Figure \ref{fig:fig-cpue-mrfss}).

\newpage

\begin{table}

\caption{\label{tab:tab-region-mrfss}Samples of vermilion rockfish in the northern model by subregion used in the index.}
\centering
\begin{tabular}[t]{lrrl}
\toprule
Subregion & Positive Samples & Samples & Percent Positive\\
\midrule
\cellcolor{gray!6}{C} & \cellcolor{gray!6}{442} & \cellcolor{gray!6}{585} & \cellcolor{gray!6}{76\%}\\
N & 320 & 498 & 64\%\\
\bottomrule
\end{tabular}
\end{table}

\begin{table}

\caption{\label{tab:tab-year-mrfss}Samples of vermilion rockfish in the northern model by year.}
\centering
\begin{tabular}[t]{lrrl}
\toprule
Year & Positive Samples & Samples & Percent Positive\\
\midrule
\cellcolor{gray!6}{1980} & \cellcolor{gray!6}{31} & \cellcolor{gray!6}{57} & \cellcolor{gray!6}{54\%}\\
1981 & 14 & 32 & 44\%\\
\cellcolor{gray!6}{1982} & \cellcolor{gray!6}{24} & \cellcolor{gray!6}{41} & \cellcolor{gray!6}{59\%}\\
1983 & 19 & 33 & 58\%\\
\cellcolor{gray!6}{1984} & \cellcolor{gray!6}{34} & \cellcolor{gray!6}{59} & \cellcolor{gray!6}{58\%}\\
\addlinespace
1985 & 54 & 98 & 55\%\\
\cellcolor{gray!6}{1986} & \cellcolor{gray!6}{50} & \cellcolor{gray!6}{87} & \cellcolor{gray!6}{57\%}\\
1987 & 27 & 36 & 75\%\\
\cellcolor{gray!6}{1988} & \cellcolor{gray!6}{38} & \cellcolor{gray!6}{48} & \cellcolor{gray!6}{79\%}\\
1989 & 29 & 42 & 69\%\\
\addlinespace
\cellcolor{gray!6}{1995} & \cellcolor{gray!6}{31} & \cellcolor{gray!6}{41} & \cellcolor{gray!6}{76\%}\\
1996 & 104 & 129 & 81\%\\
\cellcolor{gray!6}{1997} & \cellcolor{gray!6}{127} & \cellcolor{gray!6}{162} & \cellcolor{gray!6}{78\%}\\
1998 & 98 & 119 & 82\%\\
\cellcolor{gray!6}{1999} & \cellcolor{gray!6}{82} & \cellcolor{gray!6}{99} & \cellcolor{gray!6}{83\%}\\
\bottomrule
\end{tabular}
\end{table}

\FloatBarrier

\begin{table}

\caption{\label{tab:tab-model-select-mrfss}Model selection for the MRFSS dockside survey index for vermilion rockfish in the northern model.}
\centering
\begin{tabular}[t]{lrr}
\toprule
Model & Binomial $\Delta$AIC & Lognormal $\Delta$AIC\\
\midrule
\cellcolor{gray!6}{1} & \cellcolor{gray!6}{65.99} & \cellcolor{gray!6}{106.17}\\
YEAR + SubRegion & 0.00 & 0.89\\
\cellcolor{gray!6}{YEAR + SubRegion + WAVE} & \cellcolor{gray!6}{1.77} & \cellcolor{gray!6}{3.03}\\
YEAR + SubRegion + WAVE + AREA X & 3.76 & 1.85\\
\cellcolor{gray!6}{YEAR + WAVE} & \cellcolor{gray!6}{22.05} & \cellcolor{gray!6}{21.16}\\
\addlinespace
YEAR + AREA X & 20.13 & 14.44\\
\cellcolor{gray!6}{YEAR + WAVE + AREA X} & \cellcolor{gray!6}{22.67} & \cellcolor{gray!6}{16.13}\\
YEAR + SubRegion + AREA X & 2.00 & 0.00\\
\bottomrule
\end{tabular}
\end{table}

\FloatBarrier

\begin{table}

\caption{\label{tab:tab-index-mrfss}Standardized index for the MRFSS dockside survey index with log-scale standard errors and 95\% highest
       posterior density (HPD) intervals for vermilion in the northern model.}
\centering
\begin{tabular}[t]{rrrrr}
\toprule
Year & Index & logSE & lower HPD & upper HPD\\
\midrule
\cellcolor{gray!6}{1980} & \cellcolor{gray!6}{0.05} & \cellcolor{gray!6}{0.21} & \cellcolor{gray!6}{0.03} & \cellcolor{gray!6}{0.08}\\
1981 & 0.04 & 0.32 & 0.02 & 0.07\\
\cellcolor{gray!6}{1982} & \cellcolor{gray!6}{0.05} & \cellcolor{gray!6}{0.23} & \cellcolor{gray!6}{0.03} & \cellcolor{gray!6}{0.07}\\
1983 & 0.07 & 0.26 & 0.04 & 0.11\\
\cellcolor{gray!6}{1984} & \cellcolor{gray!6}{0.09} & \cellcolor{gray!6}{0.20} & \cellcolor{gray!6}{0.06} & \cellcolor{gray!6}{0.13}\\
\addlinespace
1985 & 0.06 & 0.16 & 0.04 & 0.08\\
\cellcolor{gray!6}{1986} & \cellcolor{gray!6}{0.07} & \cellcolor{gray!6}{0.16} & \cellcolor{gray!6}{0.05} & \cellcolor{gray!6}{0.10}\\
1987 & 0.08 & 0.21 & 0.05 & 0.12\\
\cellcolor{gray!6}{1988} & \cellcolor{gray!6}{0.11} & \cellcolor{gray!6}{0.17} & \cellcolor{gray!6}{0.08} & \cellcolor{gray!6}{0.15}\\
1989 & 0.09 & 0.21 & 0.06 & 0.13\\
\addlinespace
\cellcolor{gray!6}{1995} & \cellcolor{gray!6}{0.08} & \cellcolor{gray!6}{0.20} & \cellcolor{gray!6}{0.05} & \cellcolor{gray!6}{0.12}\\
1996 & 0.09 & 0.11 & 0.07 & 0.11\\
\cellcolor{gray!6}{1997} & \cellcolor{gray!6}{0.23} & \cellcolor{gray!6}{0.11} & \cellcolor{gray!6}{0.19} & \cellcolor{gray!6}{0.29}\\
1998 & 0.17 & 0.12 & 0.13 & 0.21\\
\cellcolor{gray!6}{1999} & \cellcolor{gray!6}{0.09} & \cellcolor{gray!6}{0.12} & \cellcolor{gray!6}{0.07} & \cellcolor{gray!6}{0.11}\\
\bottomrule
\end{tabular}
\end{table}

\FloatBarrier

\begin{figure}
\includegraphics[width=0.6\linewidth]{C:/Stock_Assessments/VRML_Assessment_2021/Indices_of_Abundance/MRFSS_dockside/NCA/2021-07-19/MRFSS_dockside_SM_species} \caption{Species coefficients (blue bars) from the binomial GLM for presence/absence of vermilion rockfish in the CRFS private boat data. Horizontal black bars are $95\%$ confidence intervals.}\label{fig:fig-sm-mrfss}
\end{figure}

\FloatBarrier

\begin{figure}
\centering
\includegraphics{C:/Stock_Assessments/VRML_Assessment_2021/GitHub/Vermilion_2021/doc/indices/vermilion_MRFSS_dockside_writeup_NCA_files/figure-latex/fig-areacpue-mrfss-1.pdf}
\caption{\label{fig:fig-areacpue-mrfss}Arithmetic mean of CPUE by region for vermilion from the filtered data.}
\end{figure}

\begin{figure}
\centering
\includegraphics{C:/Stock_Assessments/VRML_Assessment_2021/GitHub/Vermilion_2021/doc/indices/vermilion_MRFSS_dockside_writeup_NCA_files/figure-latex/fig-dist-fits-mrfss-1.pdf}
\caption{\label{fig:fig-dist-fits-mrfss}Q-Q plot (top) of the positive observations fit to lognormal and gamma distributions, and fitted values vs residuals for the Lognormal model (bottom).}
\end{figure}

\begin{figure}
\centering
\includegraphics{C:/Stock_Assessments/VRML_Assessment_2021/GitHub/Vermilion_2021/doc/indices/vermilion_MRFSS_dockside_writeup_NCA_files/figure-latex/fig-posterior-mean-mrfss-1.pdf}
\caption{\label{fig:fig-posterior-mean-mrfss}Posterior predictive draws of the mean (x-axis) by year in replicate data sets generated by the delta model with a vertical line representing the observed mean in the data.}
\end{figure}

\FloatBarrier

\begin{figure}
\centering
\includegraphics{C:/Stock_Assessments/VRML_Assessment_2021/GitHub/Vermilion_2021/doc/indices/vermilion_MRFSS_dockside_writeup_NCA_files/figure-latex/fig-posterior-sd-mrfss-1.pdf}
\caption{\label{fig:fig-posterior-sd-mrfss}Posterior predictive draws of the standard deviation by year (x-axis) in replicate data sets generated by the delta model with a vertical line representing the observed standard deviation in the data.}
\end{figure}

\begin{figure}
\centering
\includegraphics{C:/Stock_Assessments/VRML_Assessment_2021/GitHub/Vermilion_2021/doc/indices/vermilion_MRFSS_dockside_writeup_NCA_files/figure-latex/fig-propzero-mrfss-1.pdf}
\caption{\label{fig:fig-propzero-mrfss}Posterior predictive distribution of the proportion of zero observations (x-axis) in replicate data sets generated by the delta model with a vertical line representing the observed average proportion of zeros in the data.}
\end{figure}

\FloatBarrier

\begin{figure}
\centering
\includegraphics{C:/Stock_Assessments/VRML_Assessment_2021/GitHub/Vermilion_2021/doc/indices/vermilion_MRFSS_dockside_writeup_NCA_files/figure-latex/fig-Dbin-marginal-mrfss-1.pdf}
\caption{\label{fig:fig-Dbin-marginal-mrfss}Binomial model marginal effects.}
\end{figure}

\begin{figure}
\centering
\includegraphics{C:/Stock_Assessments/VRML_Assessment_2021/GitHub/Vermilion_2021/doc/indices/vermilion_MRFSS_dockside_writeup_NCA_files/figure-latex/fig-Dpos-marginal-mrfss-1.pdf}
\caption{\label{fig:fig-Dpos-marginal-mrfss}Positive model marginal effects.}
\end{figure}

\begin{figure}
\centering
\includegraphics{C:/Stock_Assessments/VRML_Assessment_2021/GitHub/Vermilion_2021/doc/indices/vermilion_MRFSS_dockside_writeup_NCA_files/figure-latex/fig-cpue-mrfss-1.pdf}
\caption{\label{fig:fig-cpue-mrfss}Standardized index and arithmetic mean of the CPUE from the filtered data. Each timeseries is scaled to its respective means.}
\end{figure}

\newpage

\tagstructbegin{tag=H2}\tagmcbegin{tag=H2}

\hypertarget{cpfv-index}{%
\subsection{California Onboard CPFV Index of Abundance}\label{cpfv-index}}

\leavevmode\tagmcend\tagstructend

\renewcommand{\thepage}{C\arabic{page}}
\renewcommand{\thefigure}{C\arabic{figure}}
\renewcommand{\thetable}{C\arabic{table}}
\setcounter{page}{1}
\setcounter{figure}{0}
\setcounter{table}{0}

\textbf{California Onboard Observer Survey, 1999-2019}

The state of California implemented a statewide onboard observer sampling program in 1999 {\tagstructbegin{tag=Reference}\tagmcbegin{tag=Reference}(Monk et al. 2014)\leavevmode\tagmcend\tagstructend}. California Polytechnic State University (Cal Poly) has conducted an independent onboard sampling program as of 2003 for boats in Port San Luis and Morro Bay, and follows the protocols established in Reilly et al. {\tagstructbegin{tag=Reference}\tagmcbegin{tag=Reference}(1998)\leavevmode\tagmcend\tagstructend}.

During an onboard observer trip the sampler rides along on the CPFV and records location-specific catch and discard information to the species level for a subset of anglers onboard the vessel. The subset of observed anglers is usually a maximum of 15 people the observed anglers change during each fishing stop.\\
The catch cannot be linked to an individual, but rather to a specific fishing location. The sampler also records the starting and ending time, number of anglers observed, starting and ending depth, and measures discarded fish. The fine-scale catch and effort data allow us to better filter the data for indices to fishing stops within suitable habitat for vermilion rockfish . Cal Poly has modified protocols reflect sampling changes that CDFW has also adopted, e.g., observing fish as they are encountered instead of at the level of a fisher's bag. Therefore, the Cal Poly data area incorporated in the same index as the CDFW data from 1999-2019. The only difference is that Cal Poly measures the length of both retained and discarded fish.

Due to the COVID-19 pandemic, there are no onboard observer samples from either CDFW or Cal Poly in 2020.

\textbf{California CPFV CPUE Index: Data Preparation, Filtering, and Sample Sizes}

As described above the CDFW and Cal Poly onboard observer programs are identical in that the same protocols are followed. The only difference is that Cal Poly measures both retained and discarded fish from the observed anglers and CDFW measures only discarded fish from the observed anglers. CDFW measures retained fish as part of the angler interview at the bag and trip level. This index selectivity is mirrored to the recreational fleet in the stock assessment model, which represent only retained (dead) fish. Therefore, only retained fish were modeled in this index. The length from CDFW sampling are contained in the RecFIN database and included in the length composition for the recreational fleet in the assessment model.

A number of filters are applied to these data. All of the Cal Poly data were QA/QC-ed once key-punched, whereas a number of errors remain in the data from CDFW. Data sheets from CDFW are not available prior to 2012 and staff constraints have also prevented a quality control review of the data.

Each drift was assigned to a reef (hard bottom). Hard bottom was extracted from the {\tagstructbegin{tag=Link}\tagmcbegin{tag=Link}\href{http://seafloor.otterlabs.org/index.html}{California Seafloor Mapping Project}\leavevmode\tagmcend\tagstructend}, with bathymetric data from state waters available at a 2 m resolution. Reefs were developed based on a number of factors described in the supplemental material (``Reef Delineation''). Depth restrictions in the recreational fishery were fairly consistent from 2004-2016. Starting in 2017, depth restrictions eased in districts north of Point Conception and the recreational fleet targeted these depths (Figure \ref{fig:fig-depthfished-cpfvonboard}). The deeper waters (40-50 fm) are outside of the mapped hard bottom habitat, but could be assigend to the larger areas considered as a factor in the index.

We retained 4481 drifts for index standardization, with 1706 drifts encountering vermilion rockfish (Table \ref{tab:tab-data-filter-cpfvonboard}).

Sample sizes by factors selected to model, excluding WAVE can be found in Tables \ref{tab:tab-region-cpfvonboard}, \ref{tab:tab-depth-cpfvonboard}, and \ref{tab:tab-year-cpfvonboard}.

\textbf{California CPFV CPUE Index: Model Selection, Fits, and Diagnostics}

We modeled retained catch per angler hour (CPUE; number of fish per angler hour) a Bayesian delta-GLM model. Indices with a year and area interaction were not considered in model selection; trends in the average CPUE by region were similar in the filtered data set (Figure \ref{fig:fig-areacpue-cpfvonboard}).

A Lognormal model was selected over a over a Gamma model for the positive observations by a {\tagstructbegin{tag=Formula}\tagmcbegin{tag=Formula}\(\Delta AIC\)\leavevmode\tagmcend\tagstructend} of 122.41, and supported by Q-Q plots of the positive observations fit to both distributions (Figure \ref{fig:fig-dist-fits-cpfvonboard}). The delta-GLM method allows the linear predictors to differ between the binomial and positive models. Based on AIC values from maximum likelihood fits (Table \ref{tab:tab-model-select-cpfvonboard}), a main effects model including YEAR and WAVE and DEPTH bin was fit for the binomial model and a main effects model including YEAR and WAVE and DEPTH bin was fit for the Lognormal model. Models were fit using the ``rstanarm'' R package (version 2.21.1). Posterior predictive checks of the Bayesian model fit for the binomial model and the positive model were all reasonable (Figures \ref{fig:fig-posterior-mean-cpfvonboard} and \ref{fig:fig-posterior-sd-cpfvonboard}). The binomial model generated data sets with the proportion zeros similar to the 62\% zeroes in the observed data (Figure \ref{fig:fig-propzero-cpfvonboard}). The predicted marginal effects from both the binomial and Lognormal models can be found in Figures \ref{fig:fig-Dbin-marginal-cpfvonboard} and \ref{fig:fig-Dpos-marginal-cpfvonboard}. The final index (Table \ref{tab:tab-index-cpfvonboard}) represents a similar trend to the arithmetic mean of the annual CPUE (Figure \ref{fig:fig-cpue-cpfvonboard}).

\FloatBarrier

\begin{table}

\caption{\label{tab:tab-data-filter-cpfvonboard}Data filtering steps for theCA CPFV onboard survey index for vermilion rockfish in the northern model. The last row in the table represents the number of trips used 
      to develop the index.}
\centering
\begin{tabular}[t]{>{\raggedright\arraybackslash}p{8em}>{\raggedright\arraybackslash}p{15em}c>{\centering\arraybackslash}p{8em}>{\centering\arraybackslash}p{8em}}
\toprule
Filter & Desciption & Trip & Positive Trips & Percent drifts retained\\
\midrule
\cellcolor{gray!6}{All} & \cellcolor{gray!6}{Download from SQL; identifiable errors filtered} & \cellcolor{gray!6}{6901} & \cellcolor{gray!6}{1755} & \cellcolor{gray!6}{25\%}\\
Fishery closed & Removed samples when target fish fishery closed & 5922 & 1736 & 29\%\\
\cellcolor{gray!6}{Ocean only} & \cellcolor{gray!6}{Removed samples from major bays} & \cellcolor{gray!6}{5780} & \cellcolor{gray!6}{1736} & \cellcolor{gray!6}{30\%}\\
Catch & Removed samples with zero catch of any species & 5335 & 1736 & 33\%\\
\cellcolor{gray!6}{Depth} & \cellcolor{gray!6}{Removed samples in less than max depth of species} & \cellcolor{gray!6}{5287} & \cellcolor{gray!6}{1736} & \cellcolor{gray!6}{33\%}\\
\addlinespace
Time fished & Removed upper two percent of time fished & 5180 & 1722 & 33\%\\
\cellcolor{gray!6}{Percent groundfish in samples} & \cellcolor{gray!6}{Removed samples with fewer groundfish than when the target observed} & \cellcolor{gray!6}{4481} & \cellcolor{gray!6}{1706} & \cellcolor{gray!6}{38\%}\\
\bottomrule
\end{tabular}
\end{table}

\FloatBarrier

\begin{table}

\caption{\label{tab:tab-depth-cpfvonboard}Positive samples of vermilion rockfish in the northern model by depth (fm).}
\centering
\begin{tabular}[t]{lrrl}
\toprule
Year & Positive Samples & Samples & Percent Positive\\
\midrule
\cellcolor{gray!6}{(0,10]} & \cellcolor{gray!6}{40} & \cellcolor{gray!6}{346} & \cellcolor{gray!6}{12\%}\\
(10,15] & 139 & 559 & 25\%\\
\cellcolor{gray!6}{(15,20]} & \cellcolor{gray!6}{279} & \cellcolor{gray!6}{808} & \cellcolor{gray!6}{35\%}\\
(20,25] & 226 & 588 & 38\%\\
\cellcolor{gray!6}{(25,30]} & \cellcolor{gray!6}{219} & \cellcolor{gray!6}{601} & \cellcolor{gray!6}{36\%}\\
\addlinespace
(30,35] & 159 & 373 & 43\%\\
\cellcolor{gray!6}{(35,40]} & \cellcolor{gray!6}{216} & \cellcolor{gray!6}{450} & \cellcolor{gray!6}{48\%}\\
(40,65] & 428 & 756 & 57\%\\
\bottomrule
\end{tabular}
\end{table}

\FloatBarrier

\begin{table}

\caption{\label{tab:tab-region-cpfvonboard}Samples of vermilion rockfish in the northern model by subregion used in the index.}
\centering
\begin{tabular}[t]{lrrl}
\toprule
Subregion & Positive Samples & Samples & Percent Positive\\
\midrule
\cellcolor{gray!6}{CA/OR border to Santa Cruz (V1)} & \cellcolor{gray!6}{238} & \cellcolor{gray!6}{1213} & \cellcolor{gray!6}{20\%}\\
Moss Landing to Big Sur (V2) & 146 & 511 & 29\%\\
\cellcolor{gray!6}{San Luis Obsipso to Morro Bay (V3)} & \cellcolor{gray!6}{591} & \cellcolor{gray!6}{1044} & \cellcolor{gray!6}{57\%}\\
South Morro Bay to Point Conception (V4) & 643 & 1180 & 54\%\\
\cellcolor{gray!6}{Offshore (V5)} & \cellcolor{gray!6}{88} & \cellcolor{gray!6}{533} & \cellcolor{gray!6}{17\%}\\
\bottomrule
\end{tabular}
\end{table}

\FloatBarrier

\begin{table}

\caption{\label{tab:tab-year-cpfvonboard}Samples of vermilion rockfish in the northern model by year.}
\centering
\begin{tabular}[t]{lrrl}
\toprule
Year & Positive Samples & Samples & Percent Positive\\
\midrule
\cellcolor{gray!6}{1999} & \cellcolor{gray!6}{13} & \cellcolor{gray!6}{60} & \cellcolor{gray!6}{22\%}\\
2000 & 6 & 38 & 16\%\\
\cellcolor{gray!6}{2001} & \cellcolor{gray!6}{11} & \cellcolor{gray!6}{71} & \cellcolor{gray!6}{15\%}\\
2002 & 17 & 60 & 28\%\\
\cellcolor{gray!6}{2003} & \cellcolor{gray!6}{117} & \cellcolor{gray!6}{276} & \cellcolor{gray!6}{42\%}\\
\addlinespace
2004 & 192 & 400 & 48\%\\
\cellcolor{gray!6}{2005} & \cellcolor{gray!6}{67} & \cellcolor{gray!6}{153} & \cellcolor{gray!6}{44\%}\\
2006 & 121 & 265 & 46\%\\
\cellcolor{gray!6}{2007} & \cellcolor{gray!6}{126} & \cellcolor{gray!6}{268} & \cellcolor{gray!6}{47\%}\\
2008 & 47 & 155 & 30\%\\
\addlinespace
\cellcolor{gray!6}{2009} & \cellcolor{gray!6}{54} & \cellcolor{gray!6}{198} & \cellcolor{gray!6}{27\%}\\
2010 & 79 & 193 & 41\%\\
\cellcolor{gray!6}{2011} & \cellcolor{gray!6}{62} & \cellcolor{gray!6}{182} & \cellcolor{gray!6}{34\%}\\
2012 & 66 & 220 & 30\%\\
\cellcolor{gray!6}{2013} & \cellcolor{gray!6}{29} & \cellcolor{gray!6}{160} & \cellcolor{gray!6}{18\%}\\
\addlinespace
2014 & 47 & 221 & 21\%\\
\cellcolor{gray!6}{2015} & \cellcolor{gray!6}{75} & \cellcolor{gray!6}{219} & \cellcolor{gray!6}{34\%}\\
2016 & 79 & 321 & 25\%\\
\cellcolor{gray!6}{2017} & \cellcolor{gray!6}{226} & \cellcolor{gray!6}{426} & \cellcolor{gray!6}{53\%}\\
2018 & 146 & 295 & 49\%\\
\addlinespace
\cellcolor{gray!6}{2019} & \cellcolor{gray!6}{126} & \cellcolor{gray!6}{300} & \cellcolor{gray!6}{42\%}\\
\bottomrule
\end{tabular}
\end{table}

\FloatBarrier

\begin{table}

\caption{\label{tab:tab-model-select-cpfvonboard}Model selection for the CA CPFV onboard survey index for vermilion rockfish in the northern model.}
\centering
\begin{tabular}[t]{lrr}
\toprule
Model & Binomial $\Delta$AIC & Lognormal $\Delta$AIC\\
\midrule
\cellcolor{gray!6}{1} & \cellcolor{gray!6}{797.52} & \cellcolor{gray!6}{436.25}\\
YEAR + SubRegion & 129.05 & 60.03\\
\cellcolor{gray!6}{YEAR + SubRegion + WAVE} & \cellcolor{gray!6}{120.54} & \cellcolor{gray!6}{58.72}\\
YEAR + SubRegion + WAVE + DEPTH bin & 0.00 & 0.00\\
\cellcolor{gray!6}{YEAR + WAVE + DEPTH bin} & \cellcolor{gray!6}{285.69} & \cellcolor{gray!6}{66.16}\\
\addlinespace
YEAR + DEPTH bin & 316.83 & 74.00\\
\cellcolor{gray!6}{YEAR + SubRegion + DEPTH bin} & \cellcolor{gray!6}{10.87} & \cellcolor{gray!6}{6.06}\\
\bottomrule
\end{tabular}
\end{table}

\FloatBarrier

\begin{table}

\caption{\label{tab:tab-index-cpfvonboard}Standardized index for the CA CPFV onboard survey index with log-scale standard errors and 95\% highest
       posterior density (HPD) intervals for vermilion in the northern model.}
\centering
\begin{tabular}[t]{rrrrr}
\toprule
Year & Index & logSE & lower HPD & upper HPD\\
\midrule
\cellcolor{gray!6}{1999} & \cellcolor{gray!6}{0.02} & \cellcolor{gray!6}{0.53} & \cellcolor{gray!6}{0.01} & \cellcolor{gray!6}{0.05}\\
2000 & 0.02 & 0.65 & 0.00 & 0.04\\
\cellcolor{gray!6}{2001} & \cellcolor{gray!6}{0.01} & \cellcolor{gray!6}{0.53} & \cellcolor{gray!6}{0.00} & \cellcolor{gray!6}{0.02}\\
2002 & 0.02 & 0.42 & 0.01 & 0.05\\
\cellcolor{gray!6}{2003} & \cellcolor{gray!6}{0.05} & \cellcolor{gray!6}{0.33} & \cellcolor{gray!6}{0.02} & \cellcolor{gray!6}{0.09}\\
\addlinespace
2004 & 0.07 & 0.28 & 0.04 & 0.11\\
\cellcolor{gray!6}{2005} & \cellcolor{gray!6}{0.04} & \cellcolor{gray!6}{0.38} & \cellcolor{gray!6}{0.02} & \cellcolor{gray!6}{0.08}\\
2006 & 0.05 & 0.36 & 0.02 & 0.09\\
\cellcolor{gray!6}{2007} & \cellcolor{gray!6}{0.06} & \cellcolor{gray!6}{0.35} & \cellcolor{gray!6}{0.03} & \cellcolor{gray!6}{0.11}\\
2008 & 0.03 & 0.38 & 0.01 & 0.05\\
\addlinespace
\cellcolor{gray!6}{2009} & \cellcolor{gray!6}{0.04} & \cellcolor{gray!6}{0.37} & \cellcolor{gray!6}{0.02} & \cellcolor{gray!6}{0.07}\\
2010 & 0.05 & 0.37 & 0.02 & 0.09\\
\cellcolor{gray!6}{2011} & \cellcolor{gray!6}{0.04} & \cellcolor{gray!6}{0.37} & \cellcolor{gray!6}{0.02} & \cellcolor{gray!6}{0.08}\\
2012 & 0.03 & 0.38 & 0.01 & 0.06\\
\cellcolor{gray!6}{2013} & \cellcolor{gray!6}{0.02} & \cellcolor{gray!6}{0.42} & \cellcolor{gray!6}{0.01} & \cellcolor{gray!6}{0.04}\\
\addlinespace
2014 & 0.02 & 0.38 & 0.01 & 0.04\\
\cellcolor{gray!6}{2015} & \cellcolor{gray!6}{0.04} & \cellcolor{gray!6}{0.37} & \cellcolor{gray!6}{0.02} & \cellcolor{gray!6}{0.07}\\
2016 & 0.03 & 0.37 & 0.01 & 0.06\\
\cellcolor{gray!6}{2017} & \cellcolor{gray!6}{0.04} & \cellcolor{gray!6}{0.36} & \cellcolor{gray!6}{0.02} & \cellcolor{gray!6}{0.08}\\
2018 & 0.05 & 0.37 & 0.02 & 0.09\\
\addlinespace
\cellcolor{gray!6}{2019} & \cellcolor{gray!6}{0.04} & \cellcolor{gray!6}{0.37} & \cellcolor{gray!6}{0.02} & \cellcolor{gray!6}{0.08}\\
\bottomrule
\end{tabular}
\end{table}

\FloatBarrier

\begin{figure}
\centering
\includegraphics{C:/Stock_Assessments/VRML_Assessment_2021/GitHub/Vermilion_2021/doc/indices/vermilion_CA_CPFV_onboard_writeup_NCA_files/figure-latex/fig-depthfished-cpfvonboard-1.pdf}
\caption{\label{fig:fig-depthfished-cpfvonboard}Boxplots of depths fished by year in the filtered data.}
\end{figure}

\FloatBarrier

\begin{figure}
\centering
\includegraphics{C:/Stock_Assessments/VRML_Assessment_2021/GitHub/Vermilion_2021/doc/indices/vermilion_CA_CPFV_onboard_writeup_NCA_files/figure-latex/fig-areacpue-cpfvonboard-1.pdf}
\caption{\label{fig:fig-areacpue-cpfvonboard}Arithmetic mean of CPUE by region for vermilion from the filtered data. The areas used are in the text.}
\end{figure}

\begin{figure}
\centering
\includegraphics{C:/Stock_Assessments/VRML_Assessment_2021/GitHub/Vermilion_2021/doc/indices/vermilion_CA_CPFV_onboard_writeup_NCA_files/figure-latex/fig-dist-fits-cpfvonboard-1.pdf}
\caption{\label{fig:fig-dist-fits-cpfvonboard}Q-Q plot (top) of the positive observations lognormal gamma distributions and fitted values vs residuals for the Lognormal model (bottom).}
\end{figure}

\begin{figure}
\centering
\includegraphics{C:/Stock_Assessments/VRML_Assessment_2021/GitHub/Vermilion_2021/doc/indices/vermilion_CA_CPFV_onboard_writeup_NCA_files/figure-latex/fig-posterior-mean-cpfvonboard-1.pdf}
\caption{\label{fig:fig-posterior-mean-cpfvonboard}Posterior predictive draws of the mean (x-axis) by year in replicate data sets generated by the delta model with a vertical line representing the observed mean in the data.}
\end{figure}

\FloatBarrier

\begin{figure}
\centering
\includegraphics{C:/Stock_Assessments/VRML_Assessment_2021/GitHub/Vermilion_2021/doc/indices/vermilion_CA_CPFV_onboard_writeup_NCA_files/figure-latex/fig-posterior-sd-cpfvonboard-1.pdf}
\caption{\label{fig:fig-posterior-sd-cpfvonboard}Posterior predictive draws of the standard deviation by year (x-axis) in replicate data sets generated by the delta model with a vertical line representing the observed standard deviation in the data.}
\end{figure}

\begin{figure}
\centering
\includegraphics{C:/Stock_Assessments/VRML_Assessment_2021/GitHub/Vermilion_2021/doc/indices/vermilion_CA_CPFV_onboard_writeup_NCA_files/figure-latex/fig-propzero-cpfvonboard-1.pdf}
\caption{\label{fig:fig-propzero-cpfvonboard}Posterior predictive distribution of the proportion of zero observations (x-axis) in replicate data sets generated by the delta model with a vertical line representing the observed average proportion of zeros in the data.}
\end{figure}

\FloatBarrier

\begin{figure}
\centering
\includegraphics{C:/Stock_Assessments/VRML_Assessment_2021/GitHub/Vermilion_2021/doc/indices/vermilion_CA_CPFV_onboard_writeup_NCA_files/figure-latex/fig-cpue-cpfvonboard-1.pdf}
\caption{\label{fig:fig-cpue-cpfvonboard}Standardized index and arithmetic mean of the CPUE from the filtered data. Each timeseries is scaled to its respective mean.}
\end{figure}

\begin{figure}
\centering
\includegraphics{C:/Stock_Assessments/VRML_Assessment_2021/GitHub/Vermilion_2021/doc/indices/vermilion_CA_CPFV_onboard_writeup_NCA_files/figure-latex/fig-Dbin-marginal-cpfvonboard-1.pdf}
\caption{\label{fig:fig-Dbin-marginal-cpfvonboard}Marginal effects from the binomial model of the delta-GLM.}
\end{figure}

\begin{figure}
\centering
\includegraphics{C:/Stock_Assessments/VRML_Assessment_2021/GitHub/Vermilion_2021/doc/indices/vermilion_CA_CPFV_onboard_writeup_NCA_files/figure-latex/fig-Dpos-marginal-cpfvonboard-1.pdf}
\caption{\label{fig:fig-Dpos-marginal-cpfvonboard}Marginal effects from the positive model of the delta-GLM.}
\end{figure}

\newpage

\tagstructbegin{tag=H2}\tagmcbegin{tag=H2}

\hypertarget{debwv-index}{%
\subsection{Deb Wilson-Vandenberg Onboard CPFV Index of Abundance}\label{debwv-index}}

\leavevmode\tagmcend\tagstructend

\renewcommand{\thepage}{D\arabic{page}}
\renewcommand{\thefigure}{D\arabic{figure}}
\renewcommand{\thetable}{D\arabic{table}}
\setcounter{page}{1}
\setcounter{figure}{0}
\setcounter{table}{0}

\textbf{Deb Wilson-Vandenberg Index}

The Deb Wilson-Vandenberg data set is an onboard observer survey data conducted by CDFW survey in central California from 1987-1998 and referred to as the Deb Wilson-Vandenberg onboard observer survey, {\tagstructbegin{tag=Reference}\tagmcbegin{tag=Reference}(Reilly et al. 1998)\leavevmode\tagmcend\tagstructend}. During an onboard observer trip the sampler rides along on the CPFV and records location-specific catch and discard information to the species level for a subset of anglers onboard the vessel. The subset of observed anglers is usually a maximum of 15 people the observed anglers change during each fishing stop. The catch cannot be linked to an individual, but rather to a specific fishing location. The sampler also records the starting and ending time, number of anglers observed, starting and ending depth, and measures discarded fish. The fine-scale catch and effort data allow us to better filter the data for indices to fishing stops within suitable habitat for the target species.

\textbf{Deb Wilson-Vandenberg Index: Data Preparation, Filtering, and Sample Sizes}

A large effort was made by the SWFSC to recover data from the original data sheets for this survey and developed into a relational database {\tagstructbegin{tag=Reference}\tagmcbegin{tag=Reference}(Monk et al. 2016)\leavevmode\tagmcend\tagstructend}. The specific fishing locations at each fishing stop were recorded at a finer scale than the catch data for this survey. We aggregated the relevant location information (time and number of observed anglers) to match the available catch information. Between April 1987 and July 1992 the number of observed anglers was not recorded for each fishing stop, but the number of anglers aboard the vessel is available. We imputed the number of observed anglers using the number of anglers aboard the vessel and the number of observed anglers at each fishing stop from the August 1992-December 1998 data (see Supplemental materials for details). In 1987, trips were only observed in Monterey, CA and were therefore excluded from the analysis (Table \ref{tab:tab-data-filter-debwv}). Sampling targeted areas of central California. Of the 2,256 trips observed, only 12 of those launched from port in District 6, which was removed from the analysis.

Each fishing location was assigned to a reef based on the on the bathymetric maps and interpretation of hard bottom was extracted from the {\tagstructbegin{tag=Link}\tagmcbegin{tag=Link}\href{http://seafloor.otterlabs.org/index.html}{California Seafloor Mapping Project}\leavevmode\tagmcend\tagstructend}. Reefs were aggregated to four regions produce adequate sample sizes; Ft. Bragg to Santa Cruz (V1), Moss Landing to Big Sur (V2), San Luis Obispo to Pt. Conception (V3), and Offshore (deeper) locations including the Farallon Islands and reefs of Half Moon Bay and Monterey Bay (V4). The ports in San Luis Obispo county were sampled more frequently than other regions and the arithmetic mean of CPUE by year was higher also higher in this area (Figure \ref{fig:fig-areacpue-debwv})

We retained 6597 drifts for index standardization, with 2016 fishing location encountering vermilion rockfish.\\
Tables of the number of samples and positive observervations by factors depth, region and year, can be found in Tables \ref{tab:tab-depth-debwv}, \ref{tab:tab-region-debwv}, and \ref{tab:tab-year-debwv}.

\textbf{Deb Wilson-Vandenberg Index: Model Selection, Fits, and Diagnostics}

A Lognormal model was over a Gamma model selected for the positive observations by a {\tagstructbegin{tag=Formula}\tagmcbegin{tag=Formula}\(\Delta AIC\)\leavevmode\tagmcend\tagstructend} of 313.12 and supported by Q-Q plots of the positive observations fit to both distributions (Figure \ref{fig:fig-dist-fits-debwv}). The delta-GLM method allows the linear predictors to differ between the binomial and positive models. Based on AIC values from maximum likelihood fits Table \ref{tab:tab-model-select-debwv}), a main effects model including YEAR and WAVE and DEPTH bin was fit for the binomial model and a main effects model including YEAR and WAVE and DEPTH bin was fit for the Lognormal model. Models were fit using the ``rstanarm'' R package (version 2.21.1). Posterior predictive checks of the Bayesian model fit for the binomial model and the positive model were all reasonable (Figures \ref{fig:fig-posterior-mean-debwv} and \ref{fig:fig-posterior-sd-debwv}). The binomial model generated data sets with the proportion zeros similar to the 69\% zeroes in the observed data (Figure \ref{fig:fig-propzero-debwv}). The predicted marginal effects from both the binomial and Lognormal models can be found in (Figures \ref{fig:fig-Dbin-marginal-debwv} and \ref{fig:fig-Dpos-marginal-debwv}). The final index (Table \ref{tab:tab-index-debwv}) represents a similar trend to the arithmetic mean of the annual CPUE (Figure \ref{fig:fig-cpue-debwv}).

\FloatBarrier

\begin{table}

\caption{\label{tab:tab-data-filter-debwv}Data filtering steps for theDebWV onboard survey index for vermilion rockfish in the northern model. The last row in the table represents the number of trips used 
      to develop the index.}
\centering
\begin{tabular}[t]{>{\raggedright\arraybackslash}p{8em}>{\raggedright\arraybackslash}p{15em}c>{\centering\arraybackslash}p{8em}>{\centering\arraybackslash}p{8em}}
\toprule
Filter & Desciption & Trip & Positive Trips & Percent drifts retained\\
\midrule
\cellcolor{gray!6}{All} & \cellcolor{gray!6}{None} & \cellcolor{gray!6}{7566} & \cellcolor{gray!6}{2593} & \cellcolor{gray!6}{34\%}\\
No catch & Remove no catch trips & 7041 & 2068 & 29\%\\
\cellcolor{gray!6}{Sparse data} & \cellcolor{gray!6}{Remove District 6 and 1987} & \cellcolor{gray!6}{6697} & \cellcolor{gray!6}{2022} & \cellcolor{gray!6}{30\%}\\
Time fished & Remove drifts fished less than 6 minutes & 6597 & 2016 & 31\%\\
\bottomrule
\end{tabular}
\end{table}

\begin{table}

\caption{\label{tab:tab-depth-debwv}Positive samples of vermilion rockfish in the northern model by depth (fm).}
\centering
\begin{tabular}[t]{lrrl}
\toprule
Year & Positive Samples & Samples & Percent Positive\\
\midrule
\cellcolor{gray!6}{(0,10]} & \cellcolor{gray!6}{113} & \cellcolor{gray!6}{478} & \cellcolor{gray!6}{24\%}\\
(10,20] & 455 & 1344 & 34\%\\
\cellcolor{gray!6}{(20,30]} & \cellcolor{gray!6}{410} & \cellcolor{gray!6}{1198} & \cellcolor{gray!6}{34\%}\\
(30,40] & 465 & 1331 & 35\%\\
\cellcolor{gray!6}{(40,50]} & \cellcolor{gray!6}{347} & \cellcolor{gray!6}{1067} & \cellcolor{gray!6}{33\%}\\
\addlinespace
(50,60] & 172 & 617 & 28\%\\
\cellcolor{gray!6}{(60,70]} & \cellcolor{gray!6}{36} & \cellcolor{gray!6}{263} & \cellcolor{gray!6}{14\%}\\
(70,118] & 18 & 299 & 6\%\\
\bottomrule
\end{tabular}
\end{table}

\FloatBarrier

\begin{table}

\caption{\label{tab:tab-region-debwv}Samples of vermilion rockfish in the northern model by subregion used in the index.}
\centering
\begin{tabular}[t]{lrrl}
\toprule
Subregion & Positive Samples & Samples & Percent Positive\\
\midrule
\cellcolor{gray!6}{V1} & \cellcolor{gray!6}{362} & \cellcolor{gray!6}{1317} & \cellcolor{gray!6}{27\%}\\
V2 & 322 & 1448 & 22\%\\
\cellcolor{gray!6}{V3} & \cellcolor{gray!6}{924} & \cellcolor{gray!6}{1668} & \cellcolor{gray!6}{55\%}\\
V4 & 408 & 2164 & 19\%\\
\bottomrule
\end{tabular}
\end{table}

\begin{table}

\caption{\label{tab:tab-year-debwv}Samples of vermilion rockfish in the northern model by year.}
\centering
\begin{tabular}[t]{lrrl}
\toprule
Year & Positive Samples & Samples & Percent Positive\\
\midrule
\cellcolor{gray!6}{1988} & \cellcolor{gray!6}{136} & \cellcolor{gray!6}{422} & \cellcolor{gray!6}{32\%}\\
1989 & 170 & 446 & 38\%\\
\cellcolor{gray!6}{1990} & \cellcolor{gray!6}{65} & \cellcolor{gray!6}{122} & \cellcolor{gray!6}{53\%}\\
1991 & 73 & 135 & 54\%\\
\cellcolor{gray!6}{1992} & \cellcolor{gray!6}{168} & \cellcolor{gray!6}{467} & \cellcolor{gray!6}{36\%}\\
\addlinespace
1993 & 196 & 485 & 40\%\\
\cellcolor{gray!6}{1994} & \cellcolor{gray!6}{189} & \cellcolor{gray!6}{555} & \cellcolor{gray!6}{34\%}\\
1995 & 247 & 791 & 31\%\\
\cellcolor{gray!6}{1996} & \cellcolor{gray!6}{238} & \cellcolor{gray!6}{963} & \cellcolor{gray!6}{25\%}\\
1997 & 323 & 1312 & 25\%\\
\addlinespace
\cellcolor{gray!6}{1998} & \cellcolor{gray!6}{211} & \cellcolor{gray!6}{899} & \cellcolor{gray!6}{23\%}\\
\bottomrule
\end{tabular}
\end{table}

\FloatBarrier

\begin{table}

\caption{\label{tab:tab-model-select-debwv}Model selection for the DebWV onboard survey index for vermilion rockfish in the northern model.}
\centering
\begin{tabular}[t]{lrr}
\toprule
Model & Binomial $\Delta$AIC & Lognormal $\Delta$AIC\\
\midrule
\cellcolor{gray!6}{1} & \cellcolor{gray!6}{1011.38} & \cellcolor{gray!6}{422.42}\\
YEAR + MegaReef & 169.08 & 52.50\\
\cellcolor{gray!6}{YEAR + MegaReef + WAVE} & \cellcolor{gray!6}{120.32} & \cellcolor{gray!6}{42.13}\\
YEAR + MegaReef + WAVE + DEPTH bin & 0.00 & 0.00\\
\cellcolor{gray!6}{YEAR + WAVE + DEPTH bin} & \cellcolor{gray!6}{611.73} & \cellcolor{gray!6}{260.44}\\
\addlinespace
YEAR + DEPTH bin & 642.50 & 272.83\\
\cellcolor{gray!6}{YEAR + MegaReef + DEPTH bin} & \cellcolor{gray!6}{55.30} & \cellcolor{gray!6}{7.28}\\
\bottomrule
\end{tabular}
\end{table}

\FloatBarrier

\begin{table}

\caption{\label{tab:tab-index-debwv}Standardized index for the DebWV onboard survey index with log-scale standard errors and 95\% highest
       posterior density (HPD) intervals for vermilion in the northern model.}
\centering
\begin{tabular}[t]{rrrrr}
\toprule
Year & Index & logSE & lower HPD & upper HPD\\
\midrule
\cellcolor{gray!6}{1988} & \cellcolor{gray!6}{0.02} & \cellcolor{gray!6}{0.22} & \cellcolor{gray!6}{0.01} & \cellcolor{gray!6}{0.03}\\
1989 & 0.03 & 0.20 & 0.02 & 0.04\\
\cellcolor{gray!6}{1990} & \cellcolor{gray!6}{0.06} & \cellcolor{gray!6}{0.23} & \cellcolor{gray!6}{0.04} & \cellcolor{gray!6}{0.10}\\
1991 & 0.03 & 0.25 & 0.02 & 0.05\\
\cellcolor{gray!6}{1992} & \cellcolor{gray!6}{0.02} & \cellcolor{gray!6}{0.20} & \cellcolor{gray!6}{0.01} & \cellcolor{gray!6}{0.03}\\
\addlinespace
1993 & 0.03 & 0.20 & 0.02 & 0.04\\
\cellcolor{gray!6}{1994} & \cellcolor{gray!6}{0.02} & \cellcolor{gray!6}{0.20} & \cellcolor{gray!6}{0.01} & \cellcolor{gray!6}{0.03}\\
1995 & 0.02 & 0.20 & 0.01 & 0.03\\
\cellcolor{gray!6}{1996} & \cellcolor{gray!6}{0.02} & \cellcolor{gray!6}{0.20} & \cellcolor{gray!6}{0.01} & \cellcolor{gray!6}{0.02}\\
1997 & 0.02 & 0.20 & 0.01 & 0.03\\
\addlinespace
\cellcolor{gray!6}{1998} & \cellcolor{gray!6}{0.02} & \cellcolor{gray!6}{0.20} & \cellcolor{gray!6}{0.01} & \cellcolor{gray!6}{0.03}\\
\bottomrule
\end{tabular}
\end{table}

\FloatBarrier

\begin{figure}
\centering
\includegraphics{C:/Stock_Assessments/VRML_Assessment_2021/GitHub/Vermilion_2021/doc/indices/vermilion_DebWV_onboard_writeup_NCA_files/figure-latex/fig-areacpue-debwv-1.pdf}
\caption{\label{fig:fig-areacpue-debwv}Arithmetic mean of CPUE by region for vermilion from the filtered data.}
\end{figure}

\begin{figure}
\centering
\includegraphics{C:/Stock_Assessments/VRML_Assessment_2021/GitHub/Vermilion_2021/doc/indices/vermilion_DebWV_onboard_writeup_NCA_files/figure-latex/fig-dist-fits-debwv-1.pdf}
\caption{\label{fig:fig-dist-fits-debwv}Q-Q plot (top) of the positive observations lognormal gamma distributions and fitted values vs residuals for the Lognormal model (bottom).}
\end{figure}

\begin{figure}
\centering
\includegraphics{C:/Stock_Assessments/VRML_Assessment_2021/GitHub/Vermilion_2021/doc/indices/vermilion_DebWV_onboard_writeup_NCA_files/figure-latex/fig-posterior-mean-debwv-1.pdf}
\caption{\label{fig:fig-posterior-mean-debwv}Posterior predictive draws of the mean (x-axis) by year in replicate data sets generated by the delta model with a vertical line representing the observed mean in the data.}
\end{figure}

\FloatBarrier

\begin{figure}
\centering
\includegraphics{C:/Stock_Assessments/VRML_Assessment_2021/GitHub/Vermilion_2021/doc/indices/vermilion_DebWV_onboard_writeup_NCA_files/figure-latex/fig-posterior-sd-debwv-1.pdf}
\caption{\label{fig:fig-posterior-sd-debwv}Posterior predictive draws of the standard deviation by year (x-axis) in replicate data sets generated by the delta model with a vertical line representing the observed standard deviation in the data.}
\end{figure}

\begin{figure}
\centering
\includegraphics{C:/Stock_Assessments/VRML_Assessment_2021/GitHub/Vermilion_2021/doc/indices/vermilion_DebWV_onboard_writeup_NCA_files/figure-latex/fig-propzero-debwv-1.pdf}
\caption{\label{fig:fig-propzero-debwv}Posterior predictive distribution of the proportion of zero observations (x-axis) in replicate data sets generated by the delta model with a vertical line representing the observed average proportion of zeros in the data.}
\end{figure}

\begin{figure}
\centering
\includegraphics{C:/Stock_Assessments/VRML_Assessment_2021/GitHub/Vermilion_2021/doc/indices/vermilion_DebWV_onboard_writeup_NCA_files/figure-latex/fig-Dbin-marginal-debwv-1.pdf}
\caption{\label{fig:fig-Dbin-marginal-debwv}Binomial marginal effects from the final model}
\end{figure}

\begin{figure}
\centering
\includegraphics{C:/Stock_Assessments/VRML_Assessment_2021/GitHub/Vermilion_2021/doc/indices/vermilion_DebWV_onboard_writeup_NCA_files/figure-latex/fig-Dpos-marginal-debwv-1.pdf}
\caption{\label{fig:fig-Dpos-marginal-debwv}Positive model marginal effects from the final model.}
\end{figure}

\begin{figure}
\centering
\includegraphics{C:/Stock_Assessments/VRML_Assessment_2021/GitHub/Vermilion_2021/doc/indices/vermilion_DebWV_onboard_writeup_NCA_files/figure-latex/fig-cpue-debwv-1.pdf}
\caption{\label{fig:fig-cpue-debwv}Standardized index and arithmetic mean of the CPUE from the filtered data. Each timeseries is scaled to its respective means.}
\end{figure}

\newpage

\tagstructbegin{tag=H2}\tagmcbegin{tag=H2}

\hypertarget{pr-index}{%
\subsection{CRFS PR Dockside Index of Abundance}\label{pr-index}}

\leavevmode\tagmcend\tagstructend

\renewcommand{\thepage}{E\arabic{page}}
\renewcommand{\thefigure}{E\arabic{figure}}
\renewcommand{\thetable}{E\arabic{table}}
\setcounter{page}{1}
\setcounter{figure}{0}
\setcounter{table}{0}

\textbf{CRFS Dockside Private Boat Index}

Catch and effort data from CRFS dockside sampling of private boats, 2004-2018, were provided by CDFW for use in this assessment. The data include catch (number of fish) by species, number of anglers (i.e.~effort units are angler trips), angler-reported distance from shore (Area X: inside/outside of 3 nm), county, port, interview site, year, month, and CRFS district. The sample size of the unfiltered private boat CPUE data is much larger than the crfspr CPFV data set, with 391,279 trips statewide, 120,655 in southern California (south of Point Conception), and 270,064 north of Point Conception.

\emph{CRFS Private Boat Index: Data Preparation, Filtering, and Sample Sizes} Records were limited to ``PR1'' sites, and only the hook-and-line gear type (Table \ref{tab:tab-data-filter-crfspr}). Since this is a dockside index lacking precise fishing location information, we use the percent of groundfish within the catch from a trip as a proxy for retaining trips for index standardization. Similar to the CRFSS onboard index, we partitioned the data into areas north and south of Point Conception and applied the method separately to each data set.

Since 2005, the recreational fishery for shelf rockfish north of Point Conception has been closed from January through part of April and May.Angler reported distance from shore had no samples in the ``outside 3 nm'' category (Area X = 2) from 2004-2011, but was retained in the index standardization due to the relaxation of depth restrictions beginning in 2017. We retained 57647 drifts for index standardization, with 21464 drifts encountering vermilion (Table \ref{tab:tab-data-filter-crfspr}).

\emph{Northern California CRFS Private Boat Index: Model Selection, Fits, and Diagnostics}

Sample sizes by factors selected to model, excluding WAVE can be found in Tables \ref{tab:tab-region-crfspr} and \ref{tab:tab-year-crfspr}. We modeled retained catch per angler hour (CPUE; number of fish per angler hour) a Bayesian delta-GLM model. Indices with a year and area interaction were not considered in model selection; trends in the average CPUE by region were similar in the filtered data set (Figure \ref{fig:fig-areacpue-crfspr}).

A Lognormal model was selected for the positive observation GLM by a {\tagstructbegin{tag=Formula}\tagmcbegin{tag=Formula}\(\Delta AIC\)\leavevmode\tagmcend\tagstructend} of 3457.72 over a Gamma model and supported by Q-Q plots of the positive observations fit to both distributions (Figure \ref{fig:fig-dist-fits-crfspr}). The delta-GLM method allows the linear predictors to differ between the binomial and positive models. Based on AIC values from maximum likelihood fits Table \ref{tab:tab-model-select-crfspr}), a main effects model including YEAR and DISTRICT and WAVE and AREA X was fit for the binomial model and a main effects model including YEAR and DISTRICT and WAVE and AREA X was fit for the Lognormal model. Models were fit using the ``rstanarm'' R package (version 2.21.1). Posterior predictive checks of the Bayesian model fit for the binomial model and the positive model were all reasonable (Figures \ref{fig:fig-posterior-mean-crfspr} and \ref{fig:fig-posterior-sd-crfspr}). The binomial model generated data sets with the proportion zeros similar to the 63\% zeroes in the observed data (Figure \ref{fig:fig-propzero-crfspr}). The predicted marginal effects from both the binomial and Lognormal models can be found in (Figures \ref{fig:fig-Dbin-marginal-crfspr} and \ref{fig:fig-Dpos-marginal-crfspr}). The final index (Table \ref{tab:tab-index-crfspr}) represents a similar trend to the arithmetic mean of the annual CPUE (Figure \ref{fig:fig-cpue-crfspr}).

\FloatBarrier

\begin{table}

\caption{\label{tab:tab-data-filter-crfspr}Data filtering steps for theCRFS PR dockside survey index for vermilion rockfish in the northern model. The last row in the table represents the number of trips used 
      to develop the index.}
\centering
\begin{tabular}[t]{>{\raggedright\arraybackslash}p{8em}>{\raggedright\arraybackslash}p{15em}c>{\centering\arraybackslash}p{8em}>{\centering\arraybackslash}p{8em}}
\toprule
Filter & Desciption & Trip & Positive Trips & Percent drifts retained\\
\midrule
\cellcolor{gray!6}{All data} & \cellcolor{gray!6}{Pre-filtered for drifts with marked for exclusion} & \cellcolor{gray!6}{78855} & \cellcolor{gray!6}{24932} & \cellcolor{gray!6}{32\%}\\
Year 2020 & Remove 2020 due to decreased sampling. & 77109 & 24404 & 32\%\\
\cellcolor{gray!6}{Months samples} & \cellcolor{gray!6}{Remove waves less than 2 due to small sample sizes and fishery closures.} & \cellcolor{gray!6}{76979} & \cellcolor{gray!6}{24344} & \cellcolor{gray!6}{32\%}\\
Groundfish & Removed trips with no observed groundfish & 66621 & 24344 & 37\%\\
\cellcolor{gray!6}{HMS} & \cellcolor{gray!6}{Remove trips with more than half the catch composed of HMS species} & \cellcolor{gray!6}{66609} & \cellcolor{gray!6}{24341} & \cellcolor{gray!6}{37\%}\\
\addlinespace
Final trips & Retained trips with at least 0.95 groundfish. & 57647 & 21464 & 37\%\\
\bottomrule
\end{tabular}
\end{table}

\begin{table}

\caption{\label{tab:tab-region-crfspr}Samples of vermilion rockfish in the northern model by subregion used in the index.}
\centering
\begin{tabular}[t]{lrrl}
\toprule
Subregion & Positive Samples & Samples & Percent Positive\\
\midrule
\cellcolor{gray!6}{3} & \cellcolor{gray!6}{12234} & \cellcolor{gray!6}{24086} & \cellcolor{gray!6}{51\%}\\
4 & 4504 & 11933 & 38\%\\
\cellcolor{gray!6}{5} & \cellcolor{gray!6}{1706} & \cellcolor{gray!6}{4527} & \cellcolor{gray!6}{38\%}\\
6 & 3020 & 17101 & 18\%\\
\bottomrule
\end{tabular}
\end{table}

\begin{table}

\caption{\label{tab:tab-year-crfspr}Samples of vermilion rockfish in the northern model by year.}
\centering
\begin{tabular}[t]{lrrl}
\toprule
Year & Positive Samples & Samples & Percent Positive\\
\midrule
\cellcolor{gray!6}{2004} & \cellcolor{gray!6}{1076} & \cellcolor{gray!6}{2487} & \cellcolor{gray!6}{43\%}\\
2005 & 1433 & 3568 & 40\%\\
\cellcolor{gray!6}{2006} & \cellcolor{gray!6}{1934} & \cellcolor{gray!6}{4508} & \cellcolor{gray!6}{43\%}\\
2007 & 1342 & 3328 & 40\%\\
\cellcolor{gray!6}{2008} & \cellcolor{gray!6}{1023} & \cellcolor{gray!6}{3414} & \cellcolor{gray!6}{30\%}\\
\addlinespace
2009 & 1004 & 3722 & 27\%\\
\cellcolor{gray!6}{2010} & \cellcolor{gray!6}{883} & \cellcolor{gray!6}{2442} & \cellcolor{gray!6}{36\%}\\
2011 & 1037 & 2831 & 37\%\\
\cellcolor{gray!6}{2012} & \cellcolor{gray!6}{920} & \cellcolor{gray!6}{2785} & \cellcolor{gray!6}{33\%}\\
2013 & 1134 & 3380 & 34\%\\
\addlinespace
\cellcolor{gray!6}{2014} & \cellcolor{gray!6}{1271} & \cellcolor{gray!6}{4065} & \cellcolor{gray!6}{31\%}\\
2015 & 1802 & 4924 & 37\%\\
\cellcolor{gray!6}{2016} & \cellcolor{gray!6}{1658} & \cellcolor{gray!6}{4357} & \cellcolor{gray!6}{38\%}\\
2017 & 1567 & 4122 & 38\%\\
\cellcolor{gray!6}{2018} & \cellcolor{gray!6}{1638} & \cellcolor{gray!6}{3954} & \cellcolor{gray!6}{41\%}\\
\addlinespace
2019 & 1742 & 3760 & 46\%\\
\bottomrule
\end{tabular}
\end{table}

\FloatBarrier

\begin{table}

\caption{\label{tab:tab-model-select-crfspr}Model selection for the CRFS PR dockside survey index for vermilion rockfish in the northern model.}
\centering
\begin{tabular}[t]{lrr}
\toprule
Model & Binomial $\Delta$AIC & Lognormal $\Delta$AIC\\
\midrule
\cellcolor{gray!6}{1} & \cellcolor{gray!6}{6137.96} & \cellcolor{gray!6}{1832.84}\\
YEAR + DISTRICT & 469.50 & 83.21\\
\cellcolor{gray!6}{YEAR + DISTRICT + WAVE} & \cellcolor{gray!6}{425.71} & \cellcolor{gray!6}{34.01}\\
YEAR + DISTRICT + WAVE + AREA X & 0.00 & 0.00\\
\cellcolor{gray!6}{YEAR + WAVE} & \cellcolor{gray!6}{5198.73} & \cellcolor{gray!6}{1446.58}\\
\addlinespace
YEAR + AREA X & 5353.08 & 1527.86\\
\cellcolor{gray!6}{YEAR + WAVE + AREA X} & \cellcolor{gray!6}{5024.99} & \cellcolor{gray!6}{1440.38}\\
YEAR + DISTRICT + AREA X & 42.53 & 47.71\\
\bottomrule
\end{tabular}
\end{table}

\FloatBarrier

\begin{table}

\caption{\label{tab:tab-index-crfspr}Standardized index for the CRFS PR dockside survey index with log-scale standard errors and 95\% highest
       posterior density (HPD) intervals for vermilion in the northern model.}
\centering
\begin{tabular}[t]{rrrrr}
\toprule
Year & Index & logSE & lower HPD & upper HPD\\
\midrule
\cellcolor{gray!6}{2004} & \cellcolor{gray!6}{0.80} & \cellcolor{gray!6}{0.05} & \cellcolor{gray!6}{0.72} & \cellcolor{gray!6}{0.87}\\
2005 & 0.80 & 0.05 & 0.73 & 0.88\\
\cellcolor{gray!6}{2006} & \cellcolor{gray!6}{0.86} & \cellcolor{gray!6}{0.04} & \cellcolor{gray!6}{0.78} & \cellcolor{gray!6}{0.93}\\
2007 & 0.81 & 0.05 & 0.73 & 0.88\\
\cellcolor{gray!6}{2008} & \cellcolor{gray!6}{0.58} & \cellcolor{gray!6}{0.05} & \cellcolor{gray!6}{0.52} & \cellcolor{gray!6}{0.65}\\
\addlinespace
2009 & 0.51 & 0.05 & 0.46 & 0.56\\
\cellcolor{gray!6}{2010} & \cellcolor{gray!6}{0.62} & \cellcolor{gray!6}{0.05} & \cellcolor{gray!6}{0.55} & \cellcolor{gray!6}{0.68}\\
2011 & 0.63 & 0.05 & 0.57 & 0.70\\
\cellcolor{gray!6}{2012} & \cellcolor{gray!6}{0.52} & \cellcolor{gray!6}{0.05} & \cellcolor{gray!6}{0.47} & \cellcolor{gray!6}{0.58}\\
2013 & 0.44 & 0.05 & 0.40 & 0.49\\
\addlinespace
\cellcolor{gray!6}{2014} & \cellcolor{gray!6}{0.49} & \cellcolor{gray!6}{0.05} & \cellcolor{gray!6}{0.44} & \cellcolor{gray!6}{0.54}\\
2015 & 0.54 & 0.05 & 0.49 & 0.58\\
\cellcolor{gray!6}{2016} & \cellcolor{gray!6}{0.57} & \cellcolor{gray!6}{0.05} & \cellcolor{gray!6}{0.52} & \cellcolor{gray!6}{0.62}\\
2017 & 0.53 & 0.05 & 0.48 & 0.58\\
\cellcolor{gray!6}{2018} & \cellcolor{gray!6}{0.63} & \cellcolor{gray!6}{0.04} & \cellcolor{gray!6}{0.58} & \cellcolor{gray!6}{0.69}\\
\addlinespace
2019 & 0.77 & 0.04 & 0.71 & 0.84\\
\bottomrule
\end{tabular}
\end{table}

\FloatBarrier

\begin{figure}
\centering
\includegraphics{C:/Stock_Assessments/VRML_Assessment_2021/GitHub/Vermilion_2021/doc/indices/vermilion_CRFS_PR_dockside_writeup_NCA_files/figure-latex/fig-dist-fits-crfspr-1.pdf}
\caption{\label{fig:fig-dist-fits-crfspr}Q-Q plot (top) of the positive observations lognormal gamma distributions and fitted values vs residuals for the Lognormal model (bottom).}
\end{figure}

\begin{figure}
\centering
\includegraphics{C:/Stock_Assessments/VRML_Assessment_2021/GitHub/Vermilion_2021/doc/indices/vermilion_CRFS_PR_dockside_writeup_NCA_files/figure-latex/fig-areacpue-crfspr-1.pdf}
\caption{\label{fig:fig-areacpue-crfspr}Arithmetic mean of CPUE by region for vermilion from the filtered data.}
\end{figure}

\begin{figure}
\centering
\includegraphics{C:/Stock_Assessments/VRML_Assessment_2021/GitHub/Vermilion_2021/doc/indices/vermilion_CRFS_PR_dockside_writeup_NCA_files/figure-latex/fig-posterior-mean-crfspr-1.pdf}
\caption{\label{fig:fig-posterior-mean-crfspr}Posterior predictive draws of the mean (x-axis) by year in replicate data sets generated by the delta model with a vertical line representing the observed mean in the data.}
\end{figure}

\FloatBarrier

\begin{figure}
\centering
\includegraphics{C:/Stock_Assessments/VRML_Assessment_2021/GitHub/Vermilion_2021/doc/indices/vermilion_CRFS_PR_dockside_writeup_NCA_files/figure-latex/fig-posterior-sd-crfspr-1.pdf}
\caption{\label{fig:fig-posterior-sd-crfspr}Posterior predictive draws of the standard deviation by year (x-axis) in replicate data sets generated by the delta model with a vertical line representing the observed standard deviation in the data.}
\end{figure}

\begin{figure}
\centering
\includegraphics{C:/Stock_Assessments/VRML_Assessment_2021/GitHub/Vermilion_2021/doc/indices/vermilion_CRFS_PR_dockside_writeup_NCA_files/figure-latex/fig-propzero-crfspr-1.pdf}
\caption{\label{fig:fig-propzero-crfspr}Posterior predictive distribution of the proportion of zero observations (x-axis) in replicate data sets generated by the delta model with a vertical line representing the observed average proportion of zeros in the data.}
\end{figure}

\begin{figure}
\centering
\includegraphics{C:/Stock_Assessments/VRML_Assessment_2021/GitHub/Vermilion_2021/doc/indices/vermilion_CRFS_PR_dockside_writeup_NCA_files/figure-latex/fig-Dbin-marginal-crfspr-1.pdf}
\caption{\label{fig:fig-Dbin-marginal-crfspr}Binomial marginal effects from the final model.}
\end{figure}

\begin{figure}
\centering
\includegraphics{C:/Stock_Assessments/VRML_Assessment_2021/GitHub/Vermilion_2021/doc/indices/vermilion_CRFS_PR_dockside_writeup_NCA_files/figure-latex/fig-Dpos-marginal-crfspr-1.pdf}
\caption{\label{fig:fig-Dpos-marginal-crfspr}Positive model marginal effects from the final model.}
\end{figure}

\begin{figure}
\centering
\includegraphics{C:/Stock_Assessments/VRML_Assessment_2021/GitHub/Vermilion_2021/doc/indices/vermilion_CRFS_PR_dockside_writeup_NCA_files/figure-latex/fig-cpue-crfspr-1.pdf}
\caption{\label{fig:fig-cpue-crfspr}Standardized index and arithmetic mean of the CPUE from the filtered data. Each timeseries is scaled to its respective means.}
\end{figure}

\newpage

\tagstructbegin{tag=H2}\tagmcbegin{tag=H2}

\hypertarget{ccfrp-index}{%
\subsection{CCFRP Index of Abundance}\label{ccfrp-index}}

\leavevmode\tagmcend\tagstructend

\renewcommand{\thepage}{F\arabic{page}}
\renewcommand{\thefigure}{F\arabic{figure}}
\renewcommand{\thetable}{F\arabic{table}}
\setcounter{page}{1}
\setcounter{figure}{0}
\setcounter{table}{0}

\textbf{California Collaborative Fisheries Research Program Index}

The California Collaborative Fisheries Research Program, {\tagstructbegin{tag=Link}\tagmcbegin{tag=Link}\href{https://www.mlml.calstate.edu/ccfrp/}{CCFRP}\leavevmode\tagmcend\tagstructend}, is a fishery-independent hook-and-line survey designed to monitor nearshore fish populations at a series of sampling locations both inside and adjacent to MPAs along the central California coast {\tagstructbegin{tag=Reference}\tagmcbegin{tag=Reference}(Wendt and Starr 2009, Starr et al. 2015)\leavevmode\tagmcend\tagstructend}. The CCFRP survey began in 2007 and was originally designed as a statewide program in collaboration with NMFS scientists and fishermen. From 2007-2016 the CCFRP project was focused on the central California coast, and has monitored four MPAs consistently. In 2017, the program was expanded coastwide within California. The index of abundance was developed from the four MPAs sampled consistently (Año Nuevo and Point Lobos by Moss Landing Marine Labs; Point Buchon and Piedras Blancas by Cal Poly).

The survey design for CCFRP consists a number 500 x 500 m cells both within and outside each MPA. On any given survey day site cells are randomly selected within a stratum (MPA and/or reference cells). CPFVs are chartered for the survey and the fishing captain is allowed to search within the cell for a fishing location. During a sampling event, each cell is fished for a total of 30-45 minutes by volunteer anglers. Each fish encountered is recorded, measured, and can be linked back to a particular angler, and released (or descended to depth). Starting in 2017, a subset of fish have been retained to collect otoliths and fin clips that provide needed biological information for nearshore species. For the index of abundance, CPUE was modeled at the level of the drift, similar to the fishery-dependent onboard observer survey described above.

\emph{CCFRP Index: Data Preparation, Filtering, and Sample Sizes}

The CCFRP data are quality controlled at the time they are key punched and little filtering was needed for the index. Cells not consistently sampled over time were excluded as well as cells that never encountered vermilion rockfish. CCFRP samples shallower depths to avoid barotrauma-induced mortality. We retained 5444 drifts for index standardization, with 1927 drifts encountering vermilion rockfish.

\emph{CCFRP Index: Model Selection, Fits, and Diagnostics}

Sample sizes by factors selected to model, excluding WAVE can be found in Tables \ref{tab:tab-region-ccfrp} and \ref{tab:tab-year-ccfrp}. We modeled retained catch per angler hour (CPUE; number of fish per angler hour) a Bayesian delta-GLM model. Indices with a year and area interaction were not considered in model selection; trends in the average CPUE by region were similar in the filtered data set (Figure \ref{fig:fig-areacpue-ccfrp}). Plots of the arithmetic mean by inside (MPA) and outside (REF) MPAs over time is in Figure \ref{fig:fig-sitecpue-ccfrp}.

A negative binomial model was fit to the drift-level data (catch with a log offset for angler hours). Because the averaged observed CPUE inside MPAs and in the reference sites exhibited differing trends, we explored a YEAR:SITE interaction, which was selected as the best fit model by AIC Table \ref{tab:tab-model-select-ccfrp}), The final model included YEAR and AREA and SITE and DEPTH\_bin and YEAR:SITE and offset(logEffort). The model was fit using the ``rstanarm'' R package (version 2.21.1). Posterior predictive checks of the Bayesian model fit for the binomial model and the positive model were all reasonable (Figures \ref{fig:fig-posterior-mean-ccfrp} and \ref{fig:fig-posterior-sd-ccfrp}). The negative binomial model generated data sets with the proportion zeros similar to the 65\% zeroes in the observed data (Figure \ref{fig:fig-propzero-ccfrp}). The predicted marginal effects from the model can be found in (Figures \ref{fig:fig-Dnbin-marginal-ccfrp}).

Based on work completed at the SWFSC, we estimate that the percent of rocky reef habitat from Point Conception to the California border within California state waters is 892 {\tagstructbegin{tag=Formula}\tagmcbegin{tag=Formula}\(km^2\)\leavevmode\tagmcend\tagstructend}, of which approximately 23\% is in MPAs that prohibit the harvest of groundfish (pers comm. Rebecca Miller, UCSC). There is recreational fishing outside of state waters, but habitat maps are not available at the same 2-m resolution and do not allow for direct comparisons. High-resolution habitat maps are not available for the state waters south of Point Conception.

The final index was weighted, giving 20\% of the model weight to MPAs and 80\% of model weight to the ``open'' areas within the state. The CCFRP index includes all of the MPAs currently sampled from 2017-2020 and the core central California sampling sites from 2007-2016. Trends among all of the MPAs sampled increased along the entire coast from 2017-2020. The final index (Table \ref{tab:tab-index-ccfrp}) represents a similar trend to the arithmetic mean of the annual CPUE (Figure \ref{fig:fig-cpue-ccfrp}).

To visualize the affect of weighting on the index, Figure (\ref{fig:fig-weighted-cpue-ccfrp}) shows the unweighted index and the index with 10-60\% of the weight given to MPAs versus open areas. Each of these indices are scaled to their means to allow for direct comparison.

\newpage

\begin{table}

\caption{\label{tab:tab-region-ccfrp}Samples of vermilion rockfish in the northern model by subregion used in the index.}
\centering
\begin{tabular}[t]{lrrl}
\toprule
Subregion & Positive Samples & Samples & Percent Positive\\
\midrule
\cellcolor{gray!6}{South Cape Mendocino} & \cellcolor{gray!6}{474} & \cellcolor{gray!6}{1854} & \cellcolor{gray!6}{26\%}\\
Ten Mile & 364 & 1343 & 27\%\\
\cellcolor{gray!6}{Stewarts Point} & \cellcolor{gray!6}{599} & \cellcolor{gray!6}{932} & \cellcolor{gray!6}{64\%}\\
Bodega Head & 490 & 1315 & 37\%\\
\bottomrule
\end{tabular}
\end{table}

\begin{table}

\caption{\label{tab:tab-depth-ccfrp}Positive samples of vermilion rockfish in the northern model by depth (fm).}
\centering
\begin{tabular}[t]{lrrl}
\toprule
Year & Positive Samples & Samples & Percent Positive\\
\midrule
\cellcolor{gray!6}{(0,10]} & \cellcolor{gray!6}{356} & \cellcolor{gray!6}{1589} & \cellcolor{gray!6}{22\%}\\
(10,15] & 925 & 2438 & 38\%\\
\cellcolor{gray!6}{(15,20]} & \cellcolor{gray!6}{646} & \cellcolor{gray!6}{1417} & \cellcolor{gray!6}{46\%}\\
\bottomrule
\end{tabular}
\end{table}

\begin{table}

\caption{\label{tab:tab-year-ccfrp}Samples of vermilion rockfish in the northern model by year.}
\centering
\begin{tabular}[t]{lrrl}
\toprule
Year & Positive Samples & Samples & Percent Positive\\
\midrule
\cellcolor{gray!6}{2007} & \cellcolor{gray!6}{92} & \cellcolor{gray!6}{539} & \cellcolor{gray!6}{17\%}\\
2008 & 123 & 563 & 22\%\\
\cellcolor{gray!6}{2009} & \cellcolor{gray!6}{113} & \cellcolor{gray!6}{366} & \cellcolor{gray!6}{31\%}\\
2010 & 163 & 416 & 39\%\\
\cellcolor{gray!6}{2011} & \cellcolor{gray!6}{139} & \cellcolor{gray!6}{366} & \cellcolor{gray!6}{38\%}\\
\addlinespace
2012 & 161 & 394 & 41\%\\
\cellcolor{gray!6}{2013} & \cellcolor{gray!6}{109} & \cellcolor{gray!6}{426} & \cellcolor{gray!6}{26\%}\\
2014 & 161 & 448 & 36\%\\
\cellcolor{gray!6}{2015} & \cellcolor{gray!6}{98} & \cellcolor{gray!6}{224} & \cellcolor{gray!6}{44\%}\\
2016 & 168 & 411 & 41\%\\
\addlinespace
\cellcolor{gray!6}{2017} & \cellcolor{gray!6}{155} & \cellcolor{gray!6}{366} & \cellcolor{gray!6}{42\%}\\
2018 & 148 & 301 & 49\%\\
\cellcolor{gray!6}{2019} & \cellcolor{gray!6}{164} & \cellcolor{gray!6}{306} & \cellcolor{gray!6}{54\%}\\
2020 & 133 & 318 & 42\%\\
\bottomrule
\end{tabular}
\end{table}

\FloatBarrier

\begin{table}

\caption{\label{tab:tab-model-select-ccfrp}Model selection for the CCFRP survey index for vermilion rockfish in the northern model.}
\centering
\begin{tabular}[t]{lr}
\toprule
Model & $\Delta$AIC\\
\midrule
\cellcolor{gray!6}{1 + offset(logEffort)} & \cellcolor{gray!6}{1191.26}\\
YEAR + AREA + offset(logEffort) & 653.34\\
\cellcolor{gray!6}{YEAR + AREA + SITE + offset(logEffort)} & \cellcolor{gray!6}{188.88}\\
YEAR + AREA + SITE + DEPTH bin + offset(logEffort) & 62.28\\
\cellcolor{gray!6}{YEAR + SITE + offset(logEffort)} & \cellcolor{gray!6}{579.86}\\
\addlinespace
YEAR + DEPTH bin + offset(logEffort) & 760.91\\
\cellcolor{gray!6}{YEAR + SITE + DEPTH bin + offset(logEffort)} & \cellcolor{gray!6}{397.53}\\
YEAR + AREA + DEPTH bin + offset(logEffort) & 478.98\\
\cellcolor{gray!6}{YEAR + AREA + SITE + DEPTH bin + YEAR:SITE + offset(logEffort)} & \cellcolor{gray!6}{0.00}\\
\bottomrule
\end{tabular}
\end{table}

\FloatBarrier

\begin{table}

\caption{\label{tab:tab-index-ccfrp}Standardized index for the CCFRP survey index with log-scale standard errors and 95\% highest
       posterior density (HPD) intervals for vermilion in the northern model.}
\centering
\begin{tabular}[t]{rrrrr}
\toprule
Year & Index & logSE & lower HPD & upper HPD\\
\midrule
\cellcolor{gray!6}{2007} & \cellcolor{gray!6}{0.11} & \cellcolor{gray!6}{0.13} & \cellcolor{gray!6}{0.09} & \cellcolor{gray!6}{0.14}\\
2008 & 0.10 & 0.11 & 0.08 & 0.12\\
\cellcolor{gray!6}{2009} & \cellcolor{gray!6}{0.16} & \cellcolor{gray!6}{0.11} & \cellcolor{gray!6}{0.13} & \cellcolor{gray!6}{0.19}\\
2010 & 0.19 & 0.10 & 0.15 & 0.23\\
\cellcolor{gray!6}{2011} & \cellcolor{gray!6}{0.16} & \cellcolor{gray!6}{0.10} & \cellcolor{gray!6}{0.13} & \cellcolor{gray!6}{0.19}\\
\addlinespace
2012 & 0.16 & 0.10 & 0.13 & 0.19\\
\cellcolor{gray!6}{2013} & \cellcolor{gray!6}{0.08} & \cellcolor{gray!6}{0.12} & \cellcolor{gray!6}{0.06} & \cellcolor{gray!6}{0.10}\\
2014 & 0.14 & 0.10 & 0.11 & 0.17\\
\cellcolor{gray!6}{2015} & \cellcolor{gray!6}{0.19} & \cellcolor{gray!6}{0.13} & \cellcolor{gray!6}{0.14} & \cellcolor{gray!6}{0.24}\\
2016 & 0.14 & 0.10 & 0.11 & 0.17\\
\addlinespace
\cellcolor{gray!6}{2017} & \cellcolor{gray!6}{0.14} & \cellcolor{gray!6}{0.10} & \cellcolor{gray!6}{0.11} & \cellcolor{gray!6}{0.17}\\
2018 & 0.18 & 0.10 & 0.15 & 0.22\\
\cellcolor{gray!6}{2019} & \cellcolor{gray!6}{0.22} & \cellcolor{gray!6}{0.10} & \cellcolor{gray!6}{0.18} & \cellcolor{gray!6}{0.26}\\
2020 & 0.24 & 0.11 & 0.19 & 0.29\\
\bottomrule
\end{tabular}
\end{table}

\FloatBarrier

\begin{figure}
\centering
\includegraphics{C:/Stock_Assessments/VRML_Assessment_2021/GitHub/Vermilion_2021/doc/indices/vermilion_CCFRP_writeup_NCA_files/figure-latex/fig-areacpue-ccfrp-1.pdf}
\caption{\label{fig:fig-areacpue-ccfrp}Arithmetic mean of CPUE by region for vermilion from the filtered data. The areas used are in the text.}
\end{figure}

\begin{figure}
\centering
\includegraphics{C:/Stock_Assessments/VRML_Assessment_2021/GitHub/Vermilion_2021/doc/indices/vermilion_CCFRP_writeup_NCA_files/figure-latex/fig-sitecpue-ccfrp-1.pdf}
\caption{\label{fig:fig-sitecpue-ccfrp}Arithmetic mean of CPUE by inside/outside MPAs for vermilion from the filtered data. The areas used are in the text.}
\end{figure}

\begin{figure}
\centering
\includegraphics{C:/Stock_Assessments/VRML_Assessment_2021/GitHub/Vermilion_2021/doc/indices/vermilion_CCFRP_writeup_NCA_files/figure-latex/fig-propzero-ccfrp-1.pdf}
\caption{\label{fig:fig-propzero-ccfrp}Posterior predictive distribution of the proportion of zero observations (x-axis) in replicate data sets generated by the delta model with a vertical line representing the observed average in the data.}
\end{figure}

\begin{figure}
\centering
\includegraphics{C:/Stock_Assessments/VRML_Assessment_2021/GitHub/Vermilion_2021/doc/indices/vermilion_CCFRP_writeup_NCA_files/figure-latex/fig-posterior-mean-ccfrp-1.pdf}
\caption{\label{fig:fig-posterior-mean-ccfrp}Posterior predictive draws of the mean (x-axis) by year in replicate data sets generated by the negative binomial model with a vertical line representing the observed mean in the data.}
\end{figure}

\begin{figure}
\centering
\includegraphics{C:/Stock_Assessments/VRML_Assessment_2021/GitHub/Vermilion_2021/doc/indices/vermilion_CCFRP_writeup_NCA_files/figure-latex/fig-posterior-sd-ccfrp-1.pdf}
\caption{\label{fig:fig-posterior-sd-ccfrp}Posterior predictive draws of the standard deviation (x-axis) in replicate data sets generated by the negative binomial model with a vertical line representing the observed mean in the data.}
\end{figure}

\begin{figure}
\centering
\includegraphics{C:/Stock_Assessments/VRML_Assessment_2021/GitHub/Vermilion_2021/doc/indices/vermilion_CCFRP_writeup_NCA_files/figure-latex/fig-Dnbin-marginal-ccfrp-1.pdf}
\caption{\label{fig:fig-Dnbin-marginal-ccfrp}Negative ninomial marginal effects from the unweighted model.}
\end{figure}

\begin{figure}
\centering
\includegraphics{C:/Stock_Assessments/VRML_Assessment_2021/GitHub/Vermilion_2021/doc/indices/vermilion_CCFRP_writeup_NCA_files/figure-latex/fig-cpue-ccfrp-1.pdf}
\caption{\label{fig:fig-cpue-ccfrp}Standardized index and arithmetic mean of the CPUE from the filtered data. Each timeseries is scaled to its respective means.}
\end{figure}

\begin{figure}
\centering
\includegraphics{C:/Stock_Assessments/VRML_Assessment_2021/GitHub/Vermilion_2021/doc/indices/vermilion_CCFRP_writeup_NCA_files/figure-latex/fig-weighted-cpue-ccfrp-1.pdf}
\caption{\label{fig:fig-weighted-cpue-ccfrp}Standardized index with differing weighting to the MPAs from 10\% to 60\%. Each index is scaled to its respective means.}
\end{figure}

\newpage

\tagstructbegin{tag=H2}\tagmcbegin{tag=H2}

\hypertarget{wcgbts-index}{%
\subsection{WCGBTS Index of Abundance}\label{wcgbts-index}}

\leavevmode\tagmcend\tagstructend

\renewcommand{\thepage}{G\arabic{page}}
\renewcommand{\thefigure}{G\arabic{figure}}
\renewcommand{\thetable}{G\arabic{table}}
\setcounter{page}{1}
\setcounter{figure}{0}
\setcounter{table}{0}

\tagstructbegin{tag=H3}\tagmcbegin{tag=H3}

\hypertarget{northwest-fisheries-science-center-west-coast-groundfish-bottom-trawl-survey}{%
\subsubsection{Northwest Fisheries Science Center West Coast Groundfish Bottom Trawl Survey}\label{northwest-fisheries-science-center-west-coast-groundfish-bottom-trawl-survey}}

\leavevmode\tagmcend\tagstructend

In 2003, the NWFSC expanded the ongoing slope survey to include the continental shelf. This survey, referred to in this document as the West Coast Groundfish Bottom Trawl Survey (WCGBT Survey or WCGBTS), is conducted annually. It uses a r andom-grid design covering the coastal waters from a depth of 55 m to 1,280 m from late-May to early-October {\tagstructbegin{tag=Reference}\tagmcbegin{tag=Reference}(Keller et al. 2017)\leavevmode\tagmcend\tagstructend}. Four chartered industry vessels are used in most years.

*\textbf{WCGBTS Index: Data Preparation, Filtering, and Sample Sizes}

Vermilion rockfish were found during the WCGBTS, mainly off the coast of California. Haul-level information collected during the survey was extracted from the {\tagstructbegin{tag=Link}\tagmcbegin{tag=Link}\href{https://www.webapps.nwfsc.noaa.gov/data}{Northwest Fisheries Science Center database}\leavevmode\tagmcend\tagstructend} using code within the \texttt{nwfscSurvey} package, providing information on catches (kg), vessel, year, latitude (decimal degrees), and area swept (hectares).

Just two records with positive tows were located north of the California-Oregon border and were excluded from this analysis. Most of the positive tows were found in waters less than 200 m depth (Table @ref\{tab:ndepth\}), and thus, this analysis was truncated to waters with a depth of 300 m or less. Positive tows were found south of 32.45 decimal degrees, which was used to represent the California-Mexico border. This left, fifty-eight positive tows north of 34.50 decimal degrees and one hundred twenty-three positive tows south of 34.50 decimal degrees. Positive encounters were just 7 and 15 percent of all tows for these two areas, respectively.

\emph{WCGBTS Index: Model Selection, Fits, and Diagnostics}

Sample sizes by factors selected to model, excluding WAVE can be found in Tables \ref{tab:tab-depth-wcgbts} and \ref{tab:tab-year-wcgbts}. We modeled retained catch per angler hour (CPUE; number of fish per angler hour) a Bayesian delta-GLM model.

A Gamma distribution was selected over a Lognormal for the positive observation GLM. The delta-GLM method allows the linear predictors to differ between the binomial and positive models. Based on AIC values from maximum likelihood fits Table \ref{tab:tab-model-select-wcgbts}), a main effects model including YEAR and DEPTH bin and LAT bin was fit for the binomial model and a main effects model including YEAR and PASS and DEPTH bin and LAT bin was fit for the Gamma model. Models were fit using the ``rstanarm'' R package (version 2.21.1). Posterior predictive checks of the Bayesian model fit for the binomial model and the positive model were all reasonable (Figures \ref{fig:fig-posterior-mean-wcgbts} and \ref{fig:fig-posterior-sd-wcgbts}). The binomial model generated data sets with the proportion zeros similar to the 92\% zeroes in the observed data (Figure \ref{fig:fig-propzero-wcgbts}). The predicted marginal effects from both the binomial and Gamma models can be found in (Figures \ref{fig:fig-Dbin-marginal-wcgbts} and \ref{fig:fig-Dpos-marginal-wcgbts}). The final index (Table \ref{tab:tab-index-wcgbts}) represents a similar trend to the arithmetic mean of the annual CPUE (Figure \ref{fig:fig-cpue-wcgbts}).

\newpage

\begin{table}

\caption{\label{tab:tab-region-wcgbts}Samples of vermilion rockfish in the northern model by subregion used in the index.}
\centering
\begin{tabular}[t]{lrrl}
\toprule
Subregion & Positive Samples & Samples & Percent Positive\\
\midrule
\cellcolor{gray!6}{34} & \cellcolor{gray!6}{12} & \cellcolor{gray!6}{125} & \cellcolor{gray!6}{10\%}\\
35 & 15 & 132 & 11\%\\
\cellcolor{gray!6}{36} & \cellcolor{gray!6}{13} & \cellcolor{gray!6}{113} & \cellcolor{gray!6}{12\%}\\
37 & 16 & 313 & 5\%\\
\bottomrule
\end{tabular}
\end{table}

\begin{table}

\caption{\label{tab:tab-depth-wcgbts}Positive samples of vermilion rockfish in the northern model by depth (fm).}
\centering
\begin{tabular}[t]{lrrl}
\toprule
Year & Positive Samples & Samples & Percent Positive\\
\midrule
\cellcolor{gray!6}{{}[55,75]} & \cellcolor{gray!6}{10} & \cellcolor{gray!6}{121} & \cellcolor{gray!6}{8\%}\\
(75,100] & 16 & 170 & 9\%\\
\cellcolor{gray!6}{(100,150]} & \cellcolor{gray!6}{23} & \cellcolor{gray!6}{214} & \cellcolor{gray!6}{11\%}\\
(150,200] & 4 & 67 & 6\%\\
\cellcolor{gray!6}{(200,300]} & \cellcolor{gray!6}{3} & \cellcolor{gray!6}{111} & \cellcolor{gray!6}{3\%}\\
\bottomrule
\end{tabular}
\end{table}

\begin{table}

\caption{\label{tab:tab-year-wcgbts}Samples of vermilion rockfish in the northern model by year.}
\centering
\begin{tabular}[t]{lrrl}
\toprule
Year & Positive Samples & Samples & Percent Positive\\
\midrule
\cellcolor{gray!6}{2003} & \cellcolor{gray!6}{2} & \cellcolor{gray!6}{38} & \cellcolor{gray!6}{5\%}\\
2004 & 2 & 42 & 5\%\\
\cellcolor{gray!6}{2006} & \cellcolor{gray!6}{2} & \cellcolor{gray!6}{45} & \cellcolor{gray!6}{4\%}\\
2008 & 6 & 58 & 10\%\\
\cellcolor{gray!6}{2009} & \cellcolor{gray!6}{8} & \cellcolor{gray!6}{65} & \cellcolor{gray!6}{12\%}\\
\addlinespace
2010 & 5 & 59 & 8\%\\
\cellcolor{gray!6}{2012} & \cellcolor{gray!6}{3} & \cellcolor{gray!6}{64} & \cellcolor{gray!6}{5\%}\\
2013 & 4 & 30 & 13\%\\
\cellcolor{gray!6}{2014} & \cellcolor{gray!6}{5} & \cellcolor{gray!6}{56} & \cellcolor{gray!6}{9\%}\\
2015 & 3 & 48 & 6\%\\
\addlinespace
\cellcolor{gray!6}{2016} & \cellcolor{gray!6}{5} & \cellcolor{gray!6}{58} & \cellcolor{gray!6}{9\%}\\
2017 & 5 & 48 & 10\%\\
\cellcolor{gray!6}{2018} & \cellcolor{gray!6}{3} & \cellcolor{gray!6}{45} & \cellcolor{gray!6}{7\%}\\
2019 & 3 & 27 & 11\%\\
\bottomrule
\end{tabular}
\end{table}

\FloatBarrier

\begin{table}

\caption{\label{tab:tab-model-select-wcgbts}Model selection for the WCGBTS survey index for vermilion rockfish in the northern model.}
\centering
\begin{tabular}[t]{lrr}
\toprule
Model & Binomial $\Delta$AIC & Gamma $\Delta$AIC\\
\midrule
\cellcolor{gray!6}{1} & \cellcolor{gray!6}{0.00} & \cellcolor{gray!6}{67.52}\\
YEAR + PASS & 15.66 & 48.80\\
\cellcolor{gray!6}{YEAR + PASS + DEPTH bin} & \cellcolor{gray!6}{15.59} & \cellcolor{gray!6}{0.00}\\
YEAR + PASS + DEPTH bin + LAT bin & 7.42 & 4.40\\
\cellcolor{gray!6}{YEAR + DEPTH bin + LAT bin} & \cellcolor{gray!6}{12.04} & \cellcolor{gray!6}{10.96}\\
\addlinespace
YEAR + LAT bin & 16.79 & 67.52\\
\cellcolor{gray!6}{YEAR + PASS + LAT bin} & \cellcolor{gray!6}{12.89} & \cellcolor{gray!6}{53.13}\\
\bottomrule
\end{tabular}
\end{table}

\FloatBarrier

\begin{table}

\caption{\label{tab:tab-index-wcgbts}Standardized index for the WCGBTS survey index with log-scale standard errors and 95\% highest
       posterior density (HPD) intervals for vermilion in the northern model.}
\centering
\begin{tabular}[t]{rrrrr}
\toprule
Year & Index & logSE & lower HPD & upper HPD\\
\midrule
\cellcolor{gray!6}{2003} & \cellcolor{gray!6}{0.20} & \cellcolor{gray!6}{1.08} & \cellcolor{gray!6}{0.01} & \cellcolor{gray!6}{0.91}\\
2004 & 0.00 & 1.29 & 0.00 & 0.02\\
\cellcolor{gray!6}{2006} & \cellcolor{gray!6}{0.39} & \cellcolor{gray!6}{1.41} & \cellcolor{gray!6}{0.01} & \cellcolor{gray!6}{2.23}\\
2008 & 0.66 & 0.95 & 0.07 & 2.91\\
\cellcolor{gray!6}{2009} & \cellcolor{gray!6}{0.33} & \cellcolor{gray!6}{1.44} & \cellcolor{gray!6}{0.01} & \cellcolor{gray!6}{1.89}\\
\addlinespace
2010 & 0.06 & 0.83 & 0.01 & 0.23\\
\cellcolor{gray!6}{2012} & \cellcolor{gray!6}{0.13} & \cellcolor{gray!6}{0.93} & \cellcolor{gray!6}{0.01} & \cellcolor{gray!6}{0.48}\\
2013 & 0.05 & 0.88 & 0.01 & 0.17\\
\cellcolor{gray!6}{2014} & \cellcolor{gray!6}{0.07} & \cellcolor{gray!6}{1.37} & \cellcolor{gray!6}{0.00} & \cellcolor{gray!6}{0.36}\\
2015 & 0.10 & 1.02 & 0.01 & 0.42\\
\addlinespace
\cellcolor{gray!6}{2016} & \cellcolor{gray!6}{0.21} & \cellcolor{gray!6}{0.83} & \cellcolor{gray!6}{0.03} & \cellcolor{gray!6}{0.75}\\
2017 & 0.03 & 0.92 & 0.00 & 0.11\\
\cellcolor{gray!6}{2018} & \cellcolor{gray!6}{0.01} & \cellcolor{gray!6}{0.99} & \cellcolor{gray!6}{0.00} & \cellcolor{gray!6}{0.05}\\
2019 & 0.02 & 0.85 & 0.00 & 0.06\\
\bottomrule
\end{tabular}
\end{table}

\FloatBarrier

\begin{figure}
\centering
\includegraphics{C:/Stock_Assessments/VRML_Assessment_2021/GitHub/Vermilion_2021/doc/indices/vermilion_WCGBTS_writeup_NCA_files/figure-latex/fig-propzero-wcgbts-1.pdf}
\caption{\label{fig:fig-propzero-wcgbts}Posterior predictive distribution of the proportion of zero observations in replicate data sets generated by the delta model with a vertical line representing the observed average.}
\end{figure}

\begin{figure}
\centering
\includegraphics{C:/Stock_Assessments/VRML_Assessment_2021/GitHub/Vermilion_2021/doc/indices/vermilion_WCGBTS_writeup_NCA_files/figure-latex/fig-posterior-mean-wcgbts-1.pdf}
\caption{\label{fig:fig-posterior-mean-wcgbts}Posterior predictive draws of the mean by year with a vertical line of the raw data average.}
\end{figure}

\begin{figure}
\centering
\includegraphics{C:/Stock_Assessments/VRML_Assessment_2021/GitHub/Vermilion_2021/doc/indices/vermilion_WCGBTS_writeup_NCA_files/figure-latex/fig-posterior-sd-wcgbts-1.pdf}
\caption{\label{fig:fig-posterior-sd-wcgbts}Posterior predictive draws of the standard deviation by year with a vertical line representing the observed average.}
\end{figure}

\begin{figure}
\centering
\includegraphics{C:/Stock_Assessments/VRML_Assessment_2021/GitHub/Vermilion_2021/doc/indices/vermilion_WCGBTS_writeup_NCA_files/figure-latex/fig-cpue-wcgbts-1.pdf}
\caption{\label{fig:fig-cpue-wcgbts}Standardized index and arithmetic mean of the CPUE from the filtered data. Each timeseries is scaled to its respective means.}
\end{figure}

\begin{figure}
\centering
\includegraphics{C:/Stock_Assessments/VRML_Assessment_2021/GitHub/Vermilion_2021/doc/indices/vermilion_WCGBTS_writeup_NCA_files/figure-latex/fig-Dbin-marginal-wcgbts-1.pdf}
\caption{\label{fig:fig-Dbin-marginal-wcgbts}Binomial marginal effects from the final model}
\end{figure}

\begin{figure}
\centering
\includegraphics{C:/Stock_Assessments/VRML_Assessment_2021/GitHub/Vermilion_2021/doc/indices/vermilion_WCGBTS_writeup_NCA_files/figure-latex/fig-Dpos-marginal-wcgbts-1.pdf}
\caption{\label{fig:fig-Dpos-marginal-wcgbts}Positive model marginal effects from the final model.}
\end{figure}

\newpage

\tagstructbegin{tag=H2}\tagmcbegin{tag=H2}

\hypertarget{regs-graphic}{%
\subsection{Recreational Regulations}\label{regs-graphic}}

\leavevmode\tagmcend\tagstructend

\renewcommand{\thepage}{H\arabic{page}}
\renewcommand{\thefigure}{H\arabic{figure}}
\renewcommand{\thetable}{H\arabic{table}}
\setcounter{page}{1}
\setcounter{figure}{0}
\setcounter{table}{0}

\clearpage

\begin{landscape}


\begin{figure}
\includegraphics{C:/Stock_Assessments/VRML_Assessment_2021/GitHub/Vermilion_2021/doc/figures/Regs_Northern_ShelfRF.png}
\caption{Recreational depth closures for shelf rockfish in the northern California management area.\label{fig:rec-reg-n}}
\end{figure}

\FloatBarrier

\begin{figure}
\includegraphics{C:/Stock_Assessments/VRML_Assessment_2021/GitHub/Vermilion_2021/doc/figures/Regs_NorthCentral_ShelfRF.png}
\caption{Recreational depth closures for shelf rockfish in the north-central California management area.\label{fig:rec-reg-cn}}
\end{figure}

\FloatBarrier

\begin{figure}
\includegraphics{C:/Stock_Assessments/VRML_Assessment_2021/GitHub/Vermilion_2021/doc/figures/Regs_Central_ShelfRF.png}
\caption{Recreational depth closures for shelf rockfish in the central California management area.\label{fig:rec-reg-cs}}
\end{figure}


\begin{figure}
\includegraphics{C:/Stock_Assessments/VRML_Assessment_2021/GitHub/Vermilion_2021/doc/figures/Regs_Southern_ShelfRF.png}
\caption{Recreational depth closures for shelf rockfish in the southern California management area.\label{fig:rec-reg-s}}
\end{figure}


\end{landscape}

\clearpage

\newpage

\tagstructbegin{tag=H2}\tagmcbegin{tag=H2}

\hypertarget{management-boundary-analysis}{%
\subsection{Management Boundary Analysis}\label{management-boundary-analysis}}

\leavevmode\tagmcend\tagstructend

\renewcommand{\thepage}{I\arabic{page}}
\renewcommand{\thefigure}{I\arabic{figure}}
\renewcommand{\thetable}{I\arabic{table}}
\setcounter{page}{1}
\setcounter{figure}{0}
\setcounter{table}{0}

\tagstructbegin{tag=H2}\tagmcbegin{tag=H2}

\hypertarget{allocation-of-yield-among-federal-management-areas}{%
\subsection{Allocation of Yield Among Federal Management Areas}\label{allocation-of-yield-among-federal-management-areas}}

\leavevmode\tagmcend\tagstructend

The 2021 northern California base model for vermilion rockfish represents U.S. waters between $34^\circ 27^\prime N$ and the California-Oregon border $42^\circ 00^\prime N$. Federal management of the minor shelf rockfish, which includes vermilion rockfish, is based on areas north and south of $40^\circ 10^\prime N$, near Cape Mendocino. Therefore, yield estimates from the northern California base model must be divided between the northern and southern management areas in order to determine the contribution of vermilion rockfish to the minor nearshore rockfish overfishing limit (OFL).

Allocation of the OFL could, ideally, be based on a fishery-independent survey of abundance, but lacking that information several alternatives exist. Previous allocations have used catch as a proxy for abundance when no other information was available {\tagstructbegin{tag=Reference}\tagmcbegin{tag=Reference}(Dick and MacCall 2010, Dick and Maccall 2011)\leavevmode\tagmcend\tagstructend}. Recent catches of vermilion rockfish in the recreational and commercial sectors suggest that roughly 4.8\% and 2.8\%, respectively, of catches in these sectors are landed north of Cape Mendocino (Tables \ref{tab:rec-split} and \ref{tab:com-split}). Removals for the recreational fleet are in numbers of fish and removals from the commercial fleet are in mt, to be consistent with the assessment inputs.

Recent advances in habitat mapping allow us to estimate the relative amount of reef habitat within state waters (0-3 nm) in each area, e.g., the {\tagstructbegin{tag=Link}\tagmcbegin{tag=Link}\href{https://walrus.wr.usgs.gov/mapping/csmp/}{California Seafloor Mapping Project}\leavevmode\tagmcend\tagstructend}. If we assumed that average density of vermilion rockfish is constant over the assessed area, the fraction of vermilion rockfish occurring north of Cape Mendocino would be equal to the fraction of habitat in the same area: approximately 18\% (pers. comm. Rebecca Miller, UCSC). However, the assumption of equal density may not be accurate, and no direct estimates of density are available from a fishery-independent survey with adequate spatial coverage.

As was proposed in the 2017 blue/deacon rockfish complex stock assessment {\tagstructbegin{tag=Reference}\tagmcbegin{tag=Reference}(Dick et al. 2017a)\leavevmode\tagmcend\tagstructend} we combined existing habitat information with a proxy for fish density -- catch per unit effort. Although data from the CRFS onboard CPFV observer program are more precise in terms of total catch, effort, and location, relatively few samples are available north of Cape Mendocino. Sampling coverage for the dockside survey is spatially more complete, in that numerous samples exist in the northern management area. We therefore used the private boat (PR1) CPUE data to develop a spatial index (with CPUE assumed proportional to density), and multiplied the area-specific CPUE estimates by the amount of habitat to produce a spatial index of relative abundance. Data were filtered using the same methods detailed in the assessment for the CRFS private boat dockside index. Years prior to 2016 were subsequently dropped as well as 2020 due to reduced sampling during COVID-19, to create an index that is representative of recent catch rates in each area. Sample sizes (number of trips) for the final data set are shown in Table \ref{tab:number-trips}.

Vermilion rockfish is a shelf species and we recognize that there is a fraction of the population and rocky habitat outside of state waters. However, due to depth closures that began in 2002, samples from deeper waters are not available, nor is the associated habitat data. This method assumes the same proportion of habitat outside state waters north and south of Cape Mendocino. We explored limiting the data to only angler-reported trips inside state waters. However, the accuracy of the angler-reported trip location is unknown and the trip may represent catch from both inside and outside state waters. Filtering based on angler-reported area fished did not affect the final result, so we retained all data for this analysis.

We modeled CPUE (vermilion rockfish per angler trip) using a Bayesian negative binomial regression with subregion (defined as CRFS districts, see Table \ref{tab:number-trips}) as a qualitative covariate and pooling data across years 2016-2019. Including the subregion covariate reduced AIC by 2270 points relative to the null (intercept-only) model. CPUE in the Wine District subregion was lower than the other subregions in the model (Table \ref{tab:rel-cpue}). When CPUE is multiplied by the percentage of habitat area north of $40^\circ 10^\prime N$ latitude, the expected percentage of the stock that occurs north of Cape Mendocino is 4.4\% (Table \ref{tab:rel-cpue}).

\clearpage

\begin{table}

\caption{\label{tab:rec-split}California recreational total mortality (1000s of fish) for vermilion rockfish by CRFS district, 2016-2019. The Redwood district occurs mainly north of Cape Mendocino. Source:RecFIN}
\centering
\begin{tabular}[t]{ccccc>{\centering\arraybackslash}p{3cm}}
\toprule
Year & CENTRAL & BAY & WINE & REDWOOD & Percent mortality in Redwood District\\
\midrule
\cellcolor{gray!6}{2016} & \cellcolor{gray!6}{63.382} & \cellcolor{gray!6}{15.480} & \cellcolor{gray!6}{3.888} & \cellcolor{gray!6}{2.099} & \cellcolor{gray!6}{2.47\%}\\
2017 & 79.042 & 20.795 & 4.891 & 2.858 & 2.66\%\\
\cellcolor{gray!6}{2018} & \cellcolor{gray!6}{89.937} & \cellcolor{gray!6}{17.996} & \cellcolor{gray!6}{4.192} & \cellcolor{gray!6}{3.214} & \cellcolor{gray!6}{2.79\%}\\
2019 & 96.274 & 29.016 & 8.616 & 3.363 & 2.45\%\\
\bottomrule
\end{tabular}
\end{table}

\begin{table}

\caption{\label{tab:com-split}Commercial landings (mt) of vermilion rockfish in California port complexes located north (CRS+ERK) and south (MRO-BRG) of Cape Mendocino, 2016-2019. Source: CALCOM.}
\centering
\begin{tabular}[t]{ccc>{\centering\arraybackslash}p{3cm}}
\toprule
Year & MRO-BRG & CRS+ERK & Percent landings in CRS+ERK\\
\midrule
\cellcolor{gray!6}{2016} & \cellcolor{gray!6}{12.477} & \cellcolor{gray!6}{0.888} & \cellcolor{gray!6}{1.33\%}\\
2017 & 12.738 & 1.550 & 2.32\%\\
\cellcolor{gray!6}{2018} & \cellcolor{gray!6}{17.650} & \cellcolor{gray!6}{2.010} & \cellcolor{gray!6}{3.00\%}\\
2019 & 16.579 & 3.052 & 4.56\%\\
\bottomrule
\end{tabular}
\end{table}

\begin{table}

\caption{\label{tab:number-trips}Number of trips sampled in the PR1 mode by year and CRFS District.}
\centering
\begin{tabular}[t]{ccccc}
\toprule
YEAR & Central & Bay & Wine & Redwood\\
\midrule
\cellcolor{gray!6}{2016} & \cellcolor{gray!6}{2175} & \cellcolor{gray!6}{795} & \cellcolor{gray!6}{279} & \cellcolor{gray!6}{1108}\\
2017 & 1782 & 800 & 392 & 1148\\
\cellcolor{gray!6}{2018} & \cellcolor{gray!6}{1783} & \cellcolor{gray!6}{677} & \cellcolor{gray!6}{345} & \cellcolor{gray!6}{1149}\\
2019 & 1724 & 681 & 204 & 1151\\
\bottomrule
\end{tabular}
\end{table}

\begin{table}

\caption{\label{tab:rel-cpue}Estimated CPUE, percent habitat area, and relative abundance by CRFS District.}
\centering
\begin{tabular}[t]{lcccrl}
\toprule
CRFS District & CPUE & Area & Percent of Area & CPUExAREA & Relative Abundance\\
\midrule
\cellcolor{gray!6}{Central} & \cellcolor{gray!6}{0.833} & \cellcolor{gray!6}{315.912} & \cellcolor{gray!6}{35.56\%} & \cellcolor{gray!6}{0.296} & \cellcolor{gray!6}{59.32\%}\\
Bay & 0.448 & 271.279 & 30.54\% & 0.137 & 27.45\%\\
\cellcolor{gray!6}{Wine} & \cellcolor{gray!6}{0.286} & \cellcolor{gray!6}{136.937} & \cellcolor{gray!6}{15.42\%} & \cellcolor{gray!6}{0.044} & \cellcolor{gray!6}{8.82\%}\\
Redwood & 0.122 & 164.193 & 18.48\% & 0.022 & 4.41\%\\
\bottomrule
\end{tabular}
\end{table}

\newpage

\tagstructbegin{tag=H1}\tagmcbegin{tag=H1}

\hypertarget{references}{%
\section{References}\label{references}}

\leavevmode\tagmcend\tagstructend

\hypertarget{refs}{}
\begin{CSLReferences}{1}{0}
\leavevmode\vadjust pre{\hypertarget{ref-Abrams2014}{}}%
Abrams, J.L. 2014. {The effect of local fishing pressure on the size and age structure of fishes associated with rocky habitats along California's north coast}. \emph{In} Master's T. Humboldt State University.

\leavevmode\vadjust pre{\hypertarget{ref-Albin1993}{}}%
Albin, D., and Karpov, K.A. 1993. {Effort and catch estimates for northern and central California marine recreational fisheries, 1981-1986}. State of California Department of Fish; Game, Marine Resources Division.

\leavevmode\vadjust pre{\hypertarget{ref-Ally1991}{}}%
Ally, J.R.R., Ono, D.S., Read, R.B., and Wallace, M. 1991. {Status of major southern California marine sport fish species with management recommendations, based on analyses of catch and size composition data collected on board commercial passenger fishing vessels from 1985 through 1987}. Marine Resources Division Administrative Report No. 90-2 Califronia Departent of Fish and Game.

\leavevmode\vadjust pre{\hypertarget{ref-Alverson1964}{}}%
Alverson, D.L., Pruter, a.T., and Ronholt, L.L. 1964. {A Study of Demersal Fishes and Fisheries of the Northeastern Pacific Ocean}. Institute of Fisheries, University of British Columbia.

\leavevmode\vadjust pre{\hypertarget{ref-Baskett2006}{}}%
Baskett, M.L., Yoklavich, M., and Love, M.S. 2006. {Predation, competition, and the recovery of overexploited fish stocks in marine reserves}. Canadian Journal of Fisheries and Aquatic Sciences \textbf{63}(6): 1214--1229. doi: \href{https://doi.org/10.1139/F06-013}{10.1139/F06-013}.

\leavevmode\vadjust pre{\hypertarget{ref-Berger2017}{}}%
Berger, A.M., Goethel, D.R., Lynch, P.D., Terrance, Q.I., Mormede, S., Mckenzie, J., and Dunn, A. 2017. {Space oddity: The mission for spatial integration}. Canadian Journal of Fisheries and Aquatic Sciences \textbf{74}: 1698--1716.

\leavevmode\vadjust pre{\hypertarget{ref-Brown2001}{}}%
Brown, L.D., Cai, T.T., and DasGupta, A. 2001. {Interval estimation of binomial proportion}. Statistical Science \textbf{16}(2): 101--133. doi: \href{https://doi.org/10.1002/sim.2930}{10.1002/sim.2930}.

\leavevmode\vadjust pre{\hypertarget{ref-Budrick2016}{}}%
Budrick, J. 2016. {Evolutionary processes contributing to population structure in the rockfishes of the subgenus genus \emph{Rosicola}: implications for fishery management, stock assessment and prioritization of future analyses of structure in the genus \emph{Sebastes}.} PhD thesis, University of California, Berkeley.

\leavevmode\vadjust pre{\hypertarget{ref-Cadrin2020}{}}%
Cadrin, S.X. 2020. {Defining spatial structure for fishery stock assessment}. Fisheries Research \textbf{221}(October 2019). doi: \href{https://doi.org/10.1016/j.fishres.2019.105397}{10.1016/j.fishres.2019.105397}.

\leavevmode\vadjust pre{\hypertarget{ref-Collins1978}{}}%
Collins, R.A., and Crooke, S.J. (n.d.). {An evaluation of the commercial passenger fishing vessel record system and the results of sampling the Southern California catch for species and size composition, 1975-1978}. Unpublished report.

\leavevmode\vadjust pre{\hypertarget{ref-Croker1940}{}}%
Croker, R.S. 1940. {Three Years of Fisheries Statistics on Marine Sport Fishing in California}. Transactions of the American Fisheries Society \textbf{69}(1).

\leavevmode\vadjust pre{\hypertarget{ref-Dark1994}{}}%
Dark, T.A., and Wilkins, M.E. 1994. {Distribution, abundance, and biological charracteristics of groundfish off the coast of Washington, Oregon, and California, 1977-1986}. U.S. Department of Commerce, National Oceanic; Atmospheric Administration, National Marine Fisheries Service.

\leavevmode\vadjust pre{\hypertarget{ref-Dick2017a}{}}%
Dick, E.J., Berger, A., Bizzarro, J., Bosley, K., Cope, J., Field, J., Gilbert-Horvath, L., Grunloh, N., and Ivens-Duran, M. 2017a. {The combined status of blue and deacon rockfishes in U.S. waters off California and Oregon in 2017}. Pacific Ficheries Management Council, Portland, OR.

\leavevmode\vadjust pre{\hypertarget{ref-Dick2017}{}}%
Dick, E.J., Beyer, S., Mangel, M., and Ralston, S. 2017b. {A meta-analysis of fecundity in rockfishes (genus genus\emph{Sebastes})}. Fisheries Research \textbf{187}: 73--85. Elsevier B.V. doi: \href{https://doi.org/10.1016/j.fishres.2016.11.009}{10.1016/j.fishres.2016.11.009}.

\leavevmode\vadjust pre{\hypertarget{ref-Dick2011}{}}%
Dick, E.J., and Maccall, A.D. 2011. {Depletion-Based Stock Reduction Analysis : A catch-based method for determining sustainable yields for data-poor fish stocks}. Fisheries Research \textbf{110}(2): 331--341. Elsevier B.V. doi: \href{https://doi.org/10.1016/j.fishres.2011.05.007}{10.1016/j.fishres.2011.05.007}.

\leavevmode\vadjust pre{\hypertarget{ref-Dick2010}{}}%
Dick, E.J., and MacCall, A.D. 2010. {Estimates of sustainable yield for 50 data-poor stocks in the pacific coast groundfish fishery management plan}. NOAA technical memorandum NOAA-TM-NMFS-SWFSC 460.

\leavevmode\vadjust pre{\hypertarget{ref-Dick2007}{}}%
Dick, E.J., Ralston, S., and Pearson, D. 2007. {Status of cowcod, genus\emph{Sebastes levis}, in the Southern California Bight}.

\leavevmode\vadjust pre{\hypertarget{ref-Echeverria1987}{}}%
Echeverria, T.W. 1987. {Thirty-four species of California rockfishes: maturity and seasonality of reproduction}. Fishery Bulletin \textbf{85}(2): 229--250.

\leavevmode\vadjust pre{\hypertarget{ref-Field2021}{}}%
Field, J.C., Miller, R.R., Santora, J.A., Tolimieri, N., Haltuch, M.A., Brodeur, R.D., Auth, T.D., Dick, E.J., Monk, M.H., Sakuma, K.M., and Wells, B.K. 2021. {Spatiotemporal patterns of variability in the abundance and distribution of winter-spawned pelagic juvenile rockfish in the California Current}. PLoS ONE \textbf{16}(5): 1--25. doi: \href{https://doi.org/10.1371/journal.pone.0251638}{10.1371/journal.pone.0251638}.

\leavevmode\vadjust pre{\hypertarget{ref-Field2006}{}}%
Field, J.C., Punt, A.E., Methot, R.D., and Thomson, C.J. 2006, December. {Does MPA mean 'Major Problem for Assessments'? Considering the consequences of place-based management systems}. John Wiley \& Sons, Ltd. doi: \href{https://doi.org/10.1111/j.1467-2979.2006.00226.x}{10.1111/j.1467-2979.2006.00226.x}.

\leavevmode\vadjust pre{\hypertarget{ref-Francis2011}{}}%
Francis, R.I.C.C. 2011. {Data weighting in statistical fisheries stock assessment models}. (July). doi: \href{https://doi.org/10.1139/F2011-025}{10.1139/F2011-025}.

\leavevmode\vadjust pre{\hypertarget{ref-Frey1971}{}}%
Frey, H.W. 1971. {California's Living Marine Resources and Their Utilization. California Department of Fish and Game}.

\leavevmode\vadjust pre{\hypertarget{ref-Friedman2018}{}}%
Friedman, W.R., Santora, J.A., Schroeder, I.D., Huff, D.D., Brodeur, R.D., Field, J.C., and Wells, B.K. 2018. {Environmental and geographic relationships among salmon forage assemblages along the continental shelf of the California Current}. Marine Ecology Progress Series \textbf{596}(May): 181--198. doi: \href{https://doi.org/10.3354/meps12598}{10.3354/meps12598}.

\leavevmode\vadjust pre{\hypertarget{ref-Hamel2015}{}}%
Hamel, O.S. 2015. {A method for calculating a meta-analytical prior for the natural mortality rate using multiple life history correlates}. ICES Journal of Marine Science \textbf{72}(1): 62--69. doi: \href{https://doi.org/doi:10.1093/icesjms/fsu131}{doi:10.1093/icesjms/fsu131}.

\leavevmode\vadjust pre{\hypertarget{ref-Hannah2011}{}}%
Hannah, R.W., and Rankin, P.S. 2011. {Site fidelity and movement of eight species of pacific rockfish at a high-relief rocky reef on the Oregon coast}. North American Journal of Fisheries Management \textbf{31}(3): 483--494. doi: \href{https://doi.org/10.1080/02755947.2011.591239}{10.1080/02755947.2011.591239}.

\leavevmode\vadjust pre{\hypertarget{ref-Harry1961}{}}%
Harry, G., and Morgan, A.R. 1961. {History of the trawl fishery, 1884-1961}. Oregon Fish Commission Research Briefs \textbf{19}: 5--26.

\leavevmode\vadjust pre{\hypertarget{ref-Hastie2007}{}}%
Hastie, J., and Ralston, S. 2007. {Pre-recruit survey workshop}. Santa Cruz, CA. pp. 23 p.

\leavevmode\vadjust pre{\hypertarget{ref-Hyde2007}{}}%
Hyde, J. 2007. {The origin, evolution, and diversification of rockfishes of the genus Sebastes (Cuvier): insights into speciation and biogeography of temperate reef fishes}. PhD thesis, University of California San Diego.

\leavevmode\vadjust pre{\hypertarget{ref-Hyde2008b}{}}%
Hyde, J.R.; Kimbrell, C. A.; Budrick, J. E.; Lynn, E. A.; Vetter, R.D. 2008. {Cryptic speciation in the vermilion rockfish (\emph{Sebastes miniatus}) and the role of bathymetry in the speciation process}. Molecular Ecology \textbf{17}: 1122--1136. doi: \href{https://doi.org/10.1111/j.1365-294X.2007.03653.x}{10.1111/j.1365-294X.2007.03653.x}.

\leavevmode\vadjust pre{\hypertarget{ref-Hyde2009}{}}%
Hyde, J.R., and Vetter, R.D. 2009. {Population genetic structure in the redefined vermilion rockfish (\emph{Sebastes miniatus}) indicates limited larval dispersal and reveals natural management units}. Canadian Journal of Fisheries and Aquatic Sciences \textbf{66}(9): 1569--1581. doi: \href{https://doi.org/10.1139/F09-104}{10.1139/F09-104}.

\leavevmode\vadjust pre{\hypertarget{ref-Karpov1995}{}}%
Karpov, K.A., Albin, D.P., and Van Buskirk, W.H. 1995. {The marine recreational fishery in northern and central California a historical Comparison (1958--86), status of stocks (1980--86), and effects of changes in the California current}. Fish Bulletin: 192. Available from \url{http://www.psmfc.org/$/sim$wade/pub/bull176/bull176.htm}.

\leavevmode\vadjust pre{\hypertarget{ref-Keller2017}{}}%
Keller, A.A., Wallace, J.R., and Methot, R.D. 2017. {The northwest fisheries science center's west coast groundfish bottom trawl survey: history, design, and description}. National Oceanic; Atmospheric Administration. doi: \href{https://doi.org/10.7289/V5/TM-NWFSC-136}{10.7289/V5/TM-NWFSC-136}.

\leavevmode\vadjust pre{\hypertarget{ref-Lea1999}{}}%
Lea, R.N., McAllister, R.D., and VenTresca, D.A. 1999. {Biological aspects of nearshore rockfishes of the \emph{Sebastes} from central California: with notes on ecologically related sport fishes.} Fish Bulletin No. 177: 112.

\leavevmode\vadjust pre{\hypertarget{ref-Lenarz1987}{}}%
Lenarz, W.H. 1987. {A history of California rockfish fisheries. In Proceedings of the International Rockfish Symposium.} \emph{In} International rockfish symposium.

\leavevmode\vadjust pre{\hypertarget{ref-Love1990b}{}}%
Love, M.S., Morris, P., McCrae, M., and Collins, R. 1990. {Life history aspects of 19 rockfish species (Scorpaenidae:\emph{Sebastes}) from the Southern Califronia Bight}. NOAA Technical Report NMFS 87.

\leavevmode\vadjust pre{\hypertarget{ref-Love2012a}{}}%
Love, M.S., Nishimoto, M., Clark, S., and Schroeder, D.M. 2012. {Recruitment of young-of-the-year fishes to natural and artificial offshore structure within central and southern California waters, 2008-2010}. Bulletin of Marine Science \textbf{88}(4): 863--882. doi: \href{https://doi.org/10.5343/bms.2011.1101}{10.5343/bms.2011.1101}.

\leavevmode\vadjust pre{\hypertarget{ref-Love2002}{}}%
Love, M., Yoklavich, M.M., and Thorsteinson, L. 2002. {The rockfishes of the northeast Pacific}. University of California Press, Berkeley, CA, USA.

\leavevmode\vadjust pre{\hypertarget{ref-Lowe2009}{}}%
Lowe, C.G., Anthony, K.M., Jarvis, E.T., Bellquist, L.F., and Love, M.S. 2009. {Site fidelity and movement patterns of groundfish associated with offshore petroleum platforms in the Santa Barbara Channel}. Marine and Coastal Fisheries \textbf{1}(1): 71--89. doi: \href{https://doi.org/10.1577/c08-047.1}{10.1577/c08-047.1}.

\leavevmode\vadjust pre{\hypertarget{ref-MacCall2002}{}}%
MacCall, A.D. 2002. {Fishery-management and stock-rebuilding prospects under conditions of low-frequency environmental variability and species interactions}. Bulletin of Marine Science \textbf{70}(2): 613--628.

\leavevmode\vadjust pre{\hypertarget{ref-McAllister1997}{}}%
McAllister, Murdoch K.; Ianelli, J.N. 1997. {Bayesian stock assessment using catch-age data and the sampling - importance resampling algorithm}. Canadian Journal of Fisheries and Aquatic Sciences \textbf{54}: 284--300.

\leavevmode\vadjust pre{\hypertarget{ref-McGilliard2015}{}}%
McGilliard, C.R., Punt, A.E., Methot, R.D., and Hilborn, R. 2014. {Accounting for marine reserves using spatial stock assessments}. Canadian Journal of Fisheries and Aquatic Sciences \textbf{72}: 262--280. doi: \href{https://doi.org/10.1139/cjfas-2013-0364}{10.1139/cjfas-2013-0364}.

\leavevmode\vadjust pre{\hypertarget{ref-Methot2020}{}}%
Methot, R. D., Jr., Wetzel, C.R., Taylor, I.G., and Doering, K. 2020. {Stock Synthesis User Manual Version 3.30.15}. U.S. Department of Commerce, NOAA Processed Report NMFS-NWFSC-PR-2020-05.

\leavevmode\vadjust pre{\hypertarget{ref-Methot2013}{}}%
Methot, R.D., and Wetzel, C.R. 2013. {Stock synthesis: a biological and statistical framework for fish stock assessment and fishery management}. Fisheries Research \textbf{142}: 86--99. Elsevier B.V. doi: \href{https://doi.org/10.1016/j.fishres.2012.10.012}{10.1016/j.fishres.2012.10.012}.

\leavevmode\vadjust pre{\hypertarget{ref-Miller2014}{}}%
Miller, R.R., Field, J.C., Santora, J.A., Schroeder, I.D., Huff, D.D., Key, M., Pearson, D.E., and MacCall, A.D. 2014. {A spatially distinct history of the development of California groundfish fisheries}. PLoS ONE \textbf{9}(6). Public Library of Science. doi: \href{https://doi.org/10.1371/journal.pone.0099758}{10.1371/journal.pone.0099758}.

\leavevmode\vadjust pre{\hypertarget{ref-Monk2014}{}}%
Monk, M.H., Dick, E.J., and Pearson, D. 2014. {Documentation of a relational database for the California recreational fisheries survey onboard observer sampling program, 1999-2011}. NOAA-TM-NMFS-SWFSC-529.

\leavevmode\vadjust pre{\hypertarget{ref-Monk2016}{}}%
Monk, M.H., Miller, R.R., Field, J., Dick, E.J., Wilson-Vandenberg, D., and Reilly, P. 2016. {Documentation for California Department of Fish and Wildlife's Onboard Sampling of the Rockfish and Lingcod Commercial Passenger Fishing Vessel Industry in Northern and Central California (1987-1998) as a relational database}. NOAA-TM-NMFS-SWFSC-558.

\leavevmode\vadjust pre{\hypertarget{ref-PFMC2002}{}}%
Pacific Fishery Management Council. 2002. {Status of the Pacific Coast Groundfish Fishery Through 2001 and Acceptable Biological Catches for 2002: Stock Assessment and Fishery Evaluation.} Pacific Fishery Management Council, Portland, OR.

\leavevmode\vadjust pre{\hypertarget{ref-PFMC2004}{}}%
Pacific Fishery Management Council. 2004. {Pacific coast groundfish fishery management plan: fishery management plan for the California, Oregon, and Washington groundfish fishery as amended through Amendment 17}. Pacific Fishery Management Council, Portland, OR.

\leavevmode\vadjust pre{\hypertarget{ref-Pearson1997}{}}%
Pearson, D.E.D., and Erwin, B. 1997. {Documentation of California's Commercial Market Sampling Data Entry and Expansion Programs}. National Marine Fisheries Service.

\leavevmode\vadjust pre{\hypertarget{ref-Pearson2008}{}}%
Pearson, D.E., Erwin, B., and Key, M. 2008. {Reliability of California's groundfish landing estimates from 1969-2006}. Tational Oceanic; Atmospheric Administration.

\leavevmode\vadjust pre{\hypertarget{ref-Phillips1964}{}}%
Phillips, J.B. 1964. {Life history studies on ten species of rockfish (genus \emph{Sebastodes})}. Fish Bulletin \textbf{126}.

\leavevmode\vadjust pre{\hypertarget{ref-Punt2020}{}}%
Punt, A.E., Dunn, A., Elvarsson, B.Þ., Hampton, J., Hoyle, S.D., Maunder, M.N., Methot, R.D., and Nielsen, A. 2020. {Essential features of the next-generation integrated fisheries stock assessment package: A perspective}. Fisheries Research \textbf{229}. Elsevier. doi: \href{https://doi.org/10.1016/j.fishres.2020.105617}{10.1016/j.fishres.2020.105617}.

\leavevmode\vadjust pre{\hypertarget{ref-Punt2004}{}}%
Punt, A.E., and Methot, R.D. 2004. {Effects of marine protected areas on the assessment of marine fishes}. \emph{In} Aquatic protected areas as fisheries management tools. American fisheries society. Quebec, Canada. pp. 133--154.

\leavevmode\vadjust pre{\hypertarget{ref-Ralston2010b}{}}%
Ralston, S., and MacFarlane, B.R. 2010. {Population estimation of bocaccio (\emph{Sebastes paucispinis}) based on larval production}. Canadian Journal of Fisheries and Aquatic Sciences \textbf{67}(6): 1005--1020. doi: \href{https://doi.org/10.1139/F10-039}{10.1139/F10-039}.

\leavevmode\vadjust pre{\hypertarget{ref-Ralston2010}{}}%
Ralston, S., Pearson, D.E., Field, J.C., and Key, M. 2010. {Documentation of the California catch reconstruction project}.

\leavevmode\vadjust pre{\hypertarget{ref-Ralston2013}{}}%
Ralston, S., Sakuma, K.M., and Field, J.C. 2013. {Interannual variation in pelagic juvenile rockfish (\emph{Sebastes} spp.) abundance - going with the flow}. Fisheries Oceanography \textbf{22}(4): 288--308. doi: \href{https://doi.org/10.1111/fog.12022}{10.1111/fog.12022}.

\leavevmode\vadjust pre{\hypertarget{ref-Reilly1998}{}}%
Reilly, P.N., Wilson-Vandenberg, D., Wilson, C.E., and Mayer, K. 1998. {Onboard sampling of the rockfish and lingcod commercial passenger fishing vessel industry in northern and central California, January through December 1995.} Marine region, Admin. Rep. \textbf{98-1}: 1--110.

\leavevmode\vadjust pre{\hypertarget{ref-Roedel1948}{}}%
Roedel, P.M. 1948. {Common Marine Fishes of California}. California Department of Fish; Game Bulletin No. 68.

\leavevmode\vadjust pre{\hypertarget{ref-Sakuma2016}{}}%
Sakuma, K.M., Field, J.C., Mantua, N.J., Ralston, S., Marinovic, B.B., and Carrion, C.N. 2016. {Anomalous epipelagic micronekton assemblage patterns in the neritic waters of the California Current in spring 2015 during a period of extreme ocean conditions}. CalCOFI Report \textbf{57}: 163--183.

\leavevmode\vadjust pre{\hypertarget{ref-Schroeder2019}{}}%
Schroeder, I.D., Santora, J.A., Bograd, S.J., Hazen, E.L., Sakuma, K.M., Moore, A.M., Edwards, C.A., Wells, B.K., and Field, J.C. 2019. {Source water variability as a driver of rockfish recruitment in the California current ecosystem: implications for climate change and fisheries management}. Canadian Journal of Fisheries and Aquatic Sciences \textbf{76}(6): 950--960. doi: \href{https://doi.org/10.1139/cjfas-2017-0480}{10.1139/cjfas-2017-0480}.

\leavevmode\vadjust pre{\hypertarget{ref-Sette1928}{}}%
Sette, O.E., and Fiedler, R.H. 1927. {Fishery industries of the United States, 1927}. \emph{In} Report of the united states commissioner of fisheries for the fiscal year 1928. U.S. Bureau of Fisheries.

\leavevmode\vadjust pre{\hypertarget{ref-Somers2020}{}}%
Somers, K.A., Jannot, J., Richerson, K., Riley, N.B., Tuttle, V., and McVeigh, J. 2020. {Estimated discard and catch of groundfish species in the 2019 U.S. west coast fisheries. e}. NOAA Fisheries, NWFSC Observer Program, 2725 Montlake Blvd E. NOAA Fisheries, NWFSC Observer Program, 2725 Montlake Blvd E., Seattle, WA.

\leavevmode\vadjust pre{\hypertarget{ref-Stachura2014}{}}%
Stachura, M.M., Essington, T.E., Mantua, N.J., Hollowed, A.B., Haltuch, M.A., Spencer, P.D., Branch, T.A., and Doyle, M.J. 2014. {Linking Northeast Pacific recruitment synchrony to environmental variability}. Fisheries Oceanography \textbf{23}(5): 389--408. doi: \href{https://doi.org/10.1111/fog.12066}{10.1111/fog.12066}.

\leavevmode\vadjust pre{\hypertarget{ref-Starr2015}{}}%
Starr, R.M., Wendt, D.E., Barnes, C.L., Marks, C.I., Malone, D., Waltz, G., Schmidt, K.T., Chiu, J., Launer, A.L., and Hall, N.C. 2015. {Variation in responses of fishes across multiple reserves within a network of marine protected areas in temperate waters}. PLoS ONE \textbf{10}(3): 1--24. doi: \href{https://doi.org/10.5061/dryad.6hk4h.Funding}{10.5061/dryad.6hk4h.Funding}.

\leavevmode\vadjust pre{\hypertarget{ref-Stephens2004}{}}%
Stephens, A., and MacCall, A. 2004. {A multispecies approach to subsetting logbook data for purposes of estimating CPUE}. Fisheries Research \textbf{70}(2-3 SPEC. ISS.): 299--310. doi: \href{https://doi.org/10.1016/j.fishres.2004.08.009}{10.1016/j.fishres.2004.08.009}.

\leavevmode\vadjust pre{\hypertarget{ref-Stierhoff2013}{}}%
Stierhoff, K., and Cutter, G. 2013. {Rockfish (/emph{Sebastes spp.}) training and validation image dataset: NOAA Southwest Fisheries Science Center remotely operated vehicle (ROV) digital still images.}

\leavevmode\vadjust pre{\hypertarget{ref-Then2018}{}}%
Then, A.Y., Hoenig, J.M., Hall, N.G., and Hewitt, D.A. 2018. {Evaluating the predictive performance of empirical estimators of natural mortality rate using information on over 200 fish species}. ICES Journal of Marine Science \textbf{75}(4): 1509. doi: \href{https://doi.org/10.1093/icesjms/fsx199}{10.1093/icesjms/fsx199}.

\leavevmode\vadjust pre{\hypertarget{ref-Thompson2017}{}}%
Thompson, A.R., Chen, D.C., Guo, L.W., Hyde, J.R., and Watson, W. 2017. {Larval abundances of rockfishes that were historically targeted by fishing increased over 16 years in association with a large marine protected area}. Royal Society Open Science \textbf{4}(9). doi: \href{https://doi.org/10.1098/rsos.170639}{10.1098/rsos.170639}.

\leavevmode\vadjust pre{\hypertarget{ref-Thompson2016}{}}%
Thompson, A.R., Hyde, J.R., Watson, W., Chen, D.C., and Guo, L.W. 2016. {Rockfish assemblage structure and spawning locations in southern California identified through larval sampling}. Marine Ecology Progress Series \textbf{547}: 177--192. doi: \href{https://doi.org/10.3354/meps11633}{10.3354/meps11633}.

\leavevmode\vadjust pre{\hypertarget{ref-Thorson2017}{}}%
Thorson, J.T., and Barnett, L.A.K. 2017. {Comparing estimates of abundance trends and distribution shifts using single- and multispecies models of fishes and biogenic habitat}. ICES Journal of Marine Science \textbf{74}(5): 1311--1321. doi: \href{https://doi.org/10.1093/icesjms/fsw193}{10.1093/icesjms/fsw193}.

\leavevmode\vadjust pre{\hypertarget{ref-Thorson2012}{}}%
Thorson, J.T., Stewart, I.J., and Punt, A.E. 2012. {Development and application of an agent-based model to evaluate methods for estimating relative abundance indices for shoaling fish such as Pacific rockfish (\emph{Sebastes} spp.)}. ICES Journal of Marine Science \textbf{69}(4): 635--647.

\leavevmode\vadjust pre{\hypertarget{ref-Thorson2014}{}}%
Thorson, J.T., and Ward, E.J. 2014. {Accounting for vessel effects when standardizing catch rates from cooperative surveys}. Fisheries Research \textbf{155}: 168--176. Elsevier B.V. doi: \href{https://doi.org/10.1016/j.fishres.2014.02.036}{10.1016/j.fishres.2014.02.036}.

\leavevmode\vadjust pre{\hypertarget{ref-Walters2001}{}}%
Walters, C., and Kitchell, J.F. 2001. {Cultivation/depensation effects on juvenile survival and recruitment: Implications for the theory of fishing}. Canadian Journal of Fisheries and Aquatic Sciences \textbf{58}(1): 39--50. doi: \href{https://doi.org/10.1139/f00-160}{10.1139/f00-160}.

\leavevmode\vadjust pre{\hypertarget{ref-Wendt2009}{}}%
Wendt, D.E., and Starr, R.M. 2009. {Collaborative research: an effective way to collect data for stock assessments and evaluate marine protected areas in California}. Marine and Coastal Fisheries \textbf{1}(1): 315--324. doi: \href{https://doi.org/10.1577/c08-054.1}{10.1577/c08-054.1}.

\leavevmode\vadjust pre{\hypertarget{ref-Vandenberg2014}{}}%
Wilson-Vandenberg, D., Larinto, T., and Key, M. 2014. {Implementing California's Nearshore Fishery Management Plan - twelve year later}. California Fish and Game \textbf{100}(2): 186--214.

\leavevmode\vadjust pre{\hypertarget{ref-Witzig1992}{}}%
Witzig, J.F., Holliday, M.C., Essig, R.J., and Sutherland, D.L. 1992. {Marine Recreational Fishery Statistics Survey, Pacific Coast, 1987-1989}. National Oceanic; Atmospheric Administration.

\leavevmode\vadjust pre{\hypertarget{ref-Yoklavich2007}{}}%
Yoklavich, M.M., Love, M.S., and Forney, K.A. 2007. {A fishery-independent assessment of an overfished rockfish stock, cowcod (\emph{Sebastes levis}), using direct observations from an occupied submersible}. Canadian Journal of Fisheries and Aquatic Sciences \textbf{64}(12): 1795--1804. doi: \href{https://doi.org/10.1139/F07-145}{10.1139/F07-145}.

\leavevmode\vadjust pre{\hypertarget{ref-Young1969}{}}%
Young, P.H. 1969. {The California Partyboat Fishery 1947-1967}. Fish Bulletin \textbf{145}.

\end{CSLReferences}
\end{document}
