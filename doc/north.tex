% Options for packages loaded elsewhere
\PassOptionsToPackage{unicode}{hyperref}
\PassOptionsToPackage{hyphens}{url}
%
\documentclass[
  english,
  a4paper,
]{article}
\title{DRAFT The status of Vermilion Rockfish (\emph{Sebastes miniatus}) and Sunset Rockfish (\emph{Sebastes crocotulus}) in U.S. waters off the coast of California north of Pt. Conception in 2021}
\author{true \and true \and true \and true \and true}
\date{}

\usepackage{amsmath,amssymb}
\usepackage{lmodern}
\usepackage{iftex}
\ifPDFTeX
  \usepackage[T1]{fontenc}
  \usepackage[utf8]{inputenc}
  \usepackage{textcomp} % provide euro and other symbols
\else % if luatex or xetex
  \usepackage{unicode-math}
  \defaultfontfeatures{Scale=MatchLowercase}
  \defaultfontfeatures[\rmfamily]{Ligatures=TeX,Scale=1}
\fi
% Use upquote if available, for straight quotes in verbatim environments
\IfFileExists{upquote.sty}{\usepackage{upquote}}{}
\IfFileExists{microtype.sty}{% use microtype if available
  \usepackage[]{microtype}
  \UseMicrotypeSet[protrusion]{basicmath} % disable protrusion for tt fonts
}{}
\makeatletter
\@ifundefined{KOMAClassName}{% if non-KOMA class
  \IfFileExists{parskip.sty}{%
    \usepackage{parskip}
  }{% else
    \setlength{\parindent}{0pt}
    \setlength{\parskip}{6pt plus 2pt minus 1pt}}
}{% if KOMA class
  \KOMAoptions{parskip=half}}
\makeatother
\usepackage{xcolor}
\IfFileExists{xurl.sty}{\usepackage{xurl}}{} % add URL line breaks if available
\IfFileExists{bookmark.sty}{\usepackage{bookmark}}{\usepackage{hyperref}}
\hypersetup{
  pdftitle={DRAFT The status of Vermilion Rockfish (Sebastes miniatus) and Sunset Rockfish (Sebastes crocotulus) in U.S. waters off the coast of California north of Pt. Conception in 2021},
  pdflang={en},
  hidelinks,
  pdfcreator={LaTeX via pandoc}}
\urlstyle{same} % disable monospaced font for URLs
\usepackage[margin=1in]{geometry}
\usepackage{longtable,booktabs,array}
\usepackage{calc} % for calculating minipage widths
% Correct order of tables after \paragraph or \subparagraph
\usepackage{etoolbox}
\makeatletter
\patchcmd\longtable{\par}{\if@noskipsec\mbox{}\fi\par}{}{}
\makeatother
% Allow footnotes in longtable head/foot
\IfFileExists{footnotehyper.sty}{\usepackage{footnotehyper}}{\usepackage{footnote}}
\makesavenoteenv{longtable}
\usepackage{graphicx}
\makeatletter
\def\maxwidth{\ifdim\Gin@nat@width>\linewidth\linewidth\else\Gin@nat@width\fi}
\def\maxheight{\ifdim\Gin@nat@height>\textheight\textheight\else\Gin@nat@height\fi}
\makeatother
% Scale images if necessary, so that they will not overflow the page
% margins by default, and it is still possible to overwrite the defaults
% using explicit options in \includegraphics[width, height, ...]{}
\setkeys{Gin}{width=\maxwidth,height=\maxheight,keepaspectratio}
% Set default figure placement to htbp
\makeatletter
\def\fps@figure{htbp}
\makeatother
\setlength{\emergencystretch}{3em} % prevent overfull lines
\providecommand{\tightlist}{%
  \setlength{\itemsep}{0pt}\setlength{\parskip}{0pt}}
\setcounter{secnumdepth}{5}
\newlength{\cslhangindent}
\setlength{\cslhangindent}{1.5em}
\newlength{\csllabelwidth}
\setlength{\csllabelwidth}{3em}
\newlength{\cslentryspacingunit} % times entry-spacing
\setlength{\cslentryspacingunit}{\parskip}
\newenvironment{CSLReferences}[2] % #1 hanging-ident, #2 entry spacing
 {% don't indent paragraphs
  \setlength{\parindent}{0pt}
  % turn on hanging indent if param 1 is 1
  \ifodd #1
  \let\oldpar\par
  \def\par{\hangindent=\cslhangindent\oldpar}
  \fi
  % set entry spacing
  \setlength{\parskip}{#2\cslentryspacingunit}
 }%
 {}
\usepackage{calc}
\newcommand{\CSLBlock}[1]{#1\hfill\break}
\newcommand{\CSLLeftMargin}[1]{\parbox[t]{\csllabelwidth}{#1}}
\newcommand{\CSLRightInline}[1]{\parbox[t]{\linewidth - \csllabelwidth}{#1}\break}
\newcommand{\CSLIndent}[1]{\hspace{\cslhangindent}#1}
\usepackage{booktabs}
\usepackage{longtable}
\usepackage{array}
\usepackage{multirow}
\usepackage{wrapfig}
\usepackage{float}
\usepackage{colortbl}
\usepackage{pdflscape}
\usepackage{tabu}
\usepackage{threeparttable}
\usepackage[normalem]{ulem}
\usepackage{makecell}
\usepackage{placeins}
\usepackage{caption}
\ifXeTeX
  % Load polyglossia as late as possible: uses bidi with RTL langages (e.g. Hebrew, Arabic)
  \usepackage{polyglossia}
  \setmainlanguage[]{english}
\else
  \usepackage[main=english]{babel}
% get rid of language-specific shorthands (see #6817):
\let\LanguageShortHands\languageshorthands
\def\languageshorthands#1{}
\fi
\ifLuaTeX
  \usepackage{selnolig}  % disable illegal ligatures
\fi

\begin{document}
\maketitle

{
\setcounter{tocdepth}{2}
\tableofcontents
}
\newcommand\CapeM{$40^\circ 10^\prime N$}
\newcommand\PtC{$34^\circ 27^\prime N$}
\newcommand\CAOR{$42^\circ 00^\prime N$}

\newpage

\includegraphics{cover_photo.png}
Two fish of the vermilion/sunset cryptic species pair. Confirmation of
species can only be determined via genetic analysis and species identification
of these two fish caught in the Santa Barbara channel at approximately 250 ft depth
is unknown. Photo courtesy of Sabrina Beyer (UCSC/NOAA).

\pagebreak
\pagenumbering{roman}
\setcounter{page}{1}

\renewcommand{\thetable}{\roman{table}}
\renewcommand{\thefigure}{\roman{figure}}

\setlength\parskip{0.5em plus 0.1em minus 0.2em}

\hypertarget{executive-summary}{%
\section*{Executive Summary}\label{executive-summary}}
\addcontentsline{toc}{section}{Executive Summary}

\hypertarget{stock}{%
\subsection*{Stock}\label{stock}}
\addcontentsline{toc}{subsection}{Stock}

This assessment reports the combined status of the vermilion rockfish (\emph{Sebastes miniatus}) and sunset rockfish (\emph{Sebastes crocotulus}), referred to as ``vermilion rockfish'' throughout, in U.S. waters off the coast of California north
of Point Conception ($34^\circ 27^\prime N$) using data through 2020. Genetic evidence suggests
overlapping distributions for the two species, with the majority of the sunset rockfish population occupying waters south of Point Conception. Alternative spatial structures for the vermilion rockfish assessment should be considered if additional data on stock structure and the distribution of the two species become available.

\hypertarget{catches}{%
\subsection*{Catches}\label{catches}}
\addcontentsline{toc}{subsection}{Catches}

Over the past decade, vermilion rockfish in the assessed area off the coast of
California in have been
primarily caught by the recreational fishery (Table \ref{tab:removalsES}).
Annual total mortality of catch and discards of vermilion rockfish have ranged between
76-204
mt, with total mortality (catch + discards) in 2020 of 139 mt.
Vermilion and sunset rockfishes landings from all sectors have historically been recorded as
``vermilion rockfish'' and sampling programs in California currently do not
differentiate between the two species.

Recreational removals in California prior to 2004 were only estimated at large
spatial scales (north and south of Point Conception) following the design of
the Marine Recreational Fisheries Statistics Survey (MRFSS). Recent sampling
(2004 -- present) by the California Recreational Fisheries Survey (CRFS) produces
estimates of vermilion landings and discard at a finer spatial resolution. Total
removals north of Point Conception increased steadily following World
War II, peaking in the late 1970s and 1980s with annual removals of
365 mt per year (Figure \ref{fig:catchES}). Recent
years have seen a steady increase in landings, with recreational fleets accounting
for the majority of total mortality.

\FloatBarrier

\begin{figure}
\centering
\includegraphics[width=1\textwidth,height=1\textheight]{C:/Stock_Assessments/VRML_Assessment_2021/Model_files/NCA/FINAL_base/Verm_NoCA_base_files_Day1_Request1_drop2020PR_weight_CCFRP_lengths_hessian/plots/catch2 landings stacked.png}
\caption{Catch histories by fleet used in the base model
(Commercial hook-and-line = COM\_HKL,
Commercial trawl = COM\_TWL, Commercial net = COM\_NET,
Recreational party/charter retained = REC\_PC, Recreational
private/rental retained = REC\_PR, Recreational party/charter
dead discards = REC\_PC\_DIS, Recreational private/rental dead
discards = REC\_PR\_DIS).\label{fig:catchES}}
\end{figure}

\begin{table}[H]

\caption{\label{tab:removalsES}Recent mortality (mt) by fleet and total landings summed across 
                all fleets in the model.}
\centering
\resizebox{\linewidth}{!}{
\begin{tabular}[t]{c>{\centering\arraybackslash}p{.6in}>{\centering\arraybackslash}p{.6in}>{\centering\arraybackslash}p{.6in}>{\centering\arraybackslash}p{.6in}>{\centering\arraybackslash}p{.6in}>{\centering\arraybackslash}p{.6in}>{\centering\arraybackslash}p{.6in}>{\centering\arraybackslash}p{.6in}}
\toprule
\multicolumn{1}{c}{\textbf{ }} & \multicolumn{3}{c}{\textbf{Commercial}} & \multicolumn{4}{c}{\textbf{Recreational}} & \multicolumn{1}{c}{\textbf{ }} \\
\multicolumn{4}{c}{ } & \multicolumn{2}{c}{Party/charter} & \multicolumn{2}{c}{Private/rental} & \multicolumn{1}{c}{ } \\
\cmidrule(l{3pt}r{3pt}){5-6} \cmidrule(l{3pt}r{3pt}){7-8}
Year & Hook-and-line & Trawl & Net & Retained &  Dead discards & Retained & Dead discards & Total Mortality\\
\midrule
\cellcolor{gray!6}{2011} & \cellcolor{gray!6}{10.047} & \cellcolor{gray!6}{0.000} & \cellcolor{gray!6}{0.000} & \cellcolor{gray!6}{40.278} & \cellcolor{gray!6}{0.293} & \cellcolor{gray!6}{49.362} & \cellcolor{gray!6}{0.145} & \cellcolor{gray!6}{100.124}\\
2012 & 9.400 & 0.006 & 0.000 & 36.031 & 0.241 & 41.178 & 0.226 & 87.083\\
\cellcolor{gray!6}{2013} & \cellcolor{gray!6}{13.845} & \cellcolor{gray!6}{0.005} & \cellcolor{gray!6}{0.000} & \cellcolor{gray!6}{21.101} & \cellcolor{gray!6}{0.121} & \cellcolor{gray!6}{40.642} & \cellcolor{gray!6}{0.123} & \cellcolor{gray!6}{75.837}\\
2014 & 14.139 & 0.015 & 0.023 & 21.088 & 0.042 & 41.698 & 0.214 & 77.220\\
\cellcolor{gray!6}{2015} & \cellcolor{gray!6}{18.172} & \cellcolor{gray!6}{0.410} & \cellcolor{gray!6}{0.010} & \cellcolor{gray!6}{40.022} & \cellcolor{gray!6}{0.109} & \cellcolor{gray!6}{64.580} & \cellcolor{gray!6}{0.226} & \cellcolor{gray!6}{123.528}\\
2016 & 13.271 & 0.094 & 0.000 & 37.986 & 0.192 & 60.276 & 0.261 & 112.080\\
\cellcolor{gray!6}{2017} & \cellcolor{gray!6}{14.226} & \cellcolor{gray!6}{0.062} & \cellcolor{gray!6}{0.002} & \cellcolor{gray!6}{92.320} & \cellcolor{gray!6}{0.384} & \cellcolor{gray!6}{58.287} & \cellcolor{gray!6}{0.279} & \cellcolor{gray!6}{165.560}\\
2018 & 19.041 & 0.619 & 0.000 & 88.035 & 0.144 & 72.411 & 0.245 & 180.495\\
\cellcolor{gray!6}{2019} & \cellcolor{gray!6}{19.593} & \cellcolor{gray!6}{0.039} & \cellcolor{gray!6}{0.000} & \cellcolor{gray!6}{91.995} & \cellcolor{gray!6}{0.310} & \cellcolor{gray!6}{91.878} & \cellcolor{gray!6}{0.630} & \cellcolor{gray!6}{204.445}\\
2020 & 19.930 & 0.017 & 0.000 & 55.376 & 0.159 & 63.260 & 0.264 & 139.006\\
\bottomrule
\end{tabular}}
\end{table}

\FloatBarrier

\hypertarget{data-and-assessment}{%
\subsection*{Data and Assessment}\label{data-and-assessment}}
\addcontentsline{toc}{subsection}{Data and Assessment}

A full assessment was attempted in 2005, but not accepted for management and a
data moderate assessment in 2013 was not reviewed. As such, this is the first benchmark assessment for vermilion and sunset rockfishes. The 2021 assessment uses
Stock Synthesis 3 (version V3.30.17.0). The assessment is a two-sex model, with
the population spanning from Point Conception ($34^\circ 27^\prime N$) to the California/Oregon border ($42^\circ 00^\prime N$).
The assessment model operates on an annual time step covering the period 1875 to
2020 (not including forecast years) and assumes an unfished
population prior to 1875. Population dynamics are modeled for ages 0 through 70,
with age-70 being the accumulator age.

The model is conditioned on catch from two sectors (commercial and recreational)
divided among seven fleets, and is informed by five abundance indices
(one fishery-independent survey, two CPUE indices from shore-based recreational fishery sampling programs,
and two CPUE indices from recreational onboard party/charter boat observer programs). The model is also fit to length composition data from fishery-independent and fishery-dependent sources, as well as age compositions conditioned on length. Discards for the commercial
fleets are not included in the model. Commercial discards of vermilion are a small
fraction of the total mortality and data on commercial discard length composition is
limited. The recreational fishery is split into four fleets, one discard and one
retained fish fleet each for the private/rental and the party/charter boat modes. The model also incorporates an updated length-weight relationship, length-based maturity schedule, and fecundity-at-length function.

The assessment estimates parameters for natural mortality of females and males,
and sex-specific
growth parameters. Year class strength is estimated as deviations from a Beverton-Holt stock-recruitment relationship beginning in 1970. Steepness of
the Beverton-Holt stock-recruitment relationship is fixed at the mean of the prior,
0.72.

\FloatBarrier

\hypertarget{stock-biomass}{%
\subsection*{Stock Biomass}\label{stock-biomass}}
\addcontentsline{toc}{subsection}{Stock Biomass}

Spawning output of vermilion rockfish was estimated to be
1145 million eggs in 2021 (95\% asymptotic interval:
915 - 1376 million eggs) or
43\%
(95\% asymptotic interval:
25\% - 61\% million eggs)
of unfished spawning output (``depletion,'' Table \ref{tab:ssbES}). Depletion
is a ratio of the estimated spawning output in a particular year relative to
estimated
unfished, equilibrium spawning output.

In northern California, spawning output declined rapidly in the 1970s and early
1980s, likely falling below the minimum stock size threshold for a number of
years in the 1990s and early 2000s, followed by a steady recovery since the
late 2000s (Figures \ref{fig:ssbES} and \ref{fig:deplES}). The spawning output
in 2021 is just above the management target (40\% of unfished spawning output).

\begin{figure}
\centering
\includegraphics[width=1\textwidth,height=1\textheight]{C:/Stock_Assessments/VRML_Assessment_2021/Model_files/NCA/FINAL_base/Verm_NoCA_base_files_Day1_Request1_drop2020PR_weight_CCFRP_lengths_hessian//plots/ts7_Spawning_output_with_95_asymptotic_intervals_intervals.png}
\caption{Estimated time series of spawning output (solid line with circles) with approximate 95\% asymptotic confidence intervals (dashed lines).\label{fig:ssbES}}
\end{figure}

\begin{figure}
\centering
\includegraphics[width=1\textwidth,height=1\textheight]{C:/Stock_Assessments/VRML_Assessment_2021/Model_files/NCA/FINAL_base/Verm_NoCA_base_files_Day1_Request1_drop2020PR_weight_CCFRP_lengths_hessian//plots/ts9_Relative_spawning_output_intervals.png}
\caption{Estimated time series of spawning output relative to unfished spawning output (solid line with circles) with approximate 95\% asymptotic confidence intervals (dashed lines).\label{fig:deplES}}
\end{figure}

\begin{table}[H]

\caption{\label{tab:ssbES}Estimated recent trend in spawning output and the fraction unfished and the approximate 95\% asymtotic confidence intervals.}
\centering
\begin{tabular}[t]{c>{\centering\arraybackslash}p{.6in}>{\centering\arraybackslash}p{.6in}>{\centering\arraybackslash}p{.6in}|>{\centering\arraybackslash}p{.6in}>{\centering\arraybackslash}p{.6in}>{\centering\arraybackslash}p{.6in}}
\toprule
\multicolumn{1}{c}{\textbf{ }} & \multicolumn{3}{c}{\textbf{Spawning Output}} & \multicolumn{3}{c}{\textbf{Fraction Unfished}} \\
\cmidrule(l{3pt}r{3pt}){2-4} \cmidrule(l{3pt}r{3pt}){5-7}
Year & Estimate & Lower Interval & Upper Interval & Estimate & Lower Interval & Upper Interval\\
\midrule
2011 & 431.973 & 244.002 & 619.944 & 0.377 & 0.227 & 0.527\\
2012 & 435.431 & 244.955 & 625.907 & 0.380 & 0.229 & 0.531\\
2013 & 442.395 & 249.226 & 635.564 & 0.386 & 0.234 & 0.539\\
2014 & 454.034 & 257.314 & 650.754 & 0.396 & 0.241 & 0.552\\
2015 & 469.146 & 267.897 & 670.395 & 0.410 & 0.251 & 0.568\\
2016 & 479.639 & 273.578 & 685.700 & 0.419 & 0.257 & 0.581\\
2017 & 490.602 & 279.902 & 701.302 & 0.428 & 0.263 & 0.594\\
2018 & 490.707 & 275.944 & 705.470 & 0.428 & 0.260 & 0.597\\
2019 & 487.751 & 269.376 & 706.126 & 0.426 & 0.254 & 0.598\\
2020 & 482.178 & 260.377 & 703.979 & 0.421 & 0.246 & 0.596\\
2021 & 489.439 & 263.228 & 715.650 & 0.427 & 0.249 & 0.606\\
\bottomrule
\end{tabular}
\end{table}

\FloatBarrier

\hypertarget{recruitment}{%
\subsection*{Recruitment}\label{recruitment}}
\addcontentsline{toc}{subsection}{Recruitment}

Recruitment deviations were estimated from 1970-2020 with a recent, strong
recruitment in 2016 that has contributed to the recent increase in vermilion
biomass in northern California (Table \ref{tab:recrES}; Figure \ref{fig:recruitsES}).
The second highest estimated recruitment occurred in 1985 and is more certain
than the estimated 2016 recruitment. Overall, variability in recruitment is
average (to low) in the years following 2016.

\begin{figure}
\centering
\includegraphics[width=1\textwidth,height=1\textheight]{C:/Stock_Assessments/VRML_Assessment_2021/Model_files/NCA/FINAL_base/Verm_NoCA_base_files_Day1_Request1_drop2020PR_weight_CCFRP_lengths_hessian//plots/ts11_Age-0_recruits_(1000s)_with_95_asymptotic_intervals.png}
\caption{Age-0 recruits (1,000s) with approximate 95\% asymptotic confidence intervals.\label{fig:recruitsES}}
\end{figure}

\begin{table}[H]

\caption{\label{tab:recrES}Estimated recent trend in recruitment and recruitment 
                deviations and the approximate 95\% asymptotic confidence intervals.}
\centering
\begin{tabular}[t]{r>{\raggedleft\arraybackslash}p{.6in}>{\raggedleft\arraybackslash}p{.6in}>{\raggedleft\arraybackslash}p{.6in}|>{\raggedleft\arraybackslash}p{.6in}>{\raggedleft\arraybackslash}p{.6in}>{\raggedleft\arraybackslash}p{.6in}}
\toprule
\multicolumn{1}{c}{\textbf{ }} & \multicolumn{3}{c}{\textbf{Recruitment}} & \multicolumn{3}{c}{\textbf{Recruitment Deviations}} \\
\cmidrule(l{3pt}r{3pt}){2-4} \cmidrule(l{3pt}r{3pt}){5-7}
Year & Estimate & Lower Interval & Upper Interval & Estimate & Lower Interval & Upper Interval\\
\midrule
2011 & 225 & 116 & 437 & -0.397 & -0.956 & 0.163\\
2012 & 408 & 224 & 741 & 0.196 & -0.279 & 0.672\\
2013 & 466 & 242 & 896 & 0.326 & -0.220 & 0.872\\
2014 & 476 & 239 & 946 & 0.341 & -0.248 & 0.930\\
2015 & 277 & 125 & 616 & -0.215 & -0.937 & 0.506\\
2016 & 1536 & 814 & 2901 & 1.472 & 0.963 & 1.980\\
2017 & 163 & 65 & 409 & -0.800 & -1.680 & 0.081\\
2018 & 387 & 147 & 1022 & 0.048 & -0.892 & 0.988\\
2019 & 373 & 138 & 1004 & 0.003 & -0.964 & 0.970\\
2020 & 374 & 138 & 1010 & 0.009 & -0.961 & 0.978\\
2021 & 372 & 140 & 991 & 0.000 & -0.980 & 0.980\\
\bottomrule
\end{tabular}
\end{table}

\FloatBarrier

\hypertarget{exploitation-status}{%
\subsection*{Exploitation Status}\label{exploitation-status}}
\addcontentsline{toc}{subsection}{Exploitation Status}

The annual (equilibrium) spawning potential ratio (SPR) for vermilion was above target from 2017-2019 (Table \ref{tab:exploitES}, Figure \ref{fig:1-sprES}). Prior to 2011, the fishing intensity exceeded the target for a number of years, regularly reaching levels 50\% above target in the 1980s and 1990s (Figure \ref{fig:1-sprES}). As with current estimates of spawning output, recent estimates of exploitation status are highly uncertain, ranging from 68\% to 129\% of target in 2020 (Table \ref{tab:exploitES}). As a percentage of total biomass (ages 4+), California harvest rates peaked in the 1980s and 1990s, but have since declined to levels below 10\% for the past decade (Figure \ref{fig:FmortalityES}). Harvest rates in northern California were near target in 2020, but above target in the three previous years, and the stock is just below the target biomass (Figure \ref{fig:phaseES}). However, the harvest rate in 2019 was above target, and may be more representative of future catches, all else equal, given reductions in fishing activity during the 2020 pandemic. The equilibrium yield curve is shifted left, as expected from the Beverton-Holt steepness parameter fixed at 0.72 (Figure \ref{fig:yield2ES}).

\begin{table}[H]

\caption{\label{tab:exploitES}Estimated recent trend in the relative fishing intensity
                ($\frac{1-SPR}{1-SPR_{50\%}}$, 
                where SPR is the spawning potential ratio) and the exploitation rate, 
                with approximate 95\% asymptotic confidence intervals.}
\centering
\begin{tabular}[t]{r>{\raggedleft\arraybackslash}p{.6in}>{\raggedleft\arraybackslash}p{.6in}>{\raggedleft\arraybackslash}p{.6in}|>{\raggedleft\arraybackslash}p{.6in}>{\raggedleft\arraybackslash}p{.6in}>{\raggedleft\arraybackslash}p{.6in}}
\toprule
\multicolumn{1}{c}{\textbf{ }} & \multicolumn{3}{c}{\textbf{Relative Fishing Intensity}} & \multicolumn{3}{c}{\textbf{Exploitation Rate}} \\
\cmidrule(l{3pt}r{3pt}){2-4} \cmidrule(l{3pt}r{3pt}){5-7}
Year & Estimate & Lower Interval & Upper Interval & Estimate & Lower Interval & Upper Interval\\
\midrule
2011 & 0.939 & 0.653 & 1.224 & 0.061 & 0.037 & 0.085\\
2012 & 0.826 & 0.558 & 1.094 & 0.051 & 0.031 & 0.071\\
2013 & 0.715 & 0.469 & 0.961 & 0.041 & 0.025 & 0.056\\
2014 & 0.701 & 0.461 & 0.941 & 0.040 & 0.024 & 0.055\\
2015 & 0.966 & 0.684 & 1.249 & 0.062 & 0.038 & 0.087\\
2016 & 0.905 & 0.629 & 1.181 & 0.058 & 0.035 & 0.080\\
2017 & 1.108 & 0.808 & 1.408 & 0.077 & 0.045 & 0.108\\
2018 & 1.164 & 0.861 & 1.467 & 0.081 & 0.047 & 0.115\\
2019 & 1.248 & 0.943 & 1.554 & 0.094 & 0.054 & 0.133\\
2020 & 0.990 & 0.684 & 1.296 & 0.061 & 0.035 & 0.088\\
\bottomrule
\end{tabular}
\end{table}

\begin{figure}
\centering
\includegraphics[width=1\textwidth,height=1\textheight]{C:/Stock_Assessments/VRML_Assessment_2021/Model_files/NCA/FINAL_base/Verm_NoCA_base_files_Day1_Request1_drop2020PR_weight_CCFRP_lengths_hessian//plots/SPR3_ratiointerval.png}
\caption{Timeseries of relative fishing intensity (\(\frac{1-SPR}{1-SPR_{50\%}}\) where SPR is the spawning potential ratio) with approximate 95\% asymptotic confidence intervals (dashed lines).\label{fig:1-sprES}}
\end{figure}

\begin{figure}
\centering
\includegraphics[width=1\textwidth,height=1\textheight]{C:/Stock_Assessments/VRML_Assessment_2021/Model_files/NCA/FINAL_base/Verm_NoCA_base_files_Day1_Request1_drop2020PR_weight_CCFRP_lengths_hessian//plots/ts_summaryF.png}
\caption{Time-series of estimated summary harvest rate (total catch divided by age-4 and older biomass) for the base case model with approximate 95\% asymptotic confidence intervals (veritcal lines).\label{fig:FmortalityES}}
\end{figure}

\begin{figure}
\centering
\includegraphics[width=1\textwidth,height=1\textheight]{C:/Stock_Assessments/VRML_Assessment_2021/Model_files/NCA/FINAL_base/Verm_NoCA_base_files_Day1_Request1_drop2020PR_weight_CCFRP_lengths_hessian//plots/SPR4_phase.png}
\caption{Phase plot of the relative biomass (also referred to as fraction unfished) versus the SPR ratio where each point represents the biomass ratio at the start of the year and the relative fishing intensity in that same year. Lines through the final point show the 95\% intervals based on the asymptotic uncertainty for each dimension. The shaded ellipse is a 95\% region which accounts for the estimated correlations between the biomass ratio and SPR ratio. Fishing intensity in 2020 was reduced to due the pandemic.\label{fig:phaseES}}
\end{figure}

\begin{figure}
\centering
\includegraphics[width=1\textwidth,height=1\textheight]{C:/Stock_Assessments/VRML_Assessment_2021/Model_files/NCA/FINAL_base/Verm_NoCA_base_files_Day1_Request1_drop2020PR_weight_CCFRP_lengths_hessian//plots/yield2_yield_curve_with_refpoints.png}
\caption{Equilibrium yield curve for the base case model with management quantities. Values are based on the 2020
fishery selectivities.\label{fig:yield2ES}}
\end{figure}

\FloatBarrier

\hypertarget{ecosystem-considerations}{%
\subsection*{Ecosystem Considerations}\label{ecosystem-considerations}}
\addcontentsline{toc}{subsection}{Ecosystem Considerations}

In this assessment, ecosystem considerations were not explicitly included in analyses.
This is primarily due to a lack of relevant data that could contribute ecosystem-related quantitative information for the assessment.

Vermilion/sunset rockfish are described as feeding on a wide range of both
pelagic and benthic prey items, including forage fish species such as anchovies
and mesopelagic fishes, squid, krill and octopus, as well as sporadically abundant
pelagic organisms such as pyrosomes, salps and pelagic red crabs.

As with most other rockfish and groundfish in the California Current, recruitment,
or cohort (year-class) strength appears to be highly variable for the vermilion/sunset
rockfish complex, with only a modest apparent relationship to estimated levels of spawning output. Oceanographic and ecosystem factors are widely recognized to be key drivers of
recruitment variability for most species of groundfish, as well as most elements
of California Current food webs. Although it is feasible that
ecosystem factors, the results of pre-recruit surveys for co-occurring species,
or the results of other groundfish assessments might ultimately be used to
forecast recruitment for more data-limited stocks such as vermilion/sunset. Such approaches would require more
development and evaluation. Consequently, environmental factors are not
explicitly considered in this assessment.

\FloatBarrier

\hypertarget{reference-points}{%
\subsection*{Reference Points}\label{reference-points}}
\addcontentsline{toc}{subsection}{Reference Points}

Reference point and management quantities for the vermilion rockfish base case
model can be found in Table \ref{tab:referenceES}. In 2021, spawning output
relative to unfished spawning output (``depletion'') is estimated at
43\% (95\% asymptotic interval:
25\% - 61\%).
This stock assessment estimates that vermilion rockfish in the north is above
the biomass target (\(SB_{40\%}\)), and well above the minimum stock size
threshold (\(SB_{25\%}\)). Unfished age four-plus biomass is estimated to be
6342 mt in the base case model (95\% asymptotic interval:
5667 - 7017 mt).
The target spawning output (\(SB_{40\%}\)) is 458 million eggs
(95\% asymptotic interval: 366 - 550 million eggs),
which corresponds with an equilibrium yield of 146 mt
(95\% asymptotic interval: 123 - 168 mt).
Equilibrium yield at the proxy \(F_{MSY}\)
harvest rate corresponding to \(SPR_{50\%}\) is 139 mt
(95\% asymptotic interval: 118 - 160 mt),
Table \ref{tab:referenceES} and Figure \ref{fig:yield2ES}).

\begin{table}[H]

\caption{\label{tab:referenceES}Summary of reference points and management quantities including estimates of the approximate 95\% asymtotic confidence intervals.}
\centering
\resizebox{\linewidth}{!}{
\begin{tabular}[t]{lrrr}
\toprule
 & Estimate & Lower Interval & Upper Interval\\
\midrule
Unfished Spawning Output & 1145.180 & 914.835 & 1375.525\\
Unfished Age 4+ Biomass (mt) & 6341.790 & 5666.596 & 7016.984\\
Unfished Recruitment ($R_0$) & 420.186 & 299.040 & 541.332\\
Spawning Output (2021) & 489.439 & 263.228 & 715.650\\
Fraction Unfished (2021) & 0.427 & 0.249 & 0.606\\
\addlinespace[0.3em]
\multicolumn{4}{l}{\textbf{Reference Points Based on $SB_{40\%}$}}\\
\hspace{1em}Proxy Spawning Output $SB_{40\%}$ & 458.073 & 365.935 & 550.211\\
\hspace{1em}SPR Resulting in $SB_{40\%}$ & 0.458 & 0.458 & 0.458\\
\hspace{1em}Exploitation Rate Resulting in $SB_{40\%}$ & 0.071 & 0.060 & 0.083\\
\hspace{1em}Yield with SPR Based On $SB_{40\%}$ (mt) & 145.614 & 123.238 & 167.990\\
\addlinespace[0.3em]
\multicolumn{4}{l}{\textbf{Reference Points Based on SPR Proxy for MSY}}\\
\hspace{1em}Proxy Spawning Output ($SPR_{50\%}$) & 510.928 & 408.159 & 613.697\\
\hspace{1em}$SPR_{50\%}$ & 0.500 &  & \\
\hspace{1em}Exploitation Rate Corresponding to $SPR_{50\%}$ & 0.062 & 0.052 & 0.073\\
\hspace{1em}Yield with $SPR_{50\%}$ at $SB_{SPR}$ (mt) &  &  & \\
\addlinespace[0.3em]
\multicolumn{4}{l}{\textbf{Reference Points Based on Estimated MSY Values}}\\
\hspace{1em}Spawning Output at MSY ($SB_{MSY}$) & 308.931 & 249.480 & 368.382\\
\hspace{1em}$SPR_{MSY}$ & 0.341 & 0.332 & 0.349\\
\hspace{1em}Exploitation Rate Corresponding to $SPR_{MSY}$ & 0.104 & 0.087 & 0.121\\
\hspace{1em}MSY (mt) & 155.029 & 130.706 & 179.352\\
\bottomrule
\end{tabular}}
\end{table}

\FloatBarrier

\hypertarget{management-performance}{%
\subsection*{Management Performance}\label{management-performance}}
\addcontentsline{toc}{subsection}{Management Performance}

Vermilion rockfish have been managed as part of the minor shelf rockfish
complex in the Pacific Coast Groundfish Fishery Management Plan. North of $40^\circ 10^\prime N$,
total mortality of the minor shelf rockfish complex has exceeded the OFL since
2011. South of $40^\circ 10^\prime N$, total mortality of the minor shelf rockfish complex has
exceeded the OFL since 2015, and exceeded the ABC in most years since 2011.
Total mortality estimates from the NWFSC are not yet available for 2020.
A summary of these values as well as other base case summary results can be found
in Tables \ref{tab:summaryES} and \ref{tab:managementES}.

Results from post-STAR base models in all areas (southern California, northern
California, Oregon, and Washington) are presented in Table \ref{tab:CombinedRefPtsES}.
The fraction of the northern CA model allocated to the northern management area
(north of $40^\circ 10^\prime N$) is based on an Appendix in northern CA assessment.

\begin{table}[H]

\caption{\label{tab:summaryES}Summary of recent estimates and managment quantities for vermilion rockfish in the assessed area.}
\centering
\resizebox{\linewidth}{!}{
\begin{tabular}[t]{lrrrrrrrrrrr}
\toprule
Quantity & 2011 & 2012 & 2013 & 2014 & 2015 & 2016 & 2017 & 2018 & 2019 & 2020 & 2021\\
\midrule
Total catch (mt) & 100.124 & 87.083 & 75.837 & 77.220 & 123.528 & 112.080 & 165.560 & 180.495 & 204.445 & 139.006 & \\
$(1-SPR)/(1-SPR_{50\%})$ & 0.939 & 0.826 & 0.715 & 0.701 & 0.966 & 0.905 & 1.108 & 1.164 & 1.248 & 0.990 & \\
Annual F & 0.061 & 0.051 & 0.041 & 0.040 & 0.062 & 0.058 & 0.077 & 0.081 & 0.094 & 0.061 & \\
Age 4+ Biomass (mt) & 2741.110 & 2813.220 & 2961.290 & 3037.340 & 3087.710 & 3118.040 & 3173.250 & 3184.580 & 3135.420 & 3393.480 & 6335.880\\
\addlinespace[0.3em]
\multicolumn{12}{l}{\textbf{Spawning Output ($10^6$)}}\\
\hspace{1em}Estimate & 431.973 & 435.431 & 442.395 & 454.034 & 469.146 & 479.639 & 490.602 & 490.707 & 487.751 & 482.178 & 489.439\\
\hspace{1em}Lower Interval & 244.002 & 244.955 & 249.226 & 257.314 & 267.897 & 273.578 & 279.902 & 275.944 & 269.376 & 260.377 & 263.228\\
\hspace{1em}Upper Interval & 619.944 & 625.907 & 635.564 & 650.754 & 670.395 & 685.700 & 701.302 & 705.470 & 706.126 & 703.979 & 715.650\\
\addlinespace[0.3em]
\multicolumn{12}{l}{\textbf{Recruits (1,000s)}}\\
\hspace{1em}Estimate & 224.973 & 407.824 & 465.847 & 475.537 & 277.184 & 1536.160 & 162.592 & 387.483 & 372.609 & 373.837 & 371.777\\
\hspace{1em}Lower Interval & 115.906 & 224.497 & 242.276 & 238.986 & 124.805 & 813.510 & 64.605 & 146.879 & 138.265 & 138.332 & 139.533\\
\hspace{1em}Upper Interval & 436.670 & 740.858 & 895.729 & 946.231 & 615.609 & 2900.748 & 409.194 & 1022.226 & 1004.144 & 1010.280 & 990.579\\
\addlinespace[0.3em]
\multicolumn{12}{l}{\textbf{Fraction Unfished}}\\
\hspace{1em}Estimate & 0.377 & 0.380 & 0.386 & 0.396 & 0.410 & 0.419 & 0.428 & 0.428 & 0.426 & 0.421 & 0.427\\
\hspace{1em}Lower Interval & 0.227 & 0.229 & 0.234 & 0.241 & 0.251 & 0.257 & 0.263 & 0.260 & 0.254 & 0.246 & 0.249\\
\hspace{1em}Upper Interval & 0.527 & 0.531 & 0.539 & 0.552 & 0.568 & 0.581 & 0.594 & 0.597 & 0.598 & 0.596 & 0.606\\
\bottomrule
\end{tabular}}
\end{table}

\begin{table}[H]

\caption{\label{tab:managementES}Annual estimates of total mortality, overfishing limit (OFL), acceptable biological catch (ABC), annual catch limit (ACL) for vermilion in the minor shelf rockfish complex as reported in the GEMM report (NWFSC).}
\centering
\resizebox{\linewidth}{!}{
\begin{tabular}[t]{lrrrrrrrrrrrr}
\toprule
 & 2011 & 2012 & 2013 & 2014 & 2015 & 2016 & 2017 & 2018 & 2019 & 2020 & 2021 & 2022\\
\midrule
\addlinespace[0.3em]
\multicolumn{13}{l}{\textbf{North of 40°10' N}}\\
\hspace{1em}OFL & 11.127 & 11.127 & 9.717 & 9.717 & 9.717 & 9.717 & 9.720 & 9.720 & 9.720 & 9.720 & 9.700 & 9.700\\
\hspace{1em}ABC & 5.564 & 5.564 & 8.104 & 8.104 & 8.104 & 8.104 & 8.104 & 8.104 & 8.104 & 8.104 & 7.547 & 7.547\\
\hspace{1em}Total landings & 15.249 & 18.695 & 14.149 & 10.504 & 13.472 & 12.104 & 20.602 & 22.949 & 25.696 &  &  & \\
\hspace{1em}CA rec landings & 4.209 & 4.867 & 2.657 & 2.950 & 5.018 & 4.549 & 6.490 & 7.631 & 7.884 &  &  & \\
\hspace{1em}OR rec landings & 6.102 & 9.150 & 6.305 & 3.949 & 4.653 & 3.689 & 8.798 & 9.199 & 9.252 &  &  & \\
\hspace{1em}WA rec landings & 1.001 & 0.911 & 1.279 & 0.960 & 1.141 & 0.997 & 0.731 & 1.151 & 2.497 &  &  & \\
\hspace{1em}Commercial landings & 3.935 & 3.767 & 3.906 & 2.644 & 2.661 & 2.799 & 4.557 & 4.966 & 6.063 &  &  & \\
\hspace{1em}Research & 0.002 &  & 0.002 & 0.002 &  & 0.069 & 0.026 & 0.002 &  &  &  & \\
\midrule
\addlinespace[0.3em]
\multicolumn{13}{l}{\textbf{South of 40°10' N}}\\
\hspace{1em}OFL & 308.359 & 308.359 & 269.276 & 269.276 & 269.276 & 269.276 & 269.280 & 269.280 & 269.280 & 269.280 & 269.280 & 269.280\\
\hspace{1em}ABC & 154.179 & 154.179 & 224.576 & 224.576 & 224.576 & 224.576 & 224.580 & 224.580 & 224.580 & 224.580 & 209.515 & 209.515\\
\hspace{1em}Total landings & 210.310 & 235.216 & 237.074 & 197.043 & 334.984 & 292.375 & 341.207 & 344.454 & 484.967 &  &  & \\
\hspace{1em}CA rec landings & 191.437 & 216.480 & 208.198 & 167.572 & 291.779 & 260.162 & 287.493 & 278.158 & 413.946 &  &  & \\
\hspace{1em}Commercial landings & 16.928 & 16.642 & 26.601 & 26.607 & 39.669 & 29.148 & 48.195 & 59.644 & 67.189 &  &  & \\
\hspace{1em}Research & 1.944 & 2.094 & 2.275 & 2.863 & 3.536 & 3.065 & 5.519 & 6.652 & 3.832 &  &  & \\
\bottomrule
\end{tabular}}
\end{table}

\FloatBarrier

\hypertarget{unresolved-problems-and-major-uncertainties}{%
\subsection*{Unresolved Problems and Major Uncertainties}\label{unresolved-problems-and-major-uncertainties}}
\addcontentsline{toc}{subsection}{Unresolved Problems and Major Uncertainties}

The stratification of assessment areas was based on consideration of population structure identified in genetic analyses, differences in historical exploitation, differences in length composition within fleets, and availability of data sources. The STAR Panel discussed the potential for alternative stratifications such as north and south of Cape Mendocino depending on the results of future analyses of population structure.

Natural mortality remains the primary axis of uncertainty across assessment areas. Additional collection of otoliths from across the range of the stock and continued ageing of available otoliths may help reduce uncertainty in the future. In the relatively data-rich southern model, steepness was estimated and uncertainties in both natural mortality and steepness were considered when determining alternative states of nature.

\FloatBarrier

\hypertarget{decision-table-and-forecasts}{%
\subsection*{Decision Table and Forecasts}\label{decision-table-and-forecasts}}
\addcontentsline{toc}{subsection}{Decision Table and Forecasts}

The forecasts of stock abundance and yield were developed using the post-STAR
base model, with the forecast projections presented in
Table \ref{tab:projectionES}. The total catches in 2021 and 2022 are set to
the projected catch from the California Department of Fish and Wildlife (CDFW) by
sector and model region, i.e., allocated north and south of $34^\circ 27^\prime N$ in California.

Uncertainty in the forecasts is based upon the three states of nature agreed upon
at the STAR panel, reflecting three different natural mortality rates. The steepness
parameter of the Beverton-Holt stock-recruit curve was fixed in the base model
and in all of the forecasts. The northern California model is not data rich and
while there is uncertainty in steepness, it was not well estimated in the base
model when natural mortality was also estimated. The alternative states of nature
maintain the
female to male natural mortality rate ratio from the base model. To capture the
75\% interval around the negative log-likelihood, alternate states were identified
within 0.66 negative log-likelihood points from the base model where female
\(M\) = 0.0856 and male \(M\) = 0.0805.
The high state of nature fixes female \(M\) = 0.0956 and male \(M\) = 0.08989.
For the low state of nature, female \(M\) = 0.0769 and male \(M\) = 0.07231.

For reference, the base model predicted \(\sigma\) = 0.246. The buffers between the OFL and ABC were calculated assuming a category 2 stock, with \(\sigma\) = 1.0
and a \(p^*\) = 0.45. Alternative catch streams (rows in the table) include \(\sigma\) = 1.0
with a \(p^*\) = 0.4, and removals of long-term equilibrium catch with and without a buffer assuming \(\sigma\) = 1.0 with a \(p^*\) = 0.45. The buffer multiplier with \(p^*\) = 0.45 ranges from 0.874 in 2023 ramping to 0.803 in 2032.

Current forecasts based on the alternative states of nature and requested catch streams project that the stock will remain above the target threshold of 40\% in 2032 (Table \ref{tab:DecisionES}). In all of the scenarios of the low state of nature, the stock remains below the
target threshold of 40\% until 2026 or 2027.
The base model with the base catches results in an increasing stock over the period from
2023-2032. In all scenarios the catch significantly decreases from 2022 to 2023 in a
all catch scenarios; assumed catch in 2022 is 227 mt, and 2023 catches from the base
model range from 118-139 mt. The base model includes a portion of the stock within
the northern management unit (north of $40^\circ 10^\prime N$). An analysis based on the private/rental
mode index through 2019 suggests
that 4.44\% of the catches from this model should be apportioned to the northern
management unit for vermilion rockfish.

The STAT cautions that the GMT projections for catches in 2021-2022 (22 mt per year) exceed the maximum sustainable yield according to both proxies (\(B_{40\%}\) and \(SPR_{50\%}\)) as well as the MSY value based on the estimated value of steepness (Table \ref{tab:referenceES}). The northern California stock is just above target biomass in 2021 (43\% of unfished spawning ouptut), so these catch levels are unlikely to result in significant stock declines over a short period of time. However, similar catch levels would exceed the overfishing limits (OFL) if carried forward for 2023 and beyond (Table \ref{tab:CombinedRefPtsES}), and would be unsustainable in the long term. Given recent and projected near-term exploitation levels, and especially if vermilion and sunset rockfishes continue to be managed as part of the minor shelf rockfish complex, the STAT recommends regular monitoring of total mortality for these two species to avoid excessive stock depletion and potential loss of yield.

\begin{table}[H]

\caption{\label{tab:projectionES}Projections of potential OFLs (mt), ABCs (mt), estimated age 4+ biomass (mt), estimated spawning output ($10^6$ eggs) and fraction unfished.}
\centering
\begin{tabular}[t]{c>{\centering\arraybackslash}p{.8in}>{\centering\arraybackslash}p{.8in}>{\centering\arraybackslash}p{.8in}>{\centering\arraybackslash}p{.8in}>{\centering\arraybackslash}p{.8in}}
\toprule
Year & Predicted OFL & ABC Catch & Age 4+ Biomass & Spawning Output & Fraction Unfished\\
\midrule
\cellcolor{gray!6}{2021} & \cellcolor{gray!6}{148.994} & \cellcolor{gray!6}{148.994} & \cellcolor{gray!6}{3459.01} & \cellcolor{gray!6}{489.439} & \cellcolor{gray!6}{0.427389}\\
2022 & 156.383 & 156.383 & 3539.37 & 501.884 & 0.438257\\
\cellcolor{gray!6}{2023} & \cellcolor{gray!6}{161.401} & \cellcolor{gray!6}{141.065} & \cellcolor{gray!6}{3590.72} & \cellcolor{gray!6}{518.613} & \cellcolor{gray!6}{0.452865}\\
2024 & 164.761 & 142.519 & 3638.45 & 538.451 & 0.470188\\
\cellcolor{gray!6}{2025} & \cellcolor{gray!6}{166.078} & \cellcolor{gray!6}{142.328} & \cellcolor{gray!6}{3666.95} & \cellcolor{gray!6}{555.898} & \cellcolor{gray!6}{0.485423}\\
2026 & 165.981 & 140.918 & 3681.40 & 569.855 & 0.497611\\
\cellcolor{gray!6}{2027} & \cellcolor{gray!6}{165.064} & \cellcolor{gray!6}{138.819} & \cellcolor{gray!6}{3686.31} & \cellcolor{gray!6}{580.383} & \cellcolor{gray!6}{0.506804}\\
2028 & 163.786 & 136.434 & 3685.38 & 587.989 & 0.513445\\
\cellcolor{gray!6}{2029} & \cellcolor{gray!6}{162.453} & \cellcolor{gray!6}{134.186} & \cellcolor{gray!6}{3681.30} & \cellcolor{gray!6}{593.289} & \cellcolor{gray!6}{0.518074}\\
2030 & 161.233 & 131.889 & 3675.80 & 596.847 & 0.521181\\
\cellcolor{gray!6}{2031} & \cellcolor{gray!6}{160.217} & \cellcolor{gray!6}{129.775} & \cellcolor{gray!6}{3670.30} & \cellcolor{gray!6}{599.185} & \cellcolor{gray!6}{0.523222}\\
2032 & 159.428 & 128.020 & 3665.54 & 600.698 & 0.524543\\
\bottomrule
\end{tabular}
\end{table}

\FloatBarrier

\begin{table}

\caption{\label{tab:DecisionES}Decision table summarizing 12-year projections 
                (2021 to 2032) for vermilion based on three alternative 
                states of nature spanning quantiles of spawning output in 
                2021.  Columns range over low, medium, and high state of 
                nature, and rows range over different assumptions of total 
                catch levels corresponding to the forecast catches from 
                each state of nature.  Catches in 2021 and 2022 are fixed 
                at catches provided by the CDFW.}
\centering
\resizebox{\linewidth}{!}{
\begin{tabular}[t]{>{\centering\arraybackslash}p{1in}|>{}c|>{}c|>{}c|>{\centering\arraybackslash}p{.8in}>{\centering\arraybackslash}p{.8in}|>{\centering\arraybackslash}p{.8in}>{\centering\arraybackslash}p{.8in}|>{\centering\arraybackslash}p{.8in}>{\centering\arraybackslash}p{.8in}}
\toprule
\multicolumn{4}{c}{ } & \multicolumn{2}{c}{Low Productivity} & \multicolumn{2}{c}{Base Model} & \multicolumn{2}{c}{High Productivity} \\
\cmidrule(l{3pt}r{3pt}){5-6} \cmidrule(l{3pt}r{3pt}){7-8} \cmidrule(l{3pt}r{3pt}){9-10}
\multicolumn{4}{c}{ } & \multicolumn{2}{c}{Female M = 0.0769} & \multicolumn{2}{c}{Female M = 0.0856} & \multicolumn{2}{c}{Female M = 0.0956} \\
\multicolumn{4}{c}{ } & \multicolumn{2}{c}{Male M = 0.0723} & \multicolumn{2}{c}{Male M = 0.0805} & \multicolumn{2}{c}{Male M = 0.0899} \\
\multicolumn{4}{c}{ } & \multicolumn{2}{c}{NLL = 1031.34} & \multicolumn{2}{c}{NLL = 1032} & \multicolumn{2}{c}{NLL = 1031.34} \\
\cmidrule(l{3pt}r{3pt}){5-6} \cmidrule(l{3pt}r{3pt}){7-8} \cmidrule(l{3pt}r{3pt}){9-10}
  & Year & Buffer & Catch (mt) & Spawning Output & Fraction Unfished & Spawning Output & Fraction Unfished & Spawning Output & Fraction Unfished\\
\midrule
 & 2021 & 1.000 & 227 & 437 & 0.362 & 489 & 0.427 & 554 & 0.506\\

 & 2022 & 1.000 & 227 & 435 & 0.361 & 491 & 0.429 & 558 & 0.510\\

 & 2023 & 0.874 & 135 & 438 & 0.363 & 497 & 0.434 & 568 & 0.519\\

 & 2024 & 0.865 & 136 & 453 & 0.376 & 516 & 0.451 & 591 & 0.540\\

 & 2025 & 0.857 & 137 & 467 & 0.387 & 533 & 0.466 & 612 & 0.559\\

 & 2026 & 0.849 & 136 & 477 & 0.396 & 547 & 0.478 & 629 & 0.575\\

 & 2027 & 0.841 & 134 & 485 & 0.402 & 558 & 0.487 & 642 & 0.587\\

 & 2028 & 0.833 & 132 & 491 & 0.407 & 566 & 0.494 & 652 & 0.595\\

 & 2029 & 0.826 & 130 & 496 & 0.411 & 572 & 0.500 & 658 & 0.602\\

 & 2030 & 0.818 & 128 & 499 & 0.414 & 577 & 0.504 & 663 & 0.606\\

 & 2031 & 0.810 & 127 & 502 & 0.416 & 580 & 0.507 & 666 & 0.608\\

\multirow{-12}{1in}{\centering\arraybackslash $p^\ast = 0.45,  \sigma = 1$} & 2032 & 0.803 & 125 & 505 & 0.418 & 583 & 0.509 & 667 & 0.610\\
\cmidrule{1-10}
 & 2021 & 1.000 & 227 & 437 & 0.362 & 489 & 0.427 & 554 & 0.506\\

 & 2022 & 1.000 & 227 & 435 & 0.361 & 491 & 0.429 & 558 & 0.510\\

 & 2023 & 0.762 & 118 & 438 & 0.363 & 497 & 0.434 & 568 & 0.519\\

 & 2024 & 0.747 & 118 & 456 & 0.378 & 519 & 0.453 & 593 & 0.542\\

 & 2025 & 0.733 & 118 & 472 & 0.392 & 539 & 0.470 & 616 & 0.563\\

 & 2026 & 0.719 & 117 & 487 & 0.404 & 556 & 0.485 & 636 & 0.581\\

 & 2027 & 0.706 & 115 & 499 & 0.414 & 570 & 0.498 & 652 & 0.595\\

 & 2028 & 0.693 & 113 & 509 & 0.422 & 581 & 0.508 & 664 & 0.607\\

 & 2029 & 0.680 & 111 & 518 & 0.429 & 591 & 0.516 & 674 & 0.615\\

 & 2030 & 0.667 & 108 & 525 & 0.436 & 599 & 0.523 & 681 & 0.622\\

 & 2031 & 0.654 & 106 & 533 & 0.442 & 606 & 0.529 & 686 & 0.627\\

\multirow{-12}{1in}{\centering\arraybackslash $p^\ast = 0.40,  \sigma = 1$} & 2032 & 0.642 & 105 & 539 & 0.447 & 612 & 0.534 & 691 & 0.631\\
\cmidrule{1-10}
 & 2021 & 1.000 & 227 & 437 & 0.362 & 489 & 0.427 & 554 & 0.506\\

 & 2022 & 1.000 & 227 & 435 & 0.361 & 491 & 0.429 & 558 & 0.510\\

 & 2023 & 1.000 & 139 & 438 & 0.363 & 497 & 0.434 & 568 & 0.519\\

 & 2024 & 1.000 & 139 & 453 & 0.376 & 516 & 0.451 & 590 & 0.539\\

 & 2025 & 1.000 & 139 & 467 & 0.387 & 533 & 0.465 & 610 & 0.558\\

 & 2026 & 1.000 & 139 & 477 & 0.396 & 546 & 0.477 & 627 & 0.573\\

 & 2027 & 1.000 & 139 & 485 & 0.402 & 557 & 0.486 & 639 & 0.584\\

 & 2028 & 1.000 & 139 & 491 & 0.407 & 564 & 0.493 & 647 & 0.591\\

 & 2029 & 1.000 & 139 & 495 & 0.410 & 569 & 0.497 & 652 & 0.596\\

 & 2030 & 1.000 & 139 & 497 & 0.412 & 572 & 0.499 & 654 & 0.598\\

 & 2031 & 1.000 & 139 & 98 & 0.413 & 573 & 0.500 & 655 & 0.598\\

\multirow{-12}{1in}{\centering\arraybackslash Long-term Equil. Yield (MSY proxy, $SPR_{50\%}$), no buffer} & 2032 & 1.000 & 139 & 499 & 0.414 & 573 & 0.501 & 654 & 0.597\\
\cmidrule{1-10}
 & 2021 & 1.000 & 227 & 437 & 0.362 & 489 & 0.427 & 554 & 0.506\\

 & 2022 & 1.000 & 227 & 435 & 0.361 & 491 & 0.429 & 558 & 0.510\\

 & 2023 & 0.874 & 122 & 438 & 0.363 & 497 & 0.434 & 568 & 0.519\\

 & 2024 & 0.865 & 120 & 456 & 0.378 & 518 & 0.453 & 593 & 0.542\\

 & 2025 & 0.857 & 119 & 472 & 0.392 & 538 & 0.470 & 616 & 0.563\\

 & 2026 & 0.849 & 118 & 486 & 0.403 & 555 & 0.485 & 635 & 0.580\\

 & 2027 & 0.841 & 117 & 498 & 0.413 & 569 & 0.497 & 651 & 0.595\\

 & 2028 & 0.833 & 116 & 508 & 0.421 & 580 & 0.507 & 663 & 0.606\\

 & 2029 & 0.826 & 116 & 516 & 0.428 & 589 & 0.515 & 672 & 0.614\\

 & 2030 & 0.818 & 115 & 522 & 0.433 & 596 & 0.521 & 678 & 0.620\\

 & 2031 & 0.810 & 114 & 528 & 0.438 & 602 & 0.526 & 682 & 0.624\\

\multirow{-12}{1in}{\centering\arraybackslash Long-term Equil. Yield (MSY proxy, $SPR_{50\%}$), with buffer} & 2032 & 0.803 & 113 & 533 & 0.442 & 606 & 0.529 & 685 & 0.626\\
\bottomrule
\end{tabular}}
\end{table}

\FloatBarrier

\begin{table}[H]

\caption{\label{tab:CombinedRefPtsES}Combined reference points for the four stock 
                assessments conducted for vermilion and sunset rockfishes in 2021. The fraction of the northern California stock that is estimated to be north of $40^\circ 10^\prime N${} is 4.44\% (see the appendix in the northern CA model for more details). The projected OFLs (2023-2032) assume full attainment of GMT-projected catches for 2021-22, and catches based on the PFMC harvest control rule given $p\ast$ =  0.45 and $\sigma$ = 1.}
\centering
\resizebox{\linewidth}{!}{
\begin{tabular}[t]{>{\centering\arraybackslash}p{2.5in}|>{\centering\arraybackslash}p{.5in}>{\centering\arraybackslash}p{.5in}>{\centering\arraybackslash}p{.7in}>{\centering\arraybackslash}p{.7in}|>{\centering\arraybackslash}p{.7in}>{\centering\arraybackslash}p{.5in}>{\centering\arraybackslash}p{.5in}>{\centering\arraybackslash}p{.5in}}
\toprule
Description & CA South model & CA North model & $34^\circ 27^\prime N${} to $40^\circ 10^\prime N$& South of $40^\circ 10^\prime N$& $40^\circ 10^\prime N${} to CA/OR border & OR model & WA model & North of $40^\circ 10^\prime N$\\
\midrule
Unfished spawning output ($10^6$ eggs) & 977.8 & 1145.2 & 1094.8 & 2072.6 & 50.4 & 29.2 & 2.8 & 82.4\\
Total Biomass, mt & 6263.3 & 6458.0 & 6173.8 & 12437.1 & 284.1 & 439.4 & 36.6 & 760.2\\
Unfished  Recruitment (1000s of fish) & 809.3 & 420.2 & 401.7 & 1211.0 & 18.5 & 16.3 & 2.5 & 37.3\\
Spawning Output (2021, $10^6$ eggs) & 471.2 & 489.4 & 467.9 & 939.1 & 21.5 & 21.4 & 1.5 & 44.4\\
Fraction Unfished (2021) & 0.5 & 0.4 &  &  &  & 0.7 & 0.6 & \\
\midrule
\addlinespace[0.3em]
\multicolumn{9}{l}{\textbf{Reference Points Based on $SPR_{50\%}$}}\\
\hspace{1em}Proxy Spawning Output ($10^6$ eggs) & 439.0 & 510.9 & 488.4 & 927.5 & 22.5 & 13.0 & 1.2 & 36.7\\
\hspace{1em}Proxy MSY, mt & 148.3 & 139.0 & 132.9 & 281.2 & 6.1 & 7.9 & 0.8 & 14.8\\
\midrule
GMT Projected Catch, 2021, mt & 210.3 & 226.8 & 216.8 & 427.1 & 10.0 & 13.0 & 2.7 & 25.6\\
GMT Projected Catch, 2022, mt & 210.3 & 226.8 & 216.8 & 427.1 & 10.0 & 13.0 & 3.3 & 26.2\\
OFL 2023, mt & 159.4 & 154.2 & 147.4 & 306.8 & 6.8 & 13.5 & 0.7 & 21.0\\
OFL 2024, mt & 158.8 & 157.8 & 150.9 & 309.7 & 6.9 & 13.4 & 0.7 & 21.0\\
OFL 2025, mt & 158.8 & 159.5 & 152.5 & 311.3 & 7.0 & 13.2 & 0.7 & 20.9\\
OFL 2026, mt & 159.0 & 159.9 & 152.8 & 311.8 & 7.0 & 12.9 & 0.7 & 20.6\\
OFL 2027, mt & 159.3 & 159.4 & 152.4 & 311.7 & 7.0 & 12.6 & 0.7 & 20.3\\
OFL 2028, mt & 159.6 & 158.7 & 151.7 & 311.3 & 7.0 & 12.3 & 0.7 & 20.0\\
OFL 2029, mt & 159.9 & 157.8 & 150.8 & 310.7 & 6.9 & 12.0 & 0.7 & 19.7\\
OFL 2030, mt & 160.3 & 157.0 & 150.1 & 310.3 & 6.9 & 11.8 & 0.8 & 19.4\\
OFL 2031, mt & 160.6 & 156.3 & 149.5 & 310.1 & 6.9 & 11.5 & 0.8 & 19.2\\
OFL 2032, mt & 161.1 & 155.9 & 149.0 & 310.1 & 6.9 & 11.3 & 0.8 & 18.9\\
\bottomrule
\end{tabular}}
\end{table}

\newpage

\hypertarget{research-and-data-needs}{%
\subsection*{Research and Data Needs}\label{research-and-data-needs}}
\addcontentsline{toc}{subsection}{Research and Data Needs}

The following are high priority research and data needs for this assessment. Additional details for each topic can be found in the full assessment.

We recommend the following research be conducted before the next assessment:

\begin{itemize}
\item
  Develop a coastwide hook-and-line survey to provide indices of abundance and associated biological sampling providing representative data in untrawlable habitats.
\item
  Examine the available tools more fully in cases when a survey's footprint is abruptly changed as a result of management action. These tools may include (but are not limited to), treating the ``new'' and ``old'' surveys as completely separate (aka breaking the survey), using selectivity blocks, or spatial/temporal modeling approaches. This avenue is important for many fishery-independent and -dependent indices, as they are subjected to numerous spatial management changes which in turn can affect the veracity of the data collected. Additional efforts are needed to investigate how fishery selectivity changes with management changes and how best to address the effects of management changes on length composition and indices.
\item
  Expansion of the California Collaborative Fisheries Research Project from the current 120 ft depth or starting similar surveys that sample in deeper waters outside, if not inside MPAs and other closed areas to encompass the full depth distribution of vermilion and sunset rockfish or other shallow shelf rockfish species would provide valuable data for future assessments.
\item
  Conduct additional investigations to resolve uncertainties in historical catch reconstructions would improve estimates of the scale of assessments and provide more representative removal estimates.
\item
  Explore appropriate methods of including catches as numbers of fish vs.~biomass.
\item
  Connectedness of this stock with Southern California (below point Conception) is an unresolved uncertainty as outlined in the STAT report and elsewhere in this report. Further studies on larval/juvenile/adult movement via tagging or other methods are warranted. Additionally population substructure investigations, particularly north and south of Cape Mendocino are also recommended.
\item
  Development of a more comprehensive fishery-independent index is a priority for this region. This could involve expansion of the CCFRP across depths and latitudes or expansion of the NWFSC hook-and-line survey northward.
\end{itemize}

\pagebreak
\setlength{\parskip}{5mm plus1mm minus1mm}
\pagenumbering{arabic}
\setcounter{page}{1}
\renewcommand{\thefigure}{\arabic{figure}}
\renewcommand{\thetable}{\arabic{table}}
\setcounter{table}{0}
\setcounter{figure}{0}

\hypertarget{introduction}{%
\section{Introduction}\label{introduction}}

Note to readers: Text in this section is the same in both California vermilion rockfish assessment
documents.

\hypertarget{basic-information-and-life-history}{%
\subsection{Basic Information and Life History}\label{basic-information-and-life-history}}

\emph{Note: Prior to the identification of sunset rockfish as a separate species (Hyde, J.R.; Kimbrell, C. A.; Budrick, J. E.; Lynn, E. A.; Vetter 2008), historical studies of ``vermilion'' rockfish, particularly those conducted south of Point Conception ($34^\circ 27^\prime N$), California, could have included a mixture of both species. Also, many current studies and data sets (e.g., landing statistics) do not distinguish between the species. In this document, we refer simply to ``vermilion rockfish'' when no species-specific information is available.}

Vermilion rockfish (\emph{Sebastes miniatus}) range from Prince William Sound, Alaska, to central Baja California at
depths of 6 m to 436 m (Love et al. 2002). However, they are most commonly found from central Oregon
to Punta Baja, Mexico (Hyde and Vetter 2009) at depths of 50 m to 150 m (Hyde and Vetter 2009). Hyde and Vetter
(2009) describe vermilion rockfish as residents of shallower depths (\textless100 m) than their sibling species,
sunset rockfish (\emph{Sebastes crocotulus}). Adult fish tend to cluster on high relief rocky outcrops (Love et al. 2002)
and kelp forests (Hyde and Vetter 2009). North of Point Conception, California, some adults reside in shallower,
living in caves and cracks (Love et al. 2002). Vermilion rockfish have shown high site fidelity
(Hannah and Rankin 2011 (only tagged 1 vermilion), Lea et al. 1999), and low to average larval dispersal
distance (Hyde and Vetter 2009). Lowe et al. (2009) suggested that vermilion rockfish
have a lower site fidelity than previously believed, but acknowledged that their
observations of movements to different depths may have been due to differences in depth distribution between the species.
Vermilion rockfish have been aged to over 80 years, but few fish have been aged above 60 years, with females growing larger than their male counterparts. Fifty percent of females are mature at 5 years and about
37 cm, with males likely maturing at shorter lengths than females (Love et al. 2002).

Vermilion rockfish are viviparous, and females produce an estimated 63,000 to 2,600,000 eggs per brood, with larger fish releasing a substantially larger number of larvae.
In southern California, vermilion rockfish larvae are released between July and March.
In central and northern California, this release occurs in September, December, and
April-June (Love et al. 2002). Larval release in fall and winter is not common among other
rockfish species. Hyde and Vetter (2009) suggest that low larval dispersal may
be due to weak poleward flow of nearshore waters corresponding with peak vermilion larval release.

Young-of-the-year vermilion rockfish settle out of the water column during two primary recruitment
periods per year, first from February to April and a second from August to October,
and settlement has been observed in May off southern California (Love et al. 2002). Young-of-the-year vermilion and sunset rockfish are both mottled brown with areas of black, and older juveniles turn a mottled orange or red color (Love et al. 2012). Larvae
measure about 4.3 mm and juvenile fish are found in depths of 6-36 m, living near sand and structure. After two months, juveniles travel deeper and live on low relief rocky outcrops and
other structures (Love et al. 2002).

Adult vermilion rockfish predominantly eat smaller fish, though sometimes they pursue
euphausiids and other various macroplankton (Phillips 1964). Love et al. (2002) noted
their diet includes octopuses, salps, shrimps, and pelagic red crabs.

\emph{Population Structure and Multi-species Assessment Considerations}

This assessment represents the aggregate population dynamics of the cryptic species pair vermilion rockfish
and sunset rockfish.
Hyde (2007) examined seven mitochondrial and two nuclear genes, which upon analysis suggested
three species within the subgenus \emph{Rosicola}. Hyde et al. (2008) described sunset rockfish as a distinct species noting depth separation
of the adult populations of the two species using nine microsatellite loci.
Adult sunset rockfish are mainly distributed at depths
greater than 50 fm (100 m) and are predominantly located south of Point Conception ($34^\circ 27^\prime N$).
Hyde and Vetter (2009) and Budrick (2016) identified species using mtDNA assays and microsatellite loci,
respectively. The mtDNA analyses proved to be subject to errors in species identification due to introgression resulting from mating between the two species post-divergence.
Additional population clusters of vermilion were found north of Point Conception, but neither
study detected population structure between Half Moon Bay, California and Brookings,
Oregon (Hyde and Vetter 2009, Budrick 2016).

Vermilion and sunset rockfishes are morphologically very similar, with color being
the most commonly
cited differentiating feature. Hyde and Vetter (2009) noted differences in three of six morphological
parameters examined, but none of them can readily be used for field identification.

In all historical and current recreational and commercial catches, sunset and
vermilion rockfish are both recorded as vermilion rockfish. Future studies,
such as the one described below will provide data needed to compare biological
parameters between the two species as well as habitats and distributions.

\emph{Ongoing Population Structure Research (Provided by John Harms, NWFSC)}

A group of researchers from the NWFSC and SWFSC is collaborating on a project to
genotype tissue specimens collected from the vermilion and sunset rockfish cryptic
pair captured during the West Coast Groundfish Bottom Trawl (WCGBT) Survey and the Southern
CA Shelf Rockfish Hook-and-Line Survey for the years 2004 - 2019. Funding for this
project was obtained through the Saltonstall-Kennedy program for FY 2020 through a
proposal led by representatives from Pacific States Marine Fisheries Commission and
the commercial passenger fishing vessel industry in southern California.

After combining with specimens obtained through other collection efforts along
the West Coast, approximately 25,000 tissue specimens will be analyzed. Some
earlier efforts to separate this cryptic pair to species used mitochondrial DNA
(mtDNA) markers. However, due to a one-way mitochondrial introgression from
the vermilion genome into the sunset genome, a portion of the sunset rockfish
population contains mitochondrial DNA sequences consistent with vermilion rockfish
resulting in incorrect species assignments for these introgressed individuals
during the prior research project. The current research has identified a robust
suite of genetic markers within the nuclear genomes of the two species that
definitively separates vermilion and sunset rockfish (including introgressed
sunset rockfish), canary rockfish, first generation vermilion-sunset hybrids,
and identifies emerging patterns of intraspecific stock structure within the
two sister species.

Once the collected specimens have been genotyped, any species-specific differences
in spatial and depth distribution, size composition, weight-length relationships,
and other biological characteristics will be identified. Using previously
collected otoliths and ovaries, the demographics of the two species including
age and growth and reproductive biology parameters such as length and age at
50\% maturity and the prevalence of skip spawning will be explored and compared.
These new genotyping results will be combined with data from the prior mtDNA
work to evaluate whether introgressed sunset rockfish represent a biologically
intermediate subform of the species complex. The effort also proposes to develop
and test the efficacy of models to predict the relative proportion of the two
species based upon explanatory variables including latitude, depth, species of
co-occurrence, oceanographic parameters, habitat descriptors and/or other
information. The anticipated completion of the genotyping of all specimens
is approximately December 2021 with provision of final results by the end of FY 2022.

This research is aimed at providing information to support the successful stock
assessment of this commercially and recreationally valuable cryptic species pair
and is responsive to any data gaps identified by the assessment community. If
successful, this research, conducted in close communication with stock assessors,
may also assist the PFMC in establishing best practices for the assessment and
management of cryptic species complexes. Though this project will only focus
on nominal vermilion rockfish specimens collected through the 2019 survey
field season, it may be advisable that tissue specimens collected aboard
fishery-independent surveys as well as through fishery-dependent programs
continue to be genotyped on an ongoing basis to support continued and timely
monitoring of this economically and ecologically important species complex.

\hypertarget{map}{%
\subsection{Map}\label{map}}

A map showing the scope of the two California vermilion rockfish assessments and depicting a
boundary
at Point Conception ($34^\circ 27^\prime N$) that separates the two assessments is provided as Figure \ref{fig:assess-area}. The northern California
model is bounded in the north by the California/Oregon border ($42^\circ 00^\prime N$) and the southern California model is
bounded by the U.S./ Mexico border in the south (Figure \ref{fig:assess-area}).
Cape Mendocino ($40^\circ 10^\prime N$) is also noted as it is a management boundary for the
Pacific Fishery Management Council (PFMC) ``minor shelf rockfish'' stock complex.

\hypertarget{ecosystem-considerations-1}{%
\subsection{Ecosystem Considerations}\label{ecosystem-considerations-1}}

This stock assessment does not explicitly incorporate trophic interactions,
habitat factors (other than as they inform relative abundance indices) or environmental
factors into the assessment model, but a brief description of likely or potential
ecosystem considerations are provided below.

Vermilion/sunset rockfish are described as feeding on a wide range of both
pelagic and benthic prey items, including forage fish species such as anchovies
and mesopelagic fishes, squid, krill and octopus, as well as sporadically abundant
pelagic organisms such as pyrosomes, salps and pelagic red crabs
(Phillips 1964, Love et al. 2002). Interestingly, other rockfishes (either juvenile or
adult stages) have not been
documented as prey for vermilion, as they have been for other large \emph{Sebastes}
species such as cowcod, bocaccio, and yelloweye rockfish. For the latter species,
the idea of ``cultivation effects,'' in which adults crop down forage species that
are potential competitors/predators of their own juveniles (Walters and Kitchell 2001),
has been suggested by Baskett et al. (2006). For example, Baskett et al. (2006)
found that in such scenarios there could be alternative stable states in which
either the overfished species or the smaller prey species could dominate. While
the sparse diet data for vermilion/sunset rockfish do not suggest such a process
for this species complex, food habits data for vermilion/sunset are not robust,
and the larger community processes on these rocky reef communities may also influence
productivity and community composition regardless of the direct predation interactions.
Pelagic and benthic juvenile vermilion and sunset rockfish are likely preyed upon by
the same wide range of predators that prey on juveniles and adults of other rockfish
species, including seabirds, piscivorous fishes, and marine mammals.

As with most other rockfish and groundfish in the California Current, recruitment,
or cohort (year-class) strength appears to be highly variable for the vermilion/sunset
rockfish complex, with only a modest apparent relationship to estimated levels of spawning output. Oceanographic and ecosystem factors are widely recognized to be key drivers of
recruitment variability for most species of groundfish, as well as most elements
of California Current food webs. Empirical estimates of recruitment from pelagic
juvenile rockfish surveys have been used to inform incoming year class strength for
some of these stocks, however vermilion and sunset rockfish are rarely encountered
in these surveys. Specifically, only 47 of nearly 300,000 total juvenile \emph{Sebastes}
encountered in juvenile surveys since 2001 were identified as vermilion or sunset
rockfish (Field et al. 2021). Despite this, the results here suggest that at least a
reasonable fraction of recruitment variability for sunset and vermilion rockfish
is shared with other rockfish and groundfish stocks throughout the California Current,
many of which also had strong year classes in 1984, 1999 and 2015-2016. Previous studies
have demonstrated that large-scale oceanographic drivers, such as the relative transport
of subarctic waters (typically indicated by relative sea level) tend to relate to a
substantial fraction of overall groundfish recruitment trends and ecosystem
productivity Schroeder et al. (2019). Although it is feasible that
ecosystem factors, the results of pre-recruit surveys for co-occurring species,
or the results of other groundfish assessments might ultimately be used to
forecast recruitment for more data-limited stocks such as vermilion/sunset,
as suggested by (Thorson and Ward 2014), such approaches would require more
development and evaluation. Consequently, environmental factors are not
explicitly considered in this assessment.

\hypertarget{historical-and-current-fishery-information}{%
\subsection{Historical and Current Fishery Information}\label{historical-and-current-fishery-information}}

\emph{Commercial Fishery}\\
The commercial groundfish fishery off California developed in the late 19th
century and consisted mainly of hook and line gear types (Figure \ref{fig:catch}).
At the turn of the century, total rockfish landings were estimated to be between
2,000 to 3,500 tons statewide, with slightly over half of the catch during this
period coming from waters south of Point Conception, and most of the remaining
catch from central California ports (particularly San Francisco and Monterey).
Catches declined through the 1930s as a result of the rapid expansion of the
California sardine fishery, which tended to be more profitable (Love et al. 2002).
The rockfish trawl fishery rapidly expanded into California in the early 1940s,
after the introduction of the `balloon trawl,' and when the United States became
involved in World War II and wartime shortage of red
meat created an increased demand for other sources of protein (Harry and Morgan 1961, Alverson et al. 1964, Lenarz 1987). Trawl landings have been restricted in most of southern
California for decades (Frey 1971), and trawl gear north of Point Conception has
not recently been a major component of the landings for vermilion, with the
highest reported landings in the 1970s. The commercial setnet fishery has never
been a large component of the vermilion rockfish landings and has essentially
been non-existent for vermilion since 2002 when the state of California prohibited
setnet gear in 60 fm or less. The largest net landings for vermilion were in the 1980s.

Vermilion have been landed in the commercial live-fish fishery that developed
off the coast of
California in the 1990s, but have not been a major target of that fishery due
to their susceptibility to barotrauma. The fraction of the total catch
from the live fish fleet is small, concentrated in northern California, and
included in the commercial hook-and-line fleet in the northern California
assessment models. The STAT also learned that vermilion targeted for the live-fish
fishery, but landed dead due to barotrauma, remain valuable and may be sold dead.
Separation of catch and size compositions for the live and dead catch is therefore
less informative and was not pursued further.

Miller et al. (2014) described the spatial and temporal development of the
California commercial groundfish fishery based on historical CDFW fish ticket
and block summary data. They analyzed a spatially-explicit database of
landings in California dating back to 1933, finding that groundfish fishing effort
has shifted from shallow, coastal areas to deeper depths, greater distances from
port, and in areas of more inclement weather over time. That general result was
also found with limited data from recreational fisheries. Sampling of commercial
species compositions in Southern California began in 1983, a time when the
groundfish fleet was already fishing in deeper depths.
Both historical reconstructions used these data to represent species compositions of
total rockfish catch during earlier periods of the fishery. As a result, the
reconstructions may overestimate the percentage of deep-water species in earlier
fisheries that operated closer to port and in shallower depths.

\emph{Recreational Fishery}\\
Vermilion rockfish are a targeted species in California's recreational fishery
and have always ranked high in terms of catch among rockfish species, both in
the party/charter boat and private/rental sectors. The Commercial Passenger
Fishing Vessel (CPFV; aka `party' and `charter' boat) fleet began circa 1919
in California, although recreational fishing effort for
fishes other than Tunas, other gamefish, and salmon was minimal until about
1930. The CPFV fleet numbered about 200 vessels in 1939 {[}(Croker 1940), cited
in Young (1969)). After a hiatus in most operations during WWII, the
fleet increased to about 590 vessels by 1953, then declined to approximately
256 vessels around 1963. {]} \& 139 \& 559 \& 25\%\textbackslash{}
\cellcolor{gray!6}{(15,20]} \& \cellcolor{gray!6}{279} \& \cellcolor{gray!6}{808} \& \cellcolor{gray!6}{35\%}\textbackslash{}
(20,25{]} \& 226 \& 588 \& 38\%\textbackslash{}
\cellcolor{gray!6}{(25,30]} \& \cellcolor{gray!6}{219} \& \cellcolor{gray!6}{601} \& \cellcolor{gray!6}{36\%}\textbackslash{}
\addlinespace
(30,35{]} \& 159 \& 373 \& 43\%\textbackslash{}
\cellcolor{gray!6}{(35,40]} \& \cellcolor{gray!6}{216} \& \cellcolor{gray!6}{450} \& \cellcolor{gray!6}{48\%}\textbackslash{}
(40,65{]} \& 428 \& 756 \& 57\%\textbackslash{}
\bottomrule
\textbackslash end\{tabular\}
\textbackslash end\{table\}

\FloatBarrier

\begin{table}

\caption{\label{tab:tab-region-cpfvonboard}Samples of vermilion in the northern model by subregion used in the index.}
\centering
\begin{tabular}[t]{lrrl}
\toprule
Year & Samples & Positive Samples & Percent Positive\\
\midrule
\cellcolor{gray!6}{CA/OR border to Santa Cruz (V1)} & \cellcolor{gray!6}{238} & \cellcolor{gray!6}{1213} & \cellcolor{gray!6}{20\%}\\
Moss Landing to Big Sur (V2) & 146 & 511 & 29\%\\
\cellcolor{gray!6}{San Luis Obsipso to Morro Bay (V3)} & \cellcolor{gray!6}{591} & \cellcolor{gray!6}{1044} & \cellcolor{gray!6}{57\%}\\
South Morro Bay to Point Conception (V4) & 643 & 1180 & 54\%\\
\cellcolor{gray!6}{Offshore (V5)} & \cellcolor{gray!6}{88} & \cellcolor{gray!6}{533} & \cellcolor{gray!6}{17\%}\\
\bottomrule
\end{tabular}
\end{table}

\begin{table}

\caption{\label{tab:tab-year-cpfvonboard}Samples of vermilion in the northern model by year.}
\centering
\begin{tabular}[t]{lrrl}
\toprule
Year & Samples & Positive Samples & Percent Positive\\
\midrule
\cellcolor{gray!6}{1999} & \cellcolor{gray!6}{13} & \cellcolor{gray!6}{60} & \cellcolor{gray!6}{22\%}\\
2000 & 6 & 38 & 16\%\\
\cellcolor{gray!6}{2001} & \cellcolor{gray!6}{11} & \cellcolor{gray!6}{71} & \cellcolor{gray!6}{15\%}\\
2002 & 17 & 60 & 28\%\\
\cellcolor{gray!6}{2003} & \cellcolor{gray!6}{117} & \cellcolor{gray!6}{276} & \cellcolor{gray!6}{42\%}\\
\addlinespace
2004 & 192 & 400 & 48\%\\
\cellcolor{gray!6}{2005} & \cellcolor{gray!6}{67} & \cellcolor{gray!6}{153} & \cellcolor{gray!6}{44\%}\\
2006 & 121 & 265 & 46\%\\
\cellcolor{gray!6}{2007} & \cellcolor{gray!6}{126} & \cellcolor{gray!6}{268} & \cellcolor{gray!6}{47\%}\\
2008 & 47 & 155 & 30\%\\
\addlinespace
\cellcolor{gray!6}{2009} & \cellcolor{gray!6}{54} & \cellcolor{gray!6}{198} & \cellcolor{gray!6}{27\%}\\
2010 & 79 & 193 & 41\%\\
\cellcolor{gray!6}{2011} & \cellcolor{gray!6}{62} & \cellcolor{gray!6}{182} & \cellcolor{gray!6}{34\%}\\
2012 & 66 & 220 & 30\%\\
\cellcolor{gray!6}{2013} & \cellcolor{gray!6}{29} & \cellcolor{gray!6}{160} & \cellcolor{gray!6}{18\%}\\
\addlinespace
2014 & 47 & 221 & 21\%\\
\cellcolor{gray!6}{2015} & \cellcolor{gray!6}{75} & \cellcolor{gray!6}{219} & \cellcolor{gray!6}{34\%}\\
2016 & 79 & 321 & 25\%\\
\cellcolor{gray!6}{2017} & \cellcolor{gray!6}{226} & \cellcolor{gray!6}{426} & \cellcolor{gray!6}{53\%}\\
2018 & 146 & 295 & 49\%\\
\addlinespace
\cellcolor{gray!6}{2019} & \cellcolor{gray!6}{126} & \cellcolor{gray!6}{300} & \cellcolor{gray!6}{42\%}\\
\bottomrule
\end{tabular}
\end{table}

\FloatBarrier

\begin{table}

\caption{\label{tab:tab-model-select-cpfvonboard}Model selection for the CA CPFV onboard survey index for vermilion in the northern model .}
\centering
\begin{tabular}[t]{lrr}
\toprule
Model & Binomial $\Delta$AIC & Lognormal $\Delta$AIC\\
\midrule
\cellcolor{gray!6}{1} & \cellcolor{gray!6}{797.52} & \cellcolor{gray!6}{436.25}\\
YEAR + SubRegion & 129.05 & 60.03\\
\cellcolor{gray!6}{YEAR + SubRegion + WAVE} & \cellcolor{gray!6}{120.54} & \cellcolor{gray!6}{58.72}\\
YEAR + SubRegion + WAVE + DEPTH bin & 0.00 & 0.00\\
\cellcolor{gray!6}{YEAR + WAVE + DEPTH bin} & \cellcolor{gray!6}{285.69} & \cellcolor{gray!6}{66.16}\\
\addlinespace
YEAR + DEPTH bin & 316.83 & 74.00\\
\cellcolor{gray!6}{YEAR + SubRegion + DEPTH bin} & \cellcolor{gray!6}{10.87} & \cellcolor{gray!6}{6.06}\\
\bottomrule
\end{tabular}
\end{table}

\FloatBarrier

\begin{table}

\caption{\label{tab:tab-index-cpfvonboard}Standardized index for the CA CPFV onboard survey index with log-scale standard errors and 95% highest
       posterior density (HPD) intervals for vermilion in the northern model .}
\centering
\begin{tabular}[t]{rrrrr}
\toprule
Year & Mean & logSE & lower HPD & upper HPD\\
\midrule
\cellcolor{gray!6}{1999} & \cellcolor{gray!6}{0.02} & \cellcolor{gray!6}{0.53} & \cellcolor{gray!6}{0.01} & \cellcolor{gray!6}{0.05}\\
2000 & 0.02 & 0.65 & 0.00 & 0.04\\
\cellcolor{gray!6}{2001} & \cellcolor{gray!6}{0.01} & \cellcolor{gray!6}{0.53} & \cellcolor{gray!6}{0.00} & \cellcolor{gray!6}{0.02}\\
2002 & 0.02 & 0.42 & 0.01 & 0.05\\
\cellcolor{gray!6}{2003} & \cellcolor{gray!6}{0.05} & \cellcolor{gray!6}{0.33} & \cellcolor{gray!6}{0.02} & \cellcolor{gray!6}{0.09}\\
\addlinespace
2004 & 0.07 & 0.28 & 0.04 & 0.11\\
\cellcolor{gray!6}{2005} & \cellcolor{gray!6}{0.04} & \cellcolor{gray!6}{0.38} & \cellcolor{gray!6}{0.02} & \cellcolor{gray!6}{0.08}\\
2006 & 0.05 & 0.36 & 0.02 & 0.09\\
\cellcolor{gray!6}{2007} & \cellcolor{gray!6}{0.06} & \cellcolor{gray!6}{0.35} & \cellcolor{gray!6}{0.03} & \cellcolor{gray!6}{0.11}\\
2008 & 0.03 & 0.38 & 0.01 & 0.05\\
\addlinespace
\cellcolor{gray!6}{2009} & \cellcolor{gray!6}{0.04} & \cellcolor{gray!6}{0.37} & \cellcolor{gray!6}{0.02} & \cellcolor{gray!6}{0.07}\\
2010 & 0.05 & 0.37 & 0.02 & 0.09\\
\cellcolor{gray!6}{2011} & \cellcolor{gray!6}{0.04} & \cellcolor{gray!6}{0.37} & \cellcolor{gray!6}{0.02} & \cellcolor{gray!6}{0.08}\\
2012 & 0.03 & 0.38 & 0.01 & 0.06\\
\cellcolor{gray!6}{2013} & \cellcolor{gray!6}{0.02} & \cellcolor{gray!6}{0.42} & \cellcolor{gray!6}{0.01} & \cellcolor{gray!6}{0.04}\\
\addlinespace
2014 & 0.02 & 0.38 & 0.01 & 0.04\\
\cellcolor{gray!6}{2015} & \cellcolor{gray!6}{0.04} & \cellcolor{gray!6}{0.37} & \cellcolor{gray!6}{0.02} & \cellcolor{gray!6}{0.07}\\
2016 & 0.03 & 0.37 & 0.01 & 0.06\\
\cellcolor{gray!6}{2017} & \cellcolor{gray!6}{0.04} & \cellcolor{gray!6}{0.36} & \cellcolor{gray!6}{0.02} & \cellcolor{gray!6}{0.08}\\
2018 & 0.05 & 0.37 & 0.02 & 0.09\\
\addlinespace
\cellcolor{gray!6}{2019} & \cellcolor{gray!6}{0.04} & \cellcolor{gray!6}{0.37} & \cellcolor{gray!6}{0.02} & \cellcolor{gray!6}{0.08}\\
\bottomrule
\end{tabular}
\end{table}

\FloatBarrier

\begin{figure}
\centering
\includegraphics{C:/Stock_Assessments/VRML_Assessment_2021/GitHub/Vermilion_2021/doc/indices/vermilion_CA_CPFV_onboard_writeup_NCA_files/figure-latex/fig-dist-fits-cpfvonboard-1.pdf}
\caption{\label{fig:fig-dist-fits-cpfvonboard}Q-Q plot (top) of the positive observations lognormal gamma distributions and fitted values vs residuals for the Lognormal model (bottom).}
\end{figure}

\begin{figure}
\centering
\includegraphics{C:/Stock_Assessments/VRML_Assessment_2021/GitHub/Vermilion_2021/doc/indices/vermilion_CA_CPFV_onboard_writeup_NCA_files/figure-latex/fig-depthfished-cpfvonboard-1.pdf}
\caption{\label{fig:fig-depthfished-cpfvonboard}Boxplots of depths fished by year in the filtered data.}
\end{figure}

\begin{figure}
\centering
\includegraphics{C:/Stock_Assessments/VRML_Assessment_2021/GitHub/Vermilion_2021/doc/indices/vermilion_CA_CPFV_onboard_writeup_NCA_files/figure-latex/fig-areacpue-cpfvonboard-1.pdf}
\caption{\label{fig:fig-areacpue-cpfvonboard}Arithmetic mean of CPUE by region for vermilion from the filtered data. The areas used are in the text.}
\end{figure}

\begin{figure}
\centering
\includegraphics{C:/Stock_Assessments/VRML_Assessment_2021/GitHub/Vermilion_2021/doc/indices/vermilion_CA_CPFV_onboard_writeup_NCA_files/figure-latex/fig-propzero-cpfvonboard-1.pdf}
\caption{\label{fig:fig-propzero-cpfvonboard}Posterior predictive distribution of the proportion of zero observations in replicate data sets generated by the delta model with a vertical line representing the observed average.}
\end{figure}

\begin{figure}
\centering
\includegraphics{C:/Stock_Assessments/VRML_Assessment_2021/GitHub/Vermilion_2021/doc/indices/vermilion_CA_CPFV_onboard_writeup_NCA_files/figure-latex/fig-posterior-mean-cpfvonboard-1.pdf}
\caption{\label{fig:fig-posterior-mean-cpfvonboard}Posterior predictive draws of the mean by year with a vertical line representing the observed average.}
\end{figure}

\begin{figure}
\centering
\includegraphics{C:/Stock_Assessments/VRML_Assessment_2021/GitHub/Vermilion_2021/doc/indices/vermilion_CA_CPFV_onboard_writeup_NCA_files/figure-latex/fig-posterior-sd-cpfvonboard-1.pdf}
\caption{\label{fig:fig-posterior-sd-cpfvonboard}Posterior predictive draws of the standard deviation by year with a vertical line representing the observed average.}
\end{figure}

\begin{figure}
\centering
\includegraphics{C:/Stock_Assessments/VRML_Assessment_2021/GitHub/Vermilion_2021/doc/indices/vermilion_CA_CPFV_onboard_writeup_NCA_files/figure-latex/fig-cpue-cpfvonboard-1.pdf}
\caption{\label{fig:fig-cpue-cpfvonboard}Standardized index and arithmetic mean of the CPUE from the filtered data. Each timeseries is scaled to its respective means.}
\end{figure}

\begin{figure}
\centering
\includegraphics{C:/Stock_Assessments/VRML_Assessment_2021/GitHub/Vermilion_2021/doc/indices/vermilion_CA_CPFV_onboard_writeup_NCA_files/figure-latex/fig-Dbin-marginal-cpfvonboard-1.pdf}
\caption{\label{fig:fig-Dbin-marginal-cpfvonboard}Binomial marginal effects from the final model}
\end{figure}

\begin{figure}
\centering
\includegraphics{C:/Stock_Assessments/VRML_Assessment_2021/GitHub/Vermilion_2021/doc/indices/vermilion_CA_CPFV_onboard_writeup_NCA_files/figure-latex/fig-Dpos-marginal-cpfvonboard-1.pdf}
\caption{\label{fig:fig-Dpos-marginal-cpfvonboard}Positive model marginal effects from the final model.}
\end{figure}

\clearpage

\hypertarget{debwv-index}{%
\subsection{Deb Wilson-Vandenberg Onboard CPFV Index of Abundance}\label{debwv-index}}

\hypertarget{deb-wilson-vandenberg-index}{%
\subsubsection{Deb Wilson-Vandenberg Index}\label{deb-wilson-vandenberg-index}}

The Deb Wilson-Vanedenberg data set is an onboard observer survey data conducted
by CDFW survey in central California from 1987-1998 and referred to as the Deb
Wilson-Vandenberg onboard observer survey, (Reilly et al. 1998)). During an onboard
observer trip the sampler rides along on the CPFV and records location-specific
catch and discard information to the species level for a subset of anglers
onboard the vessel. The subset of observed anglers is usually a maximum of 15
people the observed anglers change during each fishing stop. The catch cannot be
linked to an individual, but rather to a specific fishing location. The sampler
also records the starting and ending time, number of anglers observed, starting
and ending depth, and measures discarded fish. The fine-scale catch and effort
data allow us to better filter the data for indices
to fishing stops within suitable habitat for the target species.

\textbf{Deb Wilson-Vandenberg Index: Data Preparation, Filtering, and Sample Sizes}

A large effort was made by the SWFSC to recover data from the original data
sheets for this survey and developed into a relational database (Monk et al. 2016).
The specific fishing locations at each fishing stop were recorded at a finer
scale than the catch data for this survey. We aggregated the relevant location
information (time and number of observed anglers) to match the available catch
information. Between April 1987 and July 1992 the number of observed anglers
was not recorded for each fishing stop, but the number of anglers aboard the
vessel is available. We imputed the number of observed anglers using the number
of anglers aboard the vessel and the number of observed anglers at each fishing
stop from the August 1992-December 1998 data (see Supplemental materials for
details). In 1987, trips were only observed in Monterey, CA and were therefore
excluded from the analysis (Table \ref{tab:tab-data-filter-debwv}). Sampling
targeted areas of central California. Of the 2,256 trips observed, only 12 of
those launched from port in District 6, which was removed from the analysis.

Each fishing location was assigned to a reef based on the on the bathymetric maps
and interpretation of hard bottom was extracted from
the \href{http://seafloor.otterlabs.org/index.html}{California Seafloor Mapping Project}.
Reefs were aggregated to four regions produce adequate sample sizes;
Ft. Bragg to Santa Cruz (V1), Moss Landing to Big Sur (V2), San Luis Obispo to
Pt. Conception (V3), and Offshore (deeper) locations including the Farallon
Islands and reefs of Half Moon Bay and Monterey Bay (V4). The ports in San
Luis Obispo county were sampled more frequently than other regions and the arithmetic
mean of CPUE by year was higher also higher in this area (Figure \ref{fig:fig-areacpue-debwv})

We retained 6597 drifts for index standardization, with
2016 fishing location encountering vermilion.\\
Tables of the number of samples and positive observervations by factor can be
found in Tables (\ref{tab:tab-depth-debwv}, \ref{tab:tab-region-debwv}, and
\ref{tab:tab-year-debwv}).

\textbf{Deb Wilson-Vandenberg Index: Model Selection, Fits, and Diagnostics}

A Lognormal model was
selected for the positive observation GLM by
a \(\Delta AIC\) of 313.12 over a Gamma model and supported by Q-Q plots of the positive observations fit to both distributions (Figure \ref{fig:fig-dist-fits-debwv}). The delta-GLM
method allows the linear predictors to differ between the binomial and positive models.
Based on AIC values from maximum likelihood fits Table \ref{tab:tab-model-select-debwv}),
a main effects model including
YEAR and WAVE and DEPTH bin
was fit for the binomial model and a main
effects model including
YEAR and WAVE and DEPTH bin
was fit for the Lognormal model.
Models were fit using the ``rstanarm'' R package (version 2.21.1). Posterior predictive
checks of the Bayesian model fit for the binomial model and the positive model
were all reasonable (Figures \ref{fig:fig-posterior-mean-debwv} and
\ref{fig:fig-posterior-sd-debwv}). The binomial model generated data sets with the
proportion zeros similar to the 69\% zeroes in the observed data
(Figure \ref{fig:fig-propzero-debwv}). The predicted marginal effects from
both the binomial and Lognormal models can be found in (Figures \ref{fig:fig-Dbin-marginal-debwv} and \ref{fig:fig-Dpos-marginal-debwv}). The
final index (Table \ref{tab:tab-index-debwv})
represents a similar trend to the arithmetic mean of the annual CPUE (Figure \ref{fig:fig-cpue-debwv}).

\newpage

\begin{table}

\caption{\label{tab:tab-data-filter-debwv}Data filtering steps DebWV onboard survey index for vermilion in the northern model .}
\centering
\begin{tabular}[t]{>{\raggedright\arraybackslash}p{8em}>{\raggedright\arraybackslash}p{15em}c>{\centering\arraybackslash}p{8em}>{\centering\arraybackslash}p{8em}}
\toprule
Filter & Desciption & Trip & Positive Trips & Percent drifts retained\\
\midrule
\cellcolor{gray!6}{All} & \cellcolor{gray!6}{None} & \cellcolor{gray!6}{7566} & \cellcolor{gray!6}{2593} & \cellcolor{gray!6}{34\%}\\
No catch & Remove no catch trips & 7041 & 2068 & 29\%\\
\cellcolor{gray!6}{Sparse data} & \cellcolor{gray!6}{Remove District 6 and 1987} & \cellcolor{gray!6}{6697} & \cellcolor{gray!6}{2022} & \cellcolor{gray!6}{30\%}\\
Time fished & Remove drifts fished less than 6 minutes & 6597 & 2016 & 31\%\\
\bottomrule
\end{tabular}
\end{table}

\begin{table}

\caption{\label{tab:tab-depth-debwv}Positive samples of vermilion in the northern model by depth (fm).}
\centering
\begin{tabular}[t]{lrrl}
\toprule
Year & Samples & Positive Samples & Percent Positive\\
\midrule
\cellcolor{gray!6}{(0,10]} & \cellcolor{gray!6}{113} & \cellcolor{gray!6}{478} & \cellcolor{gray!6}{24\%}\\
(10,20] & 455 & 1344 & 34\%\\
\cellcolor{gray!6}{(20,30]} & \cellcolor{gray!6}{410} & \cellcolor{gray!6}{1198} & \cellcolor{gray!6}{34\%}\\
(30,40] & 465 & 1331 & 35\%\\
\cellcolor{gray!6}{(40,50]} & \cellcolor{gray!6}{347} & \cellcolor{gray!6}{1067} & \cellcolor{gray!6}{33\%}\\
\addlinespace
(50,60] & 172 & 617 & 28\%\\
\cellcolor{gray!6}{(60,70]} & \cellcolor{gray!6}{36} & \cellcolor{gray!6}{263} & \cellcolor{gray!6}{14\%}\\
(70,118] & 18 & 299 & 6\%\\
\bottomrule
\end{tabular}
\end{table}

\begin{table}

\caption{\label{tab:tab-region-debwv}Samples of vermilion in the northern model by subregion used in the index.}
\centering
\begin{tabular}[t]{lrrl}
\toprule
Year & Samples & Positive Samples & Percent Positive\\
\midrule
\cellcolor{gray!6}{V1} & \cellcolor{gray!6}{362} & \cellcolor{gray!6}{1317} & \cellcolor{gray!6}{27\%}\\
V2 & 322 & 1448 & 22\%\\
\cellcolor{gray!6}{V3} & \cellcolor{gray!6}{924} & \cellcolor{gray!6}{1668} & \cellcolor{gray!6}{55\%}\\
V4 & 408 & 2164 & 19\%\\
\bottomrule
\end{tabular}
\end{table}

\begin{table}

\caption{\label{tab:tab-year-debwv}Samples of vermilion in the northern model by year.}
\centering
\begin{tabular}[t]{lrrl}
\toprule
Year & Samples & Positive Samples & Percent Positive\\
\midrule
\cellcolor{gray!6}{1988} & \cellcolor{gray!6}{136} & \cellcolor{gray!6}{422} & \cellcolor{gray!6}{32\%}\\
1989 & 170 & 446 & 38\%\\
\cellcolor{gray!6}{1990} & \cellcolor{gray!6}{65} & \cellcolor{gray!6}{122} & \cellcolor{gray!6}{53\%}\\
1991 & 73 & 135 & 54\%\\
\cellcolor{gray!6}{1992} & \cellcolor{gray!6}{168} & \cellcolor{gray!6}{467} & \cellcolor{gray!6}{36\%}\\
\addlinespace
1993 & 196 & 485 & 40\%\\
\cellcolor{gray!6}{1994} & \cellcolor{gray!6}{189} & \cellcolor{gray!6}{555} & \cellcolor{gray!6}{34\%}\\
1995 & 247 & 791 & 31\%\\
\cellcolor{gray!6}{1996} & \cellcolor{gray!6}{238} & \cellcolor{gray!6}{963} & \cellcolor{gray!6}{25\%}\\
1997 & 323 & 1312 & 25\%\\
\addlinespace
\cellcolor{gray!6}{1998} & \cellcolor{gray!6}{211} & \cellcolor{gray!6}{899} & \cellcolor{gray!6}{23\%}\\
\bottomrule
\end{tabular}
\end{table}

\FloatBarrier

\begin{table}

\caption{\label{tab:tab-model-select-debwv}Model selection for the DebWV onboard survey index for vermilion in the northern model .}
\centering
\begin{tabular}[t]{lrr}
\toprule
Model & Binomial $\Delta$AIC & Lognormal $\Delta$AIC\\
\midrule
\cellcolor{gray!6}{1} & \cellcolor{gray!6}{1011.38} & \cellcolor{gray!6}{422.42}\\
YEAR + MegaReef & 169.08 & 52.50\\
\cellcolor{gray!6}{YEAR + MegaReef + WAVE} & \cellcolor{gray!6}{120.32} & \cellcolor{gray!6}{42.13}\\
YEAR + MegaReef + WAVE + DEPTH bin & 0.00 & 0.00\\
\cellcolor{gray!6}{YEAR + WAVE + DEPTH bin} & \cellcolor{gray!6}{611.73} & \cellcolor{gray!6}{260.44}\\
\addlinespace
YEAR + DEPTH bin & 642.50 & 272.83\\
\cellcolor{gray!6}{YEAR + MegaReef + DEPTH bin} & \cellcolor{gray!6}{55.30} & \cellcolor{gray!6}{7.28}\\
\bottomrule
\end{tabular}
\end{table}

\FloatBarrier

\begin{table}

\caption{\label{tab:tab-index-debwv}Standardized index for the DebWV onboard survey index with log-scale standard errors and 95% highest
       posterior density (HPD) intervals for vermilion in the northern model .}
\centering
\begin{tabular}[t]{rrrrr}
\toprule
Year & Mean & logSE & lower HPD & upper HPD\\
\midrule
\cellcolor{gray!6}{1988} & \cellcolor{gray!6}{0.02} & \cellcolor{gray!6}{0.22} & \cellcolor{gray!6}{0.01} & \cellcolor{gray!6}{0.03}\\
1989 & 0.03 & 0.20 & 0.02 & 0.04\\
\cellcolor{gray!6}{1990} & \cellcolor{gray!6}{0.06} & \cellcolor{gray!6}{0.23} & \cellcolor{gray!6}{0.04} & \cellcolor{gray!6}{0.10}\\
1991 & 0.03 & 0.25 & 0.02 & 0.05\\
\cellcolor{gray!6}{1992} & \cellcolor{gray!6}{0.02} & \cellcolor{gray!6}{0.20} & \cellcolor{gray!6}{0.01} & \cellcolor{gray!6}{0.03}\\
\addlinespace
1993 & 0.03 & 0.20 & 0.02 & 0.04\\
\cellcolor{gray!6}{1994} & \cellcolor{gray!6}{0.02} & \cellcolor{gray!6}{0.20} & \cellcolor{gray!6}{0.01} & \cellcolor{gray!6}{0.03}\\
1995 & 0.02 & 0.20 & 0.01 & 0.03\\
\cellcolor{gray!6}{1996} & \cellcolor{gray!6}{0.02} & \cellcolor{gray!6}{0.20} & \cellcolor{gray!6}{0.01} & \cellcolor{gray!6}{0.02}\\
1997 & 0.02 & 0.20 & 0.01 & 0.03\\
\addlinespace
\cellcolor{gray!6}{1998} & \cellcolor{gray!6}{0.02} & \cellcolor{gray!6}{0.20} & \cellcolor{gray!6}{0.01} & \cellcolor{gray!6}{0.03}\\
\bottomrule
\end{tabular}
\end{table}

\FloatBarrier

\begin{figure}
\centering
\includegraphics{C:/Stock_Assessments/VRML_Assessment_2021/GitHub/Vermilion_2021/doc/indices/vermilion_DebWV_onboard_writeup_NCA_files/figure-latex/fig-dist-fits-debwv-1.pdf}
\caption{\label{fig:fig-dist-fits-debwv}Q-Q plot (top) of the positive observations lognormal gamma distributions and fitted values vs residuals for the Lognormal model (bottom).}
\end{figure}

\begin{figure}
\centering
\includegraphics{C:/Stock_Assessments/VRML_Assessment_2021/GitHub/Vermilion_2021/doc/indices/vermilion_DebWV_onboard_writeup_NCA_files/figure-latex/fig-areacpue-debwv-1.pdf}
\caption{\label{fig:fig-areacpue-debwv}Arithmetic mean of CPUE by region for vermilion from the filtered data.}
\end{figure}

\begin{figure}
\centering
\includegraphics{C:/Stock_Assessments/VRML_Assessment_2021/GitHub/Vermilion_2021/doc/indices/vermilion_DebWV_onboard_writeup_NCA_files/figure-latex/fig-propzero-debwv-1.pdf}
\caption{\label{fig:fig-propzero-debwv}Posterior predictive distribution of the proportion of zero observations in replicate data sets generated by the delta model with a vertical line representing the observed average.}
\end{figure}

\begin{figure}
\centering
\includegraphics{C:/Stock_Assessments/VRML_Assessment_2021/GitHub/Vermilion_2021/doc/indices/vermilion_DebWV_onboard_writeup_NCA_files/figure-latex/fig-posterior-mean-debwv-1.pdf}
\caption{\label{fig:fig-posterior-mean-debwv}Posterior predictive draws of the mean by year with a vertical line representing the observed average.}
\end{figure}

\begin{figure}
\centering
\includegraphics{C:/Stock_Assessments/VRML_Assessment_2021/GitHub/Vermilion_2021/doc/indices/vermilion_DebWV_onboard_writeup_NCA_files/figure-latex/fig-posterior-sd-debwv-1.pdf}
\caption{\label{fig:fig-posterior-sd-debwv}Posterior predictive draws of the standard deviation by year with a vertical line representing the observed average.}
\end{figure}

\begin{figure}
\centering
\includegraphics{C:/Stock_Assessments/VRML_Assessment_2021/GitHub/Vermilion_2021/doc/indices/vermilion_DebWV_onboard_writeup_NCA_files/figure-latex/fig-cpue-debwv-1.pdf}
\caption{\label{fig:fig-cpue-debwv}Standardized index and arithmetic mean of the CPUE from the filtered data. Each timeseries is scaled to its respective means.}
\end{figure}

\begin{figure}
\centering
\includegraphics{C:/Stock_Assessments/VRML_Assessment_2021/GitHub/Vermilion_2021/doc/indices/vermilion_DebWV_onboard_writeup_NCA_files/figure-latex/fig-Dbin-marginal-debwv-1.pdf}
\caption{\label{fig:fig-Dbin-marginal-debwv}Binomial marginal effects from the final model}
\end{figure}

\begin{figure}
\centering
\includegraphics{C:/Stock_Assessments/VRML_Assessment_2021/GitHub/Vermilion_2021/doc/indices/vermilion_DebWV_onboard_writeup_NCA_files/figure-latex/fig-Dpos-marginal-debwv-1.pdf}
\caption{\label{fig:fig-Dpos-marginal-debwv}Positive model marginal effects from the final model.}
\end{figure}

\hypertarget{refs}{}
\begin{CSLReferences}{1}{0}
\leavevmode\vadjust pre{\hypertarget{ref-Alverson1964}{}}%
Alverson, D.L., Pruter, a.T., and Ronholt, L.L. 1964. {A Study of Demersal Fishes and Fisheries of the Northeastern Pacific Ocean}. Institute of Fisheries, University of British Columbia.

\leavevmode\vadjust pre{\hypertarget{ref-Baskett2006}{}}%
Baskett, M.L., Yoklavich, M., and Love, M.S. 2006. {Predation, competition, and the recovery of overexploited fish stocks in marine reserves}. Canadian Journal of Fisheries and Aquatic Sciences \textbf{63}(6): 1214--1229. doi: \href{https://doi.org/10.1139/F06-013}{10.1139/F06-013}.

\leavevmode\vadjust pre{\hypertarget{ref-Budrick2016}{}}%
Budrick, J. 2016. {Evolutionary processes contributing to population structure in the rockfishes of the subgenus genus \emph{Rosicola}: implications for fishery management, stock assessment and prioritization of future analyses of structure in the genus \emph{Sebastes}.} PhD thesis, University of California, Berkeley.

\leavevmode\vadjust pre{\hypertarget{ref-Croker1940}{}}%
Croker, R.S. 1940. {Three Years of Fisheries Statistics on Marine Sport Fishing in California}. Transactions of the American Fisheries Society \textbf{69}(1).

\leavevmode\vadjust pre{\hypertarget{ref-Field2021}{}}%
Field, J.C., Miller, R.R., Santora, J.A., Tolimieri, N., Haltuch, M.A., Brodeur, R.D., Auth, T.D., Dick, E.J., Monk, M.H., Sakuma, K.M., and Wells, B.K. 2021. {Spatiotemporal patterns of variability in the abundance and distribution of winter-spawned pelagic juvenile rockfish in the California Current}. PLoS ONE \textbf{16}(5): 1--25. doi: \href{https://doi.org/10.1371/journal.pone.0251638}{10.1371/journal.pone.0251638}.

\leavevmode\vadjust pre{\hypertarget{ref-Frey1971}{}}%
Frey, H.W. 1971. {California's Living Marine Resources and Their Utilization. California Department of Fish and Game}.

\leavevmode\vadjust pre{\hypertarget{ref-Hannah2011}{}}%
Hannah, R.W., and Rankin, P.S. 2011. {Site fidelity and movement of eight species of pacific rockfish at a high-relief rocky reef on the Oregon coast}. North American Journal of Fisheries Management \textbf{31}(3): 483--494. doi: \href{https://doi.org/10.1080/02755947.2011.591239}{10.1080/02755947.2011.591239}.

\leavevmode\vadjust pre{\hypertarget{ref-Harry1961}{}}%
Harry, G., and Morgan, A.R. 1961. {History of the trawl fishery, 1884-1961}. Oregon Fish Commission Research Briefs \textbf{19}: 5--26.

\leavevmode\vadjust pre{\hypertarget{ref-Hyde2007}{}}%
Hyde, J. 2007. {The origin, evolution, and diversification of rockfishes of the genus Sebastes (Cuvier): insights into speciation and biogeography of temperate reef fishes}. PhD thesis, University of California San Diego.

\leavevmode\vadjust pre{\hypertarget{ref-Hyde2008b}{}}%
Hyde, J.R.; Kimbrell, C. A.; Budrick, J. E.; Lynn, E. A.; Vetter, R.D. 2008. {Cryptic speciation in the vermilion rockfish (\emph{Sebastes miniatus}) and the role of bathymetry in the speciation process}. Molecular Ecology \textbf{17}: 1122--1136. doi: \href{https://doi.org/10.1111/j.1365-294X.2007.03653.x}{10.1111/j.1365-294X.2007.03653.x}.

\leavevmode\vadjust pre{\hypertarget{ref-Hyde2009}{}}%
Hyde, J.R., and Vetter, R.D. 2009. {Population genetic structure in the redefined vermilion rockfish (\emph{Sebastes miniatus}) indicates limited larval dispersal and reveals natural management units}. Canadian Journal of Fisheries and Aquatic Sciences \textbf{66}(9): 1569--1581. doi: \href{https://doi.org/10.1139/F09-104}{10.1139/F09-104}.

\leavevmode\vadjust pre{\hypertarget{ref-Lea1999}{}}%
Lea, R.N., McAllister, R.D., and VenTresca, D.A. 1999. {Biological aspects of nearshore rockfishes of the \emph{Sebastes} from central California: with notes on ecologically related sport fishes.} Fish Bulletin No. 177: 112.

\leavevmode\vadjust pre{\hypertarget{ref-Lenarz1987}{}}%
Lenarz, W.H. 1987. {A history of California rockfish fisheries. In Proceedings of the International Rockfish Symposium.} \emph{In} International rockfish symposium.

\leavevmode\vadjust pre{\hypertarget{ref-Love2012a}{}}%
Love, M.S., Nishimoto, M., Clark, S., and Schroeder, D.M. 2012. {Recruitment of young-of-the-year fishes to natural and artificial offshore structure within central and southern California waters, 2008-2010}. Bulletin of Marine Science \textbf{88}(4): 863--882. doi: \href{https://doi.org/10.5343/bms.2011.1101}{10.5343/bms.2011.1101}.

\leavevmode\vadjust pre{\hypertarget{ref-Love2002}{}}%
Love, M., Yoklavich, M.M., and Thorsteinson, L. 2002. {The rockfishes of the northeast Pacific}. University of California Press, Berkeley, CA, USA.

\leavevmode\vadjust pre{\hypertarget{ref-Lowe2009}{}}%
Lowe, C.G., Anthony, K.M., Jarvis, E.T., Bellquist, L.F., and Love, M.S. 2009. {Site fidelity and movement patterns of groundfish associated with offshore petroleum platforms in the Santa Barbara Channel}. Marine and Coastal Fisheries \textbf{1}(1): 71--89. doi: \href{https://doi.org/10.1577/c08-047.1}{10.1577/c08-047.1}.

\leavevmode\vadjust pre{\hypertarget{ref-MacCall2002}{}}%
MacCall, A.D. 2002. {Fishery-management and stock-rebuilding prospects under conditions of low-frequency environmental variability and species interactions}. Bulletin of Marine Science \textbf{70}(2): 613--628.

\leavevmode\vadjust pre{\hypertarget{ref-Miller2014}{}}%
Miller, R.R., Field, J.C., Santora, J.A., Schroeder, I.D., Huff, D.D., Key, M., Pearson, D.E., and MacCall, A.D. 2014. {A spatially distinct history of the development of California groundfish fisheries}. PLoS ONE \textbf{9}(6). Public Library of Science. doi: \href{https://doi.org/10.1371/journal.pone.0099758}{10.1371/journal.pone.0099758}.

\leavevmode\vadjust pre{\hypertarget{ref-Monk2016}{}}%
Monk, M.H., Miller, R.R., Field, J., Dick, E.J., Wilson-Vandenberg, D., and Reilly, P. 2016. {Documentation for California Department of Fish and Wildlife's Onboard Sampling of the Rockfish and Lingcod Commercial Passenger Fishing Vessel Industry in Northern and Central California (1987-1998) as a relational database}. NOAA-TM-NMFS-SWFSC-558.

\leavevmode\vadjust pre{\hypertarget{ref-Phillips1964}{}}%
Phillips, J.B. 1964. {Life history studies on ten species of rockfish (genus \emph{Sebastodes})}. Fish Bulletin \textbf{126}.

\leavevmode\vadjust pre{\hypertarget{ref-Reilly1998}{}}%
Reilly, P.N., Wilson-Vandenberg, D., Wilson, C.E., and Mayer, K. 1998. {Onboard sampling of the rockfish and lingcod commercial passenger fishing vessel industry in northern and central California, January through December 1995.} Marine region, Admin. Rep. \textbf{98-1}: 1--110.

\leavevmode\vadjust pre{\hypertarget{ref-Schroeder2019}{}}%
Schroeder, I.D., Santora, J.A., Bograd, S.J., Hazen, E.L., Sakuma, K.M., Moore, A.M., Edwards, C.A., Wells, B.K., and Field, J.C. 2019. {Source water variability as a driver of rockfish recruitment in the California current ecosystem: implications for climate change and fisheries management}. Canadian Journal of Fisheries and Aquatic Sciences \textbf{76}(6): 950--960. doi: \href{https://doi.org/10.1139/cjfas-2017-0480}{10.1139/cjfas-2017-0480}.

\leavevmode\vadjust pre{\hypertarget{ref-Stachura2014}{}}%
Stachura, M.M., Essington, T.E., Mantua, N.J., Hollowed, A.B., Haltuch, M.A., Spencer, P.D., Branch, T.A., and Doyle, M.J. 2014. {Linking Northeast Pacific recruitment synchrony to environmental variability}. Fisheries Oceanography \textbf{23}(5): 389--408. doi: \href{https://doi.org/10.1111/fog.12066}{10.1111/fog.12066}.

\leavevmode\vadjust pre{\hypertarget{ref-Thorson2014}{}}%
Thorson, J.T., and Ward, E.J. 2014. {Accounting for vessel effects when standardizing catch rates from cooperative surveys}. Fisheries Research \textbf{155}: 168--176. Elsevier B.V. doi: \href{https://doi.org/10.1016/j.fishres.2014.02.036}{10.1016/j.fishres.2014.02.036}.

\leavevmode\vadjust pre{\hypertarget{ref-Walters2001}{}}%
Walters, C., and Kitchell, J.F. 2001. {Cultivation/depensation effects on juvenile survival and recruitment: Implications for the theory of fishing}. Canadian Journal of Fisheries and Aquatic Sciences \textbf{58}(1): 39--50. doi: \href{https://doi.org/10.1139/f00-160}{10.1139/f00-160}.

\leavevmode\vadjust pre{\hypertarget{ref-Young1969}{}}%
Young, P.H. 1969. {The California Partyboat Fishery 1947-1967}. Fish Bulletin \textbf{145}.

\end{CSLReferences}

\end{document}
