% Options for packages loaded elsewhere
\PassOptionsToPackage{unicode}{hyperref}
\PassOptionsToPackage{hyphens}{url}
%
\documentclass[
]{article}
\title{West Coast Groundfish Bottom Trawl Survey for vermilion in 2021}
\author{Melissa H. Monk}
\date{August 24, 2021}

\usepackage{amsmath,amssymb}
\usepackage{lmodern}
\usepackage{iftex}
\ifPDFTeX
  \usepackage[T1]{fontenc}
  \usepackage[utf8]{inputenc}
  \usepackage{textcomp} % provide euro and other symbols
\else % if luatex or xetex
  \usepackage{unicode-math}
  \defaultfontfeatures{Scale=MatchLowercase}
  \defaultfontfeatures[\rmfamily]{Ligatures=TeX,Scale=1}
\fi
% Use upquote if available, for straight quotes in verbatim environments
\IfFileExists{upquote.sty}{\usepackage{upquote}}{}
\IfFileExists{microtype.sty}{% use microtype if available
  \usepackage[]{microtype}
  \UseMicrotypeSet[protrusion]{basicmath} % disable protrusion for tt fonts
}{}
\makeatletter
\@ifundefined{KOMAClassName}{% if non-KOMA class
  \IfFileExists{parskip.sty}{%
    \usepackage{parskip}
  }{% else
    \setlength{\parindent}{0pt}
    \setlength{\parskip}{6pt plus 2pt minus 1pt}}
}{% if KOMA class
  \KOMAoptions{parskip=half}}
\makeatother
\usepackage{xcolor}
\IfFileExists{xurl.sty}{\usepackage{xurl}}{} % add URL line breaks if available
\IfFileExists{bookmark.sty}{\usepackage{bookmark}}{\usepackage{hyperref}}
\hypersetup{
  pdftitle={West Coast Groundfish Bottom Trawl Survey for vermilion in 2021},
  pdfauthor={Melissa H. Monk},
  hidelinks,
  pdfcreator={LaTeX via pandoc}}
\urlstyle{same} % disable monospaced font for URLs
\usepackage[margin=1in]{geometry}
\usepackage{longtable,booktabs,array}
\usepackage{calc} % for calculating minipage widths
% Correct order of tables after \paragraph or \subparagraph
\usepackage{etoolbox}
\makeatletter
\patchcmd\longtable{\par}{\if@noskipsec\mbox{}\fi\par}{}{}
\makeatother
% Allow footnotes in longtable head/foot
\IfFileExists{footnotehyper.sty}{\usepackage{footnotehyper}}{\usepackage{footnote}}
\makesavenoteenv{longtable}
\usepackage{graphicx}
\makeatletter
\def\maxwidth{\ifdim\Gin@nat@width>\linewidth\linewidth\else\Gin@nat@width\fi}
\def\maxheight{\ifdim\Gin@nat@height>\textheight\textheight\else\Gin@nat@height\fi}
\makeatother
% Scale images if necessary, so that they will not overflow the page
% margins by default, and it is still possible to overwrite the defaults
% using explicit options in \includegraphics[width, height, ...]{}
\setkeys{Gin}{width=\maxwidth,height=\maxheight,keepaspectratio}
% Set default figure placement to htbp
\makeatletter
\def\fps@figure{htbp}
\makeatother
\setlength{\emergencystretch}{3em} % prevent overfull lines
\providecommand{\tightlist}{%
  \setlength{\itemsep}{0pt}\setlength{\parskip}{0pt}}
\setcounter{secnumdepth}{5}
\usepackage{booktabs}
\usepackage{longtable}
\usepackage{array}
\usepackage{multirow}
\usepackage{wrapfig}
\usepackage{float}
\usepackage{colortbl}
\usepackage{pdflscape}
\usepackage{tabu}
\usepackage{threeparttable}
\usepackage[normalem]{ulem}
\usepackage{makecell}
\usepackage{xcolor}
\usepackage{placeins}
\ifLuaTeX
  \usepackage{selnolig}  % disable illegal ligatures
\fi

\begin{document}
\maketitle

{
\setcounter{tocdepth}{2}
\tableofcontents
}
\hypertarget{northwest-fisheries-science-center-west-coast-groundfish-bottom-trawl-survey}{%
\subsubsection{Northwest Fisheries Science Center West Coast Groundfish Bottom Trawl Survey}\label{northwest-fisheries-science-center-west-coast-groundfish-bottom-trawl-survey}}

In 2003, the NWFSC expanded the ongoing slope survey to include the continental
shelf. This survey, referred to in this document as the West Coast Groundfish
Bottom Trawl Survey (WCGBT Survey or WCGBTS), is conducted annually. It uses a r
andom-grid design covering the coastal waters from a depth of 55 m to 1,280 m
from late-May to early-October {[}@Keller2017{]}. Four chartered industry vessels
are used in most years.

*\textbf{WCGBTS Index: Data Preparation, Filtering, and Sample Sizes}

Vermilion rockfish were found during the WCGBTS, mainly off the coast of
California. Haul-level information collected during the
survey was extracted from the
\href{https://www.webapps.nwfsc.noaa.gov/data}{Northwest Fisheries Science Center database}
using code within the \texttt{nwfscSurvey} package, providing information on
catches (kg),
vessel,
year,
latitude (decimal degrees), and
area swept (hectares).

Just
two
records with positive tows were located north of the California-Oregon border
and were excluded from this analysis. Most of the positive tows were found in waters less than 200 m depth
(Table @ref\{tab:ndepth\}), and thus,
this analysis was truncated to waters with a depth of 300 m or less.
Positive tows were found south of 32.45 decimal degrees, which was used to represent the California-Mexico border.
This left,
fifty-eight
positive tows north of 34.50 decimal degrees and
one hundred twenty-three
positive tows south of 34.50 decimal degrees.
Positive encounters were just
7 and 15
percent of all tows for these two areas, respectively.

\emph{WCGBTS Index: Model Selection, Fits, and Diagnostics}

Sample sizes by factors selected to model, excluding WAVE can be found in Tables
\ref{tab:tab-depth-wcgbts} and \ref{tab:tab-year-wcgbts}.
We modeled retained catch per angler hour (CPUE; number of fish per angler hour)
a Bayesian delta-GLM model.

A Lognormal distribution was selected over a Gamma for the positive observation GLM.
The delta-GLM
method allows the linear predictors to differ between the binomial and positive models.
Based on AIC values from maximum likelihood fits Table \ref{tab:tab-model-select-wcgbts}),
a main effects model including
YEAR and LAT bin
was fit for the binomial model and a main
effects model including
YEAR and DEPTH bin and LAT bin
was fit for the Lognormal model.
Models were fit using the ``rstanarm'' R package (version 2.21.1). Posterior predictive
checks of the Bayesian model fit for the binomial model and the positive model
were all reasonable (Figures \ref{fig:fig-posterior-mean-wcgbts} and
\ref{fig:fig-posterior-sd-wcgbts}). The binomial model generated data sets with the
proportion zeros similar to the 84\% zeroes in the observed data
(Figure \ref{fig:fig-propzero-wcgbts}). The predicted marginal effects from
both the binomial and Lognormal models can be found in (Figures \ref{fig:fig-Dbin-marginal-wcgbts} and \ref{fig:fig-Dpos-marginal-wcgbts}). The
final index (Table \ref{tab:tab-index-wcgbts})
represents a similar trend to the arithmetic mean of the annual CPUE (Figure \ref{fig:fig-cpue-wcgbts}).

\newpage

\begin{table}

\caption{\label{tab:tab-region-wcgbts}Samples of vermilion rockfish in the southern model by subregion used in the index.}
\centering
\begin{tabular}[t]{lrrl}
\toprule
Subregion & Positive Samples & Samples & Percent Positive\\
\midrule
\cellcolor{gray!6}{32} & \cellcolor{gray!6}{14} & \cellcolor{gray!6}{64} & \cellcolor{gray!6}{22\%}\\
33 & 46 & 340 & 14\%\\
\cellcolor{gray!6}{34} & \cellcolor{gray!6}{58} & \cellcolor{gray!6}{339} & \cellcolor{gray!6}{17\%}\\
\bottomrule
\end{tabular}
\end{table}

\begin{table}

\caption{\label{tab:tab-depth-wcgbts}Positive samples of vermilion rockfish in the southern model by depth (fm).}
\centering
\begin{tabular}[t]{lrrl}
\toprule
Year & Positive Samples & Samples & Percent Positive\\
\midrule
\cellcolor{gray!6}{{}[55,75]} & \cellcolor{gray!6}{28} & \cellcolor{gray!6}{87} & \cellcolor{gray!6}{32\%}\\
(75,100] & 52 & 203 & 26\%\\
\cellcolor{gray!6}{(100,150]} & \cellcolor{gray!6}{31} & \cellcolor{gray!6}{156} & \cellcolor{gray!6}{20\%}\\
(150,200] & 5 & 127 & 4\%\\
\cellcolor{gray!6}{(200,300]} & \cellcolor{gray!6}{2} & \cellcolor{gray!6}{170} & \cellcolor{gray!6}{1\%}\\
\bottomrule
\end{tabular}
\end{table}

\begin{table}

\caption{\label{tab:tab-year-wcgbts}Samples of vermilion rockfish in the southern model by year.}
\centering
\begin{tabular}[t]{lrrl}
\toprule
Year & Positive Samples & Samples & Percent Positive\\
\midrule
\cellcolor{gray!6}{2003} & \cellcolor{gray!6}{3} & \cellcolor{gray!6}{32} & \cellcolor{gray!6}{9\%}\\
2005 & 5 & 38 & 13\%\\
\cellcolor{gray!6}{2006} & \cellcolor{gray!6}{3} & \cellcolor{gray!6}{45} & \cellcolor{gray!6}{7\%}\\
2007 & 7 & 50 & 14\%\\
\cellcolor{gray!6}{2008} & \cellcolor{gray!6}{7} & \cellcolor{gray!6}{47} & \cellcolor{gray!6}{15\%}\\
\addlinespace
2009 & 6 & 59 & 10\%\\
\cellcolor{gray!6}{2010} & \cellcolor{gray!6}{11} & \cellcolor{gray!6}{55} & \cellcolor{gray!6}{20\%}\\
2011 & 2 & 49 & 4\%\\
\cellcolor{gray!6}{2012} & \cellcolor{gray!6}{12} & \cellcolor{gray!6}{53} & \cellcolor{gray!6}{23\%}\\
2013 & 7 & 29 & 24\%\\
\addlinespace
\cellcolor{gray!6}{2014} & \cellcolor{gray!6}{8} & \cellcolor{gray!6}{52} & \cellcolor{gray!6}{15\%}\\
2015 & 9 & 53 & 17\%\\
\cellcolor{gray!6}{2016} & \cellcolor{gray!6}{15} & \cellcolor{gray!6}{52} & \cellcolor{gray!6}{29\%}\\
2017 & 9 & 50 & 18\%\\
\cellcolor{gray!6}{2018} & \cellcolor{gray!6}{10} & \cellcolor{gray!6}{53} & \cellcolor{gray!6}{19\%}\\
\addlinespace
2019 & 4 & 26 & 15\%\\
\bottomrule
\end{tabular}
\end{table}

\FloatBarrier

\begin{table}

\caption{\label{tab:tab-model-select-wcgbts}Model selection for the WCGBTS survey index for vermilion rockfish in the southern model.}
\centering
\begin{tabular}[t]{lrr}
\toprule
Model & Binomial $\Delta$AIC & Lognormal $\Delta$AIC\\
\midrule
\cellcolor{gray!6}{1} & \cellcolor{gray!6}{79.86} & \cellcolor{gray!6}{12.60}\\
YEAR + PASS & 88.32 & 12.30\\
\cellcolor{gray!6}{YEAR + PASS + DEPTH bin} & \cellcolor{gray!6}{2.07} & \cellcolor{gray!6}{1.87}\\
YEAR + PASS + DEPTH bin + LAT bin & 1.97 & 0.00\\
\cellcolor{gray!6}{YEAR + DEPTH bin + LAT bin} & \cellcolor{gray!6}{0.00} & \cellcolor{gray!6}{1.15}\\
\addlinespace
YEAR + LAT bin & 86.61 & 11.82\\
\cellcolor{gray!6}{YEAR + PASS + LAT bin} & \cellcolor{gray!6}{88.41} & \cellcolor{gray!6}{11.24}\\
\bottomrule
\end{tabular}
\end{table}

\FloatBarrier

\begin{table}

\caption{\label{tab:tab-index-wcgbts}Standardized index for the WCGBTS survey index with log-scale standard errors and 95\% highest
       posterior density (HPD) intervals for vermilion in the southern model.}
\centering
\begin{tabular}[t]{rrrrr}
\toprule
Year & Index & logSE & lower HPD & upper HPD\\
\midrule
\cellcolor{gray!6}{2003} & \cellcolor{gray!6}{0.78} & \cellcolor{gray!6}{1.26} & \cellcolor{gray!6}{0.03} & \cellcolor{gray!6}{4.10}\\
2005 & 0.34 & 1.87 & 0.00 & 2.40\\
\cellcolor{gray!6}{2006} & \cellcolor{gray!6}{0.13} & \cellcolor{gray!6}{1.16} & \cellcolor{gray!6}{0.01} & \cellcolor{gray!6}{0.62}\\
2007 & 0.47 & 1.17 & 0.02 & 2.34\\
\cellcolor{gray!6}{2008} & \cellcolor{gray!6}{0.83} & \cellcolor{gray!6}{1.01} & \cellcolor{gray!6}{0.07} & \cellcolor{gray!6}{3.58}\\
\addlinespace
2009 & 0.27 & 1.04 & 0.02 & 1.23\\
\cellcolor{gray!6}{2010} & \cellcolor{gray!6}{0.19} & \cellcolor{gray!6}{1.01} & \cellcolor{gray!6}{0.02} & \cellcolor{gray!6}{0.79}\\
2011 & 0.04 & 1.08 & 0.00 & 0.19\\
\cellcolor{gray!6}{2012} & \cellcolor{gray!6}{1.81} & \cellcolor{gray!6}{1.46} & \cellcolor{gray!6}{0.04} & \cellcolor{gray!6}{10.72}\\
2013 & 1.00 & 0.85 & 0.13 & 3.66\\
\addlinespace
\cellcolor{gray!6}{2014} & \cellcolor{gray!6}{3.72} & \cellcolor{gray!6}{1.01} & \cellcolor{gray!6}{0.30} & \cellcolor{gray!6}{16.34}\\
2015 & 0.10 & 0.93 & 0.01 & 0.41\\
\cellcolor{gray!6}{2016} & \cellcolor{gray!6}{0.22} & \cellcolor{gray!6}{0.88} & \cellcolor{gray!6}{0.03} & \cellcolor{gray!6}{0.82}\\
2017 & 0.16 & 0.77 & 0.03 & 0.52\\
\cellcolor{gray!6}{2018} & \cellcolor{gray!6}{0.61} & \cellcolor{gray!6}{0.91} & \cellcolor{gray!6}{0.07} & \cellcolor{gray!6}{2.34}\\
\addlinespace
2019 & 0.17 & 0.98 & 0.02 & 0.70\\
\bottomrule
\end{tabular}
\end{table}

\FloatBarrier

\begin{figure}
\centering
\includegraphics{C:/Stock_Assessments/VRML_Assessment_2021/GitHub/Vermilion_2021/doc/indices/vermilion_WCGBTS_writeup_SCA_files/figure-latex/fig-propzero-wcgbts-1.pdf}
\caption{\label{fig:fig-propzero-wcgbts}Posterior predictive distribution of the proportion of zero observations in replicate data sets generated by the delta model with a vertical line representing the observed average.}
\end{figure}

\begin{figure}
\centering
\includegraphics{C:/Stock_Assessments/VRML_Assessment_2021/GitHub/Vermilion_2021/doc/indices/vermilion_WCGBTS_writeup_SCA_files/figure-latex/fig-posterior-mean-wcgbts-1.pdf}
\caption{\label{fig:fig-posterior-mean-wcgbts}Posterior predictive draws of the mean by year with a vertical line of the raw data average.}
\end{figure}

\begin{figure}
\centering
\includegraphics{C:/Stock_Assessments/VRML_Assessment_2021/GitHub/Vermilion_2021/doc/indices/vermilion_WCGBTS_writeup_SCA_files/figure-latex/fig-posterior-sd-wcgbts-1.pdf}
\caption{\label{fig:fig-posterior-sd-wcgbts}Posterior predictive draws of the standard deviation by year with a vertical line representing the observed average.}
\end{figure}

\begin{figure}
\centering
\includegraphics{C:/Stock_Assessments/VRML_Assessment_2021/GitHub/Vermilion_2021/doc/indices/vermilion_WCGBTS_writeup_SCA_files/figure-latex/fig-cpue-wcgbts-1.pdf}
\caption{\label{fig:fig-cpue-wcgbts}Standardized index and arithmetic mean of the CPUE from the filtered data. Each timeseries is scaled to its respective means.}
\end{figure}

\begin{figure}
\centering
\includegraphics{C:/Stock_Assessments/VRML_Assessment_2021/GitHub/Vermilion_2021/doc/indices/vermilion_WCGBTS_writeup_SCA_files/figure-latex/fig-Dbin-marginal-wcgbts-1.pdf}
\caption{\label{fig:fig-Dbin-marginal-wcgbts}Binomial marginal effects from the final model}
\end{figure}

\begin{figure}
\centering
\includegraphics{C:/Stock_Assessments/VRML_Assessment_2021/GitHub/Vermilion_2021/doc/indices/vermilion_WCGBTS_writeup_SCA_files/figure-latex/fig-Dpos-marginal-wcgbts-1.pdf}
\caption{\label{fig:fig-Dpos-marginal-wcgbts}Positive model marginal effects from the final model.}
\end{figure}

\end{document}
