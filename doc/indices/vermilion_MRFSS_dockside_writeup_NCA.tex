% Options for packages loaded elsewhere
\PassOptionsToPackage{unicode}{hyperref}
\PassOptionsToPackage{hyphens}{url}
%
\documentclass[
]{article}
\usepackage{amsmath,amssymb}
\usepackage{lmodern}
\usepackage{iftex}
\ifPDFTeX
  \usepackage[T1]{fontenc}
  \usepackage[utf8]{inputenc}
  \usepackage{textcomp} % provide euro and other symbols
\else % if luatex or xetex
  \usepackage{unicode-math}
  \defaultfontfeatures{Scale=MatchLowercase}
  \defaultfontfeatures[\rmfamily]{Ligatures=TeX,Scale=1}
\fi
% Use upquote if available, for straight quotes in verbatim environments
\IfFileExists{upquote.sty}{\usepackage{upquote}}{}
\IfFileExists{microtype.sty}{% use microtype if available
  \usepackage[]{microtype}
  \UseMicrotypeSet[protrusion]{basicmath} % disable protrusion for tt fonts
}{}
\makeatletter
\@ifundefined{KOMAClassName}{% if non-KOMA class
  \IfFileExists{parskip.sty}{%
    \usepackage{parskip}
  }{% else
    \setlength{\parindent}{0pt}
    \setlength{\parskip}{6pt plus 2pt minus 1pt}}
}{% if KOMA class
  \KOMAoptions{parskip=half}}
\makeatother
\usepackage{xcolor}
\IfFileExists{xurl.sty}{\usepackage{xurl}}{} % add URL line breaks if available
\IfFileExists{bookmark.sty}{\usepackage{bookmark}}{\usepackage{hyperref}}
\hypersetup{
  pdftitle={MRFSS Dockside CPFV Index, 1980-1999, for vermilion in 2021},
  pdfauthor={Melissa H. Monk},
  hidelinks,
  pdfcreator={LaTeX via pandoc}}
\urlstyle{same} % disable monospaced font for URLs
\usepackage[margin=1in]{geometry}
\usepackage{longtable,booktabs,array}
\usepackage{calc} % for calculating minipage widths
% Correct order of tables after \paragraph or \subparagraph
\usepackage{etoolbox}
\makeatletter
\patchcmd\longtable{\par}{\if@noskipsec\mbox{}\fi\par}{}{}
\makeatother
% Allow footnotes in longtable head/foot
\IfFileExists{footnotehyper.sty}{\usepackage{footnotehyper}}{\usepackage{footnote}}
\makesavenoteenv{longtable}
\usepackage{graphicx}
\makeatletter
\def\maxwidth{\ifdim\Gin@nat@width>\linewidth\linewidth\else\Gin@nat@width\fi}
\def\maxheight{\ifdim\Gin@nat@height>\textheight\textheight\else\Gin@nat@height\fi}
\makeatother
% Scale images if necessary, so that they will not overflow the page
% margins by default, and it is still possible to overwrite the defaults
% using explicit options in \includegraphics[width, height, ...]{}
\setkeys{Gin}{width=\maxwidth,height=\maxheight,keepaspectratio}
% Set default figure placement to htbp
\makeatletter
\def\fps@figure{htbp}
\makeatother
\setlength{\emergencystretch}{3em} % prevent overfull lines
\providecommand{\tightlist}{%
  \setlength{\itemsep}{0pt}\setlength{\parskip}{0pt}}
\setcounter{secnumdepth}{5}
\usepackage{booktabs}
\usepackage{longtable}
\usepackage{array}
\usepackage{multirow}
\usepackage{wrapfig}
\usepackage{float}
\usepackage{colortbl}
\usepackage{pdflscape}
\usepackage{tabu}
\usepackage{threeparttable}
\usepackage[normalem]{ulem}
\usepackage{makecell}
\usepackage{xcolor}
\usepackage{placeins}
\ifLuaTeX
  \usepackage{selnolig}  % disable illegal ligatures
\fi

\title{MRFSS Dockside CPFV Index, 1980-1999, for vermilion in 2021}
\author{Melissa H. Monk}
\date{June 20, 2021}

\begin{document}
\maketitle

{
\setcounter{tocdepth}{2}
\tableofcontents
}
\hypertarget{mrfss-dockside-cpfv-index-1980-1999}{%
\subsubsection{MRFSS Dockside CPFV Index, 1980-1999}\label{mrfss-dockside-cpfv-index-1980-1999}}

From 1980 to 2003 the MRFSS program conducted dockside intercept surveys of
recreational CPFV fishing fleet. No MRFSS CPUE data are available for the years
1990-1992, due to a hiatus in sampling related to funding issues. Sampling of
California CPFVs north of Point Conception was further delayed, and CPFV samples
n 1993 and 1994 are limited to San Luis Obispo County.
For purposes of this assessment, the MRFSS time series was truncated at 1999 due
to sampling overlap with the
onboard observer program (i.e., the same observer samples the catch while
onboard the vessel and also conducts the dockside intercept survey for
the same vessel).

Each entry in the RecFIN Type 3 database corresponds to a
single fish examined by a sampler at a particular survey site. Since only a
subset of the catch may be sampled, each record also
identifies the total number of that species possessed by the group of anglers
being interviewed. The number of anglers and the hours fished are also recorded.
The data, as they exist in RecFIN, do not indicate which records
belong to the same boat trip. A description of the algorithms and process used to
aggregate the RecFIN records to the trip level is outlined in the Supplemental Materials
(``Identifying Trips in RecFIN'').

\textbf{MRFSS CPUE Index: Data Preparation, Filtering, and Sample Sizes}

Trips recorded as having the primary area fished in Mexico or occurring in bays, e.g.,
San Francisco Bay, were excluded before any filtering on species composition.
For indices representing only north of Pt. Conception, the years 1993-1994 were
excluded due to limited spatial coverage.

The Stephens-MacCall {[}-@Stephens2004{]} filtering approach was used to predict the
probability of of catching vermilion, based
on the species composition of the sampler observed catch in a given trip. Prior
to applying the Stephens-MacCall filter, we identified potentially informative
predictor species, i.e., species with sufficient sample sizes and temporal coverage
(at least 5\% of all trips) to inform the binomial model. The remaining
25 all co-occurred with vermilion in at least one trip
and were retained for the Stephens-MacCall logistic regression. Coefficients
from the Stephens-MacCall analysis (a binomial GLM) are positive
for species that are more likely to co-occur with vermilion,
and negative for species that are less likely to be caught with vermilion
(Figure \ref{fig:fig-sm-mrfss}).
The top five species with high probability of co-occurrence with vermilion include
Gopher, Flag, Copper, Canary, and Starry rockfishes, all of which are associated with rocky reef and kelp
habitats. The five species with the lowest probability of co-occurrence were
Chinook salmon, Widow and Greenspotted rockfishes, Chub mackerel and Rosy rockfish.

While the filter is useful in identifying co-occurring or non-occurring species
assuming all effort was exerted in pursuit of a single target, the targeting of
more than one species or species complex (``mixed trips'') can result in co-occurrence of species in the catch
that do not truly co-occur in terms of habitat
associations informative for an index of abundance. Stephens and MacCall
{[}-@Stephens2004{]} recommended including all trips above a threshold where the
false negatives and false positives are equally balanced. However, this does
not have any biological relevance and for this data set, we assume that if a
vermilion was landed, the anglers had to have fished in appropriate habitat,
especially given vermilion is strongly associated with rocky habitat.

The Stephens-MacCall filtering method identified the probability of occurrence
at which the rate of ``false
positives'' equals ``false negatives'' of 0.35. The
trips selected using this criteria were compared to an alternative method
including all the ``false positive'' trips, regardless of the probability of
encountering vermilion.
This assumes that if vermilion were caught, the anglers must have fished in
appropriate habitat during the trip. The catch included in this index is
``sampler-examined'' and the samplers are well trained in species identification.

Stephens and MacCall proposed filtering (excluding) trips from the index
standardization based on a criterion of balancing the number of false positives
and false negatives. False positives (FP) are trips that are predicted to catch
a vermilion based on the species composition of the catch, but did not. False
negatives (FN) are trips that were not predicted to catch a vermilion, given the
catch composition, but caught at least one. The threshold probability that
balances FP and FN excludes
1182
trips that did not catch a vermilion (52\%
of the trips), and 188
trips (8\% of the data) that
caught a vermilion. We retained the latter set of trips (FN), assuming that
catching a vermilion indicates that a non-negligible fraction of the fishing
effort occurred in habitat where vermilion occur. Only ``true negatives''
(the 1182
trips that neither caught vermilion, nor were predicted to catch them by the model)
were excluded from the index standardization. The final dataset selected included
1083 trips, 70\%
of which encountered vermilion. Sample sizes by the factors selected to model are in Tables
\ref{tab:tab-region-mrfss} and \ref{tab:tab-year-mrfss}.

\textbf{MRFSS CPUE Index: Model Selection, Fits, and Diagnostics}

Initial exploration of negative binomial models for this dataset proved to be
ill-fitting and the proportion of zeroes predicted by the Bayesian negative binomial
models were different enough from the fraction of zeroes in the raw data, that
a negative binomial model was not considered for model selection. We modeled catch
per angler hour (CPUE; number of fish per angler hour) a Bayesian delta-GLM model.
Models incorporating temporal (year, 2-month waves)
and geographic (region and primary area fished (inshore \textless3 nm, offshore \textgreater3 nm)
factors were evaluated. Two regions were defined based on counties, 1) Del Norte
to Santa Cruz (``N'') and 2) Monterey to San Luis Obispo (``C'') north of Pt. Conception.
For models that span counties north and south of Pt. Conception, Santa Barbara to
San Diego counties compose a third region (``S''). For models tha exclusively south
of Pt. Conception, the region represent individual counties. Indices with a year
and area interaction were not considered in model selection; trends in the average
CPUE by region were similar in the filtered data set (Figure \ref{fig:fig-areacpue-mrfss}).

A Lognormal model was
selected for the positive observation GLM by
a \(\Delta AIC\) of 62.35 over a Gamma model and supported by Q-Q plots of the positive observations fit to both distributions (Figure \ref{fig:fig-dist-fits-mrfss}). The delta-GLM
method allows the linear predictors to differ between the binomial and positive models.
Based on AIC values from maximum likelihood fits Table \ref{tab:tab-model-select-mrfss}),
a main effects model including
YEAR and SubRegion
was fit for the binomial model and a main
effects model including
NA
was fit for the Lognormal model.
Models were fit using the ``rstanarm'' R package (version 2.21.1). Posterior predictive
checks of the Bayesian model fit for the binomial model and the positive model
were all reasonable (Figures \ref{fig:fig-posterior-mean-mrfss} and
\ref{fig:fig-posterior-sd-mrfss}). The binomial model generated data sets with the
proportion zeros similar to the
30\%
zeroes in the observed data (Figure \ref{fig:fig-propzero-mrfss}).
The predicted marginal effects from both the binomial and Lognormal models
can be found in (Figures \ref{fig:fig-Dbin-marginal-mrfss} and
\ref{fig:fig-Dpos-marginal-mrfss}). The final index (Table \ref{tab:tab-index-mrfss})
represents a similar trend to the arithmetic mean of the annual CPUE (Figure \ref{fig:fig-cpue-mrfss}).

\newpage

\begin{table}

\caption{\label{tab:tab-region-mrfss}Samples of vermilion in the northern model by subregion used in the index.}
\centering
\begin{tabular}[t]{lrrl}
\toprule
Year & Samples & Positive Samples & Percent Positive\\
\midrule
\cellcolor{gray!6}{C} & \cellcolor{gray!6}{442} & \cellcolor{gray!6}{585} & \cellcolor{gray!6}{76\%}\\
N & 320 & 498 & 64\%\\
\bottomrule
\end{tabular}
\end{table}

\begin{table}

\caption{\label{tab:tab-year-mrfss}Samples of vermilion in the northern model by year.}
\centering
\begin{tabular}[t]{lrrl}
\toprule
Year & Samples & Positive Samples & Percent Positive\\
\midrule
\cellcolor{gray!6}{1980} & \cellcolor{gray!6}{31} & \cellcolor{gray!6}{57} & \cellcolor{gray!6}{54\%}\\
1981 & 14 & 32 & 44\%\\
\cellcolor{gray!6}{1982} & \cellcolor{gray!6}{24} & \cellcolor{gray!6}{41} & \cellcolor{gray!6}{59\%}\\
1983 & 19 & 33 & 58\%\\
\cellcolor{gray!6}{1984} & \cellcolor{gray!6}{34} & \cellcolor{gray!6}{59} & \cellcolor{gray!6}{58\%}\\
\addlinespace
1985 & 54 & 98 & 55\%\\
\cellcolor{gray!6}{1986} & \cellcolor{gray!6}{50} & \cellcolor{gray!6}{87} & \cellcolor{gray!6}{57\%}\\
1987 & 27 & 36 & 75\%\\
\cellcolor{gray!6}{1988} & \cellcolor{gray!6}{38} & \cellcolor{gray!6}{48} & \cellcolor{gray!6}{79\%}\\
1989 & 29 & 42 & 69\%\\
\addlinespace
\cellcolor{gray!6}{1995} & \cellcolor{gray!6}{31} & \cellcolor{gray!6}{41} & \cellcolor{gray!6}{76\%}\\
1996 & 104 & 129 & 81\%\\
\cellcolor{gray!6}{1997} & \cellcolor{gray!6}{127} & \cellcolor{gray!6}{162} & \cellcolor{gray!6}{78\%}\\
1998 & 98 & 119 & 82\%\\
\cellcolor{gray!6}{1999} & \cellcolor{gray!6}{82} & \cellcolor{gray!6}{99} & \cellcolor{gray!6}{83\%}\\
\bottomrule
\end{tabular}
\end{table}

\FloatBarrier

\begin{table}

\caption{\label{tab:tab-model-select-mrfss}Model selection for the MRFSS dockside survey index for vermilion in the northern model .}
\centering
\begin{tabular}[t]{lrr}
\toprule
Model & Binomial $\Delta$AIC & Lognormal $\Delta$AIC\\
\midrule
\cellcolor{gray!6}{1} & \cellcolor{gray!6}{65.99} & \cellcolor{gray!6}{106.17}\\
YEAR + SubRegion & 0.00 & 0.89\\
\cellcolor{gray!6}{YEAR + SubRegion + WAVE} & \cellcolor{gray!6}{1.77} & \cellcolor{gray!6}{3.03}\\
YEAR + SubRegion + WAVE + AREA X & 3.76 & 1.85\\
\cellcolor{gray!6}{YEAR + WAVE + AREA X} & \cellcolor{gray!6}{22.67} & \cellcolor{gray!6}{16.13}\\
\addlinespace
YEAR + AREA X & 20.13 & 14.44\\
\cellcolor{gray!6}{YEAR + SubRegion + AREA X} & \cellcolor{gray!6}{2.00} & \cellcolor{gray!6}{0.00}\\
\bottomrule
\end{tabular}
\end{table}

\FloatBarrier

\begin{table}

\caption{\label{tab:tab-index-mrfss}Standardized index for the MRFSS dockside survey index with log-scale standard errors and 95% highest
       posterior density (HPD) intervals for vermilion in the northern model .}
\centering
\begin{tabular}[t]{rrrrr}
\toprule
Year & Mean & logSE & lower HPD & upper HPD\\
\midrule
\cellcolor{gray!6}{1980} & \cellcolor{gray!6}{0.05} & \cellcolor{gray!6}{0.21} & \cellcolor{gray!6}{0.04} & \cellcolor{gray!6}{0.08}\\
1981 & 0.04 & 0.32 & 0.02 & 0.07\\
\cellcolor{gray!6}{1982} & \cellcolor{gray!6}{0.05} & \cellcolor{gray!6}{0.23} & \cellcolor{gray!6}{0.03} & \cellcolor{gray!6}{0.07}\\
1983 & 0.07 & 0.26 & 0.04 & 0.11\\
\cellcolor{gray!6}{1984} & \cellcolor{gray!6}{0.09} & \cellcolor{gray!6}{0.20} & \cellcolor{gray!6}{0.06} & \cellcolor{gray!6}{0.13}\\
\addlinespace
1985 & 0.06 & 0.16 & 0.04 & 0.08\\
\cellcolor{gray!6}{1986} & \cellcolor{gray!6}{0.07} & \cellcolor{gray!6}{0.17} & \cellcolor{gray!6}{0.05} & \cellcolor{gray!6}{0.10}\\
1987 & 0.08 & 0.21 & 0.05 & 0.12\\
\cellcolor{gray!6}{1988} & \cellcolor{gray!6}{0.11} & \cellcolor{gray!6}{0.18} & \cellcolor{gray!6}{0.08} & \cellcolor{gray!6}{0.15}\\
1989 & 0.09 & 0.20 & 0.06 & 0.13\\
\addlinespace
\cellcolor{gray!6}{1995} & \cellcolor{gray!6}{0.08} & \cellcolor{gray!6}{0.19} & \cellcolor{gray!6}{0.05} & \cellcolor{gray!6}{0.12}\\
1996 & 0.09 & 0.11 & 0.07 & 0.11\\
\cellcolor{gray!6}{1997} & \cellcolor{gray!6}{0.23} & \cellcolor{gray!6}{0.11} & \cellcolor{gray!6}{0.18} & \cellcolor{gray!6}{0.29}\\
1998 & 0.17 & 0.12 & 0.13 & 0.21\\
\cellcolor{gray!6}{1999} & \cellcolor{gray!6}{0.09} & \cellcolor{gray!6}{0.12} & \cellcolor{gray!6}{0.07} & \cellcolor{gray!6}{0.11}\\
\bottomrule
\end{tabular}
\end{table}

\FloatBarrier

\FloatBarrier

\begin{figure}
\includegraphics[width=0.6\linewidth]{C:/Stock_Assessments/VRML_Assessment_2021/Indices_of_Abundance/MRFSS_dockside/NCA/2021-05-28/MRFSS_dockside_SM_species} \caption{Species coefficients (blue bars) from the binomial GLM for presence/absence of vermilion rockfish in the CRFS private boat data. Horizontal black bars are $95\%$ confidence intervals.}\label{fig:fig-sm-mrfss}
\end{figure}

\begin{figure}
\centering
\includegraphics{C:/Stock_Assessments/VRML_Assessment_2021/GitHub/Vermilion_2021/doc/indices/vermilion_MRFSS_dockside_writeup_NCA_files/figure-latex/fig-dist-fits-mrfss-1.pdf}
\caption{\label{fig:fig-dist-fits-mrfss}Q-Q plot of the positive observations lognormal gamma distributions and fitted values vs residuals for the Lognormal .}
\end{figure}

\FloatBarrier

\begin{figure}
\centering
\includegraphics{C:/Stock_Assessments/VRML_Assessment_2021/GitHub/Vermilion_2021/doc/indices/vermilion_MRFSS_dockside_writeup_NCA_files/figure-latex/fig-areacpue-mrfss-1.pdf}
\caption{\label{fig:fig-areacpue-mrfss}Arithmetic mean of CPUE by region for vermilion from the filtered data.}
\end{figure}

\begin{figure}
\centering
\includegraphics{C:/Stock_Assessments/VRML_Assessment_2021/GitHub/Vermilion_2021/doc/indices/vermilion_MRFSS_dockside_writeup_NCA_files/figure-latex/fig-propzero-mrfss-1.pdf}
\caption{\label{fig:fig-propzero-mrfss}Posterior predictive distribution of the proportion of zero observations in replicate data sets generated by the delta model with a vertical line representing the observed average.}
\end{figure}

\begin{figure}
\centering
\includegraphics{C:/Stock_Assessments/VRML_Assessment_2021/GitHub/Vermilion_2021/doc/indices/vermilion_MRFSS_dockside_writeup_NCA_files/figure-latex/fig-posterior-mean-mrfss-1.pdf}
\caption{\label{fig:fig-posterior-mean-mrfss}Posterior predictive draws of the mean by year with a vertical line representing the observed average.}
\end{figure}

\begin{figure}
\centering
\includegraphics{C:/Stock_Assessments/VRML_Assessment_2021/GitHub/Vermilion_2021/doc/indices/vermilion_MRFSS_dockside_writeup_NCA_files/figure-latex/fig-posterior-sd-mrfss-1.pdf}
\caption{\label{fig:fig-posterior-sd-mrfss}Posterior predictive draws of the standard deviation by year with a vertical line representing the observed average.}
\end{figure}

\begin{figure}
\centering
\includegraphics{C:/Stock_Assessments/VRML_Assessment_2021/GitHub/Vermilion_2021/doc/indices/vermilion_MRFSS_dockside_writeup_NCA_files/figure-latex/fig-cpue-mrfss-1.pdf}
\caption{\label{fig:fig-cpue-mrfss}Standardized index and arithmetic mean of the CPUE from the filtered data. Each timeseries is scaled to its respective means.}
\end{figure}

\begin{figure}
\centering
\includegraphics{C:/Stock_Assessments/VRML_Assessment_2021/GitHub/Vermilion_2021/doc/indices/vermilion_MRFSS_dockside_writeup_NCA_files/figure-latex/fig-Dbin-marginal-mrfss-1.pdf}
\caption{\label{fig:fig-Dbin-marginal-mrfss}Binomial marginal effects from the final model}
\end{figure}

\begin{figure}
\centering
\includegraphics{C:/Stock_Assessments/VRML_Assessment_2021/GitHub/Vermilion_2021/doc/indices/vermilion_MRFSS_dockside_writeup_NCA_files/figure-latex/fig-Dpos-marginal-mrfss-1.pdf}
\caption{\label{fig:fig-Dpos-marginal-mrfss}Positive model marginal effects from the final model.}
\end{figure}

\end{document}
