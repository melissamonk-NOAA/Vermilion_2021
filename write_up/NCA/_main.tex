\PassOptionsToPackage{enable-debug,check-declarations}{expl3}
\RequirePackage{pdfmanagement-testphase}
\DeclareDocumentMetadata {  }
\ExplSyntaxOn
\pdfmanagement_add:nnn{Catalog}{Lang}{(enUS)}
\ExplSyntaxOff

% xmp metadata for pdf
% Originally used \usepackage[a-2a]{pdfx}
% \usepackage{hyperxmp} replaced it
% \RequirePackage{pdfmanagement-testphase} replaced it

\documentclass[11pt,
  english,
  a4paper,
]{article}
\usepackage{sa4ss}
\usepackage{amsmath,amssymb,array}
\usepackage{booktabs}

% From tagged-template.latex
\usepackage{lmodern}
\usepackage{ifxetex,ifluatex}
\ifnum 0\ifxetex 1\fi\ifluatex 1\fi=0 % if pdftex
  \usepackage[T1]{fontenc}
  \usepackage[utf8]{inputenc}
  \usepackage{textcomp} % provide euro and other symbols
\else % if luatex or xetex
  \usepackage{unicode-math}
  \defaultfontfeatures{Scale=MatchLowercase}
  \defaultfontfeatures[\rmfamily]{Ligatures=TeX,Scale=1}
\fi

% Use upquote if available, for straight quotes in verbatim environments
\IfFileExists{upquote.sty}{\usepackage{upquote}}{}
\IfFileExists{microtype.sty}{% use microtype if available
  \usepackage[]{microtype}
  \UseMicrotypeSet[protrusion]{basicmath} % disable protrusion for tt fonts
}{}
\makeatletter
\@ifundefined{KOMAClassName}{% if non-KOMA class
  \IfFileExists{parskip.sty}{%
    \usepackage{parskip}
  }{% else
    \setlength{\parindent}{0pt}
    \setlength{\parskip}{6pt plus 2pt minus 1pt}}
}{% if KOMA class
  \KOMAoptions{parskip=half}}
\makeatother
\usepackage{xcolor}
\IfFileExists{xurl.sty}{\usepackage{xurl}}{} % add URL line breaks if available
\hypersetup{
  pdftitle={Status of Vermilion Rockfish (Sebastes caurinus) along the California North of Pt. Conception US West coast in 2021},
  pdflang={en},
  hidelinks,
  pdfcreator={LaTeX via pandoc}}
\urlstyle{same} % disable monospaced font for URLs
\usepackage{longtable}
% Correct order of tables after \paragraph or \subparagraph
\usepackage{etoolbox}
\makeatletter
\patchcmd\longtable{\par}{\if@noskipsec\mbox{}\fi\par}{}{}
\makeatother
% Allow footnotes in longtable head/foot
\IfFileExists{footnotehyper.sty}{\usepackage{footnotehyper}}{\usepackage{footnote}}
\makesavenoteenv{longtable}
\usepackage{graphicx}
\makeatletter
\def\maxwidth{\ifdim\Gin@nat@width>\linewidth\linewidth\else\Gin@nat@width\fi}
\def\maxheight{\ifdim\Gin@nat@height>\textheight\textheight\else\Gin@nat@height\fi}
\makeatother
% Scale images if necessary, so that they will not overflow the page
% margins by default, and it is still possible to overwrite the defaults
% using explicit options in \includegraphics[width, height, ...]{}
\setkeys{Gin}{width=\maxwidth,height=\maxheight,keepaspectratio}
% Set default figure placement to htbp
\makeatletter
\def\fps@figure{htbp}
\makeatother
\setlength{\emergencystretch}{3em} % prevent overfull lines
\providecommand{\tightlist}{%
  \setlength{\itemsep}{0pt}\setlength{\parskip}{0pt}}
\setcounter{secnumdepth}{5}
\ifxetex
  % Load polyglossia as late as possible: uses bidi with RTL langages (e.g. Hebrew, Arabic)
  \usepackage{polyglossia}
  \setmainlanguage[]{english}
\else
  \usepackage[shorthands=off,main=english]{babel}
\fi

%Define cslreferences environment, required by pandoc 2.8
%https://github.com/rstudio/rmarkdown/issues/1649
\newlength{\csllabelwidth}
\setlength{\csllabelwidth}{3em}
\newlength{\cslhangindent}
\setlength{\cslhangindent}{1.5em}
% for Pandoc 2.8 to 2.10.1
\newenvironment{cslreferences}%
  {}%
  {\par}
% For Pandoc 2.11+
\newenvironment{CSLReferences}[2] % #1 hanging-ident, #2 entry spacing
 {% don't indent paragraphs
  \setlength{\parindent}{0pt}
  % turn on hanging indent if param 1 is 1
  \ifodd #1 \everypar{\setlength{\hangindent}{\cslhangindent}}\ignorespaces\fi
  % set entry spacing
  \ifnum #2 > 0
  \setlength{\parskip}{#2\baselineskip}
  \fi
 }%
 {}
\usepackage{calc}  % for \widthof, \maxof in minipage
\newcommand{\CSLBlock}[1]{#1\hfill\break}
\newcommand{\CSLLeftMargin}[1]{\parbox[t]{\csllabelwidth}{#1}}
\newcommand{\CSLRightInline}[1]{\parbox[t]{\linewidth - \csllabelwidth}{#1}\break}
\newcommand{\CSLIndent}[1]{\hspace{\cslhangindent}#1}


\providecommand{\tightlist}{%
  \setlength{\itemsep}{0pt}\setlength{\parskip}{0pt}}


\date{}
\newcommand{\trTitle}{Status of Vermilion Rockfish (\emph{Sebastes caurinus}) along the California North of Pt. Conception US West coast in 2021}
\newcommand{\trYear}{2021}
\newcommand{\trMonth}{June}
\newcommand{\trAuthsLong}{truetruetruetrue}
\newcommand{\trAuthsBack}{Monk, M.H., E.J. Dick, J.C. Field, E.M. Saas}
\newcommand{\trCitation}{
\begin{hangparas}{1em}{1}
\trAuthsBack{}. \trYear{}. \trTitle{}. Pacific Fisheries Management Council, Portland, Oregon. \pageref{LastPage}{}\,p.
\end{hangparas}}

\AtBeginDocument{\tagstructbegin{tag=Document}}
\AtEndDocument{\tagstructend}
\pretocmd{\maketitle}{\tagstructbegin{tag=H1}\tagmcbegin{tag=H1}}{}{}
\apptocmd{\maketitle}{\tagmcend\tagstructend}{}{}

\begin{document}

%%%%% Frontmatter %%%%%

% Footnote symbols in front matter
\renewcommand*{\thefootnote}{\fnsymbol{footnote}}

\small
\thispagestyle{empty}
\pagenumbering{roman}
\noindent
\begin{center}
\title{Status of Vermilion Rockfish (\emph{Sebastes caurinus}) along the California North of Pt. Conception US West coast in 2021}
% \textnormal{\MakeTextUppercase{\trTitle{}}}
\vspace{1.5cm}
{\Large\textbf\newline{Status of Vermilion Rockfish (\emph{Sebastes caurinus}) along the California North of Pt. Conception US West coast in 2021}}
\vfill
by\\
Melissa H. Monk\textsuperscript{1}\\
E. J. Dick\textsuperscript{1}\\
John C. Field\textsuperscript{1}\\
Emma M. Saas\textsuperscript{1}\vfill
\textsuperscript{1}Southwest Fisheries Science Center, U.S. Department of Commerce, National Oceanic and Atmospheric Administration, National Marine Fisheries Service, 110 McAllister Way, Santa Cruz, California 95060\vfill
\trMonth{} \trYear{}
\end{center}
\clearpage

% Fourth page: Colophon
\thispagestyle{empty}
\vspace*{\fill}
\begin{center}
\copyright{} Pacific Fisheries Management Council, \trYear{}\\
\end{center}
\par
\bigskip
\noindent
Correct citation for this publication:
\bigskip
\par
\trCitation{}
\clearpage

% Add TOC to pdf bookmarks (clickable pdf)
\pdfbookmark[1]{\contentsname}{toc}

% Table of contents page, lists of figures and tables
\tableofcontents\clearpage
\label{TRlastRoman}
\clearpage

% Table of contents
\newpage
\thispagestyle{empty} % to remove page number

% Settings for the main document
\pagenumbering{arabic}  % Regular page numbers
\pagestyle{plain}  % No page number on first page of main document, use 'empty'
\renewcommand*{\thefootnote}{\arabic{footnote}}  % Back to numeric footnotes
\setcounter{footnote}{0}  % And start at 1
\renewcommand{\headrulewidth}{0.5pt}
\renewcommand{\footrulewidth}{0.5pt}
%\pagestyle{fancy}\fancyhead[c]{Draft: Do not cite or circulate}

\newcommand{\lt}{\ensuremath <}
\newcommand{\gt}{\ensuremath >}

\pagebreak
\pagenumbering{roman}
\setcounter{page}{1}

\renewcommand{\thetable}{\roman{table}}
\renewcommand{\thefigure}{\roman{figure}}

\setlength\parskip{0.5em plus 0.1em minus 0.2em}

\tagstructbegin{tag=H1}\tagmcbegin{tag=H1}

\hypertarget{executive-summary}{%
\section*{Executive Summary}\label{executive-summary}}
\addcontentsline{toc}{section}{Executive Summary}

\leavevmode\tagmcend\tagstructend

\tagstructbegin{tag=H2}\tagmcbegin{tag=H2}

\hypertarget{stock}{%
\subsection*{Stock}\label{stock}}
\addcontentsline{toc}{subsection}{Stock}

\leavevmode\tagmcend\tagstructend

\tagstructbegin{tag=P}\tagmcbegin{tag=P}

This assessment reports the status of vermilion rockfish (\emph{Sebastes caurinus}) off the California North of Pt. Conception US West coast using data through xxxx.

\leavevmode\tagmcend\tagstructend\par

\tagstructbegin{tag=H2}\tagmcbegin{tag=H2}

\hypertarget{landings}{%
\subsection*{Landings}\label{landings}}
\addcontentsline{toc}{subsection}{Landings}

\leavevmode\tagmcend\tagstructend

\tagstructbegin{tag=P}\tagmcbegin{tag=P}

Replace text.

\leavevmode\tagmcend\tagstructend\par

\tagstructbegin{tag=H2}\tagmcbegin{tag=H2}

\hypertarget{data-and-assessment}{%
\subsection*{Data and Assessment}\label{data-and-assessment}}
\addcontentsline{toc}{subsection}{Data and Assessment}

\leavevmode\tagmcend\tagstructend

\tagstructbegin{tag=P}\tagmcbegin{tag=P}

Replace text.

\leavevmode\tagmcend\tagstructend\par

\tagstructbegin{tag=H2}\tagmcbegin{tag=H2}

\hypertarget{stock-biomass}{%
\subsection*{Stock Biomass}\label{stock-biomass}}
\addcontentsline{toc}{subsection}{Stock Biomass}

\leavevmode\tagmcend\tagstructend

\tagstructbegin{tag=P}\tagmcbegin{tag=P}

Replace text.

\leavevmode\tagmcend\tagstructend\par

\tagstructbegin{tag=H2}\tagmcbegin{tag=H2}

\hypertarget{recruitment}{%
\subsection*{Recruitment}\label{recruitment}}
\addcontentsline{toc}{subsection}{Recruitment}

\leavevmode\tagmcend\tagstructend

\tagstructbegin{tag=P}\tagmcbegin{tag=P}

Replace text.

\leavevmode\tagmcend\tagstructend\par

\tagstructbegin{tag=H2}\tagmcbegin{tag=H2}

\hypertarget{exploitation-status}{%
\subsection*{Exploitation Status}\label{exploitation-status}}
\addcontentsline{toc}{subsection}{Exploitation Status}

\leavevmode\tagmcend\tagstructend

\tagstructbegin{tag=P}\tagmcbegin{tag=P}

Replace text.

\leavevmode\tagmcend\tagstructend\par

\tagstructbegin{tag=H2}\tagmcbegin{tag=H2}

\hypertarget{reference-points}{%
\subsection*{Reference Points}\label{reference-points}}
\addcontentsline{toc}{subsection}{Reference Points}

\leavevmode\tagmcend\tagstructend

\tagstructbegin{tag=P}\tagmcbegin{tag=P}

Replace text.

\leavevmode\tagmcend\tagstructend\par

\tagstructbegin{tag=H2}\tagmcbegin{tag=H2}

\hypertarget{management-performance}{%
\subsection*{Management Performance}\label{management-performance}}
\addcontentsline{toc}{subsection}{Management Performance}

\leavevmode\tagmcend\tagstructend

\tagstructbegin{tag=P}\tagmcbegin{tag=P}

Replace text.

\leavevmode\tagmcend\tagstructend\par

\tagstructbegin{tag=H2}\tagmcbegin{tag=H2}

\hypertarget{unresolved-problems-and-major-uncertainties}{%
\subsection*{Unresolved Problems and Major Uncertainties}\label{unresolved-problems-and-major-uncertainties}}
\addcontentsline{toc}{subsection}{Unresolved Problems and Major Uncertainties}

\leavevmode\tagmcend\tagstructend

\tagstructbegin{tag=P}\tagmcbegin{tag=P}

Replace text.

\leavevmode\tagmcend\tagstructend\par

\tagstructbegin{tag=H2}\tagmcbegin{tag=H2}

\hypertarget{decision-table}{%
\subsection*{Decision Table}\label{decision-table}}
\addcontentsline{toc}{subsection}{Decision Table}

\leavevmode\tagmcend\tagstructend

\tagstructbegin{tag=P}\tagmcbegin{tag=P}

Replace text.

\leavevmode\tagmcend\tagstructend\par

\tagstructbegin{tag=H2}\tagmcbegin{tag=H2}

\hypertarget{research-and-data-needs}{%
\subsection*{Research and Data Needs}\label{research-and-data-needs}}
\addcontentsline{toc}{subsection}{Research and Data Needs}

\leavevmode\tagmcend\tagstructend

\tagstructbegin{tag=P}\tagmcbegin{tag=P}

Replace text.

\leavevmode\tagmcend\tagstructend\par

\pagebreak
\setlength{\parskip}{5mm plus1mm minus1mm}
\pagenumbering{arabic}
\setcounter{page}{1}
\renewcommand{\thefigure}{\arabic{figure}}
\renewcommand{\thetable}{\arabic{table}}
\setcounter{table}{0}
\setcounter{figure}{0}

\tagstructbegin{tag=H1}\tagmcbegin{tag=H1}

\hypertarget{introduction}{%
\section{Introduction}\label{introduction}}

\leavevmode\tagmcend\tagstructend

\tagstructbegin{tag=H2}\tagmcbegin{tag=H2}

\hypertarget{basic-information}{%
\subsection{Basic Information}\label{basic-information}}

\leavevmode\tagmcend\tagstructend

\tagstructbegin{tag=P}\tagmcbegin{tag=P}

This assessment reports the status of vermilion rockfish (\emph{Sebastes caurinus}) off the California North of Pt. Conception US West coast using data through xxxx.

\leavevmode\tagmcend\tagstructend\par

\tagstructbegin{tag=H2}\tagmcbegin{tag=H2}

\hypertarget{life-history}{%
\subsection{Life History}\label{life-history}}

\leavevmode\tagmcend\tagstructend

\tagstructbegin{tag=P}\tagmcbegin{tag=P}

Replace text.

\leavevmode\tagmcend\tagstructend\par

\tagstructbegin{tag=H2}\tagmcbegin{tag=H2}

\hypertarget{ecosystem-considerations}{%
\subsection{Ecosystem Considerations}\label{ecosystem-considerations}}

\leavevmode\tagmcend\tagstructend

\tagstructbegin{tag=P}\tagmcbegin{tag=P}

Replace text.

\leavevmode\tagmcend\tagstructend\par

\tagstructbegin{tag=H2}\tagmcbegin{tag=H2}

\hypertarget{historical-and-current-fishery-information}{%
\subsection{Historical and Current Fishery Information}\label{historical-and-current-fishery-information}}

\leavevmode\tagmcend\tagstructend

\tagstructbegin{tag=P}\tagmcbegin{tag=P}

Replace text.

\leavevmode\tagmcend\tagstructend\par

\tagstructbegin{tag=H2}\tagmcbegin{tag=H2}

\hypertarget{summary-of-management-history-and-performance}{%
\subsection{Summary of Management History and Performance}\label{summary-of-management-history-and-performance}}

\leavevmode\tagmcend\tagstructend

\tagstructbegin{tag=P}\tagmcbegin{tag=P}

Replace text.

\leavevmode\tagmcend\tagstructend\par

\tagstructbegin{tag=H2}\tagmcbegin{tag=H2}

\hypertarget{foreign-fisheries}{%
\subsection{Foreign Fisheries}\label{foreign-fisheries}}

\leavevmode\tagmcend\tagstructend

\tagstructbegin{tag=P}\tagmcbegin{tag=P}

Replace text.

\leavevmode\tagmcend\tagstructend\par

\tagstructbegin{tag=H1}\tagmcbegin{tag=H1}

\hypertarget{data}{%
\section{Data}\label{data}}

\leavevmode\tagmcend\tagstructend

\tagstructbegin{tag=P}\tagmcbegin{tag=P}

A description of each data source is provided below (Figure \ref{fig:data-plot}).

\leavevmode\tagmcend\tagstructend\par

\tagstructbegin{tag=H2}\tagmcbegin{tag=H2}

\hypertarget{fishery-dependent-data}{%
\subsection{Fishery-Dependent Data}\label{fishery-dependent-data}}

\leavevmode\tagmcend\tagstructend

\tagstructbegin{tag=H2}\tagmcbegin{tag=H2}

\hypertarget{fishery-independent-data}{%
\subsection{Fishery-Independent Data}\label{fishery-independent-data}}

\leavevmode\tagmcend\tagstructend

\tagstructbegin{tag=H3}\tagmcbegin{tag=H3}

\hypertarget{section}{%
\subsubsection{\texorpdfstring{\acrlong{s-aslope}}{}}\label{section}}

\leavevmode\tagmcend\tagstructend

\tagstructbegin{tag=P}\tagmcbegin{tag=P}

The \gls{s-aslope} operated during the months of October to November aboard the R/V \emph{Miller Freeman}. Partial survey coverage of the US west coast occurred during the years 1988-1996 and complete coverage (north of 34\textdegree 30\textquotesingle S) during the years 1997 and 1999-2001. Typically, only these four years that are seen as complete surveys are included in assessments.

\leavevmode\tagmcend\tagstructend\par

\tagstructbegin{tag=H3}\tagmcbegin{tag=H3}

\hypertarget{section-1}{%
\subsubsection{\texorpdfstring{\acrlong{s-ccfrp}}{}}\label{section-1}}

\leavevmode\tagmcend\tagstructend

\tagstructbegin{tag=P}\tagmcbegin{tag=P}

Since 2007, the \Gls{s-ccfrp} has monitored several areas in California to evaluate the performance of \Gls{mpa}s and understand nearshore fish populations {\tagstructbegin{tag=Reference}\tagmcbegin{tag=Reference}(Wendt and Starr 2009; Starr et al. 2015)\leavevmode\tagmcend\tagstructend}. In 2017, the survey expanded beyond the four \Gls{mpa}s in central California (Año Nuevo, Point Lobos, Point Buchon, and Piedras Blancas) to include the entire California coast. Fish are collected by volunteer anglers aboard \Gls{cpfv}s guided by one of the following academic institutions based on proximity to fishing location: Humboldt State University; Bodega Marine Laboratories; Moss Landing Marine Laboratories; Cal Poly San Luis Obispo; University of California, Santa Barbara; and Scripps Institution of Oceanography.

\leavevmode\tagmcend\tagstructend\par

\tagstructbegin{tag=P}\tagmcbegin{tag=P}

Surveys consist of fishing with hook-and-line gear for 30-45 minutes within randomly chosen 500 by 500 m grid cells within and outside \Gls{mpa}s. Prior to 2017, all fish were measured for length and release or descended to depth; since then, some were sampled for otoliths and fin clips.

\leavevmode\tagmcend\tagstructend\par

\tagstructbegin{tag=H3}\tagmcbegin{tag=H3}

\hypertarget{section-2}{%
\subsubsection{\texorpdfstring{\acrlong{s-tri}}{}}\label{section-2}}

\leavevmode\tagmcend\tagstructend

\tagstructbegin{tag=P}\tagmcbegin{tag=P}

The \gls{s-tri} was first conducted by the \gls{afsc} in 1977, and the survey continued until 2004 {\tagstructbegin{tag=Reference}\tagmcbegin{tag=Reference}(Weinberg et al. 2002)\leavevmode\tagmcend\tagstructend}. Its basic design was a series of equally-spaced east-to-west transects across the continential shelf from which searches for tows in a specific depth range were initiated. The survey design changed slightly over time. In general, all of the surveys were conducted in the mid summer through early fall. The 1977 survey was conducted from early July through late September. The surveys from 1980 through 1989 were conducted from mid-July to late September. The 1992 survey was conducted from mid July through early October. The 1995 survey was conducted from early June through late August. The 1998 survey was conducted from early June through early August. Finally, the 2001 and 2004 surveys were conducted from May to July.

\leavevmode\tagmcend\tagstructend\par

\tagstructbegin{tag=P}\tagmcbegin{tag=P}

Haul depths ranged from 91-457 m during the 1977 survey with no hauls shallower than 91 m. Due to haul performance issues and truncated sampling with respect to depth, the data from 1977 were omitted from this analysis. The surveys in 1980, 1983, and 1986 covered the US West Coast south to 36.8\textdegree N latitude and a depth range of 55-366 m. The surveys in 1989 and 1992 covered the same depth range but extended the southern range to 34.5\textdegree N (near Point Conception). From 1995 through 2004, the surveys covered the depth range 55-500 m and surveyed south to 34.5\textdegree N. In 2004, the final year of the \gls{s-tri} series, the \gls{nwfsc} \gls{fram} conducted the survey following similar protocols to earlier years.

\leavevmode\tagmcend\tagstructend\par

\tagstructbegin{tag=H3}\tagmcbegin{tag=H3}

\hypertarget{section-3}{%
\subsubsection{\texorpdfstring{\acrlong{s-wcgbt}}{}}\label{section-3}}

\leavevmode\tagmcend\tagstructend

\tagstructbegin{tag=P}\tagmcbegin{tag=P}

The \Gls{s-wcgbt} is based on a random-grid design; covering the coastal waters from a depth of 55-1,280 m {\tagstructbegin{tag=Reference}\tagmcbegin{tag=Reference}(Bradburn, Keller, and Horness 2011)\leavevmode\tagmcend\tagstructend}. This design generally uses four industry-chartered vessels per year assigned to a roughly equal number of randomly selected grid cells and divided into two `passes' of the coast. Two vessels fish from north to south during each pass between late May to early October. This design therefore incorporates both vessel-to-vessel differences in catchability, as well as variance associated with selecting a relatively small number (approximately 700) of possible cells from a very large set of possible cells spread from the Mexican to the Canadian borders.

\leavevmode\tagmcend\tagstructend\par

\tagstructbegin{tag=H2}\tagmcbegin{tag=H2}

\hypertarget{biological-data}{%
\subsection{Biological Data}\label{biological-data}}

\leavevmode\tagmcend\tagstructend

\tagstructbegin{tag=H3}\tagmcbegin{tag=H3}

\hypertarget{natural-mortality}{%
\subsubsection{Natural Mortality}\label{natural-mortality}}

\leavevmode\tagmcend\tagstructend

\tagstructbegin{tag=H3}\tagmcbegin{tag=H3}

\hypertarget{maturation-and-fecundity}{%
\subsubsection{Maturation and Fecundity}\label{maturation-and-fecundity}}

\leavevmode\tagmcend\tagstructend

\tagstructbegin{tag=H3}\tagmcbegin{tag=H3}

\hypertarget{sex-ratio}{%
\subsubsection{Sex Ratio}\label{sex-ratio}}

\leavevmode\tagmcend\tagstructend

\tagstructbegin{tag=H3}\tagmcbegin{tag=H3}

\hypertarget{length-weight-relationship}{%
\subsubsection{Length-Weight Relationship}\label{length-weight-relationship}}

\leavevmode\tagmcend\tagstructend

\tagstructbegin{tag=H3}\tagmcbegin{tag=H3}

\hypertarget{growth-length-at-age}{%
\subsubsection{Growth (Length-at-Age)}\label{growth-length-at-age}}

\leavevmode\tagmcend\tagstructend

\tagstructbegin{tag=H3}\tagmcbegin{tag=H3}

\hypertarget{ageing-precision-and-bias}{%
\subsubsection{Ageing Precision and Bias}\label{ageing-precision-and-bias}}

\leavevmode\tagmcend\tagstructend

\tagstructbegin{tag=H2}\tagmcbegin{tag=H2}

\hypertarget{environmental-and-ecosystem-data}{%
\subsection{Environmental and Ecosystem Data}\label{environmental-and-ecosystem-data}}

\leavevmode\tagmcend\tagstructend

\tagstructbegin{tag=H1}\tagmcbegin{tag=H1}

\hypertarget{assessment-model}{%
\section{Assessment Model}\label{assessment-model}}

\leavevmode\tagmcend\tagstructend

\tagstructbegin{tag=H2}\tagmcbegin{tag=H2}

\hypertarget{summary-of-previous-assessments-and-reviews}{%
\subsection{Summary of Previous Assessments and Reviews}\label{summary-of-previous-assessments-and-reviews}}

\leavevmode\tagmcend\tagstructend

\tagstructbegin{tag=H3}\tagmcbegin{tag=H3}

\hypertarget{history-of-modeling-approaches-not-required-for-an-update-assessment}{%
\subsubsection{History of Modeling Approaches (not required for an update assessment)}\label{history-of-modeling-approaches-not-required-for-an-update-assessment}}

\leavevmode\tagmcend\tagstructend

\tagstructbegin{tag=H3}\tagmcbegin{tag=H3}

\hypertarget{most-recent-star-panel-and-ssc-recommendations-not-required-for-an-update-assessment}{%
\subsubsection{Most Recent STAR Panel and SSC Recommendations (not required for an update assessment)}\label{most-recent-star-panel-and-ssc-recommendations-not-required-for-an-update-assessment}}

\leavevmode\tagmcend\tagstructend

\tagstructbegin{tag=H3}\tagmcbegin{tag=H3}

\hypertarget{response-to-groundfish-subcommittee-requests-not-required-in-draft}{%
\subsubsection{Response to Groundfish Subcommittee Requests (not required in draft)}\label{response-to-groundfish-subcommittee-requests-not-required-in-draft}}

\leavevmode\tagmcend\tagstructend

\tagstructbegin{tag=H2}\tagmcbegin{tag=H2}

\hypertarget{model-structure-and-assumptions}{%
\subsection{Model Structure and Assumptions}\label{model-structure-and-assumptions}}

\leavevmode\tagmcend\tagstructend

\tagstructbegin{tag=H3}\tagmcbegin{tag=H3}

\hypertarget{model-changes-from-the-last-assessment-not-required-for-an-update-assessment}{%
\subsubsection{Model Changes from the Last Assessment (not required for an update assessment)}\label{model-changes-from-the-last-assessment-not-required-for-an-update-assessment}}

\leavevmode\tagmcend\tagstructend

\tagstructbegin{tag=H3}\tagmcbegin{tag=H3}

\hypertarget{modeling-platform-and-structure}{%
\subsubsection{Modeling Platform and Structure}\label{modeling-platform-and-structure}}

\leavevmode\tagmcend\tagstructend

\tagstructbegin{tag=P}\tagmcbegin{tag=P}

General model specifications (e.g., executable version, model structure, definition of fleets and areas)

\leavevmode\tagmcend\tagstructend\par

\tagstructbegin{tag=H3}\tagmcbegin{tag=H3}

\hypertarget{model-parameters}{%
\subsubsection{Model Parameters}\label{model-parameters}}

\leavevmode\tagmcend\tagstructend

\tagstructbegin{tag=P}\tagmcbegin{tag=P}

Describe estimated vs.~fixed parameters, priors

\leavevmode\tagmcend\tagstructend\par

\tagstructbegin{tag=H3}\tagmcbegin{tag=H3}

\hypertarget{key-assumptions-and-structural-choices}{%
\subsubsection{Key Assumptions and Structural Choices}\label{key-assumptions-and-structural-choices}}

\leavevmode\tagmcend\tagstructend

\tagstructbegin{tag=H2}\tagmcbegin{tag=H2}

\hypertarget{base-model-results}{%
\subsection{Base Model Results}\label{base-model-results}}

\leavevmode\tagmcend\tagstructend

\tagstructbegin{tag=H3}\tagmcbegin{tag=H3}

\hypertarget{parameter-estimates}{%
\subsubsection{Parameter Estimates}\label{parameter-estimates}}

\leavevmode\tagmcend\tagstructend

\tagstructbegin{tag=H3}\tagmcbegin{tag=H3}

\hypertarget{fits-to-the-data}{%
\subsubsection{Fits to the Data}\label{fits-to-the-data}}

\leavevmode\tagmcend\tagstructend

\tagstructbegin{tag=H3}\tagmcbegin{tag=H3}

\hypertarget{population-trajectory}{%
\subsubsection{Population Trajectory}\label{population-trajectory}}

\leavevmode\tagmcend\tagstructend

\tagstructbegin{tag=H3}\tagmcbegin{tag=H3}

\hypertarget{reference-points-1}{%
\subsubsection{Reference Points}\label{reference-points-1}}

\leavevmode\tagmcend\tagstructend

\tagstructbegin{tag=H2}\tagmcbegin{tag=H2}

\hypertarget{model-diagnostics}{%
\subsection{Model Diagnostics}\label{model-diagnostics}}

\leavevmode\tagmcend\tagstructend

\tagstructbegin{tag=P}\tagmcbegin{tag=P}

Describe all diagnostics

\leavevmode\tagmcend\tagstructend\par

\tagstructbegin{tag=H3}\tagmcbegin{tag=H3}

\hypertarget{convergence}{%
\subsubsection{Convergence}\label{convergence}}

\leavevmode\tagmcend\tagstructend

\tagstructbegin{tag=H3}\tagmcbegin{tag=H3}

\hypertarget{sensitivity-analyses}{%
\subsubsection{Sensitivity Analyses}\label{sensitivity-analyses}}

\leavevmode\tagmcend\tagstructend

\tagstructbegin{tag=H3}\tagmcbegin{tag=H3}

\hypertarget{retrospective-analysis}{%
\subsubsection{Retrospective Analysis}\label{retrospective-analysis}}

\leavevmode\tagmcend\tagstructend

\tagstructbegin{tag=H3}\tagmcbegin{tag=H3}

\hypertarget{likelihood-profiles}{%
\subsubsection{Likelihood Profiles}\label{likelihood-profiles}}

\leavevmode\tagmcend\tagstructend

\tagstructbegin{tag=H3}\tagmcbegin{tag=H3}

\hypertarget{unresolved-problems-and-major-uncertainties-1}{%
\subsubsection{Unresolved Problems and Major Uncertainties}\label{unresolved-problems-and-major-uncertainties-1}}

\leavevmode\tagmcend\tagstructend

\tagstructbegin{tag=H1}\tagmcbegin{tag=H1}

\hypertarget{management}{%
\section{Management}\label{management}}

\leavevmode\tagmcend\tagstructend

\tagstructbegin{tag=H2}\tagmcbegin{tag=H2}

\hypertarget{reference-points-2}{%
\subsection{Reference Points}\label{reference-points-2}}

\leavevmode\tagmcend\tagstructend

\tagstructbegin{tag=H2}\tagmcbegin{tag=H2}

\hypertarget{unresolved-problems-and-major-uncertainties-2}{%
\subsection{Unresolved Problems and Major Uncertainties}\label{unresolved-problems-and-major-uncertainties-2}}

\leavevmode\tagmcend\tagstructend

\tagstructbegin{tag=H2}\tagmcbegin{tag=H2}

\hypertarget{harvest-projections-and-decision-tables}{%
\subsection{Harvest Projections and Decision Tables}\label{harvest-projections-and-decision-tables}}

\leavevmode\tagmcend\tagstructend

\tagstructbegin{tag=H2}\tagmcbegin{tag=H2}

\hypertarget{evaluation-of-scientific-uncertainty}{%
\subsection{Evaluation of Scientific Uncertainty}\label{evaluation-of-scientific-uncertainty}}

\leavevmode\tagmcend\tagstructend

\tagstructbegin{tag=H2}\tagmcbegin{tag=H2}

\hypertarget{research-and-data-needs-1}{%
\subsection{Research and Data Needs}\label{research-and-data-needs-1}}

\leavevmode\tagmcend\tagstructend

\tagstructbegin{tag=H1}\tagmcbegin{tag=H1}

\hypertarget{acknowledgments}{%
\section{Acknowledgments}\label{acknowledgments}}

\leavevmode\tagmcend\tagstructend

\tagstructbegin{tag=P}\tagmcbegin{tag=P}

Here are all the mad props!

\leavevmode\tagmcend\tagstructend\par

\clearpage

\tagstructbegin{tag=H1}\tagmcbegin{tag=H1}

\hypertarget{references}{%
\section{References}\label{references}}

\leavevmode\tagmcend\tagstructend

\tagstructbegin{tag=BibEntry}\tagmcbegin{tag=BibEntry}

\hypertarget{refs}{}
\begin{CSLReferences}{1}{0}
\leavevmode\vadjust pre{\hypertarget{ref-bradburn_2003_2011}{}}%
Bradburn, M. J., A. A Keller, and B. H. Horness. 2011. {``The 2003 to 2008 {US} {West} {Coast} Bottom Trawl Surveys of Groundfish Resources Off {Washington}, {Oregon}, and {California}: Estimates of Distribution, Abundance, Length, and Age Composition.''} US Department of Commerce, National Oceanic; Atmospheric Administration, National Marine Fisheries Service.

\leavevmode\vadjust pre{\hypertarget{ref-Starr2015}{}}%
Starr, R. M., D. E. Wendt, C. L. Barnes, C. I. Marks, D. Malone, G. Waltz, K. T. Schmidt, et al. 2015. {``Variation in Responses of Fishes Across Multiple Reserves Within a Network of Marine Protected Areas in Temperate Waters.''} \emph{PLoS One2} 10 (3): p.e0118502.

\leavevmode\vadjust pre{\hypertarget{ref-weinberg_2001_2002}{}}%
Weinberg, K. L., M. E. Wilkins, F. R. Shaw, and M. Zimmermann. 2002. {``The 2001 {Pacific} {West} {Coast} Bottom Trawl Survey of Groundfish Resources: Estimates of Distribution, Abundance and Length and Age Composition.''} \{NOAA\} \{Technical\} \{Memorandum\} NMFS-AFSC-128. U.S. Department of Commerce.

\leavevmode\vadjust pre{\hypertarget{ref-Wendt2009}{}}%
Wendt, D. E., and R. M. Starr. 2009. {``Collaborative Research: An Effective Way to Collect Data for Stock Assessments and Evaluate Marine Protected Areas in {C}alifornia.''} \emph{Marine and Coastal Fisheries: Dynamics, Management, and Ecosystem Science.} 1: 315--24.

\end{CSLReferences}

\leavevmode\tagmcend\tagstructend

\clearpage

\tagstructbegin{tag=H1}\tagmcbegin{tag=H1}

\hypertarget{tables}{%
\section{Tables}\label{tables}}

\leavevmode\tagmcend\tagstructend

\clearpage

\tagstructbegin{tag=H1}\tagmcbegin{tag=H1}

\hypertarget{figures}{%
\section{Figures}\label{figures}}

\leavevmode\tagmcend\tagstructend

\tagstructbegin{tag=Figure,alttext={Summary of data sources used in the base model.}}\tagmcbegin{tag=Figure}

\begin{figure}
\centering
\includegraphics[width=1\textwidth,height=1\textheight]{C:/Stock_Assessments/VRML_Assessment_2021/Model_files/NCA/Verm21NoCA_045_add_GF_Ecology_ages_and_lengths/plots/data_plot2.png}
\caption{Summary of data sources used in the base model.\label{fig:data-plot}}
\end{figure}

\tagmcend\tagstructend
\end{document}
